% !TeX root = robust_global_redistr.tex
% TODO find why SA, IT stand out (inattention?): share inattentive, duration, influence of order
% TODO frame as lack of opposition rather than support
% TODO map of support
% TODO conjoint analysis correcting for inconsistent programs

\documentclass[aspectratio=169,xcolor=dvipsnames, 11pt,mathserif]{beamer} 
\input{preamble_slides.tex}

% Define title
%\texttt{
%\title{Economics by its Nobel prizes -- Keynes and the Neoclassical}
%\author[Adrien Fabre]{ \\  Adrien Fabre \\ (ETH Zürich) }
%
%              \date{Fall 2021}}
              
\begin{document}


%%%%%%%%%%%%%%%%%%%%%%%%%%%%%%%%%%%%%%%%%%%%
%%%%%%%%%%%%%%%%%%%%%%%%%%%%%%%%%%%%%%%%%%%%
% Begin of the document
% TITLE PAGE

\begin{frame}
\thispagestyle{empty}
\begin{center}
\begin{LARGE}
\textcolor{magenta}{Acceptance for global redistribution}\\~\textcolor{blue}{in high-income countries} \end{LARGE}
\vspace{1cm}

\textbf{Adrien Fabre} \\ \quad \\
%\vspace{-0.3cm}
CNRS, CIRED \\ \quad \\ \quad \\
\DTMlangsetup{showdayofmonth=false}
\textit{October 2025} 
\end{center}
\bigskip
\end{frame}

% \section{}

\begin{frame}{Literature\label{literature}}
\bbvsp
\ip \blue{ISSP (2019)}: 78\% agree disparities between countries are too large (29 countries).
\ip \blue{Carattini et al. (Nature, 2019)}: $\sim$50\% support for a global tax and dividend (U.S., UK, Australia).
\ip \blue{UNDP (2024)}: 79\% agree rich countries should help poorer countries address CC (77 countries).
\ip \blue{Andre et al. (NCC, 2024)}: \rose{69\% willing to contribute 1\% of income for climate action} (125 countries).
\ip \blue{Ghassim \& Pauli (2024)}: \rose{Majority} in 16 out of 17 countries \rose{for a global democratic government}.
\ip \blue{Fabre et al. (NHB, 2025)}: \rose{Genuine} majority \rose{support} for global redistribution (20 countries):
\bbvsp \ip \rose{Consensus for a global tax on millionaires; median preferred allocationof revenue: 30\%  to LICs}.
\ip Majorities support increased foreign aid if it really helps the poorest.
\ip Majorities support a ``Global Climate Scheme'' (GCS), i.e. a cap-and-trade with equal pc emission rights.
\ip No social desirability bias (list experiment). Similar support in a real-stake petition.
\ip Progressive candidates would not lose vote by endorsing the GCS.
\ip \rose{A political program is $\sim$10 p.p. more likely to be preferred if it contains the GCS or a global millionaire tax}.
\ee \ee
\end{frame}

\begin{frame}{Is the support for global redistribution overstated?}
\bbsp
\ip Literature results may be weakened unless we confirm some hypotheses:
\ip $\bullet$ Salience of global issues in the surveys may have fostered universalism. % created a context favorable to universalistic answers.
\ip $\Rightarrow$ H1: Global redistribution policies help a political program to be preferred.
\ip $\Rightarrow$ H2: Global issues are given substantial priority ($>\!^{2}\!/\!_{3}$ average) when allocating a budget.
\ip $\bullet$ There may not be majority support for other (more plausible or more radical) policies $\Rightarrow$ H3.
\ip $\bullet$ There may not be majority support in unsurveyed, conservative countries $\Rightarrow$ H3.
\ip $\bullet$ Support may drop below 50\% if country participation is not universal $\Rightarrow$ H4.
\ip $\bullet$ (Warm glow) Support may decrease with substitute with same more appeal $\Rightarrow$ H5a \\ 
\quad or it may hold only for as long as global redistribution seems unlikely $\Rightarrow$ H5b.
\ee
\end{frame}

\begin{frame}{Data\label{data}}
  \vspace{.1cm}
\makebox[\textwidth][c]{ \includegraphics[width=\textwidth]{../figures/maps_participation/country_coverage_progress}} \vspace{-.5cm}
\bbvs
\ip 11,865/12,000 respondents (Europe: 5,000; USA: 3,000; Japan: 2,000; Saudi Arabia, Russia: 1,000).
\ip Data collected from April 15 to July 3, 2025; except in Russia: ongoing since September 19.
\ip Representative along: gender, age, income, education, urbanicity, region. \hyperlink{representativeness}{\beamergotobutton{Representativeness tables}} 
\ip Results on control group, weighted using quota variables and aggregated with country population.
\ip Median duration: 17 min. Inattentive or fast respondents excluded.
\ip Analysis and hypotheses pre-registered.
\ee
\end{frame}

% TODO: attrition, biased

% Figures:
% 1? coverage map
% 2. survey_flow
% 3. keywords in fields (taken jointly)
% 4. revenue_split_global+few_global+few_healthcare: one point + error bar for mean per country; one global, one point for # 0% (per country + global)
% 5a. ICS: mean of variant (incl. NCS, GCS) (per country + global) 
% 5b. wealth tax by coverage: mean of variant (country + global)
% 6. conjoint: foreign aid + global tax (per country + global)
% 7. warm_glow: effect of info + display donation vs. control (per country + global)
% 8. solidarity_support (on control): heatmap
% 9. radical_redistr: heatmap sustainability, top_tax, reparations, NCQG?, vote_intl_coalition, group_defended?, my_tax_global_nation, my_tax_global_nation other source?, convergence_support
% 10. group_defended: barresN or barres?
% 11. transfer_how: heatmap (maybe just one row grouping all countries and options in columns)
% 12. average custom_redistr

% Tables:
% 1. ICS: effect of variant (incl. NCS) controlling for GCS_support (per country + global; dans régression avec 2.5 lignes pas répondant, une étant GCS et l'autre NCS) 
% 2. influence of order on split_few_global, on NCQG; of wording on NCQG


% \begin{frame}{Open-ended fields}
% \makebox[\textwidth][c]{ \includegraphics[width=\textwidth]{../figures/country_comparison/}}
% % \bbvs
% % \ip 
% % \ee
% \end{frame}
\begin{frame}{Open-ended fields}
    \begin{figure}
\vspace{-.3cm}
\begin{subfigure}{.49\textwidth}
  \caption{What are your main concerns these days?}\vspace{-.3cm}
  \makebox[\textwidth][c]{\includegraphics[height=.4\textheight]{../figures/all/concerns_field_en.pdf}}
\end{subfigure} 
\begin{subfigure}{.49\textwidth}
  \caption{What are your needs or wishes?} \vspace{-.3cm}
  \makebox[\textwidth][c]{\includegraphics[height=.4\textheight]{../figures/all/wish_field_en.pdf}}
\end{subfigure}
\begin{subfigure}{.49\textwidth}\vspace{-20cm}
  \caption{Can you name an issue that is important to you but is neglected in the public debate?} \vspace{-.3cm}
  \makebox[\textwidth][c]{\includegraphics[height=.4\textheight]{../figures/all/issue_field_en.pdf}}
\end{subfigure}
\begin{subfigure}{.49\textwidth}
  \caption{What according to you is the greatest injustice of all?}%\vspace{-.3cm}
  \makebox[\textwidth][c]{\includegraphics[height=.4\textheight]{../figures/all/injustice_field_en.pdf}}
\end{subfigure}
\end{figure}

%   \bbs \ip Four random branches:
%   \bbs \ip \blue{Wishes}: What are your needs or wishes?
%  \ip \blue{Issue}: Can you name an issue that is important to you but is neglected in the public debate?
%  \ip \blue{Concerns}: What are your main concerns these days?
%  \ip \blue{Injustice}: What according to you is the greatest injustice of all?
%   \ee
%   \ip \textit{Preliminary} findings: 
%   \bbs \ip \rose{Main concern}, issue, wish: \rose{purchasing power}.
%   \ip Main injustice: poverty.
%   \ee 
%   \ee
\end{frame}

\begin{frame}{Conjoint analysis}
\centering \vspace{.2cm}
\only<+>{\includegraphics[height=.93\textheight]{../figures/questionnaire/conjoint_US_10.png}}
\only<+>{\includegraphics[height=.93\textheight]{../figures/questionnaire/conjoint_US_5.png}}
\only<+>{\includegraphics[height=.93\textheight]{../figures/questionnaire/conjoint_US_7.png}}
\only<+>{\includegraphics[height=.93\textheight]{../figures/questionnaire/conjoint_US_9.png}}
\only<+>{\includegraphics[height=.93\textheight]{../figures/questionnaire/conjoint_US_8.png}} 
\only<+>{}
\only<+>{\includegraphics[height=.93\textheight]{../figures/questionnaire/conjoint_US_3.png}} % Cut left
\only<+>{\includegraphics[height=.93\textheight]{../figures/questionnaire/conjoint_US_6.png}} % Cut right
\only<+>{\includegraphics[height=.93\textheight]{../figures/questionnaire/conjoint_US_2.png}} % Cut in both
\only<+>{\includegraphics[height=.93\textheight]{../figures/questionnaire/conjoint_US_1.png}} % Cut in none
\end{frame}

\begin{frame}{Conjoint analysis, in the U.S. \hyperlink{conjoint_countries}{\beamergotobutton{Other countries}}\label{conjoint_country}} 
    \begin{figure} \vspace{-.15cm}
% \caption{Conjoint analysis in the U.S. (Average Marginal Component Effect) \hyperlink{conjoint_countries}{\beamergotobutton{Other countries}}} 
\includegraphics[height=.97\textheight]{../figures/all/conjoint_EN.pdf}
\end{figure}
\end{frame}

% \begin{frame}{Conjoint analysis, in France \hyperlink{conjoint_countries}{\beamergotobutton{Other countries}}\label{conjoint_country}} 
%     \begin{figure} \vspace{-.15cm}
% % \caption{Conjoint analysis in the U.S. (Average Marginal Component Effect) \hyperlink{conjoint_countries}{\beamergotobutton{Other countries}}} 
% \includegraphics[height=.97\textheight]{../figures/all/conjoint_FR.pdf}
% \end{figure}
% \end{frame}

\begin{frame}{Conjoint analysis} 
    \begin{figure}
\caption{\rose{Effect on} the \rose{likelihood that a} political \rose{program is preferred} of containing the following policy%\\(compared to no foreign policy in the program):
} \vspace{-.4cm}
\begin{subfigure}{.48\textwidth}
  \caption[]{\blue{Cut development aid}: $-$4 pp*** \quad \green{\checkmark H1a}}\vspace{-.3cm}
  \includegraphics[width=\textwidth]{../figures/country_comparison/program_preferred_by_cut_aid_in_program.pdf}
\end{subfigure} \pause
\begin{subfigure}{.50\textwidth}
  \caption[]{\blue{International tax on millionaires with 30\% financing} health and education in \blue{low-income countries}: +4 pp*** \quad \green{\checkmark H1b}} \vspace{-.3cm}
  \includegraphics[width=.95\textwidth]{../figures/country_comparison/program_preferred_by_millionaire_tax_in_program.pdf}
\end{subfigure}
\end{figure}
\end{frame}

\begin{frame}{Revenue split}
\centering How should revenue from a global tax on millionaires be allocated? (in \%) \quad \green{\checkmark H2}
\only<+>{\makebox[\textwidth][c]{
\includegraphics[height=.88\textheight]{../figures/country_comparison/split_few_bars.pdf}}}
\only<+>{\makebox[\textwidth][c]{
\includegraphics[width=\textwidth]{../figures/all/split_few.pdf}}}
\end{frame}

\begin{frame}{International wealth tax}
\centering Support for a 2\% wealth tax above [\$1 million] with 30\% financing public services in LICs,\\depending on which countries adopt it (random variant).
%     {\centering Imagine an international tax on individuals with net worth above [wealth_threshold: \$1 million]. \\
% Only wealth above [wealth_threshold: \$1 million] would be taxed, at a rate of 2\%. Each country would retain 70\% of the revenues it collects, while 30\% would be pooled at the global level to finance public services in low-income countries (in particular, access to drinking water, healthcare, and education in Africa). \\
% \\
% Say we are in 2030. [International: Imagine that some countries  (such as the European Union) adopt this policy and others (such as Japan, Canada, and China) do not.]\\
% Do you support [country_name: the United States] adopting this international tax on millionaires? }
\makebox[\textwidth][c]{ \includegraphics[width=.98\textwidth]{../figures/country_comparison/variables_wealth_tax_support_by_country}} % wealth_tax_support_positive
\rose{$-$6 p.p. when coverage is international rather than global}.
% \bbvs
% \ip 
% \ee
\end{frame}

\begin{frame}{Climate Scheme\label{cs}} 
    \bbvsp \ip 3 random branches:
      \bbvsp \ip Donation experiment: \textit{In case you win the lottery, what share of the \$100 would you donate to plant trees?}
      \ip National Climate Scheme (NCS): a cap-and-trade financing equal cash transfers.
      \ip Control group: $\varnothing$. \ee
    \ip Global Climate Scheme (GCS): \hyperlink{gcs_question}{\beamergotobutton{Question text}} 
      \bbvsp \ip Description incentivized for comprehension (73\% get the correct answer). % TODO: Say support on correct is 2.3pp lower
      \ip A \textit{global} cap-and-trade financing a \textit{global} basic income of [U.S.: \$35, France: \euro{}20]/month (\$95/tCO$_\text{2}$).
      \ip Salient cost: ``\textbf{The typical [country] person would lose out financially [U.S.: \$90, UK: £20...] per month} (as he or she would face around [U.S.: 2, UK: 1]\% in price increases, which is higher than the [U.S.: \$35] per month they would receive).'' \ee
    \ip Belief of support for the GCS. 2 random branches:
      \bbvs \ip In own country.
      \ip In the U.S. \ee
    \ip International Climate Scheme (GCS where country participation is specified). 4 random branches:
    \bbvs \ip  Country coverage variants: \textit{Low}; \textit{Mid}; \textit{High}; \textit{High color} (color: visible distributive effects). \ee
    \ee
\end{frame}

\begin{frame}{Majority support for the GCS and pluralistic ignorance\label{gcs} \hyperlink{determinants}{\beamergotobutton{Determinants}}}
\makebox[\textwidth][c]{ \includegraphics[width=\textwidth]{../figures/country_comparison/gcs_belief_mean.pdf}} 
% \ee
\end{frame}

\begin{frame}{International Climate Scheme}
    \centering 
\only<+>{\includegraphics[height=.93\textheight]{../figures/questionnaire/survey_ics_mid.png}} 
\only<+>{\centering \blue{\textbf{Low}}: Global South + EU (25-33\% of world emissions)
  \includegraphics[height=.85\textheight]{../figures/maps_participation/GCS_low.png}} 
\only<+>{\centering \blue{\textbf{Mid}} coverage: Global South + China (56\% of world emissions)
\includegraphics[height=.85\textheight]{../figures/maps_participation/GCS_mid.png}}
\only<+>{\centering \blue{\textbf{High}} coverage: Global South + China + EU + various HICs (64-72\% of world emissions)
\includegraphics[height=.85\textheight]{../figures/maps_participation/GCS_high.png}}
\only<+>{\centering \blue{\textbf{High}} coverage (64-72\% of world emissions), \rose{\textbf{color}} variant
\includegraphics[height=.85\textheight]{../figures/maps_participation/GCS_high_color.pdf}}
\only<+>{\centering \textbf{High color} as shown in the U.S.
\includegraphics[height=.85\textheight]{../figures/maps_participation/GCS_high_color_EN.pdf}}
\only<+>{\centering \textbf{High color} as shown in France
\includegraphics[height=.85\textheight]{../figures/maps_participation/GCS_high_color_FR.pdf}}
\only<+>{\centering \textbf{High color} as shown in Saudi Arabia
\includegraphics[height=.85\textheight]{../figures/maps_participation/GCS_high_color_AR.pdf}}
\end{frame}

\begin{frame}{International Climate Scheme}
\makebox[\textwidth][c]{ \includegraphics[width=.9\textwidth]{../figures/country_comparison/variables_ncs_gcs_ics_control_by_country}} % ncs_gcs_ics_all_mean
\vspace{-.7cm}
\bbvsp
% (Bechtel and Scheve 2013; Carlsson et al. 2013; Dabla-Norris et al. 2023; Gampfer, Bernauer, and Kachi 2014)
\ip \rose{Majority support} for all variants in all countries when participation map is shown. \quad \green{\checkmark H4a}
\ip \blue{+4 p.p.} when coverage expanded from \blue{Low} to \blue{High}. \quad \green{\checkmark H4b} 
\ip \rose{$-$6 p.p. when distributive effects are visible}.
\ee
\end{frame}

\begin{frame}{Plausible gobal policies\label{solidarity_support_relative}}
\centering Share of (somewhat or strong) \textit{support} among non-\textit{indifferent} answers. \quad \green{\checkmark H3} \hyperlink{solidarity_support_absolute}{\beamergotobutton{Absolute support}} \\
\makebox[\textwidth][c]{ \includegraphics[height=.89\textheight]{../figures/country_comparison/solidarity_support_share}}
% \bbvs
% \ip 
% \ee
\end{frame}

\begin{frame}{Testing warm glow} 
  \centering If \textit{warm glow}, support might dissipate when...
    \begin{figure}
% \caption{}
\begin{subfigure}{.38\textwidth}
  \caption[]{\normalsize \blue{Moral substitute}: ...the policy is replaced a substitute with the same moral appeal}
  \bbvs \ip \textit{Donation lottery} treatment: Choose share of \$100 prize to donate to plant trees.
  \ip H5a: Support for GCS no lower in the treated group.
  \ee
  \vspace{3.8cm}
  % \includegraphics[width=\textwidth]{}
\end{subfigure} \quad \pause
\begin{subfigure}{.59\textwidth}
  \caption[]{\normalsize \blue{Realism}: ...the policy materializes} \vspace{-.5cm}
  \bbvs \ip \textit{Information} treatment: ``countries have agreed to demonstrate some degree of solidarity in addressing global challenges.  Negotiations are ongoing to implement specific mechanisms for sustainable development.'' Examples:
  \bbvs \ip IMO shipping levy
  \ip 0.7\% ODA commitment
  \ip Climate finance commitment
  \ip Minimum corporate tax
  \ip Brazil proposal at G20 of taxing billionaires
  \ip UN Pact for the Future foreseeing Security Council reform
  \ip Bridgetown initiative to finance sustainable development
  \ee
  \ip H5b: Info increases belief in likelihood of global redistribution but does not reduce support for plausible global policies
  \ee
\end{subfigure}
\end{figure}
\end{frame}

\begin{frame}{No evidence of warm glow\label{warm_glow}} 
    \begin{figure}
% \caption{}
\begin{subfigure}{.47\textwidth}
  \caption[]{Effect of a \blue{\textit{Donation lottery} treatment} on support for the \rose{Global Climate Scheme}: +3p.p.*** \quad \green{\checkmark H5a}}
  \includegraphics[width=\textwidth]{../figures/country_comparison/gcs_support_by_variant_warm_glow.pdf}
\end{subfigure} \quad \pause
\begin{subfigure}{.47\textwidth}
  \caption[]{Effect of \blue{information about ongoing global redistribution} initiatives on the share of plausible \rose{global policies} supported: +1p.p.* \quad \green{\checkmark H5b} \hyperlink{2SLS}{\beamergotobutton{2SLS}}}
  \includegraphics[width=\textwidth]{../figures/country_comparison/share_solidarity_supported_by_info_solidarity.pdf}
\end{subfigure}
\end{figure}
\end{frame}

\begin{frame}{NCQG}
\centering
\only<+>{ \includegraphics[height=.9\textheight]{../figures/country_comparison/ncqg}} 
\only<+>{ \includegraphics[height=.9\textheight]{../figures/country_comparison/ncqg_full}} 
\end{frame}

\begin{frame}{BAU vs. sustainable future}
Sustainable future:
\begin{itemize}
    \item \blue{Worldwide policies} to limit global warming to +2\textdegree{}C and reduce inequality.
    \item Reduced global carbon emissions, in line with climate target.
    \item \blue{Taxes on millionaires} funding heat pumps, building insulation, and public transport.
    \item All cars electric by 2045, priced like today's gasoline cars.
    \item Heating fuel, air travel, and beef \blue{prices gradually double}. 
     %As a result, 50\% less flying and meat consumption, and more public transport use.
    \item Lower sales tax on non-polluting goods \blue{preserves overall purchasing power}.
\end{itemize}
Status quo:
\begin{itemize}
    \item No additional policies to address climate change or inequality. 
    \item Stable carbon emissions. \blue{+3\textdegree{}C} warming by 2100, causing more severe natural disasters.
    \item \blue{People maintain the same lifestyles} as in 2025, such as driving gasoline cars.
\end{itemize}
\makebox[\textwidth][c]{
\includegraphics[width=\textwidth]{../figures/country_comparison/sustainable_future_positive.pdf}} 
% \bbvs
% \ip 
% \ee
\end{frame}

\begin{frame}{Why help LICs?}
    % Some people think that high-income countries should support low-income countries. \nAmong the different reasons given, which ones do you agree with? (Multiple answers possible)
\makebox[\textwidth][c]{ \includegraphics[width=\textwidth]{../figures/country_comparison/why_hic_help_lic_positive.pdf}}
\end{frame}

\begin{frame}{Preferred means of transfers}
    \vspace*{.5cm}
\centering How do you evaluate each of these channels to transfer resources to reduce poverty in LICs?\\ Share of \textit{Right} or \textit{Best} way (other options: \textit{Wrong} or \textit{Acceptable} way).
\makebox[\textwidth][c]{ \includegraphics[width=\textwidth]{../figures/country_comparison/transfer_how_positive}}
% \bbvs
% \ip 
% \ee
\end{frame}

% Order questions in questionnaire:
% convergence
% movement
% vote_intl
% why_help
% reparations
% custom redistr
% moral circle
% my taxes

\begin{frame}{A humanist movement}
\centering If there was a worldwide movement in favor of a global program to tackle climate change, implement taxes on millionaires and fund poverty reduction in low-income countries, to what extent would you be willing to be part of that movement? (Multiple answers possible)
\makebox[\textwidth][c]{ \includegraphics[width=\textwidth]{../figures/country_comparison/global_movement_all_positive.pdf}}
$\Rightarrow$ 52\% of millionaires support the global movement.
% \bbvs
% \ip 
% \ee
\end{frame}

\begin{frame}{Importance on voting}
\makebox[\textwidth][c]{ \includegraphics[height=.9\textheight]{../figures/country_comparison/vote_intl_coalition.pdf}}
% \bbvs
% \ip 
% \ee
\end{frame}

\begin{frame}{Moral circle}
\makebox[\textwidth][c]{ \includegraphics[height=.9\textheight]{../figures/country_comparison/group_defended}}
% \bbvs
% \ip 
% \ee
\end{frame}

\begin{frame}{Radical redistribution}
\makebox[\textwidth][c]{ \includegraphics[width=\textwidth]{../figures/country_comparison/radical_redistr_few_share}}
% \bbvs
% \ip 
% \ee
\end{frame}

% TODO refaire vidéo en commençant par expliquer les axes
\begin{frame}{Custom redistribution \href{run:survey_custom_redistr.mp4}{\beamergotobutton{Video}} \href{https://bit.ly/custom_redistr}{\beamergotobutton{bit.ly/custom\_redistr}}} 
    \centering
    \only<+>{\begin{figure}
    \includegraphics[height=.93\textheight]{../figures/questionnaire/survey_custom_redistr_bottom.png}\end{figure}}
    \only<4>{\begin{figure}\caption{Redistribution with median preferred parameters (among satisfied): 
    winners 49\%; losers: 18\%; degree: 5.}
    \includegraphics[height=.83\textheight]{../figures/questionnaire/survey_custom_redistr_median_zoom.png}\end{figure}}
    \only<7>{\begin{figure}\caption{Tax rates of radical redistribution proposals tested.}
    \includegraphics[height=.83\textheight]{../figures/all/tax_radical_redistr.pdf}\end{figure}}
    \only<6>{\begin{figure}\caption{Average custom redistribution} 
      \begin{subfigure}{.2\textwidth}
        \bbs \ip \blue{Minimal income: \$243/month (i.e. \$8/day)}
        \ip \blue{5\% of world income transferred} from top 28\% \rose{to bottom 72\%}
        \ee
      \end{subfigure}\begin{subfigure}{.8\textwidth}
        \includegraphics[height=.8\textheight]{../figures/mean_custom_redistr/all_satisfied.png}
      \end{subfigure}
    \end{figure}}
    \only<2->{
        \bbsp \ip 56\% satisfied; 43\% skipped
        \ip Share satisfied among:\\non-voters: 50\%; left: 60\%, right: 53\%, far right: 58\%. \pause
        \ip \blue{48\% lose} from their custom redistribution while only \rose{9\% win}.
        \ee
    }
    % TODO update reforme_perso.js sur adrien-fabre.com
\end{frame}

\begin{frame}{Conclusion}
% \makebox[\textwidth][c]{ \includegraphics[width=\textwidth]{../figures/country_comparison/}}
\bbsp
\ip Global poverty seen as big injustice but not a salient concern.
\ip Strong majority support for globally redistributive policies.
\ip Support reduces (only) slightly when fewer countries participate.
\ip Support is genuine. No evidence of warm glow.
\ip Global redistribution is a vote-determining issue for many people. % Majority on the left? What about abstentionnist?
\ip Most agree global sustainability is a duty.
\ip Most agree they should contribute themselves.
\ip \begin{center} \textbf{Thank you for your attention!} \\   Work-in-progress at \blue{\href{https://bit.ly/support_global_redistr}{bit.ly/support\_global\_redistr}} \end{center}
\ee
\end{frame}

\appendix 
\section{Appendix}

\begin{frame}{Representativeness of pooled samples \hyperlink{data}{\beamergotobutton{Go back}} \label{representativeness}}
    \vspace{-.6cm}
    \begin{table}[h]
        \makebox[\textwidth][c]{
            \resizebox*{!}{1.4\textheight}{
            
\begin{tabular}[t]{lccccccccc}
\toprule
\multicolumn{1}{c}{} & \multicolumn{3}{c}{All} & \multicolumn{3}{c}{Eu} & \multicolumn{3}{c}{EU} \\
\cmidrule(l{3pt}r{3pt}){2-4} \cmidrule(l{3pt}r{3pt}){5-7} \cmidrule(l{3pt}r{3pt}){8-10}
  & Pop. & Sample & \makecell{Weighted\\sample} & Pop. & Sample & \makecell{Weighted\\sample} & Pop. & Sample & \makecell{Weighted\\sample}\\
\midrule
Sample size &  & 12,001 & 12,001 &  & 5,000 & 5,000 &  & 3,705 & 3,705\\
\addlinespace
Gender: Woman & .51 & .50 & .51 & .51 & .51 & .51 & .52 & .51 & .52\\
Gender: Man & .49 & .49 & .49 & .49 & .49 & .49 & .48 & .49 & .48\\
\addlinespace
Income\_quartile: Q1 & .25 & .26 & .25 & .25 & .27 & .25 & .25 & .26 & .25\\
Income\_quartile: Q2 & .25 & .24 & .25 & .25 & .26 & .25 & .25 & .26 & .25\\
Income\_quartile: Q3 & .25 & .24 & .25 & .25 & .21 & .25 & .25 & .22 & .25\\
Income\_quartile: Q4 & .25 & .26 & .25 & .25 & .26 & .25 & .25 & .26 & .25\\
\addlinespace
Age: 18-24 & .10 & .10 & .10 & .09 & .10 & .09 & .09 & .10 & .09\\
Age: 25-34 & .16 & .17 & .16 & .15 & .15 & .15 & .14 & .15 & .14\\
Age: 35-49 & .26 & .26 & .26 & .25 & .25 & .25 & .25 & .25 & .25\\
Age: 50-64 & .24 & .23 & .24 & .25 & .25 & .25 & .25 & .25 & .25\\
Age: 65+ & .25 & .24 & .25 & .26 & .25 & .26 & .27 & .25 & .27\\
\addlinespace
Diploma\_25-64: Below upper secondary & .09 & .08 & .09 & .13 & .12 & .13 & .14 & .13 & .14\\
Diploma\_25-64: Upper secondary & .31 & .29 & .31 & .26 & .26 & .26 & .28 & .27 & .28\\
Diploma\_25-64: Post secondary & .29 & .32 & .29 & .25 & .28 & .25 & .22 & .25 & .22\\
\addlinespace
Urbanity: Cities & .63 & .52 & .52 & .41 & .41 & .41 & .42 & .44 & .42\\
Urbanity: Towns and suburbs & .15 & .17 & .15 & .36 & .38 & .36 & .35 & .34 & .34\\
Urbanity: Rural & .20 & \textbf{.14} & .16 & .22 & .21 & .22 & .23 & .22 & .23\\
\addlinespace
Country: FR & .07 & .07 & .07 & .18 & .16 & .18 & .22 & .22 & .22\\
Country: DE & .08 & .09 & .08 & .23 & .21 & .23 & .28 & .28 & .28\\
Country: IT & .06 & .06 & .06 & .16 & .15 & .16 & .20 & .20 & .20\\
Country: PL & .04 & .04 & .04 & .10 & .10 & .10 & .13 & .13 & .13\\
Country: ES & .05 & .05 & .05 & .13 & .12 & .13 & .16 & .16 & .16\\
Country: GB & .07 & .07 & .07 & .18 & .17 & .18 &  &  & \\
Country: CH & .01 & \textbf{.04} & .01 & .02 & \textbf{.09} & .02 &  &  & \\
Country: JP & .13 & \textbf{.17} & .13 &  &  &  &  &  & \\
Country: RU & .14 & \textbf{.08} & .14 &  &  &  &  &  & \\
Country: SA & .03 & \textbf{.08} & .03 &  &  &  &  &  & \\
Country: US & .33 & \textbf{.25} & .33 &  &  &  &  &  & \\
\bottomrule
\end{tabular}
            }
        }
    \end{table}  
\end{frame}

\begin{frame}{Representativeness in EU countries \hyperlink{data}{\beamergotobutton{Go back}} \label{representativeness}}
    \vspace{-.6cm}
    \begin{table}[h]
        \makebox[\textwidth][c]{
            \resizebox*{!}{.97\textheight}{
            \input{../tables/FR_DE_IT_PL_ES_bold.tex}
            }
        }
    \end{table}  
\end{frame}

\begin{frame}{Representativeness in non-EU countries \hyperlink{data}{\beamergotobutton{Go back}} \label{representativeness}}
    \vspace{-.6cm}
    \begin{table}[h]
        \makebox[\textwidth][c]{
            \resizebox*{!}{.97\textheight}{
            \input{../tables/GB_CH_JP_RU_SA_US_bold.tex}
            }
        }
    \end{table}  
\end{frame}

\begin{frame}{Representativeness of vote (non-EU countries)  \hyperlink{data}{\beamergotobutton{Go back}}}
\includegraphics[height=.95\textheight]{../figures/country_comparison/vote_non_EU_pnr_out.pdf}
\end{frame}

\begin{frame}{Representativeness of vote (EU countries)  \hyperlink{data}{\beamergotobutton{Go back}}}
\includegraphics[height=.95\textheight]{../figures/country_comparison/vote_EU_pnr_out.pdf}
\end{frame}
% TODO add likelihood_global_redistr

% TODO!? move to Sincerity?
% \begin{frame}{Conjoint analyses\label{conjoint_countries} \hyperlink{conjoint_country}{\beamergotobutton{Go back}}} 
%     \begin{figure} \vspace{-.14cm}
% \only<+>{\caption{Conjoint analysis in France (Average Marginal Component Effect)} \includegraphics[width=\textwidth]{../figures/all/conjoint_EN-FR.pdf}}
% \only<+>{\caption{Conjoint analysis in Germany (Average Marginal Component Effect)} \includegraphics[width=\textwidth]{../figures/all/conjoint_EN-DE.pdf}}
% \only<+>{\caption{Conjoint analysis in Italy (Average Marginal Component Effect)} \includegraphics[width=\textwidth]{../figures/all/conjoint_EN-IT.pdf}}
% \only<+>{\caption{Conjoint analysis in Poland (Average Marginal Component Effect)} \includegraphics[width=\textwidth]{../figures/all/conjoint_EN-PL.pdf}}
% \only<+>{\caption{Conjoint analysis in Spain (Average Marginal Component Effect)} \includegraphics[width=\textwidth]{../figures/all/conjoint_EN-ES.pdf}}
% \only<+>{\caption{Conjoint analysis in the UK (Average Marginal Component Effect)} \includegraphics[width=\textwidth]{../figures/all/conjoint_EN-GB.pdf}}
% \only<+>{\caption{Conjoint analysis in Switzerland (Average Marginal Component Effect)} \includegraphics[width=\textwidth]{../figures/all/conjoint_EN-CH.pdf}}
% \only<+>{\caption{Conjoint analysis in Japan (Average Marginal Component Effect)} \includegraphics[width=\textwidth]{../figures/all/conjoint_EN-JA.pdf}}
% \only<+>{\caption{Conjoint analysis in the U.S. (Average Marginal Component Effect)} \includegraphics[width=\textwidth]{../figures/all/conjoint_EN.pdf}}
% \end{figure}
% \end{frame}

\begin{frame}{Conjoint analysis,  in France\label{conjoint_countries} \hyperlink{conjoint_country}{\beamergotobutton{Go back}}} 
    \begin{figure} \vspace{-.14cm}
\includegraphics[height=.97\textheight]{../figures/all/conjoint_EN-FR.pdf}
\end{figure}
\end{frame}

\begin{frame}{Conjoint analysis,  in Germany\label{conjoint_countries} \hyperlink{conjoint_country}{\beamergotobutton{Go back}}} 
    \begin{figure} \vspace{-.14cm}
\includegraphics[height=.97\textheight]{../figures/all/conjoint_EN-DE.pdf}
\end{figure}
\end{frame}

\begin{frame}{Conjoint analysis,  in Italy\label{conjoint_countries} \hyperlink{conjoint_country}{\beamergotobutton{Go back}}} 
    \begin{figure} \vspace{-.14cm}
\includegraphics[height=.97\textheight]{../figures/all/conjoint_EN-IT.pdf}
\end{figure}
\end{frame}

\begin{frame}{Conjoint analysis,  in Poland\label{conjoint_countries} \hyperlink{conjoint_country}{\beamergotobutton{Go back}}} 
    \begin{figure} \vspace{-.14cm}
\includegraphics[height=.97\textheight]{../figures/all/conjoint_EN-PL.pdf}
\end{figure}
\end{frame}

\begin{frame}{Conjoint analysis,  in Spain\label{conjoint_countries} \hyperlink{conjoint_country}{\beamergotobutton{Go back}}} 
    \begin{figure} \vspace{-.14cm}
\includegraphics[height=.97\textheight]{../figures/all/conjoint_EN-ES.pdf}
\end{figure}
\end{frame}

\begin{frame}{Conjoint analysis,  in the UK\label{conjoint_countries} \hyperlink{conjoint_country}{\beamergotobutton{Go back}}} 
    \begin{figure} \vspace{-.14cm}
\includegraphics[height=.97\textheight]{../figures/all/conjoint_EN-GB.pdf}
\end{figure}
\end{frame}

\begin{frame}{Conjoint analysis,  in Switzerland\label{conjoint_countries} \hyperlink{conjoint_country}{\beamergotobutton{Go back}}} 
    \begin{figure} \vspace{-.14cm}
\includegraphics[height=.97\textheight]{../figures/all/conjoint_EN-CH.pdf}
\end{figure}
\end{frame}

\begin{frame}{Conjoint analysis,  in Japan\label{conjoint_countries} \hyperlink{conjoint_country}{\beamergotobutton{Go back}}} 
    \begin{figure} \vspace{-.14cm}
\includegraphics[height=.97\textheight]{../figures/all/conjoint_EN-JA.pdf}
\end{figure}
\end{frame}

\begin{frame}{Conjoint analysis,  in the U.S.\label{conjoint_countries} \hyperlink{conjoint_country}{\beamergotobutton{Go back}}} 
    \begin{figure} \vspace{-.14cm}
\includegraphics[height=.97\textheight]{../figures/all/conjoint_EN.pdf}
\end{figure}
\end{frame}

\begin{frame}{Conjoint analysis,  in France\label{conjoint_countries} \hyperlink{conjoint_country}{\beamergotobutton{Go back}}} 
    \begin{figure} \vspace{-.14cm}
\includegraphics[height=.97\textheight]{../figures/all/conjoint_FR.pdf}
\end{figure}
\end{frame}


\begin{frame}{Global Climate Scheme question (in the U.S.)\label{gcs_question} \hyperlink{cs}{\beamergotobutton{Go back}}}
  \footnotesize
  Do you support the following policy?\\
  
  To ensure that you have attentively read the description, \textbf{we will ask some comprehension questions later in the survey: those who get correct answers can win \$100.}\\ 
  
\vspace{.2cm}
  \textbf{Global Climate Scheme:} \\ 

\vspace{.2cm}
  In 2015, all countries agreed to contain global warming ``well below +2\textdegree{}C''. To achieve this, \textbf{there is a maximum amount of greenhouse gases we can emit globally}. \\ 

\vspace{.2cm}
  To meet the climate target, a limited number of permits to emit greenhouse gases would be issued globally. Polluting firms would be required to buy permits to cover their greenhouse gas emissions. Such a policy would \textbf{make fossil fuel companies pay} for their emissions and gradually raise the price of fossil fuels. \textbf{Higher prices would encourage people and companies to use less fossil fuels, reducing greenhouse gas emissions}. \\ 

\vspace{.2cm}
  In accordance with the principle that each human has an equal right to pollute, the revenues generated by the sale of permits could finance a global basic income. \textbf{Every adult would receive \$35 per month}, thereby lifting 600 million people who earn less than \$2 a day out of extreme poverty.\\ 

  \textbf{The typical American would lose out financially \$90  per month} (as he or she would face around 2\% in price increases, which is higher than the \$35 per month they would receive).\\ 

\vspace{.2cm}
  The policy could be implemented as soon as 100 countries agree to it. Countries that would refuse to take part in the policy could face sanctions (like tariffs) from the rest of the world and would be excluded from the basic income program.\\ 
\vspace{.2cm}

  \textbf{Do you support the Global Climate Scheme?}
\end{frame}
% Version any country:
  %   To ensure that you have attentively read the description, \textbf{we will ask some comprehension questions later in the survey: those who get correct answers can win [amount_lottery: \$100].}\\ 
  
  % \textbf{Global Climate Scheme:} \\ 

  % In 2015, all countries agreed to contain global warming ``well below +2\textdegree{}C''. To achieve this, \textbf{there is a maximum amount of greenhouse gases we can emit globally}. \\ 

  % To meet the climate target, a limited number of permits to emit greenhouse gases would be issued globally. Polluting firms would be required to buy permits to cover their greenhouse gas emissions. Such a policy would \textbf{make fossil fuel companies pay} for their emissions and gradually raise the price of fossil fuels. \textbf{Higher prices would encourage people and companies to use less fossil fuels, reducing greenhouse gas emissions}. \\ 

  % In accordance with the principle that each human has an equal right to pollute, the revenues generated by the sale of permits could finance a global basic income. \textbf{Every adult would receive [amount_bi: \$35][periodicity: per month]}, thereby lifting 600 million people who earn less than \$2 a day out of extreme poverty.\\ 

  % \textbf{The typical [national: American] would lose out financially [amount_lost: \$90][periodicity: per month]} (as he or she would face around [price_increase: 2]\% in price increases, which is higher than the [amount_bi: \$35][periodicity: per month] they would receive).\\ 
  
% - People view global poverty/ineq as big injustice though not salient concern (e.g. revenue_split)
% - Strong support for global tax / GCS even with partial participation
% - No (or little) evidence of warm glow or support due to unrealism; global redistr genuinely supported (conjoint analysis)
% - People ready to sustainability and radical global redistr (for duty, not reparations)
% - People favor social protection to unconditional transfers
% - Influenced by question on revenue splits and NCQG / more or less consistent on NCQG

\begin{frame}{Plausible gobal policies\label{solidarity_support_absolute} .\hyperlink{solidarity_support_relative}{\beamergotobutton{Go back}}}
\centering Share of (somewhat or strong) \textit{support}. \\
\makebox[\textwidth][c]{ \includegraphics[width=.95\textwidth]{../figures/country_comparison/solidarity_support_positive}}
% \bbvs
% \ip 
% \ee
\end{frame}


\begin{frame}{2SLS\label{2SLS}  \hyperlink{warm_glow}{\beamergotobutton{Go back}}}  
  \small \input{../tables/iv_highlighted.tex}
\end{frame}

\begin{frame}{Determinants of support for global redistribution (selected variables)\label{determinants}  \hyperlink{gcs}{\beamergotobutton{Go back}}}  
  % Variables such as country and country:region are in the regressors but display is omitted.
      \vspace{-.18cm}
    \begin{table}[h]
        \makebox[\textwidth][c]{
            \resizebox*{!}{.97\textheight}{
            
\begin{tabular}{@{\extracolsep{5pt}}lccccccc} 
\\[-1.8ex]\hline 
\hline \\[-1.8ex] 
\\[-1.8ex] & \makecell{Share of\\plausible policies\\supported} & \makecell{Supports\\the Global\\Climate Scheme} & \makecell{Universalist\\(Group defended:\\Humans or Sentient beings)} & \makecell{More likely to\\vote for party\\in global coalition} & \makecell{Endorses convergence\\of all countries' GDP\\per capita by 2100} & \makecell{Supports an\\int'l wealth tax\\funding LICs} & \makecell{Prefers a\\sustainable\\future} \\ 
\hline \\[-1.8ex] 
Mean & 0.508 & 0.554 & 0.454 & 0.365 & 0.61 & 0.704 & 0.681  \\ \hline \\[-1.8ex]
 Vote: Center\mbox{-}right or Right & 0.017$^{*}$ & 0.004 & $-$0.084$^{***}$ & 0.030$^{**}$ & 0.039$^{***}$ & $-$0.025$^{*}$ & $-$0.061$^{***}$ \\ 
  & (0.010) & (0.015) & (0.015) & (0.013) & (0.014) & (0.014) & (0.014) \\ 
  Vote: Far right & $-$0.087$^{***}$ & $-$0.146$^{***}$ & $-$0.227$^{***}$ & $-$0.063$^{***}$ & $-$0.065$^{***}$ & $-$0.139$^{***}$ & $-$0.169$^{***}$ \\ 
  & (0.013) & (0.020) & (0.019) & (0.018) & (0.020) & (0.019) & (0.020) \\ 
  Vote: Left & 0.215$^{***}$ & 0.168$^{***}$ & 0.148$^{***}$ & 0.258$^{***}$ & 0.189$^{***}$ & 0.186$^{***}$ & 0.148$^{***}$ \\ 
  & (0.010) & (0.014) & (0.015) & (0.014) & (0.014) & (0.013) & (0.014) \\ 
  Gender: Man & 0.020$^{***}$ & 0.032$^{***}$ & $-$0.033$^{***}$ & 0.029$^{***}$ & 0.011 & 0.001 & $-$0.025$^{***}$ \\ 
  & (0.007) & (0.010) & (0.010) & (0.010) & (0.010) & (0.009) & (0.010) \\ 
  Age: 18\mbox{-}24 & 0.025$^{*}$ & 0.173$^{***}$ & 0.107$^{***}$ & 0.108$^{***}$ & 0.119$^{***}$ & 0.093$^{***}$ & 0.058$^{***}$ \\ 
  & (0.014) & (0.021) & (0.022) & (0.023) & (0.021) & (0.019) & (0.020) \\ 
  Age: 25\mbox{-}34 & 0.028$^{***}$ & 0.091$^{***}$ & 0.068$^{***}$ & 0.102$^{***}$ & 0.050$^{***}$ & 0.048$^{***}$ & 0.031$^{**}$ \\ 
  & (0.011) & (0.016) & (0.016) & (0.016) & (0.016) & (0.014) & (0.015) \\ 
  Age: 50\mbox{-}64 & $-$0.016 & $-$0.038$^{***}$ & $-$0.025$^{*}$ & $-$0.033$^{**}$ & $-$0.027$^{*}$ & $-$0.032$^{**}$ & $-$0.019 \\ 
  & (0.010) & (0.014) & (0.014) & (0.014) & (0.014) & (0.014) & (0.014) \\ 
  Age: 65+ & 0.037$^{***}$ & $-$0.011 & $-$0.002 & 0.002 & $-$0.014 & $-$0.019 & 0.007 \\ 
  & (0.013) & (0.018) & (0.018) & (0.017) & (0.018) & (0.017) & (0.017) \\ 
  Income quartile: Q2 & 0.028$^{***}$ & 0.012 & $-$0.021 & 0.016 & $-$0.007 & 0.018 & 0.011 \\ 
  & (0.010) & (0.015) & (0.015) & (0.015) & (0.015) & (0.013) & (0.014) \\ 
  Income quartile: Q3 & 0.017$^{*}$ & $-$0.005 & 0.007 & $-$0.009 & $-$0.018 & $-$0.024$^{*}$ & $-$0.001 \\ 
  & (0.010) & (0.015) & (0.015) & (0.015) & (0.015) & (0.014) & (0.015) \\ 
  Income quartile: Q4 & 0.006 & $-$0.037$^{**}$ & $-$0.024 & $-$0.032$^{*}$ & $-$0.072$^{***}$ & $-$0.080$^{***}$ & $-$0.006 \\ 
  & (0.012) & (0.017) & (0.017) & (0.017) & (0.017) & (0.016) & (0.016) \\ 
  Diploma: Upper secondary & 0.027$^{**}$ & $-$0.011 & 0.024 & 0.036$^{**}$ & 0.035$^{**}$ & 0.014 & 0.017 \\ 
  & (0.010) & (0.015) & (0.015) & (0.015) & (0.015) & (0.014) & (0.015) \\ 
  Diploma: Above upper secondary & 0.068$^{***}$ & 0.022 & 0.041$^{**}$ & 0.079$^{***}$ & 0.022 & 0.011 & 0.047$^{***}$ \\ 
  & (0.011) & (0.016) & (0.016) & (0.015) & (0.016) & (0.015) & (0.015) \\ 
  Urbanicity: Rural & $-$0.010 & $-$0.054$^{***}$ & 0.015 & $-$0.006 & $-$0.013 & $-$0.021 & $-$0.019 \\ 
  & (0.010) & (0.015) & (0.015) & (0.014) & (0.015) & (0.014) & (0.015) \\ 
  Urbanicity: Towns and suburbs & $-$0.012 & $-$0.038$^{**}$ & $-$0.024 & $-$0.023 & $-$0.013 & $-$0.024$^{*}$ & 0.026$^{*}$ \\ 
  & (0.010) & (0.015) & (0.015) & (0.015) & (0.015) & (0.014) & (0.014) \\ 
  Foreign born & 0.061 & 0.089$^{***}$ & 0.104$^{***}$ & 0.051$^{**}$ & 0.036$^{*}$ & 0.041$^{**}$ & 0.039$^{**}$ \\ 
  & (0.014) & (0.020) & (0.021) & (0.022) & (0.021) & (0.018) & (0.019) \\ 
 \hline \\[-1.8ex] 

Observations & 10,998 & 10,998 & 10,998 & 9,998 & 10,998 & 10,998 & 10,998 \\ 
R$^{2}$ & 0.161 & 0.114 & 0.109 & 0.115 & 0.098 & 0.103 & 0.078 \\ 
\hline 
\hline \\[-1.8ex] 
\end{tabular} 
            }
        }
    \end{table} 
\end{frame}

\begin{frame}{Last slide (needed)}
  \end{frame}
\end{document}