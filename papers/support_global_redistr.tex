%%%%%%%%%%%%%%%%%%%%%%%%%%%%%%%%
%%%%% WORKING PAPER FORMAT %%%%%
%%%%%%%%%%%%%%%%%%%%%%%%%%%%%%%%
%% Comment "% NCCcomment" lines, uncomment "% WPcomment" lines as well as the lines below
\documentclass[12pt,english]{article}
\usepackage[utf8]{inputenc}
\usepackage{tgpagella} % Palatino text only
\usepackage{mathpazo}  % Palatino math & text
\usepackage[left=1.5in,right=1.5in,top=1.5in,bottom=1.5in]{geometry}
% \linespread{1.5}
% \usepackage[super,compress]{natbib} % WPcomment
\usepackage{bibunits} % To get multiple bibliography
\usepackage[round,sort&compress]{natbib} % NCCcomment
\usepackage{url} % [hyphens]
\usepackage[hyperpageref]{backref} % back references biblio. Needs latexmk at compilation. 
\usepackage[pagebackref]{hyperref} 

% Lines below required to make pagebackref compatible with bibunits
\usepackage{etoolbox}
\makeatletter
\patchcmd\Hy@backout{\@auxout}{\@mainaux}{}{\fail}
\patchcmd\Hy@backout{\@auxout}{\@mainaux}{}{\fail} %yes twice the same line
\makeatother

% \usepackage{hyperref}
% \usepackage{multibib} % incompatible with backref
\hypersetup{
  colorlinks=true, % breaklinks=true,
  urlcolor=purple,    % color of external links
  linkcolor=blue,  % color of toc, list of figs etc.
  citecolor=violet,   % color of links to bibliography
}
\usepackage{bm}
\usepackage{indentfirst}
\usepackage{tocbibind}
\setcitestyle{aysep={}} 
\usepackage{amsmath}
\usepackage{tcolorbox}
\usepackage{amssymb}
\usepackage{eurosym}
\usepackage{amsfonts}
% \usepackage{fontspec} 
\usepackage{hwemoji} % for emojis
\usepackage{enumerate}
\usepackage{babel}
\usepackage{graphicx}
\usepackage{caption}
\usepackage{supertabular}
\usepackage{tabularx}
\usepackage{float}
\usepackage{dsfont}
\usepackage{fancyvrb}
\usepackage{verbatim}
\usepackage{enumitem}
\usepackage{setspace}
\usepackage{comment}
\usepackage{subcaption}
\usepackage{tikz}
\usepackage{gensymb}
\usepackage{textcomp}

\usepackage{tabulary}
\usepackage{tabularx}
\usepackage{booktabs}
\usepackage{fullpage}
\usepackage{morefloats}
\usepackage{makecell}
\usepackage{lscape}
\usepackage{pdflscape}
\usepackage{longtable}
\usepackage{rotating}
\usepackage{fancyhdr}
\usepackage{tocloft}
\usepackage{titletoc}
\usepackage[export]{adjustbox}
\usepackage[anythingbreaks]{breakurl} % for links
\usepackage{multicol}
\usepackage{lineno}
% \linenumbers
\newsavebox\ltmcbox % For net gain table over two columns
%\usepackage[nomarkers,figuresonly]{endfloat} % Figures at the end
%\usepackage[section,below]{placeins} % Floats placed in the section they appear in.
\renewcommand{\floatpagefraction}{.99}
\newenvironment{stretchpars}{\par\setlength{\parfillskip}{0pt}}{\par} % to justify a line

\newcommand\citeR[1]{\citeauthor{#1} (\citeyear{#1})}
\newcommand\citeRt[1]{\citeauthor{#1} (\citeyear{#1})}
\newcommand\citeRp[1]{(\citeauthor{#1} \citeyear{#1})}
\newcommand\citeRalp[1]{\citeauthor{#1} \citeyear{#1}}
\defaultbibliographystyle{unsrtnat}
\defaultbibliography{global_tax_attitudes}

\title{Acceptance for International Redistribution\\
in High-Income Countries
}
% \author{Adrien Fabre$^{1,2}$, Thomas Douenne$^3$ and Linus Mattauch$^{4,5,6}$} % WPcomment
\author{Adrien Fabre\footnote{CNRS, CIRED. E-mail: adrien.fabre@cnrs.fr (corresponding author).}
% ~~\thanks{The project is approved by Economics \& Business Ethics Committee (EBEC) at the University of Amsterdam (EB-1113) and %is approved by IRB at Harvard University (IRB21-0137), and 
} % NCCcomment

\date{\today} % NCCcomment

\begin{document}

\maketitle

\begin{center}
{\textbf{\href{https://github.com/bixiou/robustness_global_redistr/raw/main/papers/support_global_redistr.pdf}{Link to most recent version}}}
\end{center}


% WPcomment
% \begin{affiliations}
% \item CNRS
% \item CIRED
% \item University of Amsterdam
% \item Technical University Berlin
% \item Potsdam Institute for Climate Impact Research 
% \item University of Oxford
% \end{affiliations}

% \begin{small} % NCCcomment
\begin{abstract}

% Through an original survey on 12,000 respondents representative of eleven high-income countries (U.S., Japan, Russia, Saudi Arabia, and seven European countries), I study the extent of support for global redistribution and climate policies, and their sensitivity to policy features such as the magnitude of the transfers or country coverage. Though not a salient concern, global inequality is seen as a big injustice. There is majority support for almost all global policies in almost all countries, including for policies that would redistribute 5\% of the world income, or that would be costly to the respondents.  % from the global top 3\% to the bottom 40\%. 
% Global inequality is a vote-determining issue for many people; a political program is more likely to be preferred if it addresses it. Support for international policies decreases only slightly when the country coverage shrinks. These results confirm previous findings and suggest that a broad set of countries could work together for sustainable development. 

% Through an original survey on 12,000 respondents representative of eleven high-income countries (U.S., Japan, Russia, Saudi Arabia, and seven European countries), I study the extent of support for global redistribution and climate policies, and their sensitivity to policy features such as the magnitude of the transfers or country coverage. Though not a salient concern, global inequality is seen as a big injustice. There is majority support for almost all global policies in almost all countries, including for policies that would redistribute 5% of the world income, or that would be costly to the respondents. Global inequality is a vote-determining issue for many people; a political program is more likely to be preferred if it addresses it. Support for international policies decreases only slightly when the country coverage shrinks. These results confirm previous findings and suggest that a broad set of countries could work together for sustainable development. 

\end{abstract}

% \textbf{JEL codes:} P48, Q58, H23, Q54 % NCCcomment
% Q54 Climate • Natural Disasters and Their Management • Global Warming
% Q58 Government Policy (Q is Environmental econ)
% D78 Positive Analysis of Policy Formulation and Implementation
% H23 Externalities • Redistributive Effects • Environmental Taxes and Subsidies (H is public econ)
% P48 Political Economy • Legal Institutions • Property Rights • Natural Resources • Energy • Environment • Regional Studies (P4 is Other economic systems)
% H41 Public Goods
% H54 Infrastructures • Other Public Investment and Capital Stock

% \textbf{Keywords:} Climate change, global policies, cap-and-trade, attitudes, survey.%, inequality, wealth tax. % NCCcomment

\clearpage
\tableofcontents

\onehalfspacing % NCCcomment

%\clearpage
\begin{bibunit}

\section{Introduction}% NCCcomment

Questions that do not prime global solidarity show that this topic is not a salient concern for most people, which may explain why global redistribution proposals receives little attention in public debate, despite widespread acceptance. 

% Paragraphs in Kuziemko et al. (many citations are in footnotes):
% Context; Puzzle (citing theory/paper/evidence + graph); Hypotheses/method; Original method (interactivity); Main results; Specific result; Mechanisms/robustness (at length); Literature; Outline.
% Data: representativeness, attrition; Results; Robustness; Conclusion

\paragraph{Literature} 
% Conjoint: Example use: political candidates (e.g., Teele, Kalla, and Rosenbluth, 2018), immigrants (e.g., Hainmueller and Hopkins, 2015), and public policies (e.g., Ballard-Rosa, Martin, and Scheve, 2017), Hainmueller, Hopkins, and Yamamoto, 2014).
% Fields: Mondon corroborates (using Eurobarometer) than Immigration was the top concern in the UK over 2009-2019 but only #14 concern for "you, personally" (with cost of living being #1) 

\section{Data and design\label{sec:data}}

\paragraph{Samples.}
I conducted an original survey on 12,000 respondents representative of the adult population in eleven high-income countries (see Figure \ref{fig:country_coverage}). The countries have been chosen to span the diversity of high-income countries and the sample sizes to be commensurate to these countries' population sizes.\footnote{The sample sizes are as follows: U.S.: 3,000; Japan: 2,000; Russia: 1,000; Saudi Arabia: 1,000; Europe: 5,000, split in proportion to the countries' adult population size (except for Switzerland), that is France: 798; Germany: 1,048; Italy: 756; Spain: 603; Poland: 500; Switzerland: 469.} 
The survey was fielded online in 2025 using the companies \textit{Yandex} (for Russia), \textit{Kantar} (for Saudi Arabia), and \textit{Bilendi} (for the other countries).\footnote{For all countries but Russia, responses were collected between April 15 and July 3, 2025. For Russia, responses were collected between September 19 and TODO, 2025. Each complete response is rewarded around \euro{}3 in gift points.} 

In Russia, the questionnaire was curtailed as I could not ask the same questionnaire as in the other countries, for two reasons. First, I could not use the platform \textit{Qualtrics}, which prevented me from asking some question formats (such as constant sum scales) or from embedding Javascript (used to design an interactive question). Second, I had to cut or reword some questions due to preventive censorship by the survey company. In the other countries, the questionnaires are almost identical, though the figures in questions are adapted to the country context (e.g. when informing about the cost of the Global Climate Scheme to the average person of the country). Appendix \ref{app:specificities} lists the specificities of the questionnaire in each country.

% # 1? coverage map
\begin{figure}[h!]
    \caption[Country coverage]{Country coverage of the survey.
    }\label{fig:country_coverage}
    \makebox[\textwidth][c]{\includegraphics[width=\textwidth]{../figures/maps_participation/country_coverage_curtailed.png}} 
\end{figure}

\paragraph{Representativeness.}
The samples are stratified to be representative of the country's adult population along the following quota variables (with some exceptions\footnote{In the U.S., I also use race (4 categories) as a quota variable. In Saudi Arabia, I do not use urbanicity, but I use citizenship (Saudi vs. non-Saudi). In Russia, I do not use region nor urbanicity.}): gender, age (5 brackets), income (4), diploma (3), region (2 to 5), and urbanicity (2 to 3). Tables \ref{tab:representativeness_0}-\ref{tab:representativeness_3} in Appendix \ref{app:representativeness} show that our samples match the actual population frequencies along these dimensions, except for Saudi Arabia (where non-Saudis and non-high-school-educated people are underrepresented) and Russia (where TODO). All our results are reweighted to be fully representative of the population along our quotas (with weights trimmed between 0.25 and 4). Results aggregated at the global or European levels weigh each country in proportion to its adult population size. 

Appendix Figure \ref{fig:lmg} shows that 11\% to 17\% of the variance of our main attitudinal outcomes is explained by sociodemographic variables, and this share falls below 5\% after accounting for country and vote. In other words, even though variables such as age or diploma are sometimes significantly correlated with attitudes (see Tables \ref{tab:determinant}-\ref{tab:determinants_custom_redistr}), difference in average acceptance of a policy between (say) age groups rarely exceeds a dozen percentage points. 

Appendix \ref{app:pol} shows how our main attitudinal outcomes vary by political leaning. Non-voters exhibit attitudes close to the center of the politicalspectrum. Besides, attitudes are much less polarized in Japan compared to Europe and the U.S. % TODO: has polarization decreased since last paper?
Appendix Figures \ref{fig:vote_pnr_out}-\ref{fig:vote_representativeness} show how our % weighted or not TODO % TODO! US in Figure
samples compare to reality in terms of vote in the last election. While the share of declared non-voters is lower than in reality, votes along the three main political leanings are close to reality. Appendix \ref{app:vote} shows that our main results are robust to reweighting by vote. 

\paragraph{Data quality.} 
The median survey duration is 17 minutes (13 min in Russia). % Following TODO cite Stantcheva
Best practices have been used to ensure top-notch data quality. 
The questionnaire has been worded in a neutral and informative way;\footnote{At the end of the survey, 70\% of the respondents find it politically unbiased (Appendix Figure \ref{fig:survey_biased}).} tested on random people in public spaces to make sure it is correctly understood; translated by professional translators and double-checked by native speakers.

Of all respondents who started the questionnaire, only TODO \% dropped out. TODO respondents were allowed to pursue the survey (as their quotas were not full) and did not drop out.  The final sample is obtained after excluding of this extended sample TODO \% of respondents for suspicion of low quality: \% for failing an attention test and \% for answering the questionnaire in less than 6 min (including \% for both reasons). Appendix \ref{app:attrition} checks for differential attrition and Appendix \ref{app:extended} shows that our main results replicate on the extended sample. 
% TODO: feedback field

Whenever possible, the order of question items is randomized. Appendix \ref{app:order} studies the effect of item order on answers. The item order generally has a significant but small effect (2 to 10~p.p.). The size of the effect help identify questions for which opinions are strongly held (e.g. the preference of a sustainable scenario over the status quo) \textit{versus} weakly held (e.g. the preferred amount of climate finance). % crystallized, well-formed, consolidated, stable, ambivalent

Appendix \ref{app:comparison} compares the answers of two attitudinal questions asked in other surveys: the overall averages differ by 2 to 4~p.p. and their cross-country correlation is high: .70 \citep{global_nation_global_2023} to .86 \citep{cappelen_majority_2025}. 

\paragraph{Incentives.}
The questionnaire includes three incentivized questions, each awarded with a \$100 prize for one randomly drawn winner. First, a comprehension question on the Global Climate Scheme (GCS) checks whether respondents have understood the policy's cost. % TODO: support among those who haven't understood
Second, a donation lottery where respondents choose what part of the prize they would donate to a reforestation NGO, should they win the lottery. Third, a question on the perception of the actual support for the GCS, which rewards a correct guess.

\paragraph{Survey structure.}
While Appendix \ref{app:questionnaire} provides the full questionnaire, Figure \ref{fig:flow} depicts the survey flow with all random branches. The various treatments are all independent and uniformly distributed. Appendix \ref{app:placebo} runs placebo tests to check whether earlier treatments have an effect on unrelated outcomes. 
% # 2. survey_flow
\begin{figure}[h!]
    \caption[Survey flow]{Survey flow.
    }\label{fig:flow}
    \makebox[\textwidth][c]{\includegraphics[width=\textwidth]{../figures/questionnaire/survey_flow.pdf}} 
\end{figure}

After sociodemographic characteristics, the questionnaire starts with broad questions to grasp the prioritization and salience of global solidarity before the respondents can understand the survey's topic. First, open-ended fields on either their main concerns, wants, issues of interest, or perceived injustices. Second, a conjoint experiment where respondents have to select their preferred political program, or abstain. Both programs are random: each policy (or absence of policy) in five policy domains is taken at random from a pool of policies prominent in the country's public debate. Third, respondents are asked to allocate the revenue of a global wealth tax between five (national or global) spending items. 

Then come attitudinal questions on the main policies studied: a \textit{Climate Scheme} at the national, global, or international level; an international wealth tax funding low-income countries; and ten plausible global solidarity policies. These questions include treatments that vary the international coverage of policies or test for warm glow. 

The last part of the questionnaire explores attitudes towards more radical global redistribution scenarios and include more sophisticated questions, such as an interactive task where respondents can choose their preferred custom redistribution of global incomes by manipulating sliders. 

Finally, the survey concludes with a comprehension question, synthetic questions (e.g. on moral circle) and a feedback field.

\paragraph{Pre-registered hypotheses and data availability.}
The project is approved by the CIRED institutional review board (IRB-CIRED-2025-2) and %is approved by IRB at Harvard University (IRB21-0137), and 
was preregistered in the Open Science Foundation registry (\href{https://osf.io/7mzn4}{osf.io/7mzn4}). The study did not deviate from the registration: the questionnaires and the hypotheses tests used are the ones \href{https://osf.io/j5scn}{specified \textit{ex ante}}. All data and code as well as figures of the paper are available on \href{https://github.com/bixiou/robustness_global_redistr}{github.com/bixiou/robustness\_global\_redistr}. 


\section{Salience and prioritization of global solidarity\label{sec:salience}}
% - People view global poverty/ineq as big injustice though not salient concern (e.g. revenue_split)
In this Section, I analyze the salience of global solidarity in undirected open-ended fields; and the prioritization of global programs in a budget allocation task.

\subsection{Top-of-mind considerations}

At the beginning of the survey, respondents are randomly assigned one open-ended question among four: their main concerns, their needs or wishes, an issue important to them but neglected in public debate, or the greatest injustice of all. The questions are deliberately broad and vague to let respondents express their top-of-mind considerations without any priming.

To analyze the answers, I automatically translated each field into English.\footnote{I used \href{https://www.onlinedoctranslator.com/en/translationform}{onlinedoctranslator.com}, which is powered by \textit{Google Translate}.} Then, I used AI and my own reading of a few hundreds answers to identify the most common concepts, from which I selected 27 categories. Then, I classified each answer into one or more of these categories, both manually (Figures \ref{fig:field_manual}-\ref{fig:injustice_field}) and automatically using AI (Figure \ref{fig:field_gpt}). Finally, I manually defined a list of 47 (conjunction of) keywords and used it to automatically classify all answers.\footnote{The list of keywords is given in Appendix TODO.} I report occurrences of the 24 most common categories in Figure \ref{fig:field_keyword}. 

The three different classification methods yield consistent results but differ in accuracy. While the keyword classification allows an exact and reproducible search of concepts, the automatic search is not bound to specific words and captures more matching responses. % Although the manual classification could be less consistent than the AI one (if e.g. the interpretation of the category changes over time or among coders), 
Overall, it seems that the manual classification provides the most accurate results, with a number of matches generally between the two other methods. 
For example, to the \textit{injustice} question, 1.3\% of answers match the keywords for \textit{global inequality}, ChatGPT identifies 8.5\% of answers in this category, \textit{versus} 3.7\% according to my manual coding.\footnote{The keyword matching searches the regular expression ``global poverty|global inequal|hunger|drinking water|starv'', ignoring case. The automatic and manual classification is based on the class definition ``Inequality at the inetrnational level / Hunger or poverty in poor countries''.} 
Indeed, the AI incorrectly classifies answers like ``poverty'' in this category,\footnote{Interestingly, out of the 45 (one-word) answers ``poverty'', ChatGPT coded only 42 of them as \textit{global inequality}, showing the lack of consistency of this classifier.} while the keyword search misses answers like ``inequality among humans''. 
Given this observation, I use the manual classification as the benchmark and the two other methods as robustness checks.

While less accurate than the classifications, wordclouds (Figure \ref{fig:wordcloud}) provide a simple visualization of the most common concepts in each question. By far, the most frequent \textit{concerns} or \textit{wishes} of respondents relate to their purchasing power, with concepts such as ``money'', ``inflation'', or the ``cost of living'' mobilized in 30\% of these fields. Within country, the share of people concerned with money decreases with income: it ranges from 21\% in the top income decile to 36\% in the bottom one. % TODO define e$gdp_pc_ppp and cor money
The next most frequent \textit{concerns} are health (or the healthcare system), far right governments (or related concepts such as ``Trump'' or ``trade tariffs''), and war (either in general or a specific one such as the Gaza war). Most \textit{wishes} are personal, with the next frequent ones related to the health or peace of mind of oneself or one's relatives. Interestingly, almost none of the responses mention social considerations such as love, intimate life, friendships, loneliness, or the desire to have children (with the exception of Saudi Arabia for the latter). Further research is needed to test whether the predominance of materialistic considerations stems from the context (an impersonal survey) or truly reflects people's primary thoughts. 

% injustice...

% # 3. keywords in fields (taken jointly)
\begin{figure}[h!]
  \caption[Wordcloud of open-ended field, per variant]{Most common concepts in open-ended fields. (Questions \ref{q:concerns_field}-\ref{q:injustice_field})} \label{fig:wordcloud} % Random answers can be found on \href{http://preferences-pol.fr/fields2025.html}{bit.ly/fields2025}
  \begin{subfigure}{.49\textwidth}
    \caption[]{``What are your main concerns these days?''}
    \includegraphics[width=\textwidth]{../figures/all/concerns_field_en.pdf}
  \end{subfigure} 
  \begin{subfigure}{.49\textwidth}
    \caption[]{``What are your needs or wishes?''}
    \includegraphics[width=\textwidth]{../figures/all/wish_field_en.pdf}
  \end{subfigure}
  \begin{subfigure}{.49\textwidth}
    \caption[]{``Can you name an issue that is important to you but is neglected in the public debate?''}
    \includegraphics[width=\textwidth]{../figures/all/issue_field_en.pdf}
  \end{subfigure} 
  \begin{subfigure}{.49\textwidth}
    \caption[]{``What according to you is the greatest injustice of all?''}
    \includegraphics[width=\textwidth]{../figures/all/injustice_field_en.pdf}
  \end{subfigure} 
\end{figure}
% # 4. revenue_split: country_comparison/split_main_means_nolegend + country_comparison/split_main_nb0_nolabel
\begin{figure}[h!]
  \caption[Preferred split of revenue of an international wealth tax]{Preferred split of revenue of an international wealth tax. The first two items are from the variant \textit{Few} with 5 fixed items (the \textit{Global} one and the most preferred one are displayed); the last four items are from the variant \textit{Many} with 5 items taken at random out of 13 (the 4 \textit{Global} ones are displayed). \hfill (Questions \ref{q:revenue_split_few}-\ref{q:revenue_split_many})} \label{fig:split}

  % \begin{subfigure}{.45\textwidth}
  %   \caption[]{Average preferred allocation (in \%).}
  %   \begin{flushleft}
  %   \includegraphics[height=.36\textheight]{../figures/country_comparison/split_main_means_nolegend.pdf}\end{flushleft}
  % \end{subfigure} 
  % \begin{subfigure}{.55\textwidth}
  %   \caption[]{Share of respondents allocating 0 revenue (in \%).} 
  %   \begin{flushright}
  %   \includegraphics[height=.36\textheight]{../figures/country_comparison/split_main_nb0_nolabel.pdf}\end{flushright}

\begin{subfigure}{.62\textwidth}
    \caption[]{Average preferred allocation (in \%).}
    \includegraphics[height=.38\textheight]{../figures/country_comparison/split_main_means_nolegend.pdf}
    \end{subfigure} 
  \begin{subfigure}{.38\textwidth}
    \caption[]{Share of respondents allocating 0 revenue (in \%).} 
    \includegraphics[height=.38\textheight]{../figures/country_comparison/split_main_nb0_nolabel.pdf}

  \end{subfigure}
\end{figure}

\section{Acceptance for international policies in function of country coverage\label{sec:coverage}}
% - Strong support for global tax / GCS even with partial participation

% NB: I chose to present GCS on control. Trade-off between presenting control results (not contaminated by treatment though less precise) and overall results (more comparable to ICS as those mix treated and control, though less comparable to NCS). I prioritized presenting results for themselves rather than for comparison with ICS.

% # 5a. ICS: \includegraphics[width=\textwidth]{../figures/country_comparison/variables_ncs_gcs_ics_by_country}
\begin{figure}[h!]
    \caption[Support for the NCS, GCS, ICS]{Support for the National, Global, and International Climate Schemes (\textit{Yes}/\textit{No} question). \hfill (Questions \ref{q:ncs_support}-\ref{q:ics_support}).
    }\label{fig:ics}
    \makebox[\textwidth][c]{\includegraphics[width=\textwidth]{../figures/country_comparison/variables_ncs_gcs_ics_control_by_country.pdf}} 
\end{figure}
% # 5b. wealth tax by coverage: \includegraphics[width=\textwidth]{../figures/country_comparison/variables_wealth_tax_support_by_country}
\begin{figure}[h!]
    \caption[Support for an international wealth tax depending on country coverage]{Support for an international wealth tax with 30\% of revenue funding LICs, depending on the country coverage (\textit{Yes}/\textit{No} question). \hfill (Questions \ref{q:global_tax_support}-\ref{q:intl_tax_support}).
    }\label{fig:wealth_tax}
    \makebox[\textwidth][c]{\includegraphics[width=.9\textwidth]{../figures/country_comparison/variables_wealth_tax_support_by_country.pdf}} 
\end{figure}

\section{Sincerity of support for global redistribution\label{sec:sincerity}}
% - No (or little) evidence of warm glow or support due to unrealism; global redistr genuinely supported (conjoint analysis)

% # 6. conjoint: foreign aid + global tax (per country + global)
\begin{figure}[h!]
\caption[Conjoint analysis: effect of development aid and millionaire tax]{Effect on the likelihood that a political program is preferred of containing the following policy (compared to no foreign policy in the program). (See Figure \ref{fig:conjoint_vote} for effects by vote). \hfill (Question \ref{q:conjoint})} \label{fig:conjoint}
\begin{subfigure}{.49\textwidth}
  \caption[]{Cut development aid}
  \includegraphics[height=.36\textheight]{../figures/country_comparison/program_preferred_by_cut_aid_in_program.pdf}
\end{subfigure} 
\begin{subfigure}{.49\textwidth}
  \caption[]{International tax on millionaires with 30\% financing health and education in low-income countries}%\begin{flushright}
  \includegraphics[height=.36\textheight]{../figures/country_comparison/program_preferred_by_millionaire_tax_in_program.pdf}%\end{flushright}
\end{subfigure}
\end{figure}
% # 7. warm_glow: effect of info + display donation vs. control (per country + global)
    \begin{figure}[h!]
\caption[Testing warm glow]{Testing warm glow (negative effects would indicate the presence of warm glow).}\label{fig:warm_glow}
\begin{subfigure}{.45\textwidth}
  \caption[]{Effect of a \textit{Donation lottery} treatment on support for the Global Climate Scheme. (Questions \ref{q:donation}-\ref{q:gcs_support})\label{fig:warm_glow_substitute}}
  \includegraphics[height=.36\textheight]{../figures/country_comparison/gcs_support_by_variant_warm_glow.pdf}
\end{subfigure} \quad
\begin{subfigure}{.49\textwidth}
  \caption[]{Effect of information about ongoing global redistribution initiatives on the share of plausible global policies supported. (Questions \ref{q:info_solidarity}-\ref{q:solidarity_support})\label{fig:warm_glow_realism}}
  \includegraphics[height=.36\textheight]{../figures/country_comparison/share_solidarity_supported_by_info_solidarity.pdf}
\end{subfigure}
\end{figure}
% # 7bis: 2SLS 
\begin{table}[!htbp] 
  \caption[Effect on support for global redistribution of believing that it is likely]{Effect on support for global redistribution of believing that it is likely.}\label{tab:iv} 
  \makebox[\textwidth][c]{
% Table created by stargazer v.5.2.3 by Marek Hlavac, Social Policy Institute. E-mail: marek.hlavac at gmail.com
% Date and time: ven., sept. 26, 2025 - 18:54:05
\begin{tabular}{@{\extracolsep{5pt}}lcccc} 
\\[-1.8ex]\hline 
\hline \\[-1.8ex] 
\\[-1.8ex] & \makecell{Believes global\\redistr. likely} & \multicolumn{3}{c}{Share of plausible global policies supported} \\ 
 & IV 1st Stage & IV 2nd Stage & OLS & Direct Effect \\ 
\\[-1.8ex] & (1) & (2) & (3) & (4)\\ 
\hline \\[-1.8ex] 
 Information treatment & 0.077$^{***}$ &  &  & 0.012$^{*}$ \\ 
  & (0.009) &  &  & (0.006) \\ 
  Believes global redistribution likely &  & 0.153$^{*}$ & 0.152$^{***}$ &  \\ 
  &  & (0.081) & (0.006) &  \\ 
  (Intercept) & 0.332$^{***}$ & 0.453$^{***}$ & 0.453$^{***}$ & 0.504$^{***}$ \\ 
  & (0.006) & (0.030) & (0.004) & (0.004) \\ 
 \hline \\[-1.8ex] 
Observations & 11,872 & 11,872 & 11,872 & 11,872 \\ 
R$^{2}$ & 0.006 & 0.044 & 0.044 & 0.0003 \\ 
F Statistic (df = 1; 11870) & 76.042$^{***}$ &  & 546.559$^{***}$ & 3.406$^{*}$ \\ 
\hline 
\hline \\[-1.8ex] 
\textit{Note:}  & \multicolumn{4}{r}{$^{*}$p$<$0.1; $^{**}$p$<$0.05; $^{***}$p$<$0.01} \\ 
\end{tabular} 
}
\end{table}

\section{Breadth of international policies accepted\label{sec:breadth}}
% - People ready to sustainability and radical global redistr (for duty, not reparations)
% - People favor social protection to unconditional transfers

% # 8. solidarity_support (on control): heatmap \includegraphics[height=.89\textheight]{../figures/country_comparison/solidarity_support_share}
\begin{figure}[h!]
    \caption[Relative support for plausible global redistribution policies]{Relative support for plausible global redistribution policies (Percentage of \textit{Somewhat} or \textit{Strongly support} among non-\textit{Indifferent} responses). (Question \ref{q:solidarity_support}).
    }\label{fig:solidarity_support_share}
    \makebox[\textwidth][c]{\includegraphics[width=\textwidth]{../figures/country_comparison/solidarity_support_share.pdf}} 
\end{figure}
% # 9. radical_redistr: heatmap sustainability, top_tax, reparations, NCQG? TODO!, vote_intl_coalition, group_defended?, my_tax_global_nation, convergence_support # my_tax_global_nation other source? No, just mention corrs and means in appendix/footnote & say no sufficient confidence in their representativeness + compare w Stostad \includegraphics[width=\textwidth]{../figures/country_comparison/radical_redistr_few_share}
\begin{figure}[h!]
    \caption[Support for broad or radical global redistribution]{Support for broad action or radical proposals of global redistribution. \hfill (Questions \ref{q:sustainable_future}-\ref{q:top3_tax_support}, \ref{q:convergence_support}-\ref{q:vote_intl_coalition}, \ref{q:reparations_support}, \ref{q:my_tax_global_nation}).
    }\label{fig:radical_redistr_share} 
    \makebox[\textwidth][c]{\includegraphics[width=\textwidth]{../figures/country_comparison/radical_redistr_all_share.pdf}} % radical_redistr_share
\end{figure}
% # 10. group_defended: barresN or barres? \includegraphics[height=.9\textheight]{../figures/all/group_defended}
\begin{figure}[h!]
    \caption[Moral circle]{``Which group of people do you advocate for when you vote?'' (Question \ref{q:group_defended}).
    }\label{fig:group_defended}
    \makebox[\textwidth][c]{\includegraphics[width=.9\textwidth]{../figures/all/group_defended_nolabel.pdf}} 
\end{figure}
% # 11. transfer_how: heatmap (maybe just one row grouping all countries and options in columns) \includegraphics[width=\textwidth]{../figures/country_comparison/transfer_how_positive}
\begin{figure}[h!]
    \caption[\textit{Right} or \textit{Best} way to transfer resources to LICs (global average)]{``How do you evaluate each of these channels to transfer resources to reduce poverty in LICs?''\\ Percentage of \textit{Right} or \textit{Best} way (other options: \textit{Wrong} or \textit{Acceptable} way). (Question \ref{q:transfer_how}).
    }\label{fig:transfer_how}
    \makebox[\textwidth][c]{\includegraphics[width=\textwidth]{../figures/country_comparison/transfer_how_positive.pdf}} 
\end{figure}
% # 12. average custom_redistr \includegraphics[height=.8\textheight]{../figures/all/custom_redistr_mean.png}
\begin{figure}[h!]
   \caption[Average custom redistribution]{Average custom global redistribution. (Question \ref{q:custom_redistr}).
    }\label{fig:custom_redistr_question}
    \makebox[\textwidth][c]{\includegraphics[height=.85\textheight]{../figures/questionnaire/survey_custom_redistr_bottom.png}} % bottom2 is without the choice options satisfied/skip
\end{figure}


\section{Conclusion} % Summary, conclusion


% Appendix TODO
% - Influenced by question on revenue splits and NCQG / more or less consistent on NCQG


% \begin{methods}  % WPcomment
  \begin{small} % NCCcomment
%Put methods in here.  If you are going to subsection it, use \subsection commands.  Methods section should be less than 800 words and if it is less than 200 words, it can be incorporated into the main text.
\section*{\normalsize Methods}\label{sec:methods} % NCCcomment
\addcontentsline{toc}{section}{\nameref{sec:methods}}

% The paper utilizes two sets of surveys: the \textit{global} survey and the \textit{Western} surveys. The \textit{global} surveys consist of two U.S. surveys, \textit{US1} and \textit{US2}, and one European survey, \textit{Eu}. The \textit{global} survey was conducted from March 2021 to March 2022 on 40,680 respondents from 20 countries (with 1,465 to 2,488 respondents per country). \textit{US1} collected responses from 3,000 respondents between January and March 2023, while \textit{US2} gathered data from 2,000 respondents between March and April 2023. \textit{Eu} included 3,000 respondents and was conducted from February to March 2023. We used the survey companies \emph{Dynata} and \emph{Bilendi}. To ensure representative samples, we employed stratified quotas based on gender, age (5 brackets), income (4), region (4), education level (3), and ethnicity (3) for the U.S. We also incorporated survey weights throughout the analysis to account for any remaining imbalances. These weights were constructed using the quota variables as well as the degree of urbanization, and trimmed between 0.25 and 4. Stratified quotas followed by reweighting is the usual method to reduce selection bias from opt-in online panels, when better sampling methods (such as compulsory participation of random dwellings) are unavailable.\cite{scherpenzeel_how_2010} By applying weights, the results are fully representative of the respective countries along the above mentioned dimensions. %. 
% Appendix \ref{app:balance} shows that the treatment branches are balanced. Appendix \ref{app:placebo} runs placebo tests of the effects of each treatment on unrelated outcomes. We do not find effects of earlier treatments on unrelated outcomes arriving later in the survey. Appendix \ref{app:extended} shows that our results are unchanged when including inattentive respondents.

\section*{\normalsize Acknowledgements}
bla

% \section*{\normalsize Competing interests} Fabre declares that he also serves as treasurer of Global Redistribution Advocates.

% \end{methods} % WPcomment
\end{small}  % NCCcomment

% \bibliographystyle{naturemag_noURL} % nature class works only with style naturemag or naturemag_noURL, and naturemag bugs if there are certain URLs (like .pdf). Also, nature class only works with \cite, not \citet or \citep.  % WPcomment
\renewcommand{\url}[1]{\href{#1}{Link}} % NCCcomment
% \bibliographystyle{unsrtnat} % NCCcomment
% \bibliography{global_tax_attitudes}

\clearpage

\putbib
\end{bibunit}

\begin{bibunit}[plainnaturl_clean]

\appendix % NCCcomment
% \renewcommand{\thetable}{ED\arabic{table}}
% \renewcommand{\thefigure}{ED\arabic{figure}}
% \setcounter{figure}{0}
% \setcounter{table}{0}


\renewcommand{\thetable}{S\arabic{table}}
\renewcommand{\thefigure}{S\arabic{figure}}
\setcounter{figure}{0}
\setcounter{table}{0}

\clearpage
\section{Raw results% from the complementary surveys
}\label{app:raw_results}

% Country-specific raw results are also available as supplementary material files:  \href{https://github.com/bixiou/international_attitudes_toward_global_policies/raw/main/paper/app_desc_stats_US.pdf}{US}, \href{https://github.com/bixiou/international_attitudes_toward_global_policies/raw/main/paper/app_desc_stats_EU.pdf}{EU}, \href{https://github.com/bixiou/international_attitudes_toward_global_policies/raw/main/paper/app_desc_stats_FR.pdf}{FR}, \href{https://github.com/bixiou/international_attitudes_toward_global_policies/raw/main/paper/app_desc_stats_DE.pdf}{DE}, \href{https://github.com/bixiou/international_attitudes_toward_global_policies/raw/main/paper/app_desc_stats_ES.pdf}{ES}, \href{https://github.com/bixiou/international_attitudes_toward_global_policies/raw/main/paper/app_desc_stats_UK.pdf}{UK}.

\begin{figure}[h!]
    \caption[Keyword classification of open-ended fields]{Keyword classification of open-ended fields (matches with at least one keyword in a list). (Questions \ref{q:concerns_field}-\ref{q:injustice_field}).
    }\label{fig:field_keyword}
    \makebox[\textwidth][c]{\includegraphics[height=.85\textheight]{../figures/country_comparison/field_keyword_main_positive.pdf}} 
\end{figure}

\begin{figure}[h!]
    \caption[AI classification of open-ended fields]{AI classification of open-ended fields (using ChatGPT-4.1). (Questions \ref{q:concerns_field}-\ref{q:injustice_field}).
    }\label{fig:field_gpt}
    \makebox[\textwidth][c]{\includegraphics[height=.85\textheight]{../figures/country_comparison/field_gpt_positive.pdf}} 
\end{figure}

\begin{figure}[h!]
    \caption[Manual classification of open-ended fields]{Manual classification of open-ended fields. (Questions \ref{q:concerns_field}-\ref{q:injustice_field}).
    }\label{fig:field_manual}
    \makebox[\textwidth][c]{\includegraphics[height=.85\textheight]{../figures/country_comparison/field_manual_positive.pdf}} 
\end{figure}

\begin{figure}[h!]
    \caption[Manual classification of \textit{concerns} fields]{Manual classification of \textit{concerns} fields: ``What are your main concerns these days?'' (Question \ref{q:concerns_field}).
    }\label{fig:concerns_field}
    \makebox[\textwidth][c]{\includegraphics[height=.85\textheight]{../figures/country_comparison/field_concerns_manual_positive.pdf}} 
\end{figure}

\begin{figure}[h!]
    \caption[Manual classification of \textit{wish} fields]{Manual classification of \textit{wish} fields: ``What are your needs or wishes?'' (Question \ref{q:wish_field}).
    }\label{fig:wish_field}
    \makebox[\textwidth][c]{\includegraphics[height=.85\textheight]{../figures/country_comparison/field_wish_manual_positive.pdf}} 
\end{figure}

\begin{figure}[h!]
    \caption[Manual classification of \textit{issue} fields]{Manual classification of \textit{issue} fields: ``Can you name an issue that is important to you but is neglected in the public debate?'' (Question \ref{q:issue_field}).
    }\label{fig:issue_field}
    \makebox[\textwidth][c]{\includegraphics[height=.85\textheight]{../figures/country_comparison/field_issue_manual_positive.pdf}} 
\end{figure}

\begin{figure}[h!]
    \caption[Manual classification of \textit{injustice} fields]{Manual classification of \textit{injustice} fields: ``What according to you is the greatest injustice of all?'' (Question \ref{q:injustice_field}).
    }\label{fig:injustice_field}
    \makebox[\textwidth][c]{\includegraphics[height=.85\textheight]{../figures/country_comparison/field_concerns_manual_positive.pdf}} 
\end{figure}

\begin{figure}[h!]
    \caption[Conjoint analysis in France]{Conjoint analysis in France (Average Marginal Component Effect). \hfill (Question \ref{q:conjoint}).
    }\label{fig:conjoint_FR}
    \makebox[\textwidth][c]{\includegraphics[width=\textwidth]{../figures/FR/conjoint_EN-FR.pdf}} 
\end{figure}

\begin{figure}[h!]
    \caption[Conjoint analysis in Germany]{Conjoint analysis in Germany (Average Marginal Component Effect). \hfill (Question \ref{q:conjoint}).
    }\label{fig:conjoint_DE}
    \makebox[\textwidth][c]{\includegraphics[width=\textwidth]{../figures/DE/conjoint_EN-DE.pdf}} 
\end{figure}

\begin{figure}[h!]
    \caption[Conjoint analysis in Italy]{Conjoint analysis in Italy (Average Marginal Component Effect). \hfill (Question \ref{q:conjoint}).
    }\label{fig:conjoint_IT}
    \makebox[\textwidth][c]{\includegraphics[width=\textwidth]{../figures/IT/conjoint_EN-IT.pdf}} 
\end{figure}

\begin{figure}[h!]
    \caption[Conjoint analysis in Poland]{Conjoint analysis in Poland (Average Marginal Component Effect). \hfill (Question \ref{q:conjoint}).
    }\label{fig:conjoint_PL}
    \makebox[\textwidth][c]{\includegraphics[width=\textwidth]{../figures/PL/conjoint_EN-PL.pdf}} 
\end{figure}

\begin{figure}[h!]
    \caption[Conjoint analysis in Spain]{Conjoint analysis in Spain (Average Marginal Component Effect). \hfill (Question \ref{q:conjoint}).
    }\label{fig:conjoint_ES}
    \makebox[\textwidth][c]{\includegraphics[width=\textwidth]{../figures/ES/conjoint_EN-ES.pdf}} 
\end{figure}

\begin{figure}[h!]
    \caption[Conjoint analysis in the UK]{Conjoint analysis in the UK (Average Marginal Component Effect). \hfill (Question \ref{q:conjoint}).
    }\label{fig:conjoint_GB}
    \makebox[\textwidth][c]{\includegraphics[width=\textwidth]{../figures/GB/conjoint_EN-GB.pdf}} 
\end{figure}

\begin{figure}[h!]
    \caption[Conjoint analysis in Switzerland]{Conjoint analysis in Switzerland (Average Marginal Component Effect). \hfill (Question \ref{q:conjoint}).
    }\label{fig:conjoint_CH}
    \makebox[\textwidth][c]{\includegraphics[width=\textwidth]{../figures/CH/conjoint_EN-CH.pdf}} 
\end{figure} 

\begin{figure}[h!]
    \caption[Conjoint analysis in Japan]{Conjoint analysis in Japan (Average Marginal Component Effect). \hfill (Question \ref{q:conjoint}).
    }\label{fig:conjoint_JP}
    \makebox[\textwidth][c]{\includegraphics[width=\textwidth]{../figures/JP/conjoint_EN-JA.pdf}} 
\end{figure} 
\begin{figure}[h!]
    \caption[Conjoint analysis in the U.S.]{Conjoint analysis in the U.S. (Average Marginal Component Effect). \hfill (Question \ref{q:conjoint}).
    }\label{fig:conjoint_US}
    \makebox[\textwidth][c]{\includegraphics[width=\textwidth]{../figures/US/conjoint_EN.pdf}} 
\end{figure} 

\begin{figure}[h!]
    \caption[Conjoint analysis in France (French)]{Conjoint analysis in France (in French, Average Marginal Component Effect). \hfill (Question \ref{q:conjoint}).
    }\label{fig:conjoint_FR_original}
    \makebox[\textwidth][c]{\includegraphics[width=\textwidth]{../figures/FR/conjoint_FR.pdf}} 
\end{figure}

\begin{figure}[h!]
    \caption[Conjoint analysis in Germany (German)]{Conjoint analysis in Germany (in German, Average Marginal Component Effect). \hfill (Question \ref{q:conjoint}).
    }\label{fig:conjoint_DE_original}
    \makebox[\textwidth][c]{\includegraphics[width=\textwidth]{../figures/DE/conjoint_DE.pdf}} 
\end{figure}

\begin{figure}[h!]
    \caption[Conjoint analysis in Italy (Italian)]{Conjoint analysis in Italy (in Italian, Average Marginal Component Effect). \hfill (Question \ref{q:conjoint}).
    }\label{fig:conjoint_IT_original}
    \makebox[\textwidth][c]{\includegraphics[width=\textwidth]{../figures/IT/conjoint_IT.pdf}} 
\end{figure}

\begin{figure}[h!]
    \caption[Conjoint analysis in Poland (Polish)]{Conjoint analysis in Poland (in Polish, Average Marginal Component Effect). \hfill (Question \ref{q:conjoint}).
    }\label{fig:conjoint_PL_original}
    \makebox[\textwidth][c]{\includegraphics[width=\textwidth]{../figures/PL/conjoint_PL.pdf}} 
\end{figure}

\begin{figure}[h!]
    \caption[Conjoint analysis in Spain (Spanish)]{Conjoint analysis in Spain (in Spanish, Average Marginal Component Effect). \hfill (Question \ref{q:conjoint}).
    }\label{fig:conjoint_ES_original}
    \makebox[\textwidth][c]{\includegraphics[width=\textwidth]{../figures/ES/conjoint_ES-ES.pdf}} 
\end{figure}


\begin{figure}[h!]
    \caption[Average preferred revenue split (\textit{few})]{Average preferred revenue split for a global wealth tax (variant \textit{few}). (Question \ref{q:revenue_split_few}).
    }\label{fig:split_few_bars_nb0}
    \makebox[\textwidth][c]{\includegraphics[width=.8\textwidth]{../figures/country_comparison/split_few_bars_nb0.pdf}} 
\end{figure}

\begin{figure}[h!]
    \caption[Decomposition of preferred spendings in revenue split]{Decomposition of preferred shares for each spending item in the revenue split (\textit{All} countries together; variant \textit{few}). (Question \ref{q:revenue_split_few}).
    }\label{fig:split_few}
    \makebox[\textwidth][c]{\includegraphics[width=\textwidth]{../figures/all/split_few.pdf}} 
\end{figure}

\begin{figure}[h!]
    \caption[Decomposition of preferred spendings in revenue split]{Decomposition of preferred shares for each spending item in the revenue split (\textit{All} countries together; variant \textit{many}). (Question \ref{q:revenue_split_many}).
    }\label{fig:split_many}
    \makebox[\textwidth][c]{\includegraphics[width=.9\textwidth]{../figures/all/split_many.pdf}} 
\end{figure}

\begin{figure}[h!]
    \caption[Average preferred revenue split (\textit{many})]{Average preferred revenue split for a global wealth tax (variant \textit{many}). (Question \ref{q:revenue_split_many}).
    }\label{fig:split_many_global_mean}
    \makebox[\textwidth][c]{\includegraphics[width=\textwidth]{../figures/country_comparison/split_many_global_mean.pdf}} 
\end{figure} 

% \begin{figure}[h!]
%     \caption[]{. (Question \ref{q:gcs_support}).
%     }\label{fig:gcs_support}
%     \makebox[\textwidth][c]{\includegraphics[width=\textwidth]{../figures/country_comparison/split_few_mean.pdf}} 
% \end{figure}

% \begin{figure}[h!]
%     \caption[Amounts donated to plant trees.]{Amounts donated to plant trees, in the lottery. (Question \ref{q:donation}).
%     }\label{fig:donation}
%     \makebox[\textwidth][c]{\includegraphics[width=\textwidth]{../figures/country_comparison/donation.pdf}} 
% \end{figure}

% \begin{figure}[h!]
%     \caption[Support for the GCS and belief of support]{Support for the Global Climate Scheme and average belief regarding the support. (Questions \ref{q:gcs_support}-\ref{q:gcs_belief_own}).
%     }\label{fig:gcs_belief}
%     \makebox[\textwidth][c]{\includegraphics[width=\textwidth]{../figures/country_comparison/gcs_belief_mean.pdf}} 
% \end{figure} 

\begin{figure}[h!]
    \caption[Support for the NCS, GCS, ICS, and belief of support for GCS]{Support for the National, Global, and International Climate Schemes, and average belief regarding the support for the GCS. (Questions \ref{q:ncs_support}-\ref{q:ics_support}).
    }\label{fig:ncs_gcs_ics}
    \makebox[\textwidth][c]{\includegraphics[width=\textwidth]{../figures/country_comparison/ncs_gcs_ics_positive.pdf}} 
\end{figure} 

\begin{figure}[h!]
    \caption[Absolute support for plausible global redistribution policies]{Absolute support for plausible global redistribution policies (Percentage of \textit{Somewhat} or \textit{Strongly support}). (Question \ref{q:solidarity_support}).
    }\label{fig:solidarity_support_positive}
    \makebox[\textwidth][c]{\includegraphics[width=\textwidth]{../figures/country_comparison/solidarity_support_positive.pdf}} 
\end{figure}

\begin{figure}[h!]
    \caption[Preferred NCQG]{Preferred North-to-South climate grant funding in 2035 (NCQG, variant \textit{Short}). (Question \ref{q:ncqg}).
    }\label{fig:ncqg}
    \makebox[\textwidth][c]{\includegraphics[width=.9\textwidth]{../figures/all/ncqg.pdf}} 
\end{figure}

\begin{figure}[h!]
    \caption[Preferred NCQG]{Preferred North-to-South climate grant funding in 2035 (NCQG, variant \textit{Full}). (Question \ref{q:ncqg_full}).
    }\label{fig:ncqg_full}
    \makebox[\textwidth][c]{\includegraphics[width=.9\textwidth]{../figures/all/ncqg_full.pdf}} 
\end{figure}

\begin{figure}[h!]
    \caption[Support for an international wealth depending on country coverage]{Support for an international wealth tax with 30\% of revenue funding LICs, depending on the country coverage (\textit{Yes}/\textit{No} question). (Questions \ref{q:global_tax_support}-\ref{q:intl_tax_support}).
    }\label{fig:wealth_tax_heatmap}
    \makebox[\textwidth][c]{\includegraphics[width=\textwidth]{../figures/country_comparison/wealth_tax_support_positive.pdf}} 
\end{figure}

\begin{figure}[h!]
    \caption[Prefers a sustainable future]{Prefers a \textit{sustainable} rather than a \textit{business-as-usual} future. (Question \ref{q:sustainable_future}).
    }\label{fig:sustainable_future}
    \makebox[\textwidth][c]{\includegraphics[width=.9\textwidth]{../figures/all/sustainable_future.pdf}} 
\end{figure}

\begin{figure}[h!]
    \caption[Relative support for a global income tax on the richest to fund LICs]{Relative support for a global progressive income tax on the richest households to finance poverty reduction in the Global South (Percentage of \textit{Somewhat} or \textit{Strongly support} among non-\textit{Indifferent} responses). (Questions \ref{q:top1_tax_support}-\ref{q:top3_tax_support}).
    }\label{fig:top_tax_share}
    \makebox[\textwidth][c]{\includegraphics[width=\textwidth]{../figures/country_comparison/top_tax_share.pdf}} 
\end{figure}

\begin{figure}[h!]
    \caption[Absolute support for an income tax on top 1\% to fund LICs]{Absolute support for a global progressive income tax on the richest households to finance poverty reduction in the Global South (Percentage of \textit{Somewhat} or \textit{Strongly support}). (Questions \ref{q:top1_tax_support}-\ref{q:top3_tax_support}).
    }\label{fig:top_tax_positive}
    \makebox[\textwidth][c]{\includegraphics[width=\textwidth]{../figures/country_comparison/top_tax_positive.pdf}} 
\end{figure}

\begin{figure}[h!]
    \caption[\textit{Right} or \textit{Best} way to transfer resources to LICs]{``How do you evaluate each of these channels to transfer resources to reduce poverty in LICs?''\\ Percentage of \textit{Right} or \textit{Best} way (other options: \textit{Wrong} or \textit{Acceptable} way). (Question \ref{q:transfer_how}).
    }\label{fig:transfer_how_positive}
    \makebox[\textwidth][c]{\includegraphics[width=.9\textwidth]{../figures/country_comparison/transfer_how_positive.pdf}} 
\end{figure}

\begin{figure}[h!]
    \caption[\textit{Best} way to transfer resources to LICs]{``How do you evaluate each of these channels to transfer resources to reduce poverty in LICs?''\\ Percentage of \textit{Best} way (other options: \textit{Right}, \textit{Wrong} or \textit{Acceptable} way). (Question \ref{q:transfer_how}).
    }\label{fig:transfer_how_above_one}
    \makebox[\textwidth][c]{\includegraphics[width=.9\textwidth]{../figures/country_comparison/transfer_how_above_one.pdf}} 
\end{figure}

\begin{figure}[h!]
    \caption[\textit{Wrong} way to transfer resources to LICs]{``How do you evaluate each of these channels to transfer resources to reduce poverty in LICs?''\\ Percentage of \textit{Wrong} way (other options: \textit{Best}, \textit{Right} or \textit{Acceptable} way). (Question \ref{q:transfer_how}).
    }\label{fig:transfer_how_negative}
    \makebox[\textwidth][c]{\includegraphics[width=.9\textwidth]{../figures/country_comparison/transfer_how_negative.pdf}} 
\end{figure}

\begin{figure}[h!]
    \caption[Support for making all countries' GDP p.c. converge by 2100]{``Should governments actively cooperate to have all countries converge in terms of GDP per capita by the end of the century?'' (Question \ref{q:convergence_support}).
    }\label{fig:convergence_support}
    \makebox[\textwidth][c]{\includegraphics[width=.8\textwidth]{../figures/all/convergence_support.pdf}} 
\end{figure}

\begin{figure}[h!]
    \caption[Willingness to participate in a global movement for sustainable development]{``If there was a worldwide movement in favor of a global program to tackle climate change, implement taxes on millionaires and fund poverty reduction in low-income countries, to what extent would you be willing to be part of that movement? (Multiple answers possible)'' (Question \ref{q:global_movement}).
    }\label{fig:global_movement}
    \makebox[\textwidth][c]{\includegraphics[width=.8\textwidth]{../figures/country_comparison/global_movement_positive.pdf}} 
\end{figure}

\begin{figure}[h!]
    \caption[Would vote for a party in a global coalition for sustainable development]{``Let us call "your political party" the party you voted for in the last election, or the party that represents your views most closely.~\\\textbf{Imagine }there was \textbf{a worldwide coalition} of political parties in favor of a common program \textbf{to tackle climate change, implement taxes on millionaires and fund poverty reduction in low-income countries}.~\\\\\textbf{Would you be more likely to vote for your party if it were part of that coalition?}'' (Question \ref{q:vote_intl_coalition}).
    }\label{fig:vote_intl_coalition}
    \makebox[\textwidth][c]{\includegraphics[width=.7\textwidth]{../figures/all/vote_intl_coalition.pdf}} 
\end{figure}

\begin{figure}[h!]
    \caption[Agreement with rationales for global redistribution]{``Some people think that high-income countries should support low-income countries.~\\Among the different reasons given, which ones do you agree with? (Multiple answers possible)'' (Question \ref{q:why_hic_help_lic}).
    }\label{fig:why_hic_help_lic}
    \makebox[\textwidth][c]{\includegraphics[width=\textwidth]{../figures/country_comparison/why_hic_help_lic_positive.pdf}} 
\end{figure}

\begin{figure}[h!]
    \caption[Average custom redistribution]{Average custom global redistribution. (Question \ref{q:custom_redistr}).
    }\label{fig:custom_redistr_mean}
    \makebox[\textwidth][c]{\includegraphics[width=.9\textwidth]{../figures/all/custom_redistr_mean.png}} 
\end{figure}

\begin{figure}[h!]
    \caption[Median custom redistribution]{Global redistribution obtained from median custom parameters: 49\% of winners; 18\% of losers; degree of redistribution of 5 (out of 10). (Question \ref{q:custom_redistr}).
    }\label{fig:custom_redistr_median}
    \makebox[\textwidth][c]{\includegraphics[width=.9\textwidth]{../figures/questionnaire/survey_custom_redistr_median_zoom.png}} 
\end{figure} % TODO: tax rates: all/tax_radical_redistr

% \begin{figure}[h!]
%     \caption[Average well-being depending on the variant]{Average subjective well-being, depending on the variant. (Question \ref{q:well_being}).
%     }\label{fig:well_being}
%     \makebox[\textwidth][c]{\includegraphics[width=\textwidth]{../figures/country_comparison/well_being_mean.pdf}} 
% \end{figure}

\begin{figure}[h!]
    \caption[Comprehension question on GCS]{``\textit{Comprehension question: one respondent with the expected answer will get [amount\_lottery: \$100].}\\\\How would gasoline prices change as a result of the Global Climate Scheme? \\Gasoline prices would...'' (Correct answer: \textit{increase}) (Question \ref{q:gcs_comprehension}).
    }\label{fig:gcs_comprehension}
    \makebox[\textwidth][c]{\includegraphics[width=.7\textwidth]{../figures/all/gcs_comprehension.pdf}} 
\end{figure}

\begin{figure}[h!]
    \caption[Relative agreement: ``My taxes should go towards solving global problems'']{Relative agreement for: ``To what extent do you agree or disagree with the following statement? "My taxes should go towards solving global problems."'' (Percentage of \textit{Agree} or \textit{Strongly agree} among non-\textit{Neither agree nor disagree} responses). (Question \ref{q:my_tax_global_nation}).
    }\label{fig:my_tax_global_nation_share}
    \makebox[\textwidth][c]{\includegraphics[width=.9\textwidth]{../figures/country_comparison/my_tax_global_nation_share.pdf}} 
\end{figure}

\begin{figure}[h!]
    \caption[Absolute agreement: ``My taxes should go towards solving global problems'']{Absolute agreement for: ``To what extent do you agree or disagree with the following statement? "My taxes should go towards solving global problems."'' (Percentage of \textit{Agree} or \textit{Strongly agree}). (Question \ref{q:my_tax_global_nation}).
    }\label{fig:my_tax_global_nation_positive}
    \makebox[\textwidth][c]{\includegraphics[width=.9\textwidth]{../figures/country_comparison/my_tax_global_nation_positive.pdf}} 
\end{figure}

\begin{figure}[h!]
    \caption[Moral circle]{``Which group of people do you advocate for when you vote?'' (Question \ref{q:group_defended}).
    }\label{fig:group_defended_all}
    \makebox[\textwidth][c]{\includegraphics[width=.9\textwidth]{../figures/all/group_defended.pdf}} 
\end{figure}

% \begin{figure}[h!]
%     \caption[Moral circle (heatmap)]{``Which group of people do you advocate for when you vote?'' (Question \ref{q:group_defended}).
%     }\label{fig:group_defended_heatmap}
%     \makebox[\textwidth][c]{\includegraphics[width=.9\textwidth]{../figures/country_comparison/group_defended_5_positive.pdf}} 
% \end{figure}

\begin{figure}[h!]
    \caption[Feeling that the survey was politically biased]{``Do you feel that this survey was politically biased?'' (Question \ref{q:survey_biased}).
    }\label{fig:survey_biased}
    \makebox[\textwidth][c]{\includegraphics[width=.8\textwidth]{../figures/all/survey_biased.pdf}} 
\end{figure}

% \begin{figure}[h!]
%     \caption[]{. (Questions \ref{q:gcs_support}).
%     }\label{fig:radical_redistr_positive}
%     \makebox[\textwidth][c]{\includegraphics[width=.9\textwidth]{../figures/country_comparison/radical_redistr_positive.pdf}} 
% \end{figure} % TODO?

\begin{figure}[h!]
    \caption[Vote in the last election compared to actual results (entire population)]{Vote in the last election, compared to actual results on the entire population. (Questions \ref{q:voted}, \ref{q:vote}).
    }\label{fig:vote_representativeness}
    \makebox[\textwidth][c]{\includegraphics[width=.9\textwidth]{../figures/country_comparison/vote_representativeness.pdf}} 
\end{figure}

\begin{figure}[h!]
    \caption[Vote in the last election compared to actual results (among voters)]{Vote in the last election, compared to actual results among voters. (Questions \ref{q:voted}, \ref{q:vote}).
    }\label{fig:vote_pnr_out}
    \makebox[\textwidth][c]{\includegraphics[width=.9\textwidth]{../figures/country_comparison/vote_pnr_out.pdf}} 
\end{figure}

% \begin{figure}[h!]
%     \caption[]{. (Question \ref{q:gcs_support}).
%     }\label{fig:gcs_support}
%     \makebox[\textwidth][c]{\includegraphics[width=.9\textwidth]{../figures/country_comparison/gcs_support.pdf}} 
% \end{figure}


\renewcommand{\theenumi}{\arabic{enumi}}
\clearpage
\section{Questionnaire}\label{app:questionnaire}
The U.S. version of the questionnaire is presented. Features that vary across countries are put in square brackets within the question tex, as follows: [feature\_name: U.S. value]. Features values for each country are given in \href{https://github.com/bixiou/robustness_global_redistr/raw/main/questionnaire/sources.xlsx}{this spreadsheet}. 
Random branches or conditions for displaying the question are specified in square brackets before the question text (cf. Figure \ref{fig:flow} for the survey flow). The question text is followed by square brackets that refer to Figures and Tables presenting the question results, and the variable name(s) corresponding to the question. Finally, response options are displayed in italics. 
Unless otherwise specified, response is compulsory and a single response much be chosen.

\subsection*{Welcome} 
 \begin{enumerate} 
\item  \label{q:consent} Welcome to this survey!\\
This survey is \textbf{anonymous }and is conducted \textbf{for research} purposes on a representative sample of [sample\_size: 3,000] [nationality: American people].\\
~\\
It takes around 20 min to complete.\\
~\\
The survey contains lotteries and awards for those who get the correct answer to some comprehension questions.\\
If you are attentive and lucky, \textbf{you can win up to [amount\_lottery: \$100]}.\\
~\\
Please answer every question carefully.\\
~\\
By clicking on the button below, you consent to the terms and conditions.

\end{enumerate} 

 \subsection*{Socio-demographics} 
 \begin{enumerate}[resume] 
\item  \label{q:gender} What is your gender? [%\textit{Figure \ref{fig:gender}}; 
\verb|gender|]
  \\ \textit{Woman; Man; Other}

\item  \label{q:hidden_country} What is your country? [%\textit{Figure \ref{fig:hidden_country}}; 
\verb|hidden_country|]


\item  \label{q:age_exact} What is your age? [%\textit{Figure \ref{fig:age_exact}}; 
\verb|age_exact, age|]
  \\ \textit{Below 18; 18 to 20; 21 to 24; 25 to 29; 30 to 34; 35 to 39; 40 to 44; 45 to 49; 50 to 54; 55 to 59; 60 to 64; 65 to 69; 70 to 74; 75 to 79; 80 to 84; 85 to 89; 90 to 99; 100 or above}

\item  \label{q:foreign} Were you or your parents born in a foreign country?~ [\textit{Figure \ref{fig:foreign}}; 
\verb|foreign|]
  \\ \textit{Yes, I was born in a foreign country; Not me but both my parents were born in a foreign country; Not me but one of my parents was born in a foreign country; No, I was born in this country and my parents too}

\item  \label{q:couple} Do you live with your partner (if you have one)? [%\textit{Figure \ref{fig:couple}}; 
\verb|couple|]
  \\ \textit{Yes; No}

\item  \label{q:hh_size} How many people are there in your household? \\The household includes: \textbf{you}, your spouse, \textbf{your family members} who live with you, and your dependents (not flatmates). [%\textit{Figure \ref{fig:hh_size}}; 
\verb|hh_size|]
  \\ \textit{1; 2; 3; 4; 5 or more}

\item  \label{q:Nb_children__14} How many children under the age of 14 live with you? [%\textit{Figure \ref{fig:Nb_children__14}}; 
\verb|Nb_children__14|]
  \\ \textit{0; 1; 2; 3; 4 or more}

\item ~[new page] \label{q:race} [\textit{Only in: US}] What race or ethnicity do you identify with? (Multiple answers are possible) [%\textit{Figure \ref{fig:race}}; 
\verb|race|]
  \\ \textit{White; Black or African American; Hispanic; Asian; American Indian or Alaskan Native; Native Hawaiian or Pacific Islander; Other; Prefer not to say}

\item  \label{q:income} What is the \textbf{[periodicity\_text: monthly] [income\_type: gross] income of your household}, [income\_type\_long: after taxes and transfers]?

This includes all sources of income: wages, pensions, welfare payments, property income, dividends, self-employment earnings, Social Security benefits, and income from other sources. [%\textit{Figure \ref{fig:income}}; 
\verb|income|]
  \\ ~[\textit{All but RU, US}: Custom thresholds, taking into account household composition Questions \ref{q:couple}-\ref{q:Nb_children__14}, and corresponding to the country's deciles and quartiles of standard of living, cf. the sheet ``Income'' in \href{https://github.com/bixiou/robustness_global_redistr/raw/main/questionnaire/source.xlsx}{this spreadsheet}; \\ \textit{RU, US}: Items based on household total income deciles and quartiles, namely in US: \textit{Less than \$17,000; between \$17,001 and \$30,000; between \$30,001 and \$36,000; between \$36,001 and \$43,000; between \$43,001 and \$56,000; between \$56,001 and \$72,000; between \$72,001 and \$91,000; between \$91,001 and \$115,000; between \$115,001 and \$130,000; between \$130,001 and \$150,000; between \$150,001 and \$213,000; More than \$213,000; I prefer not to answer}]

\item  \label{q:education} What is your highest completed education level? [%\textit{Figure \ref{fig:education}}; 
\verb|education|]
  \\ ~[Country-specific, usually: 0-1 Primary or less; 2 Medium school; 2 Some high school; 3 High school diploma; 3-4 Vocational training; 5 Short-cycle tertiary; 6 Bachelor's; 7-8 Master's or higher]

\item  \label{q:employment_status} What is your employment status? [%\textit{Figure \ref{fig:employment_status}}; 
\verb|employment_status|]
  \\ \textit{Full-time employed; Part-time employed; Self-employed; Unemployed (searching for a job); Student; Retired; Inactive (not searching for a job)}

\item  \label{q:zipcode} [\textit{Only the first digits asked in RU, SA}] What is your zipcode?\\
We ask for the zipcode to balance the sample in terms of degree of urbanization (rural, town or city). The survey will be terminated if your zipcode is not recognized. [%\textit{Figure \ref{fig:zipcode}}; 
\verb|zipcode|]


\item  \label{q:home} Are you a homeowner or a tenant? (Multiple answers are possible) [%\textit{Figure \ref{fig:home}}; 
\verb|home_owner|]
  \\ \textit{Tenant; Owner; Landlord renting out property; Hosted free of charge}

\item ~[new page] \label{q:millionaire} How likely are you to become a millionaire at some point in your life? [\textit{Figure \ref{fig:millionaire}}; 
\verb|millionaire|]
  \\ \textit{Very unlikely; Unlikely; Likely; Very likely; I am already a millionaire}

\item  \label{q:voted} [\textit{Except in: RU, SA}] Did you vote in the [election: 2024 presidential election]? [\textit{Figures \ref{fig:vote_representativeness}-\ref{fig:vote_pnr_out}}; 
\verb|voted|]
  \\ \textit{Yes; No; Prefer not to say; I didn't have the right to vote in [country\_name: the United States].}

\end{enumerate} 

 \subsection*{Vote} 
 \begin{enumerate}[resume] 
\item  \label{q:nationality_SA} [\textit{Only in: SA}] What is your nationality?\\If you have both the Saudi and a foreign nationality, choose "Saudi". [%\textit{Figure \ref{fig:nationality_SA}}; 
\verb|nationality_SA|]
  \\ \textit{Saudi; India; Bangladesh; Syria; Yemen; Egypt; Pakistan; Indonesia; Philippines; Sudan; Myanmar; Jordan; Sri Lanka; Nepal; Turkey; Somalia; Lebanon; Other}

\item  \label{q:vote} [\textit{Except in: RU, SA}] [\textit{If voted}: Which candidate did you vote for in the [election: 2024 presidential election]?; \textit{Otherwise}: Even if you did not vote in the [election: 2024 presidential election], please indicate the candidate that you were most likely to have voted for or who represents your views more closely.] [\textit{Figures \ref{fig:vote_representativeness}-\ref{fig:vote_pnr_out}}; 
\verb|vote|]
  \\ ~[Candidates/parties with at least 1\% of votes, e.g. in US: \textit{Harris; Trump; Other; Prefer not to say}. In FR, IT, PL, ES, election is the 2024 European election]

% \item  \label{q:vote_GB} [text\_vote: Which candidate did you vote for in the [election: 2024 European Parliament election]?\\Even if you did not vote in the [election: 2024 European Parliament election], please indicate the candidate that you were most likely to have voted for or who represents your views more closely.] [\textit{Figure \ref{fig:vote_GB}}; 
% \verb|vote_GB|]
%   \\ \textit{Conservative; Labour; Liberal Democrats; SNP; Prefer not to say; Green; DUP; Sinn Féin; Other; Reform UK; Plaid Cymru; Alliance Party of Northern Ireland}

% \item  \label{q:vote_FR} [text\_vote: Which candidate did you vote for in the [election: 2024 European Parliament election]?\\Even if you did not vote in the [election: 2024 European Parliament election], please indicate the candidate that you were most likely to have voted for or who represents your views more closely.] [\textit{Figure \ref{fig:vote_FR}}; 
% \verb|vote_FR|]
%   \\ \textit{Renaissance, MoDem \& Horizons; Rassemblement National; La France insoumise; Les Écologistes – EÉLV; Préfère ne pas répondre; Les Républicains; Résistons (Jean Lassalle); Reconquête; Autre; Parti Socaliste \& Place publique; Parti Communiste Français; Parti animaliste}

% \item  \label{q:vote_CH} [text\_vote: Which candidate did you vote for in the [election: 2024 European Parliament election]?\\Even if you did not vote in the [election: 2024 European Parliament election], please indicate the candidate that you were most likely to have voted for or who represents your views more closely.] [\textit{Figure \ref{fig:vote_CH}}; 
% \verb|vote_CH|]
%   \\ \textit{Social Democratic Party; Swiss People's Party; The Centre; Green Liberal Party; Préfère ne pas répondre; Green Party; Evangelical People's Party; Autre; The Liberals; Federal Democratic Union}

% \item  \label{q:vote_PL} [text\_vote: Which candidate did you vote for in the [election: 2024 European Parliament election]?\\Even if you did not vote in the [election: 2024 European Parliament election], please indicate the candidate that you were most likely to have voted for or who represents your views more closely.] [\textit{Figure \ref{fig:vote_PL}}; 
% \verb|vote_PL|]
%   \\ \textit{United Right (Law and Justice, Sovereign Poland...); Civic Coalition (Civic Platform, Polish Initiative...); Polish People's Party; Prefer not to say; The Left (New Left...); Other; Confederation (New Hope, National Movement, Confederation of the Polish Crown...); Poland 2050}

% \item  \label{q:vote_IT} [text\_vote: Which candidate did you vote for in the [election: 2024 European Parliament election]?\\Even if you did not vote in the [election: 2024 European Parliament election], please indicate the candidate that you were most likely to have voted for or who represents your views more closely.] [\textit{Figure \ref{fig:vote_IT}}; 
% \verb|vote_IT|]
%   \\ \textit{PD; FdI; League; Prefer not to say; SUE; Azione; FI – NM; AVS; PTD; Libertà; M5S; Other}

% \item  \label{q:vote_ES} [text\_vote: Which candidate did you vote for in the [election: 2024 European Parliament election]?\\Even if you did not vote in the [election: 2024 European Parliament election], please indicate the candidate that you were most likely to have voted for or who represents your views more closely.] [\textit{Figure \ref{fig:vote_ES}}; 
% \verb|vote_ES|]
%   \\ \textit{PSOE; PP; Sumar; Prefer not to say; Podemos; Junts UE; Ahora Repúblicas; SALF; CEUS; Vox; Other}

% \item  \label{q:vote_DE} [text\_vote: Which candidate did you vote for in the [election: 2024 European Parliament election]?\\Even if you did not vote in the [election: 2024 European Parliament election], please indicate the candidate that you were most likely to have voted for or who represents your views more closely.] [\textit{Figure \ref{fig:vote_DE}}; 
% \verb|vote_DE|]
%   \\ \textit{AfD; CDU/CSU; BSW; Prefer not to say; Die Linke; FW; Grüne; FDP; Volt; Die Partei; SPD; Other; Tierschutzpartei}

% \item  \label{q:vote_JP} [text\_vote: Which candidate did you vote for in the [election: 2024 European Parliament election]?\\Even if you did not vote in the [election: 2024 European Parliament election], please indicate the candidate that you were most likely to have voted for or who represents your views more closely.] [\textit{Figure \ref{fig:vote_JP}}; 
% \verb|vote_JP|]
%   \\ \textit{CDP; LDP; Reiwa Shinsengumi; Prefer not to say; Sanseitō; CPJ; Komeito; JCP; SDP; Other; Ishin JIP; DPFP}

\end{enumerate} 

 \subsection*{Open-ended field} 
 [\textit{Four random branches}; \textit{Figures \ref{fig:field_keyword}-\ref{fig:injustice_field}}; 
 \verb|field, variant_field|] 
 \begin{enumerate}[resume] 
\item  \label{q:concerns_field} ~[Branch: concerns] What are your main concerns these days? [\textit{Figure \ref{fig:concerns_field}}; 
\verb|concerns_field|]


\item  \label{q:wish_field} ~[Branch: wish] What are your needs or wishes? [\textit{Figure \ref{fig:wish_field}}; 
\verb|wish_field|]


\item  \label{q:issue_field} ~[Branch: issue] Can you name an issue that is important to you but is neglected in the public debate? [\textit{Figure \ref{fig:issue_field}}; 
\verb|issue_field|]


\item  \label{q:injustice_field} ~[Branch: injustice] What according to you is the greatest injustice of all?\\ 
~[\textit{Figure \ref{fig:injustice_field}}; 
\verb|injustice_field|]


\end{enumerate} 

 \subsection*{Conjoint analysis} 
 \begin{enumerate}[resume] 
\item  \label{q:conjoint} [\textit{Except in: RU, SA}] Imagine if the two top candidates in your constituency in the next general election campaigned with the following policies in their party's platforms. \\\\Which of these candidates would you vote for?  
~\\

\begin{tabular}{@{\extracolsep{5pt}}|c|c|c|} 
    \hline \\[-1.8ex] 
    \textbf{Candidate A} & \textbf{Candidate B} & \\ \hline \\[-1.8ex]
    ~[Random policy] & [Random policy] & [Policy field in random order] \\ 
    ~[Random policy] & [Random policy] & [Policy field in random order] \\ 
    ~[Random policy] & [Random policy] & [Policy field in random order] \\ 
    ~[Random policy] & [Random policy] & [Policy field in random order] \\ 
    ~[Random policy] & [Random policy] & [Policy field in random order] \\ 
    \hline 
\end{tabular}  

~\\~[\textit{Figures \ref{fig:conjoint}, \ref{fig:conjoint_FR}-\ref{fig:conjoint_ES_original}}; 
\verb|conjoint|]
  \\ \textit{Candidate A; Candidate B; Neither of them}

\end{enumerate} 

 \subsection*{Revenue split of global tax} 
 [\textit{Two random branches};  \verb|field, variant_split|] 
 \begin{enumerate}[resume] 
\item ~[Branch: Few] \label{q:revenue_split_few} Imagine a wealth tax applied to households with a net worth above [tax\_threshold: \$5 million], implemented in every country around the world.
~\\\\ 
~[tax\_country\_name: In the U.S.], the tax revenues collected would be [tax\_revenue: \$514 billion] per year (that is, [tax\_revenue\_gdp: 2]\% of [tax\_country\_gdp: U.S. GDP]), while it would be [LIC\_revenue: \$1 billion] in all low-income countries combined (700 million people live in a low-income country, most of them in Africa).
Each country would retain part of the revenues it collects and use it for different domestic purposes. The remaining part would be pooled globally to finance sustainable development in low-income countries.
~\\\\\textbf{What percentage of the global wealth tax revenue should be allocated to each category?} \\\textbf{The total allocation must sum to 100\%.}\\\\ 
~[\textit{Figures \ref{fig:split}, \ref{fig:split_few_bars_nb0}-\ref{fig:split_few}}; 
\verb|revenue_split_few|]
  \\ \textit{Domestic: Education and Healthcare; Domestic: Social welfare programs; Domestic: Reduction in the federal income tax; Domestic: Reduction of the deficit; Global: Education, Healthcare and Renewable energy in low-income countries}

\item ~[Branch: Many] \label{q:revenue_split_many} Imagine a wealth tax applied to households with net worth above [tax\_threshold: \$5 million], implemented in all countries around the world.
~\\\\ 
~[tax\_country\_name: In the U.S.], the tax revenues collected would be [tax\_revenue: \$514 billion] per year (that is, [tax\_revenue\_gdp: 2]\% of [tax\_country\_gdp: U.S. GDP]), while it would be [LIC\_revenue: \$1 billion] in all low-income countries combined (700 million people live in a low-income country, most of them in Africa).
Each country would retain part of the revenues it collects and use it for different domestic purposes. The remaining part would be pooled globally to finance sustainable development.
~\\\\\textbf{What percentage of the global wealth tax revenue should be allocated to each category?}~\\\textbf{The total allocation must sum to
100\%.}\\\\ 
~[\textit{Figures \ref{fig:split}, \ref{fig:split_many}-\ref{fig:split_many_global_mean}}; 
\verb|revenue_split_many|]
  \\ ~[Five items are chosen at random among the 13 possible ones: \textit{Domestic: Education and Research; Domestic: Healthcare; Domestic: Defense; Domestic: Deficit reduction; Domestic: Justice and Police; Domestic: Retirement pensions; Domestic: Social welfare programs; Domestic: Infrastructure (public transport, water systems...); Domestic: Income tax reduction; Global: Education and Healthcare in low-income countries; Global: Renewable energy and infrastructure to cope with climate change; Global: Loss and Damage Fund (to rebuild after climate disasters); Global: Forestation and biodiversity projects}]


\end{enumerate} 

 \subsection*{Warm glow -- moral substitute} 
 [\textit{Three random branches: NCS; Donation; control group};  \verb|variant_warm_glow|] 
 \begin{enumerate}[resume] 
\item ~[Branch: NCS] \label{q:ncs_support} Do you agree with the following policy?
~\\
Climate Scheme:~\\
To meet the national climate target, a limited number of permits to emit greenhouse gases would be issued nationally. Polluting firms would be required to buy permits to cover their greenhouse gas emissions. Such a policy would~make fossil fuel companies pay~for their emissions and gradually raise the price of fossil fuels.~Higher prices would encourage people and companies to use less fossil fuels, reducing greenhouse gas emissions.\\
The revenues generated by the sale of permits would finance an equal cash transfer.\textbf{~}Each [country\_adjective: American] would receive [amount\_expenses: \$115][periodicity: per month], thereby offsetting~price increases for the average [country\_adjective: American].\\
~\\
\textbf{Do you support the Climate Scheme?} [\textit{Figures \ref{fig:ics}, \ref{fig:ncs_gcs_ics}}; 
\verb|ncs_support|]
  \\ \textit{Yes; No}

\item ~[Branch: Donation] \label{q:donation} By taking this survey, you will be automatically entered into a lottery to win up to [amount\_lottery: \$100]. \\Should you be selected in the lottery, you will have the option to channel a part of this additional compensation to the charity \textit{Just One Tree} to plant trees.\\\\\textbf{In case you win the lottery, what share of the [amount\_lottery: \$100 prize] would you donate to plant trees?} [\textit{Figures \ref{fig:warm_glow_substitute}, \ref{fig:donation}
}; 
\verb|donation|]
  \\ \textit{Share to plant trees}

\end{enumerate} 

 \subsection*{Cap \& Share} 
 \begin{enumerate}[resume] 
\item  \label{q:gcs_support} Do you support the following policy?\\
To ensure that you have attentively read the description,~we will ask some comprehension questions later in the survey: those who get correct answers can win [amount\_lottery: \$100].
~\\
Global Climate Scheme:~\\\\
In 2015, all countries agreed to contain global warming "well below +2~\textdegree{}C". To achieve this,~there is a maximum amount of greenhouse gases we can emit globally.~\\\\
To meet the climate target, a limited number of permits to emit greenhouse gases would be issued globally. Polluting firms would be required to buy permits to cover their greenhouse gas emissions. Such a policy would~make fossil fuel companies pay~for their emissions and gradually raise the price of fossil fuels.~Higher prices would encourage people and companies to use less fossil fuels, reducing greenhouse gas emissions.\\\\
In accordance with the principle that each human has an equal right to pollute, the revenues generated by the sale of permits could finance a global basic income.~Every adult would receive [amount\_bi: \$20][periodicity: per month], thereby lifting 600 million people who earn less than \$2 a day out of extreme poverty.\\
The typical [national: American] would lose out financially [amount\_lost: \$105][periodicity: per month]~(as he or she would face around [price\_increase: 2]\% in price increases, which is higher than the [amount\_bi: \$20][periodicity: per month] they would receive).\\\\
The policy could be implemented as soon as 100 countries agree to it. Countries that would refuse to take part in the policy could face sanctions (like tariffs) from the rest of the world and would be excluded from the basic income program.\\\\
~\\\textbf{
Do you support the Global Climate Scheme?
} [\textit{Figures \ref{fig:ics}, \ref{fig:warm_glow_substitute}, \ref{fig:ncs_gcs_ics}}; 
\verb|gcs_support|]
  \\ \textit{Yes; No}\\\\
~[new page] [\textit{Two random branches: own; US}; \textit{Figure \ref{fig:ncs_gcs_ics}}; \verb|gcs_belief, variant_belief|] 
\item ~[Branch: US] \label{q:gcs_belief_us} According to you, \textbf{what percentage of [belief\_nationality: \textit{All but US: Americans; US: Europeans}] would answer \textit{Yes }to the previous question} (considering that typical [belief\_nationality] would lose [belief\_loss: \$140] per month from the Global Climate Scheme)\textbf{?}\\ The respondent who is closest to the correct value will get [amount\_lottery: \$100]. %[\textit{Figure \ref{fig:gcs_belief_us}}; 
% \verb|gcs_belief_us|]
  \\ \textit{Percentage of [belief\_nationality] in favor of Global Climate Scheme}

\item ~[Branch: own] \label{q:gcs_belief_own} According to you, \textbf{what percentage of \textit{[nationality: fellow citizens]} would answer \textit{Yes }to the previous question?}\\ The respondent who is closest to the correct value will get [amount\_lottery: \$100]. %[\textit{Figure \ref{fig:gcs_belief_own}}; 
% \verb|gcs_belief_own|]
  \\ \textit{Percentage of [nationality: fellow citizens] in favor of Global Climate Scheme}

\end{enumerate} 

 \subsection*{Cap \& Share non-universal} 
 ~[\textit{Four random branches: low; mid; high; high\_color}; \textit{Figures \ref{fig:ics}, \ref{fig:ncs_gcs_ics}}; 
 \verb|ics_support|] 
 \begin{enumerate}[resume] 
\item ~[Branch: low]  \label{q:gcs_low} Below is a map showing a possible set of countries that would participate in the Global Climate Scheme previously described.\\
~\\
These countries include India, the European Union, as well as all Africa, Latin America, South-Asia and South-East Asia.\\
Collectively, these [nb\_countries\_low: 145] countries account for [emissions\_low\_without: 40]\% of global emissions (if [ics\_country: the U.S.] joined them, [emissions\_low\_with: 40]\% of global emissions would be covered).\\
~\\ 

\item ~[Branch: mid] \label{q:gcs_mid} Below is a map showing a possible set of countries that would participate in the Global Climate Scheme previously described.\\
~\\
These countries include China, India, as well as all Africa, Latin America, South-Asia and South-East Asia.\\
Collectively, these 119 countries account for 56\% of global emissions (if [ics\_country: the U.S.] joined them, [emissions\_mid\_with: 70]\% of global emissions would be covered).\\
~\\ 

\item ~[Branch: high]  \label{q:gcs_high} Below is a map showing a possible set of countries that would participate in the Global Climate Scheme previously described.\\
~\\
These countries include China, India, [text\_countries\_high: the European Union, Japan, the United Kingdom], Canada, South Korea, as well as all Africa, Latin America, South-Asia and South-East Asia.~\\
Collectively, these [nb\_countries\_high: 153] countries account for [emissions\_high\_without: 71]\% of global emissions (if [ics\_country: the U.S.] joined them, [emissions\_high\_with: 86]\% of global emissions would be covered).\\
~\\ 

\item ~[Branch: high\_color]  \label{q:gcs_high_color} Below is a map showing a possible set of countries that would participate in the Global Climate Scheme previously described.\\
~\\
These countries include China, India, [text\_countries\_high: the European Union, Japan, the United Kingdom], Canada, South Korea, as well as all Africa, Latin America, South-Asia and South-East Asia. \\
Collectively, these [nb\_countries\_high: 153] countries account for [emissions\_high\_without: 72]\% of global emissions (if [ics\_country: the U.S.] joined them, [emissions\_high\_with: 86]\% of global emissions would be covered).\\\\Note that a provision would prevent the Global Climate Scheme from harming low- and middle-income countries: this is why countries like China, Mexico, or Egypt are in white on the map (they would neither win nor lose financially).\\


\item  \label{q:ics_support} Do you support [ics\_country: the U.S.] joining the Global Climate Scheme, in case it is adopted by the above countries? [\textit{Figures \ref{fig:ics}, \ref{fig:ncs_gcs_ics}}; 
\verb|ics_support|]
  \\ \textit{Yes; No}

\end{enumerate} 

 \subsection*{Warm glow -- realism} 
 \begin{enumerate}[resume] 
\item ~[\textit{Two random branches: with or without this informational text.}] \label{q:info_solidarity} To ensure that you have attentively read the description below, we will ask some comprehension questions later in the survey: those who get correct answers can win \$100.

~\\\\In several international organizations, \textbf{countries have agreed to demonstrate some degree of solidarity in addressing global challenges}.\\
Negotiations are ongoing to implement specific mechanisms for sustainable development.\\\\Here are a few examples:\\🚢~In 2025, to reduce carbon emissions from shipping, \textbf{the International Maritime Organization adopted an international levy on excess emissions from maritime fuel, that should partly finance low-income countries}.\\📦~Since 1970, \textbf{developed countries have agreed to contribute 0.7\% of their GDP in foreign aid} and development assistance.\\
🌱 In international climate negotiations, \textbf{developed countries have committed to finance climate action in developing countries}. In 2009, they committed to provide \$100 billion per year by 2020. In 2023, all countries agreed to set up a fund to help vulnerable countries cope with loss and damage from climate change. In 2024, the \$100 billion goal was increased to \$300 billion per year by 2035.\\📈~In 2021, 136 countries adopted a minimum tax rate of 15\% on multinational profits.\\💎 In 2024, under the leadership of Brazil, \textbf{the G20 considered the introduction of a global tax} of 2\% \textbf{on }the wealth of \textbf{billionaires}.
~\\🌐~In 2024, the UN General Assembly adopted the Pact for the Future, which foresees a reform of the UN Security Council to limit the power of its five permanent member and expand it to new members.\\🔄 Led by the Prime Minister of Barbados and supported by the UN Secretary General, the Bridgetown initiative seeks a new financial system that would drive financial resources towards climate action and sustainable development. [\textit{Figure \ref{fig:warm_glow_realism}}; 
\verb|info_solidarity|]


\item  \label{q:likely_solidarity} According to you, how likely is it that international policies involving significant transfers from high-income countries to low-income countries will be introduced in the next 15 years? [\textit{Figure \ref{fig:warm_glow_realism}}; 
\verb|likely_solidarity|]
  \\ \textit{Very unlikely; Unlikely; Likely; Very likely}

\item  \label{q:solidarity_support} Do you support or oppose the following policies?\\
~\\ 
~[\textit{Only in PL, SA}: (As some items refer to ``developed countries'', note that we consider [Saudi Arabia] to be a developed country in this question.)] [\textit{Figures \ref{fig:solidarity_support_share}, \ref{fig:solidarity_support_positive}-\ref{fig:share_solidarity_opposed}}; 
\verb|solidarity_support|] \\
~[Item order is randomized]
\begin{itemize}
    \item Institutions like the World Bank investing in many more sustainable projects in lower-income countries, and offering lower interest rates (the Bridgetown initiative)
    \item Developed countries financing a fund to help vulnerable countries cope with loss and damage from climate change
    \item Expanding the UN Security Council (in charge of peacekeeping) to new permanent members such as India, Brazil, and the African Union, and restricting the use of the veto
    \item Raising the globally agreed minimum tax rate on profits of multinational firms from 15\% to 35\%, closing loopholes and allocating revenues to countries where sales are made
    \item Debt relief for vulnerable countries by suspending repayments until they are better able to repay, promoting their development
    \item An international levy on carbon emissions from shipping, funding national budgets in proportion to population
    \item An international levy on carbon emissions from aviation, raising ticket prices by 30\% and funding national budgets in proportion to population
    \item Developed countries providing \$300 billion a year (0.4\% of their GDP) to finance climate action in developing countries
    \item Developed countries contributing at least 0.7\% of their GDP in foreign aid and development assistance
    \item A minimum tax of 2\% on the wealth of billionaires, in voluntary countries
\end{itemize}
\textit{Strongly oppose; Somewhat oppose; Indifferent; Somewhat support; Strongly support}
\end{enumerate} 

 \subsection*{NCQG} 
 [\textit{Two random branches: Full; Short}; %\textit{Figure \ref{fig:field}}; 
 \verb|ncqg_fusion, variant_ncqg|] 
 \begin{enumerate}[resume] 
% \item  \label{q:maritime_split} This year, to meet the global climate targets, the International Maritime Organization is designing a global levy on shipping carbon emissions.\\\\\textbf{According to you, what percentage of the revenue from a maritime fuel levy should be allocated to each category below?} The total must be 100\%.\\ 
% ~[\textit{Figure \ref{fig:maritime_split}}; 
% \verb|maritime_split|]
%   \\ \textit{Fostering sustainable transition in the least developed countries and small islands states; Return revenues to shipping companies to prevent increases in shipping costs; Finance research, development and deployment for zero-emission fuels and ships}

\item ~[Branch: Full] \label{q:ncqg_full} \textbf{At international climate negotiations, developing countries call for larger provision of "climate finance": the financing of climate action from developed countries in developing countries.} [developed\_note: (Note that we consider Saudi Arabia to be a developed country in this question.)]\\\\\textbf{There are two kinds of climate finance: grants (that is, donations) and loans. In 2022, \$26 billion was provided as grants and the rest as loans, for a total of \$116 billion.~}\\\\In 2009, developed countries agreed to mobilize \$100 billion per year in climate finance by 2020. In 2024, they committed to raise this goal to \$300 billion by 2035. None of the goals specify which share should be provided as grants.\\\\Below are different positions on the amount of climate finance that should be provided in 2035, all expressed in grant-equivalent terms (that is, not counting loans):\\-~ ~ ~ ~ \$0: There should be no contributions from developed countries to climate action in developing countries.\\-~ ~ ~ \$26 billion (0.04\% of developed countries' GDP): The current amount, consistent with the old (2020) goal.\\-~ ~ \$100 billion (0.14\% of GDP): The old (2020) goal, if all climate finance were provided as grants.\\-~ ~ \$300 billion (0.43\% of GDP): The new (2035) goal, if all climate finance were provided as grants.\\-~ ~ \$600 billion (0.86\% of GDP):~The goal called for by India, a position shared by most developing countries.\\- \$1,000 billion (1.43\% of GDP): The goal called for by Climate Action Network (a network of NGOs including Greenpeace, Oxfam, and WWF).\\- \$5,000 billion (7.14\% of GDP): The goal called for by Demand Climate Justice (a network of NGOs including 350.org and~the World Council of Churches)\\\\\textbf{If you could choose the amount of climate finance provided by developed countries to developing countries in 2035, what amount would you choose (in grant-equivalent terms)?}\\ 
~[\textit{Figure \ref{fig:ncqg_full}}; 
\verb|ncqg_full|]\\
~[Item order is randomly reversed or not]
  \\ \textit{\$0; \$300 billion; \$600 billion; \$26 billion; \$100 billion; \$1,000 billion; \$5,000 billion}

\item ~[Branch: Short] \label{q:ncqg} \textbf{"Climate finance" designates the financing of climate action from developed countries in developing countries.} [developed\_note: (Note that we consider Saudi Arabia to be a developed country in this question.)]\\\\\textbf{There are two kinds of climate finance: grants (that is, donations) and loans. The large majority is currently provided as loans.~}\\\\In 2009, developed countries agreed to mobilize \$100 billion per year in climate finance. In 2024, they committed to triple this goal by 2035. None of the goals specify which share should be provided as grants.~\\At international climate negotiations, developing countries call for larger provision of climate finance, particularly in the form of grants.\\\\\textbf{If you could choose the level of climate finance provided by developed countries to developing countries in 2035, what would you choose?}\\ 
~[\textit{Figure \ref{fig:ncqg}}; 
\verb|ncqg|]\\
~[Item order is randomly flipped or not]
  \\ \textit{Stop all provision of climate finance.; \\Reduce the provision of climate finance.; \\Maintain current contributions (\$26 billion per year in grants, that is 0.04\% of developed countries' GDP, and \$80 billion in loans, or 0.1\% of GDP).; \\ Meet the newly agreed goal by tripling grants and loans (\$100 billion in grants, or 0.15\% of GDP).; \\ Increase climate finance to a level between what developed countries have agreed and what developing countries are asking for (\$300 billion in grants, or 0.45\% of GDP).; \\Increase climate finance to match what developing countries are asking for (\$600 billion in grants, or 0.9\% of GDP).; \\Increase climate finance to match what NGOs are asking for (at least \$1,000 billion per year in grants, that is 1.4\% of GDP, is what Greenpeace, Oxfam, WWF, and the World Council of Churches ask for).}

\end{enumerate} 

 \subsection*{Wealth tax depending on sets of countries} 
 [\textit{Three random branches: Global; HIC; Int'l}; \textit{Figures \ref{fig:wealth_tax}, \ref{fig:wealth_tax_heatmap}}; 
 \verb|wealth_tax_support, variant_wealth_tax|] 
 \begin{enumerate}[resume] % TODO: plus condensé ?
\item ~[Branch: Global] \label{q:global_tax_support} \textbf{Imagine an international tax on individuals with net worth above [wealth\_threshold: \$1 million].~}\\Only wealth above [wealth\_threshold: \$1 million] would be taxed, at a rate of 2\%. Each country would retain 70\% of the revenues it collects, while 30\% would be pooled at the global level to finance public services in low-income countries (in particular, access to drinking water, healthcare, and education in Africa). \\\\Say we are in 2030. \textbf{Imagine that all other countries in the world adopt this policy. \\Do you support [country\_name: the United States] adopting this international tax on millionaires?}
  \\ \textit{Yes; No}

\item ~[Branch: HIC] \label{q:hic_tax_support} \textbf{Imagine an international tax on individuals with net worth above [wealth\_threshold: \$1 million].~}\\Only wealth above [wealth\_threshold: \$1 million] would be taxed, at a rate of 2\%. Each country would retain 70\% of the revenues it collects, while 30\% would be pooled at the global level to finance public services in low-income countries (in particular, access to drinking water, healthcare, and education in Africa). \\\\Say we are in 2030. \textbf{[hic\_tax: Imagine that all other high-income countries (such as the European Union, Japan, and Canada) adopt this policy and some middle-income countries (such as China) do not.]}\textbf{~\\Do you support [country\_name: the United States] adopting this international tax on millionaires?}
  \\ \textit{Yes; No}

\item ~[Branch: Int'l] \label{q:intl_tax_support} \textbf{Imagine an international tax on individuals with net worth above [wealth\_threshold: \$1 million].~}\\Only wealth above [wealth\_threshold: \$1 million] would be taxed, at a rate of 2\%. Each country would retain 70\% of the revenues it collects, while 30\% would be pooled at the global level to finance public services in low-income countries (in particular, access to drinking water, healthcare, and education in Africa). \\\\Say we are in 2030.\textbf{ [intl\_tax: Imagine that some countries  (such as the European Union) adopt this policy and others (such as Japan, Canada, and China) do not.]\\Do you support [country\_name: the United States] adopting this international tax on millionaires?}
  \\ \textit{Yes; No}

\end{enumerate} 

 \subsection*{Scenarios \& radical tax} 
 [\textit{Scenario A \& B are randomly interverted.}]
 \begin{enumerate}[resume] 
\item  \label{q:sustainable_future} \textbf{Consider two possible scenarios for the world for the next 20 years.~\\\\Scenario A}: \\Most countries implement coordinated policies to limit global warming to +2\textdegree{}C and reduce inequality. The world greatly reduces greenhouse gas emissions and is on track to meet its climate target. Taxes on millionaires fund the installation of heat pumps, the thermal insulation of buildings, and improved public transportation. Yachts and private jets are phased out worldwide. Cars are all electric by 2045, and they are about the same price as internal combustion cars nowadays. By 2045, environmental regulations gradually double the price heating fuel or gas, air travel, and beef. As a result, people fly half as much, eat half as much meat, and use more public transportation in 2045 than they did in 2025. Despite higher prices for polluting goods, the overall purchasing power is preserved, thanks to a decrease in sales tax that reduces the prices of non-polluting goods.\\\\\textbf{Scenario B}:\\Since 2025, no additional policies are implemented to address climate change or inequality. People maintain the same lifestyles as in 2025. For example, most people continue to drive cars with internal combustion engines. Greenhouse gas emissions are stable. Global warming is expected to reach +3\textdegree{}C by 2100 and higher levels beyond that date. A warmer climate will cause more frequent and more severe droughts, heatwaves, wildfires, and floodings.\\\\Apart from the elements described, the two scenarios are the same (for example, in terms of unemployment or crime). \\\\\textbf{Which scenario do you prefer for the future?} [\textit{Figures \ref{fig:radical_redistr_share}, \ref{fig:sustainable_future}}; 
\verb|sustainable_future|]
  \\ \textit{Scenario A; Scenario B} \\\\
% \item  \label{q:sustainable_future_b} \textbf{Consider two possible scenarios for the world for the next 20 years.~\\\\Scenario A}:\\Since 2025, no additional policies are implemented to address climate change or inequality. People maintain the same lifestyles as in 2025. For example, most people continue to drive cars with internal combustion engines. Greenhouse gas emissions are stable. Global warming is expected to reach +3\textdegree{}C by 2100 and higher levels beyond that date. A warmer climate will cause more frequent and more severe droughts, heatwaves, wildfires, and floodings.\\\\\textbf{Scenario B}: \\Most countries implement coordinated policies to limit global warming to +2\textdegree{}C and reduce inequality. The world greatly reduces greenhouse gas emissions and is on track to meet its climate target. Taxes on millionaires fund the installation of heat pumps, the thermal insulation of buildings, and improved public transportation. Yachts and private jets are phased out worldwide. Cars are all electric by 2045, and they are about the same price as internal combustion cars nowadays. By 2045, environmental regulations gradually double the price of heating fuel or gas, air travel, and beef. As a result, people fly half as much, eat half as much meat, and use more public transportation in 2045 than they did in 2025. Despite higher prices for polluting goods, the overall purchasing power is preserved, thanks to a decrease in sales tax that reduces the prices of non-polluting goods.\\\\Apart from the elements described, the two scenarios are the same (for example, in terms of unemployment or crime). \\\\\textbf{Which scenario do you prefer for the future?} [\textit{Figure \ref{fig:sustainable_future_b}}; 
% \verb|sustainable_future_b|]
%   \\ \textit{Scenario A; Scenario B}
  ~[new page] [\textit{Two random branches: top1; top3}; \textit{Figures \ref{fig:radical_redistr_share}, \ref{fig:top_tax_share}-\ref{fig:top_tax_positive}}; 
\verb|top_tax_support|, \verb|variant_top_tax|]
\item ~[Branch: top1] \label{q:top1_tax_support} Currently, 2 billion people live in acute poverty, with less than [lcu\_250: \$250][periodicity: per month].\\The Sustainable Development Goals, adopted by all countries in 2015, aim to alleviate poverty and give access to healthcare, education, drinking water, and sanitation for all by 2030.~Due to lack of funding, the world is not on track to meet these poverty reduction goals.\\\\\textbf{Poverty reduction could be funded by a global tax on individual income above [lcu\_120k: \$120,000][periodicity\_tax: per year].~\\The tax rate would be 15\% for every [currency: dollar] over [lcu\_120k: \$120,000] of income} after existing taxes.~\\For example, a single person earning [lcu\_130k: \$130,000][periodicity\_tax: per year] after taxes would pay [lcu\_1500: \$1,500] in additional taxes, or 15\% of [lcu\_10k: \$10,000] = [lcu\_130k: \$130,000]~\&ndash;~[lcu\_120k: \$120,000]. Meanwhile, a married couple earning [lcu\_200k: \$200,000][periodicity\_tax: per year], [lcu\_100k: \$100,000] for each of them, would go untaxed.\\This tax would apply to the richest 1\% of the world's population. [tax\_country\_name: In the United States], it would affect the richest [affected\_top1: 8]\% and redistribute [transfer\_top1: 3]\% of GDP to lower-income countries.\\\\\textbf{Do you support or oppose such a global tax on the richest people to finance global poverty reduction?}\\ 
  \\ \textit{Strongly oppose; Somewhat support; Strongly support; Somewhat oppose; Indifferent}

\item ~[Branch: top3] \label{q:top3_tax_support} Currently, 3 billion people live in deep poverty, with less than [lcu\_400: \$400][periodicity: per month].\\The Sustainable Development Goals, adopted by all countries in 2015, aim to alleviate poverty and achieve access to healthcare, education, drinking water, and sanitation for all by 2030.~Due to lack of funding, the world is not on track to meet these poverty reduction goals.\\\\\textbf{Poverty reduction could be funded by a global tax on individual income above [lcu\_80k: \$80,000][periodicity\_tax: per year].~\\The tax rate would be 15\% for every [currency: dollar] over [lcu\_80k: \$80,000] of income} after existing taxes, \textbf{30\% over [lcu\_120k: \$120,000], and 45\% over [lcu\_1M: \$1 million].~}\\For example, a single person earning [lcu\_90k: \$90,000][periodicity\_tax: per year] after taxes would pay [lcu\_1500\_top3: \$1,500] in additional taxes, or 15\% of [lcu\_10k\_top3: \$10,000] = [lcu\_90k: \$90,000]~\&ndash;~[lcu\_80k: \$80,000]. Meanwhile, a married couple earning [lcu\_150k: \$150,000][periodicity\_tax: per year], [lcu\_75k: \$75,000] for each of them, would go untaxed.\\This tax would apply to the richest 3\% of the world's population. [tax\_country\_name: In the United States], it would affect the richest [affected\_top3: 18]\% and redistribute [transfer\_top3: 8]\% of GDP to lower-income countries.\\\\\textbf{Do you support or oppose such a global tax on the richest people to finance global poverty reduction?}\\ 
~[\textit{Figures \ref{fig:radical_redistr_share}, \ref{fig:top_tax_share}-\ref{fig:top_tax_positive}}; 
\verb|top3_tax_support|]
  \\ \textit{Strongly oppose; Somewhat support; Strongly support; Somewhat oppose; Indifferent}

\item  \label{q:attention_test} To show that you are attentive, please select "A little" in the following list: [%\textit{Figure \ref{fig:attention_test}}; 
\verb|attention_test|]
  \\ \textit{Not at all; A little; A lot; A great deal}

\end{enumerate} 

 \subsection*{Preferred transfer means to LICs} 
 \begin{enumerate}[resume] 
\item  \label{q:transfer_how} Below are different ways to transfer resources to help reduce poverty in a low-income country.~\\How do you evaluate each of these options?\\ 
~[\textit{Figures \ref{fig:transfer_how}, \ref{fig:transfer_how_positive}-\ref{fig:transfer_how_negative}}; 
\verb|transfer_how|]
~[Item order is randomly flipped or not]
\begin{itemize}
  \item Transfers to public development aid agencies which then finance suitable projects
  \item Transfers to the national government conditioned on the use of funds for poverty reduction programs
  \item Unconditional transfers to the national government
  \item Unconditional transfers to local authorities (municipality, village chief...)
  \item Transfers to local NGOs with democratic decision-making processes
  \item Cash transfers to parents (child allowances), to the disabled and to the elderly
  \item Unconditional cash transfers to each household
\end{itemize}
\textit{A wrong way; An acceptable way; A right way; The best way}

\end{enumerate} 

 \subsection*{Radical redistribution} 
 \begin{enumerate}[resume] 
\item  \label{q:convergence_support} Should governments actively cooperate to have all countries converge in terms of GDP per capita by the end of the century? [\textit{Figures \ref{fig:radical_redistr_share}, \ref{fig:convergence_support}}; 
\verb|convergence_support|]
  \\ \textit{Yes; No; I prefer not to answer}

\item  \label{q:global_movement} If there was a worldwide movement in favor of a global program to tackle climate change, implement taxes on millionaires and fund poverty reduction in low-income countries, to what extent would you be willing to be part of that movement? (Multiple answers possible) [\textit{Figures \ref{fig:radical_redistr_share}, \ref{fig:global_movement}}; 
\verb|global_movement|]
  \\ \textit{I would \textit{not} support such a movement.; I could sign a petition and spread ideas.; I could attend a demonstration.; I could go on strike.; I could donate [amount\_lottery: \$100] to a strike fund.}

\item ~[\textit{Except in: RU, SA}] \label{q:vote_intl_coalition} Let us call "your political party" the party you voted for in the last election, or the party that represents your views most closely.~\\\textbf{Imagine }there was \textbf{a worldwide coalition} of political parties in favor of a common program \textbf{to tackle climate change, implement taxes on millionaires and fund poverty reduction in low-income countries}.~\\\\\textbf{Would you be more likely to vote for your party if it were part of that coalition?}\\ 
~[\textit{Figures \ref{fig:radical_redistr_share}, \ref{fig:vote_intl_coalition}}; 
\verb|vote_intl_coalition|]
~[Item order is randomly flipped or not]
  \\ \textit{Yes, I would be \textbf{more likely} to vote for my party if it joined that coalition (or to vote for another party if only that other party joined the coalition).; \\My choice would \textbf{not depend} on which parties are part of that coalition.; \\No, I would be \textbf{less likely} to vote for my party if it joined that coalition.}

\item  \label{q:why_hic_help_lic} Some people think that high-income countries should support low-income countries.~\\Among the different reasons given, which ones do you agree with? (Multiple answers possible) [\textit{Figure \ref{fig:why_hic_help_lic}}; 
\verb|why_hic_help_lic|]
~[Order of the first three items is randomized]
  \\ \textit{High-income countries have a historical responsibility for the current situation in low-income countries.; \\In the long run, it is in the interest of high-income countries to help low-income countries.; \\Helping those in need is the right thing to do. This is also true at the international level.; \\None of the above.}

\item ~[\textit{Only in: FR, DE, IT, ES, GB, US}] \label{q:reparations_support} Some people argue that Western countries owe reparations for colonization and slavery to former colonies and descendants of slaves. \\Reparations could take the form of funding education and facilitating technology transfers, to address unequal opportunities passed down from the past. \\\\\textbf{Do you support or oppose reparations} of this kind \textbf{for colonization and slavery?~}\\ 
~[\textit{Figures \ref{fig:radical_redistr_share}, \ref{fig:reparations_support}}; 
\verb|reparations_support|]
  \\ \textit{Strongly oppose; Somewhat oppose; Indifferent; Somewhat support; Strongly support}

\end{enumerate} 

 \subsection*{[\textit{Except in: RU}] Custom redistribution} 
 \begin{enumerate}[resume] 
\item \label{q:income_exact} What is the \textit{[periodicity\_text: yearly]} income of your household \textbf{after taxes and social benefits}?\\This includes all sources of income: salaries, pensions, allowances, welfare benefits, property income, etc.\\My household earns ... [text\_unit: \$ per year] (answer with no comma, no space, no period):\\ 
~[%\textit{Figure \ref{fig:income_exact}}; % TODO
\verb|income_exact|]

\item ~[new page] \label{q:custom_redistr} If you could redistribute income at the global level, what would you do? In this question, we let you choose your preferred parameters for a redistribution of income at the world level.~\\If you prefer to skip this question, check the corresponding box at the bottom of the page.\\\\The worldwide redistribution of income would take the form of additional policies, taxes, and transfers, on top of existing ones.\\These policies would lower the income of the richest (the losers from the redistribution) and increase the income of the poorest (the winners).~\\\\Below you will find a graph of the world distribution of after-tax income and three sliders that vary it. The current distribution is in red, and your custom one is in green.~\\The first two sliders~control the proportion of winners and the proportion of losers, among all humans. The third slider controls the degree of redistribution from the richest to the poorest.~\\If you do not want new policies to reduce global inequality, you can set the third slider to zero.~\\\\\textbf{You need to move the sliders} (by holding the mouse down on the little squares and moving to the side) to make the green curve evolve: the idea is to move the sliders \textbf{until you get a green curve you are satisfied with}. \\\\

~\\Examples of income changes after your proposed redistribution:\\

\begin{tabular}{@{\extracolsep{5pt}}|c|c|} 
    \hline \\[-1.8ex] 
    \textbf{Now} & \textbf{After} \\\hline %\\[-1.8ex]
    0 [text\_unit: \$ per year] & [after\_0] [text\_unit: \$ per year] \\ 
    ~[now\_10k] [text\_unit] & [after\_10k] [text\_unit] \\ 
    ~[now\_60k] [text\_unit] & [after\_60k] [text\_unit] \\ 
    ~[now\_100k] [text\_unit] & [after\_100k] [text\_unit] \\ 
    \multicolumn{2}{c}{Your \textit{individual} income} \\ 
    ~[own] [text\_unit] & [after\_own] [text\_unit] \\ 
    \hline 
\end{tabular}  

% ~ [\textit{Figure \ref{fig:custom_redistr}}; 
% \verb|custom_redistr|]
~[\textit{Figures \ref{fig:custom_redistr_question}, \ref{fig:custom_redistr_mean}-\ref{fig:custom_redistr_median}} 
% \verb|variables_custom_redistr|
]
\textit{I am satisfied with my custom redistribution.; \\I want to skip this question.}

\end{enumerate} 

 \subsection*{Well-being (\textit{for another project})} 
  [\textit{Four random branches: gallup\_0; gallup\_1; wvs\_0; wvs\_1}; %\textit{Figure \ref{fig:well_being}}; 
 \verb|well_being, variant_well_being|] 
 \begin{enumerate}[resume] % TODO? condenser?
\item ~[Branch: gallup\_0] \label{q:well_being_gallup_0} Please imagine a ladder, with steps numbered from 0 at the bottom to 10 at the top. The top of the ladder represents the best possible life for you and the bottom of the ladder represents the worst possible life for you. \\\\On which step of the ladder would you say you personally feel you stand at this time? [%\textit{Figure \ref{fig:well_being}}; 
\verb|well_being_gallup_0|]
  \\ \textit{Worst possible 0; 1; 2; 3; 4; 5; 6; 7; 8; 9; Best possible 10}

\item ~[Branch: gallup\_1] \label{q:well_being_gallup_1} Please imagine a ladder, with steps numbered from 1 at the bottom to 10 at the top. The top of the ladder represents the best possible life for you and the bottom of the ladder represents the worst possible life for you. \\\\On which step of the ladder would you say you personally feel you stand at this time? [%\textit{Figure \ref{fig:well_being}}; 
\verb|well_being_gallup_1|]
  \\ \textit{Worst possible 1; 2; 3; 4; 5; 6; 7; 8; 9; Best possible 10}

\item ~[Branch: wvs\_0] \label{q:well_being_wvs_0} All things considered, how satisfied are you with your life as a whole these days? [%\textit{Figure \ref{fig:well_being}}; 
\verb|well_being_wvs_0|]
  \\ \textit{Completely dissatisfied 0; 1; 2; 3; 4; 5; 6; 7; 8; 9; Completely satisfied 10}

\item ~[Branch: wvs\_1] \label{q:well_being_wvs_1} All things considered, how satisfied are you with your life as a whole these days? [%\textit{Figure \ref{fig:well_being}}; 
\verb|well_being_wvs_1|]
  \\ \textit{Completely dissatisfied 1; 2; 3; 4; 5; 6; 7; 8; 9; Completely satisfied 10}

\end{enumerate} 

 \subsection*{Comprehension} 
 \begin{enumerate}[resume] 
\item  \label{q:gcs_comprehension} \textit{Comprehension question: one respondent with the expected answer will get [amount\_lottery: \$100].}\\\\How would gasoline prices change as a result of the Global Climate Scheme? \\Gasoline prices would... [\textit{Figure \ref{fig:gcs_comprehension}}; 
\verb|gcs_comprehension|]
~[Item order is randomly flipped or not]
  \\ \textit{increase; not be affected; decrease}

\end{enumerate} 

 \subsection*{Synthetic questions} 
 \begin{enumerate}[resume] 
\item  \label{q:my_tax_global_nation} To what extent do you agree or disagree with the following statement? "My taxes should go towards solving global problems." [\textit{Figures \ref{fig:radical_redistr_share}, \ref{fig:my_tax_global_nation_share}-\ref{fig:my_tax_global_nation_positive}}; 
\verb|my_tax_global_nation|]
  \\ \textit{Strongly agree; Agree; Neither agree nor disagree; Disagree; Strongly disagree}

\item  \label{q:group_defended} Which group of people do you advocate for when you vote? [\textit{Figures \ref{fig:group_defended}, \ref{fig:group_defended_all}}; 
\verb|group_defended|]
  \\ \textit{Sentient beings (humans and animals); Humans; [country\_adjective\_plural: Americans]; People from my community (for example my region, my religion, my gender…); My family and myself}

\end{enumerate} 

 \subsection*{Feedback} 
 \begin{enumerate}[resume] 
\item  \label{q:survey_biased} Do you feel that this survey was politically biased? [\textit{Figure \ref{fig:survey_biased}}; 
\verb|survey_biased|]
  \\ \textit{Yes, left-wing biased; Yes, right-wing biased; No, I do not feel it was biased}

\item  \label{q:comment_field} The survey is nearing completion. You can now enter any comments, thoughts, or suggestions in the field below. [%\textit{Figure \ref{fig:comment_field}}; % TODO
\verb|comment_field|]


% \item  \label{q:interview} Lastly, \textbf{would you be interested in participating in a 30-minute interview with a researcher (via videoconference)? }\\\textbf{If so}, please \textbf{leave your email}: [\textit{Figure \ref{fig:interview}}; 
% \verb|interview|] % TODO? leave?

 \end{enumerate} 



\clearpage
\section{Representativeness of the surveys}\label{app:representativeness}

\begin{table}[h!]
    \caption[Sample representativeness in FR, DE, IT]{Sample representativeness in France, Germany, Italy. %(Back to \ref{subsec:data}) 
    } \label{tab:representativeness_1}
    \makebox[\textwidth][c]{
        \resizebox*{!}{.60\textheight}{% 73 without notes cf. https://tex.stackexchange.com/questions/13809/resizing-a-table-by-textheight 
        
\begin{tabular}[t]{lccccccccc}
\toprule
\multicolumn{1}{c}{} & \multicolumn{3}{c}{France} & \multicolumn{3}{c}{Germany} & \multicolumn{3}{c}{Italy} \\
\cmidrule(l{3pt}r{3pt}){2-4} \cmidrule(l{3pt}r{3pt}){5-7} \cmidrule(l{3pt}r{3pt}){8-10}
  & Pop. & Sample & \makecell{Weighted\\sample} & Pop. & Sample & \makecell{Weighted\\sample} & Pop. & Sample & \makecell{Weighted\\sample}\\
\midrule
Sample size &  & 798 & 798 &  & 1,048 & 1,048 &  & 756 & 756\\
\addlinespace
Gender: Woman & .52 & .52 & .52 & .51 & .49 & .51 & .52 & .52 & .51\\
Gender: Man & .48 & .48 & .48 & .49 & .51 & .49 & .48 & .48 & .49\\
\addlinespace
Income\_quartile: Q1 & .25 & .26 & .25 & .25 & .27 & .25 & .25 & .26 & .25\\
Income\_quartile: Q2 & .25 & .26 & .25 & .25 & .27 & .25 & .25 & .26 & .25\\
Income\_quartile: Q3 & .25 & .23 & .25 & .25 & .20 & .25 & .25 & .22 & .25\\
Income\_quartile: Q4 & .25 & .25 & .25 & .25 & .26 & .25 & .25 & .25 & .25\\
\addlinespace
Age: 18-24 & .10 & .11 & .10 & .09 & .10 & .09 & .08 & .08 & .08\\
Age: 25-34 & .15 & .15 & .15 & .15 & .16 & .15 & .12 & .12 & .12\\
Age: 35-49 & .23 & .23 & .23 & .23 & .25 & .23 & .23 & .23 & .23\\
Age: 50-64 & .24 & .24 & .24 & .27 & .27 & .27 & .28 & .29 & .28\\
Age: 65+ & .27 & .27 & .27 & .27 & .22 & .27 & .29 & .28 & .29\\
\addlinespace
Diploma\_25-64: Below upper secondary & .10 & .09 & .10 & .11 & .11 & .11 & .22 & .19 & .22\\
Diploma\_25-64: Upper secondary & .26 & .26 & .26 & .32 & .32 & .32 & .28 & .28 & .28\\
Diploma\_25-64: Post secondary & .26 & .27 & .26 & .22 & .24 & .21 & .14 & .17 & .14\\
\addlinespace
Urbanity: Cities & .47 & .47 & .46 & .39 & .42 & .39 & .36 & .37 & .36\\
Urbanity: Towns and suburbs & .19 & .19 & .19 & .42 & .42 & .42 & .46 & .47 & .46\\
Urbanity: Rural & .34 & .33 & .34 & .19 & .17 & .19 & .18 & .16 & .18\\
\addlinespace
Region: 1 & .18 & .19 & .18 & .17 & .19 & .17 & .66 & .70 & .65\\
Region: 2 & .22 & .23 & .22 & .29 & .32 & .29 & .34 & .29 & .34\\
Region: 3 & .11 & .11 & .11 & .54 & .48 & .54 &  &  & \\
Region: 4 & .21 & .22 & .21 &  &  &  &  &  & \\
Region: 5 & .28 & .26 & .28 &  &  &  &  &  & \\
\bottomrule
\end{tabular}
        }
    }
    {\footnotesize \textit{Note}: This table displays summary statistics of the samples alongside actual population frequencies. 
    Detailed sources for each variable and country population frequencies, as well as the definitions of regions, diploma, urbanity, employment, and vote are available in \href{https://github.com/bixiou/robustness_global_redistr/raw/main/questionnaire/sources.xlsx}{this spreadsheet}. 
    } 
\end{table}

\begin{table}[h!]
    \caption[Sample representativeness in PL, ES, GB, CH]{Sample representativeness in Poland, Spain, the UK, Switzerland. %(Back to \ref{subsec:data}) 
    } \label{tab:representativeness_2}
    \makebox[\textwidth][c]{\resizebox*{!}{.60\textheight}{\input{../tables/PL_ES_GB_CH}}}
    {\footnotesize \textit{Note}: This table displays summary statistics of the samples alongside actual population frequencies. 
    Detailed sources for each variable and country population frequencies, as well as the definitions of regions, diploma, urbanity, employment, and vote are available in \href{https://github.com/bixiou/robustness_global_redistr/raw/main/questionnaire/sources.xlsx}{this spreadsheet}. 
    } 
\end{table}

\begin{table}[h!]
    \caption[Sample representativeness in JP, SA, US]{Sample representativeness in Japan, Saudi Arabia, the United States. %(Back to \ref{subsec:data}) 
    } \label{tab:representativeness_3}
    \makebox[\textwidth][c]{\resizebox*{!}{.80\textheight}{\input{../tables/JP_SA_US}}}
    {\footnotesize \textit{Note}: This table displays summary statistics of the samples alongside actual population frequencies. 
    Detailed sources for each variable and country population frequencies, as well as the definitions of regions, diploma, urbanity, employment, and vote are available in \href{https://github.com/bixiou/robustness_global_redistr/raw/main/questionnaire/sources.xlsx}{this spreadsheet}. 
    } 
\end{table}
\clearpage
\section{Determinants of support}\label{app:determinants}

\begin{table}[h]\label{tab:determinant}
    \caption[Correlates of support for global redistribution]{Correlates of support for global redistribution (multivariate OLS regressions). %(Back to \ref{subsec:gcs_stated_support})
    } 
    \makebox[\textwidth][c]{
\resizebox*{!}{.67\textheight}{ 
        
\begin{tabular}{@{\extracolsep{5pt}}lccccccc} 
\\[-1.8ex]\hline 
\hline \\[-1.8ex] 
\\[-1.8ex] & \makecell{Share of\\plausible\\policies\\supported} & \makecell{Supports\\the Global\\Climate\\Scheme} & \makecell{Universalist\\(Group\\defended:\\\textit{Humans} or\\\textit{Sentient beings})} & \makecell{More likely\\to vote\\for party\\in global\\coalition} & \makecell{Endorses\\convergence\\of all countries'\\GDP p.c.\\by 2100} & \makecell{Supports an\\international\\wealth tax\\funding LICs} & \makecell{Prefers a\\sustainable\\future} \\ 
\\[-1.8ex] & (1) & (2) & (3) & (4) & (5) & (6) & (7)\\ 
\hline \\[-1.8ex] 
Mean & 0.508 & 0.554 & 0.454 & 0.365 & 0.61 & 0.704 & 0.681  \\ \hline \\[-1.8ex]
 Vote: Center\mbox{-}right or Right & 0.015 & 0.008 & $-$0.083$^{***}$ & 0.029$^{**}$ & 0.041$^{***}$ & $-$0.026$^{*}$ & $-$0.061$^{***}$ \\ 
  & (0.010) & (0.015) & (0.014) & (0.013) & (0.014) & (0.014) & (0.014) \\ 
  Vote: Far right & $-$0.090$^{***}$ & $-$0.143$^{***}$ & $-$0.225$^{***}$ & $-$0.063$^{***}$ & $-$0.065$^{***}$ & $-$0.140$^{***}$ & $-$0.169$^{***}$ \\ 
  & (0.013) & (0.020) & (0.019) & (0.018) & (0.020) & (0.019) & (0.020) \\ 
  Vote: Left & 0.211$^{***}$ & 0.170$^{***}$ & 0.150$^{***}$ & 0.257$^{***}$ & 0.190$^{***}$ & 0.184$^{***}$ & 0.147$^{***}$ \\ 
  & (0.010) & (0.014) & (0.015) & (0.014) & (0.014) & (0.013) & (0.014) \\ 
  Gender: Man & 0.016$^{**}$ & 0.018$^{*}$ & $-$0.044$^{***}$ & 0.029$^{***}$ & 0.009 & $-$0.007 & $-$0.043$^{***}$ \\ 
  & (0.007) & (0.010) & (0.010) & (0.010) & (0.010) & (0.009) & (0.009) \\ 
  Age: 18\mbox{-}24 & 0.012 & 0.175$^{***}$ & 0.104$^{***}$ & 0.108$^{***}$ & 0.109$^{***}$ & 0.101$^{***}$ & 0.062$^{***}$ \\ 
  & (0.014) & (0.020) & (0.021) & (0.022) & (0.020) & (0.018) & (0.019) \\ 
  Age: 25\mbox{-}34 & 0.020$^{*}$ & 0.094$^{***}$ & 0.075$^{***}$ & 0.102$^{***}$ & 0.046$^{***}$ & 0.046$^{***}$ & 0.027$^{*}$ \\ 
  & (0.011) & (0.015) & (0.016) & (0.016) & (0.015) & (0.014) & (0.015) \\ 
  Age: 50\mbox{-}64 & $-$0.002 & $-$0.036$^{**}$ & $-$0.034$^{**}$ & $-$0.033$^{**}$ & $-$0.025$^{*}$ & $-$0.021 & $-$0.020 \\ 
  & (0.010) & (0.014) & (0.014) & (0.014) & (0.013) & (0.013) & (0.013) \\ 
  Age: 65+ & 0.041$^{***}$ & $-$0.020 & $-$0.010 & 0.002 & $-$0.021 & $-$0.018 & 0.016 \\ 
  & (0.012) & (0.018) & (0.018) & (0.017) & (0.017) & (0.016) & (0.016) \\ 
  Income quartile: Q2 & 0.018$^{*}$ & 0.004 & $-$0.025$^{*}$ & 0.016 & $-$0.014 & 0.013 & 0.010 \\ 
  & (0.010) & (0.015) & (0.015) & (0.015) & (0.014) & (0.013) & (0.014) \\ 
  Income quartile: Q3 & 0.007 & $-$0.010 & 0.019 & $-$0.009 & $-$0.024$^{*}$ & $-$0.018 & 0.002 \\ 
  & (0.010) & (0.015) & (0.015) & (0.015) & (0.014) & (0.014) & (0.014) \\ 
  Income quartile: Q4 & $-$0.010 & $-$0.042$^{***}$ & $-$0.004 & $-$0.032$^{*}$ & $-$0.075$^{***}$ & $-$0.078$^{***}$ & 0.007 \\ 
  & (0.011) & (0.016) & (0.016) & (0.017) & (0.015) & (0.015) & (0.015) \\ 
  Diploma: Upper secondary & 0.042$^{***}$ & 0.001 & 0.018 & 0.036$^{**}$ & 0.029$^{*}$ & 0.022 & 0.022 \\ 
  & (0.011) & (0.016) & (0.016) & (0.015) & (0.015) & (0.014) & (0.015) \\ 
  Diploma: Above upper secondary & 0.085$^{***}$ & 0.026 & 0.025 & 0.079$^{***}$ & 0.015 & 0.015 & 0.039$^{**}$ \\ 
  & (0.011) & (0.016) & (0.016) & (0.015) & (0.015) & (0.015) & (0.016) \\ 
  Urbanicity: Rural & $-$0.012 & $-$0.054$^{***}$ & 0.016 & $-$0.006 & $-$0.015 & $-$0.021 & $-$0.020 \\ 
  & (0.010) & (0.015) & (0.015) & (0.014) & (0.015) & (0.014) & (0.015) \\ 
  Urbanicity: Towns and suburbs & $-$0.014 & $-$0.039$^{**}$ & $-$0.022 & $-$0.023 & $-$0.016 & $-$0.024$^{*}$ & 0.026$^{*}$ \\ 
  & (0.010) & (0.015) & (0.015) & (0.015) & (0.015) & (0.014) & (0.014) \\ 
  Will become millionaire: Likely & 0.036$^{***}$ & 0.070$^{***}$ & $-$0.001 & 0.039$^{***}$ & 0.055$^{***}$ & $-$0.019$^{*}$ & $-$0.019 \\ 
  & (0.008) & (0.012) & (0.012) & (0.013) & (0.012) & (0.011) & (0.012) \\ 
  Will become millionaire: Already & $-$0.020 & $-$0.019 & 0.008 & $-$0.058$^{**}$ & $-$0.042$^{*}$ & $-$0.236$^{***}$ & $-$0.047$^{**}$ \\ 
  & (0.017) & (0.023) & (0.024) & (0.023) & (0.023) & (0.023) & (0.022) \\ 
  Foreign born & 0.065$^{***}$ & 0.083$^{***}$ & 0.088$^{***}$ & 0.051$^{**}$ & 0.037$^{*}$ & 0.040$^{**}$ & 0.030 \\ 
  & (0.014) & (0.020) & (0.021) & (0.022) & (0.020) & (0.018) & (0.019) \\ 
 \hline \\[-1.8ex] 

Observations & 12,001 & 12,001 & 12,001 & 10,000 & 12,001 & 12,001 & 12,001 \\ 
R$^{2}$ & 0.141 & 0.104 & 0.100 & 0.115 & 0.105 & 0.091 & 0.069 \\ 
\hline 
\hline \\[-1.8ex] 
\end{tabular} 
        }
    }
    {\footnotesize \textit{Note}: Robust standard errors are reported in parentheses. Covariates omitted in the Table: \textit{Country}; \textit{Employment}; \textit{Couple}; \textit{Region}; \textit{Constant}. Omitted variables are: \textit{Vote: Non-voter, PNR or Other}; \textit{Gender: Woman}; \textit{Age: 35-49}; \textit{Income\_quartile: Q1}; \textit{Diploma: Below upper secondary}; \textit{Urbanicity: City}. \hfill $^{*}$p$<$0.1; $^{**}$p$<$0.05; $^{***}$p$<$0.01.
    }
\end{table}


\clearpage
\section{Attrition analysis}\label{app:attrition}

\begin{table}[h!]\label{tab:attrition}
    \caption[Attrition analysis]{Attrition analysis.} 
    \makebox[\textwidth][c]{\resizebox*{!}{.87\textheight}{ % 73 is the max when there is a title
        
\begin{tabular}{@{\extracolsep{5pt}}lccccc} 
\\[-1.8ex]\hline 
\hline \\[-1.8ex] 
\\[-1.8ex] & \makecell{Dropped out} & \makecell{Dropped out\\after\\socio-eco} & \makecell{Failed\\attention test} & \makecell{Duration\\(in min)} & \makecell{Duration\\below\\6 min} \\ 
\\[-1.8ex] & (1) & (2) & (3) & (4) & (5)\\ 
\hline \\[-1.8ex] 
Mean & 0.166 & 0.102 & 0.088 & 53.896 & 0.087  \\ \hline \\[-1.8ex]
 Vote: Center\mbox{-}right or Right & $-$0.042$^{***}$ & $-$0.040$^{***}$ & $-$0.008 & $-$7.194 & $-$0.025$^{***}$ \\ 
  & (0.008) & (0.008) & (0.007) & (10.597) & (0.008) \\ 
  Vote: Far right & $-$0.051$^{***}$ & $-$0.050$^{***}$ & $-$0.012 & $-$18.804 & $-$0.021$^{**}$ \\ 
  & (0.010) & (0.010) & (0.009) & (15.429) & (0.009) \\ 
  Vote: Left & $-$0.029$^{***}$ & $-$0.027$^{***}$ & $-$0.013$^{*}$ & $-$16.126 & $-$0.041$^{***}$ \\ 
  & (0.008) & (0.008) & (0.007) & (10.494) & (0.008) \\ 
  Gender: Man & $-$0.043$^{***}$ & $-$0.042$^{***}$ & 0.026$^{***}$ & $-$14.234$^{**}$ & 0.004 \\ 
  & (0.005) & (0.005) & (0.005) & (7.063) & (0.005) \\ 
  Age: 18\mbox{-}24 & $-$0.030$^{***}$ & $-$0.030$^{***}$ & 0.028$^{**}$ & $-$19.220$^{***}$ & 0.089$^{***}$ \\ 
  & (0.010) & (0.010) & (0.012) & (6.900) & (0.013) \\ 
  Age: 25\mbox{-}34 & $-$0.031$^{***}$ & $-$0.031$^{***}$ & 0.018$^{**}$ & $-$4.123 & 0.051$^{***}$ \\ 
  & (0.007) & (0.007) & (0.009) & (8.929) & (0.009) \\ 
  Age: 50\mbox{-}64 & 0.011 & 0.011 & $-$0.033$^{***}$ & 9.254 & $-$0.059$^{***}$ \\ 
  & (0.008) & (0.008) & (0.007) & (10.045) & (0.007) \\ 
  Age: 65+ & 0.039$^{***}$ & 0.039$^{***}$ & $-$0.058$^{***}$ & 28.229$^{*}$ & $-$0.105$^{***}$ \\ 
  & (0.010) & (0.010) & (0.008) & (15.380) & (0.008) \\ 
  Income quartile: Q2 & $-$0.029$^{***}$ & $-$0.029$^{***}$ & $-$0.040$^{***}$ & $-$0.384 & $-$0.012 \\ 
  & (0.008) & (0.008) & (0.008) & (8.336) & (0.008) \\ 
  Income quartile: Q3 & $-$0.028$^{***}$ & $-$0.030$^{***}$ & $-$0.056$^{***}$ & $-$13.405$^{**}$ & $-$0.017$^{**}$ \\ 
  & (0.008) & (0.008) & (0.008) & (6.818) & (0.008) \\ 
  Income quartile: Q4 & $-$0.032$^{***}$ & $-$0.032$^{***}$ & $-$0.060$^{***}$ & 16.509 & $-$0.029$^{***}$ \\ 
  & (0.009) & (0.009) & (0.009) & (13.757) & (0.008) \\ 
  Diploma: Upper secondary & $-$0.018$^{**}$ & $-$0.018$^{**}$ & $-$0.052$^{***}$ & 7.862 & $-$0.004 \\ 
  & (0.009) & (0.009) & (0.009) & (9.883) & (0.008) \\ 
  Diploma: Above upper secondary & $-$0.042$^{***}$ & $-$0.043$^{***}$ & $-$0.065$^{***}$ & 1.479 & $-$0.017$^{**}$ \\ 
  & (0.009) & (0.009) & (0.009) & (10.935) & (0.008) \\ 
  Urbanicity: Rural & $-$0.004 & $-$0.005 & $-$0.008 & $-$4.275 & $-$0.003 \\ 
  & (0.008) & (0.008) & (0.007) & (8.305) & (0.007) \\ 
  Urbanicity: Towns and suburbs & 0.007 & 0.007 & $-$0.016$^{**}$ & 5.311 & 0.001 \\ 
  & (0.008) & (0.008) & (0.007) & (16.323) & (0.007) \\ 
  Country: Germany & $-$0.353 & $-$0.352 & $-$0.679$^{**}$ & 25.812 & $-$0.211$^{***}$ \\ 
  & (0.263) & (0.263) & (0.329) & (30.763) & (0.025) \\ 
  Country: Italy & $-$0.123 & $-$0.121 & $-$0.675$^{**}$ & 688.464 & $-$0.207$^{***}$ \\ 
  & (0.318) & (0.319) & (0.329) & (604.750) & (0.033) \\ 
  Country: Japan & $-$0.320 & $-$0.317 & $-$0.707$^{**}$ & 5.073 & 0.206 \\ 
  & (0.262) & (0.263) & (0.329) & (27.025) & (0.196) \\ 
  Country: Poland & $-$0.258 & $-$0.257 & $-$0.735$^{**}$ & $-$27.152 & 0.770$^{***}$ \\ 
  & (0.261) & (0.262) & (0.329) & (26.191) & (0.025) \\ 
  Country: Saudi Arabia & $-$0.072 & $-$0.069 & $-$0.260 & 18.800 & $-$0.320$^{***}$ \\ 
  & (0.280) & (0.281) & (0.354) & (31.927) & (0.038) \\ 
  Country: Spain & $-$0.299 & $-$0.296 & $-$0.704$^{**}$ & 95.626$^{***}$ & $-$0.323$^{***}$ \\ 
  & (0.262) & (0.262) & (0.329) & (26.178) & (0.026) \\ 
  Country: Switzerland & 0.001 & 0.001 & $-$0.001 & 2.543 & $-$0.004 \\ 
  & (0.025) & (0.025) & (0.020) & (13.928) & (0.020) \\ 
  Country: United Kingdom & $-$0.262 & $-$0.261 & $-$0.839$^{**}$ & 65.461$^{**}$ & $-$0.383$^{***}$ \\ 
  & (0.262) & (0.262) & (0.329) & (28.897) & (0.029) \\ 
  Country: USA & $-$0.024 & $-$0.021 & $-$0.134$^{***}$ & $-$2.372 & $-$0.147$^{***}$ \\ 
  & (0.023) & (0.023) & (0.020) & (18.225) & (0.020) \\ 
 \hline \\[-1.8ex] 

Observations & 15,013 & 15,013 & 13,261 & 12,031 & 12,031 \\ 
R$^{2}$ & 0.037 & 0.037 & 0.079 & 0.012 & 0.089 \\ 
\hline 
\hline \\[-1.8ex] 
\end{tabular} }}
    \centering {\footnotesize \textit{Note}: Robust standard errors are reported in parentheses. $^{*}$p$<$0.1; $^{**}$p$<$0.05; $^{***}$p$<$0.01.}
\end{table}

\clearpage
\section{Balance analysis}\label{app:balance}

\begin{table}[h]\label{tab:balance}
    \caption[Balance analysis]{Balance analysis.} 
    \makebox[\textwidth][c]{
\resizebox*{!}{.71\textheight}{ 
        
\begin{tabular}{@{\extracolsep{5pt}}lcccccccc} 
\\[-1.8ex]\hline 
\hline \\[-1.8ex] 
 & \multicolumn{8}{c}{Random branch:} \\ 
\cline{2-9} 
\\[-1.8ex] & \makecell{Wealth tax\\coverage:\\Global} & \makecell{Wealth tax\\coverage:\\Int'l} & \makecell{Int'l CS\\coverage:\\Low} & \makecell{Int'l CS\\coverage:\\High} & \makecell{Int'l CS\\coverage:\\High color} & \makecell{National\\CS\\asked} & \makecell{Warm glow\\substitute:\\Control} & \makecell{Warm glow\\realism: Info\\treatment} \\ 
\\[-1.8ex] & (1) & (2) & (3) & (4) & (5) & (6) & (7) & (8)\\ 
\hline \\[-1.8ex] 
Mean & 0.332 & 0.334 & 0.25 & 0.256 & 0.252 & 0.36 & 0.358 & 0.489  \\ \hline \\[-1.8ex]
 Vote: Center\mbox{-}right or Right & 0.001 & 0.018 & $-$0.011 & 0.001 & 0.011 & $-$0.013 & 0.003 & $-$0.013 \\ 
  & (0.013) & (0.013) & (0.013) & (0.012) & (0.012) & (0.014) & (0.013) & (0.014) \\ 
  Vote: Far right & $-$0.026 & 0.016 & $-$0.015 & 0.025 & 0.019 & $-$0.008 & 0.019 & $-$0.010 \\ 
  & (0.018) & (0.018) & (0.017) & (0.017) & (0.017) & (0.018) & (0.019) & (0.020) \\ 
  Vote: Left & $-$0.004 & 0.001 & $-$0.004 & $-$0.007 & 0.002 & $-$0.003 & $-$0.005 & $-$0.014 \\ 
  & (0.014) & (0.014) & (0.013) & (0.013) & (0.012) & (0.014) & (0.013) & (0.014) \\ 
  Gender: Man & $-$0.00003 & $-$0.015$^{*}$ & $-$0.013 & 0.005 & $-$0.00005 & 0.001 & $-$0.007 & 0.011 \\ 
  & (0.009) & (0.009) & (0.008) & (0.008) & (0.008) & (0.009) & (0.009) & (0.010) \\ 
  Age: 18\mbox{-}24 & $-$0.009 & 0.007 & $-$0.013 & $-$0.001 & 0.007 & 0.009 & $-$0.020 & 0.001 \\ 
  & (0.018) & (0.018) & (0.017) & (0.017) & (0.017) & (0.019) & (0.018) & (0.020) \\ 
  Age: 25\mbox{-}34 & $-$0.024$^{*}$ & 0.025$^{*}$ & $-$0.008 & 0.018 & $-$0.004 & $-$0.003 & $-$0.001 & $-$0.011 \\ 
  & (0.014) & (0.014) & (0.012) & (0.013) & (0.013) & (0.014) & (0.014) & (0.015) \\ 
  Age: 50\mbox{-}64 & $-$0.009 & 0.008 & 0.006 & 0.016 & $-$0.003 & 0.008 & 0.003 & $-$0.009 \\ 
  & (0.013) & (0.013) & (0.012) & (0.012) & (0.012) & (0.013) & (0.013) & (0.013) \\ 
  Age: 65+ & $-$0.013 & 0.031$^{*}$ & 0.011 & 0.030$^{**}$ & $-$0.023 & $-$0.014 & 0.020 & $-$0.0004 \\ 
  & (0.016) & (0.016) & (0.015) & (0.015) & (0.015) & (0.016) & (0.016) & (0.017) \\ 
  Income quartile: Q2 & $-$0.006 & 0.001 & 0.00001 & $-$0.018 & $-$0.008 & $-$0.003 & $-$0.010 & $-$0.012 \\ 
  & (0.013) & (0.013) & (0.012) & (0.012) & (0.012) & (0.013) & (0.013) & (0.014) \\ 
  Income quartile: Q3 & 0.002 & 0.0003 & 0.005 & $-$0.014 & 0.0001 & $-$0.010 & $-$0.013 & 0.004 \\ 
  & (0.013) & (0.013) & (0.012) & (0.012) & (0.012) & (0.014) & (0.014) & (0.014) \\ 
  Income quartile: Q4 & $-$0.015 & 0.012 & $-$0.006 & $-$0.001 & 0.010 & $-$0.012 & 0.001 & $-$0.006 \\ 
  & (0.015) & (0.015) & (0.013) & (0.014) & (0.013) & (0.015) & (0.015) & (0.016) \\ 
  Diploma: Upper secondary & 0.013 & 0.003 & $-$0.0003 & $-$0.010 & 0.012 & $-$0.013 & 0.018 & $-$0.021 \\ 
  & (0.014) & (0.014) & (0.013) & (0.013) & (0.013) & (0.014) & (0.014) & (0.015) \\ 
  Diploma: Above upper secondary & 0.030$^{**}$ & $-$0.008 & 0.002 & $-$0.015 & 0.020 & $-$0.006 & 0.003 & $-$0.004 \\ 
  & (0.015) & (0.014) & (0.013) & (0.014) & (0.013) & (0.015) & (0.015) & (0.015) \\ 
  Urbanicity: Rural & 0.010 & 0.012 & 0.009 & $-$0.013 & $-$0.005 & $-$0.003 & 0.012 & $-$0.006 \\ 
  & (0.014) & (0.014) & (0.013) & (0.013) & (0.013) & (0.014) & (0.014) & (0.015) \\ 
  Urbanicity: Towns and suburbs & 0.021 & $-$0.015 & 0.008 & $-$0.004 & 0.003 & 0.011 & $-$0.015 & 0.004 \\ 
  & (0.015) & (0.015) & (0.014) & (0.013) & (0.013) & (0.015) & (0.015) & (0.016) \\ 
  Will become millionaire: Likely & 0.016 & $-$0.015 & $-$0.004 & 0.014 & $-$0.013 & 0.005 & 0.0002 & 0.006 \\ 
  & (0.011) & (0.011) & (0.010) & (0.010) & (0.010) & (0.011) & (0.011) & (0.012) \\ 
  Will become millionaire: Already & 0.006 & $-$0.010 & $-$0.008 & $-$0.001 & $-$0.014 & 0.029 & $-$0.005 & $-$0.035 \\ 
  & (0.022) & (0.022) & (0.020) & (0.020) & (0.020) & (0.022) & (0.022) & (0.023) \\ 
  Foreign born & $-$0.010 & 0.017 & $-$0.014 & 0.035$^{**}$ & $-$0.012 & 0.014 & $-$0.006 & $-$0.014 \\ 
  & (0.017) & (0.017) & (0.016) & (0.017) & (0.016) & (0.018) & (0.018) & (0.018) \\ 
 \hline \\[-1.8ex] 

Observations & 12,001 & 12,001 & 11,993 & 11,993 & 11,993 & 12,001 & 12,001 & 12,001 \\ 
R$^{2}$ & 0.006 & 0.006 & 0.005 & 0.005 & 0.005 & 0.021 & 0.025 & 0.006 \\ 
\hline 
\hline \\[-1.8ex] 
\end{tabular} 
        }
    }
    \centering {\footnotesize \textit{Note}: Robust standard errors are in parentheses. \textit{CS}: \textit{Climate Scheme}. $^{*}$p$<$0.1; $^{**}$p$<$0.05; $^{***}$p$<$0.01.
    }
\end{table}
\clearpage

\section{Placebo tests}\label{app:placebo}

% \begin{table}[h]\label{tab:placebo}
%     \caption[Placebo tests]{Placebo tests.} 
%     \makebox[\textwidth][c]{
% % \resizebox*{!}{.73\textheight}{ % 73 is the max when there is a title
%         \input{../tables/placebo_tests.tex}
%         }
%     % }
%     {\footnotesize \textit{Note}: Standard errors are reported in parentheses.
%     }
% \end{table}

\section{Main results on the extended sample}\label{app:extended}

% As a robustness check, we reproduce our main results on the extended sample that includes the 14\% respondents who failed the attention check or rushed through the survey ($n = 9,318$). These results are non-weighted. They closely match the results in our main specification. For example, the support for the GCS is 54\% in the U.S. and 75\% in Europe, while the same coefficients are significant for the list experiment. % and the conjoint analyses. 

% \begin{figure}[h!] 
%     \caption[(Extended sample) Main attitudes]{[Extended sample] Main attitudes. \\ (Relative support ---unless *--- in percent in Questions \ref{q:gcs_support}, \ref{q:global_tax}, \ref{q:other_policies}, \ref{q:foreign_aid_raise_support}, \ref{q:negotiation}) \hfill (Back~to~Section~\ref{subsec:universalistic})}\label{fig:main_by_vote_alla}
%     \makebox[\textwidth][c]{\includegraphics[width=\textwidth]{../figures/country_comparison/main_alla_share.pdf}} 
% \end{figure}
 

\clearpage
\renewcommand{\url}[1]{\href{#1}{Link}} 
\putbib
\end{bibunit}

\listoftables
\listoffigures


\end{document}
