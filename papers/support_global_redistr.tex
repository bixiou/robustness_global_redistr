% TODO check AER, revision
\documentclass[12pt,english]{article}
\usepackage[utf8]{inputenc}
\usepackage{tgpagella} % Palatino text only
\usepackage{mathpazo}  % Palatino math & text
\usepackage[left=1.5in,right=1.5in,top=1.5in,bottom=1.5in]{geometry}
% \linespread{1.5}
% \usepackage[super,compress]{natbib} % WPcomment
\usepackage{bibunits} % To get multiple bibliography
\usepackage[round,sort&compress]{natbib} % NCCcomment
\usepackage{url} % [hyphens]
\usepackage[hyperpageref]{backref} % back references biblio. Needs latexmk at compilation. 
\usepackage[pagebackref]{hyperref} 

% Lines below required to make pagebackref compatible with bibunits
\usepackage{etoolbox}
\makeatletter
\patchcmd\Hy@backout{\@auxout}{\@mainaux}{}{\fail}
\patchcmd\Hy@backout{\@auxout}{\@mainaux}{}{\fail} %yes twice the same line
\makeatother

% \usepackage{hyperref}
% \usepackage{multibib} % incompatible with backref
\hypersetup{
  colorlinks=true, % breaklinks=true,
  urlcolor=purple,    % color of external links
  linkcolor=blue,  % color of toc, list of figs etc.
  citecolor=violet,   % color of links to bibliography
}
\usepackage{bm}
\usepackage{indentfirst}
\usepackage{tocbibind}
\setcitestyle{aysep={}} 
\usepackage{amsmath}
\usepackage{tcolorbox}
\usepackage{amssymb}
\usepackage{eurosym}
\usepackage{amsfonts}
% \usepackage{fontspec} 
\usepackage{hwemoji} % for emojis
\usepackage{enumerate}
\usepackage{babel}
\usepackage{graphicx}
\usepackage{caption}
\usepackage{supertabular}
\usepackage{tabularx}
\usepackage{float}
\usepackage{dsfont}
\usepackage{fancyvrb}
\usepackage{verbatim}
\usepackage{enumitem}
\usepackage{setspace}
\usepackage{comment}
\usepackage{subcaption}
\usepackage{tikz}
\usepackage{gensymb}
\usepackage{textcomp}

\usepackage{tabulary}
\usepackage{tabularx}
\usepackage{paracol}
\usepackage{booktabs}
\usepackage{fullpage}
\usepackage{morefloats}
\usepackage{makecell}
\usepackage{lscape}
\usepackage{pdflscape}
\usepackage{longtable}
\usepackage{rotating}
\usepackage{fancyhdr}
\usepackage{tocloft}
\usepackage{titletoc}
\usepackage[export]{adjustbox}
\usepackage[anythingbreaks]{breakurl} % for links
\usepackage{multicol}
\usepackage{lineno}
% \linenumbers
\newsavebox\ltmcbox % For net gain table over two columns
%\usepackage[nomarkers,figuresonly]{endfloat} % Figures at the end
%\usepackage[section,below]{placeins} % Floats placed in the section they appear in.
\renewcommand{\floatpagefraction}{.99}
\newenvironment{stretchpars}{\par\setlength{\parfillskip}{0pt}}{\par} % to justify a line

% \newcommand\citeR[1]{\citeauthor{#1} (\citeyear{#1})}
% \newcommand\cite[1]{\citeauthor{#1} (\citeyear{#1})}
% \newcommand\citep[1]{(\citeauthor{#1} \citeyear{#1})}
% \newcommand\citeRalp[1]{\citeauthor{#1} \citeyear{#1}}
\defaultbibliographystyle{plainnaturl_clean}
\defaultbibliography{global_tax_attitudes}
% \sloppy

% # TODO RU find RU$zipcode_clean[is.na(RU$region)] and update regions_ru
% # TODO representativeness RU region (more generally, check whether variable exist)
% # todo go through all fields again to fill up two new categories: "economy" and "criticize handouts / calls for lower taxes on labor income or lower welfare benefits"
% # TODO: clean files (cf. analysis.R)
% # todo: weight_control pre-compute weight_different_controls to speed up and allow use for special_levels (discarded method: reweighted_estimate)
% TODO integrate files from global_tax_attitudes in 5_questionnaire and run


\title{Public 
Acceptance of %Attitudes on 
International Redistribution\\
in High-Income Countries
}
% \author{Adrien Fabre$^{1,2}$, Thomas Douenne$^3$ and Linus Mattauch$^{4,5,6}$} % WPcomment
\author{Adrien Fabre\footnote{CNRS, CIRED. E-mail: adrien.fabre@cnrs.fr.}
% ~~\thanks{The project is approved by Economics \& Business Ethics Committee (EBEC) at the University of Amsterdam (EB-1113) and %is approved by IRB at Harvard University (IRB21-0137), and 
} % NCCcomment

\date{\today} % NCCcomment

\begin{document}

\maketitle

\begin{center}
{\textbf{\href{https://github.com/bixiou/robustness_global_redistr/raw/main/papers/support_global_redistr.pdf}{Link to most recent version}}}
\end{center}


% WPcomment
% \begin{affiliations}
% \item CNRS
% \item CIRED
% \item University of Amsterdam
% \item Technical University Berlin
% \item Potsdam Institute for Climate Impact Research 
% \item University of Oxford
% \end{affiliations}

% \begin{small} % NCCcomment
\begin{abstract}

  % 166 words
Using an original survey of 12,000 respondents representative of eleven high-income countries (the United States, Japan, Russia, Saudi Arabia, and seven European countries), I examine public support for global redistribution and climate policies, as well as its sensitivity to key policy features such as the size of transfers and country coverage. Although global inequality is not a salient concern, it is perceived as a significant injustice. 
There is majority acceptance in every country for nearly all global policies tested, including those that would redistribute 5 percent of global income or entail personal costs for respondents. 
% Global inequality is a vote-determining issue for many people, and political programs that address it are more likely to be preferred. I find no evidence of warm glow: % evidence for a status quo bias rather than warm glow?
% % pluralistic ignorance?
% support is higher among treated respondents, who perceive global redistribution as more likely. 
An information treatment shows that support for global policies causally increases among respondents who perceive them as likely; an effect opposite to warm glow. 
Support for international policies decreases only slightly as country coverage shrinks. Overall, the results reinforce previous findings and suggest that a broad coalition of countries could feasibly advance sustainable development. 

% AER (100 words; max 100 words)
% Using an original survey of 12,000 respondents representative of eleven high-income countries, I examine public support for global redistribution and climate policies. Although global inequality is not a salient concern, political programs that address it are more likely to be preferred. In every country, majorities accept nearly all global policies tested, including those that would redistribute 5 percent of global income or entail personal costs for respondents. Survey experiments demonstrate the robustness of support. In particular, an information treatment shows that support for global policies causally increases among respondents who perceive them as likely; an effect opposite to warm glow. 

% Alternative:
% Using an original survey of 12,000 respondents representative of eleven high-income countries, I examine public support for global redistribution and climate policies. Although global inequality is not a salient concern, political programs that address it are more likely to be preferred. There is majority acceptance in every country for nearly all global policies tested, including those that would redistribute 5 percent of global income or entail personal costs for respondents. I find no evidence of warm glow: acceptance increases among treated respondents, who perceive global redistribution as more likely. Finally, acceptance of %international 
% policies decreases only slightly as country coverage shrinks. 

% Acceptance increases following a treatment that boosts the belief that global redistribution is likely. 

% AI: I am a researcher and am drafting an academic paper intended for publication in the American Economic Review. I want you to help me improve the English style of that paper, paragraph by paragraph. Keep your responses concise: no need to comment on your editing suggestions, just give the suggested edited paragraph. Highlight (e.g. in bold or strikes) the changes you make. I want to stress that global policies would be *accepted* rather than supported, in the sense that the public is unlikely to mobilize against such policies should they be implemented, but is also unlikely to actively demand them. Keep that in mind while editing (i.e. prefer the use of "acceptance" over "support").

% Through an original survey on 12,000 respondents representative of eleven high-income countries (U.S., Japan, Russia, Saudi Arabia, and seven European countries), I study the extent of support for global redistribution and climate policies, and their sensitivity to policy features such as the magnitude of the transfers or country coverage. Though not a salient concern, global inequality is seen as a big injustice. There is majority support for almost all global policies in almost all countries, including for policies that would redistribute 5% of the world income, or that would be costly to the respondents. Global inequality is a vote-determining issue for many people; a political program is more likely to be preferred if it addresses it. Support for international policies decreases only slightly when the country coverage shrinks. These results confirm previous findings and suggest that a broad set of countries could work together for sustainable development. 

\end{abstract}

% \textbf{JEL codes:} P48, Q58, H23, Q54 % NCCcomment
% Q54 Climate • Natural Disasters and Their Management • Global Warming
% Q58 Government Policy (Q is Environmental econ)
% D78 Positive Analysis of Policy Formulation and Implementation
% H23 Externalities • Redistributive Effects • Environmental Taxes and Subsidies (H is public econ)
% P48 Political Economy • Legal Institutions • Property Rights • Natural Resources • Energy • Environment • Regional Studies (P4 is Other economic systems)
% H41 Public Goods
% H54 Infrastructures • Other Public Investment and Capital Stock

% \textbf{Keywords:} Climate change, global policies, cap-and-trade, attitudes, survey.%, inequality, wealth tax. % NCCcomment

\clearpage
\tableofcontents

\onehalfspacing % NCCcomment

%\clearpage
\begin{bibunit}

\section{Introduction}\label{sec:intro}% AER

The development gap between the Global North and the Global South is central in 
international relations. %, with the question of North-to-South transfers of resources and power at the forefront. 
The issue of North-to-South transfers of resources and power 
permeates negotiations in many areas, including debt restructuring, development assistance, tax cooperation, UN reform, and climate finance.\footnote{The (re)distribution of resources between countries is debated in different official fora, such as the G20, the OECD's Base Erosion and Profit Shifting project, the Conference on Financing for Development, the International Maritime Organization, the Global Solidarity Levies Taskforce, the UN Framework Convention on International Tax Cooperation, and the UN Framework Convention on Climate Change. Appendix~\ref{subsec:plausible_policies_sources} provides references on official initiatives for global redistribution.} 
In all international fora, Global South countries seek % North-to-South transfers of resources and power
a more equal world order. Indeed, redirecting just 1\% of high-income countries' output to low-income countries (LICs) would mechanically double their national income.\footnote{The GDP per capita of high-income countries (HICs) is \href{https://data.worldbank.org/indicator/NY.GDP.PCAP.PP.CD?contextual=default&end=2024&locations=EU-ZG-XD-XM-1W-IN-US-CD-BI-LU-CN&start=2024&view=bar}{28 times} greater than that of low-income countries (LICs) at Purchasing Power Parity (PPP) and \href{https://data.worldbank.org/indicator/NY.GDP.PCAP.CD?end=2024&locations=EU-ZG-XD-XM-1W-IN-US-CD-BI-LU-CN&start=2024&view=bar}{68 times} greater in nominal terms, from World Bank 2024 data. Given that 625 million people live in one of the 25 low-income countries %defined by GDP per capita at Purchasing Power Parity (PPP) below \$1,135 per year, 
and 1.42 billion people in high-income countries, %with a GDP per capita of over \$13,935.().
1\% of HICs' GDP corresponds to 60\% of LICs' GDP using PPP values and 153\% using nominal data.
}

Public attitudes in high-income countries (HICs) are key to understanding whether globally redistributive policies would be politically feasible. 
Recent large-scale surveys reveal worldwide 
support for a globally coordinated tax on billionaires \citep{cappelen_majority_2025}, % revision? remove from here as there is no international transfers in cappelen_majority_2025?
a democratic world government for global issues \citep{ghassim_who_2024}, climate action at the global (rather than national) level \citep{dechezlepretre_fighting_2025}, %increased foreign aid, 
as well as globally redistributive climate or tax policies \citep{fabre_majority_2025}. 
% Surveying the public in 40 countries, \cite{cappelen_majority_2025} document overwhelming support for a globally coordinated tax on billionaires, 

Despite strong stated support even in HICs, international redistribution is rarely discussed in public debates, let alone advocated % championed? embraced % 
by policymakers. % seldom embraced by policymakers or even activists in HICs. 
\cite{fabre_majority_2025} conduct survey experiments in the U.S. and four Western European countries to understand this mismatch, focusing on a ``Global Climate Scheme'' (GCS) costly to these countries. The authors reject hypotheses that support for the GCS might be overstated: they find no social desirability bias in a list experiment, 6 out of 10 respondents prefer a political program that includes the GCS over one that does not, and most respondents are willing to sign a petition in favor of the GCS. % or may vanish in case of a negative media cam*-+*paign. 
% TODO: more figures pro global redistr before instead of giving so prominently a counterargument here
While the authors find support reduced by 11 percentage points (p.p.) following a (fictitious) negative media campaign --- an effect size similar to the actual decrease in support for the ``Green New Deal'' after it was publicly debated \citep{gustafson_development_2019} --- a campaign effect of this magnitude would not generate majority opposition to policies favored by three-quarters of the population, such as a globally redistributive wealth tax. Therefore, \cite{fabre_majority_2025} conclude that support for global redistribution is genuine, %and their tests cannot fully explain why this topic is not more prominent in public debates. 
and another hypothesis is needed to explain the lack of prominence of this issue. A promising candidate is ``pluralistic ignorance'': the underestimation of public support. Indeed, pluralistic ignorance has been documented regarding %costly 
climate action \citep{andre_misperceived_2024,andre_globally_2024,mildenberger_beliefs_2019}, the billionaire tax \citep{cappelen_majority_2025}, and the GCS \citep{fabre_majority_2025}. 
Nevertheless, pluralistic ignorance has not prevented climate policies or a national wealth tax from entering public debates, suggesting that other mechanisms may be at play concerning global redistribution. 
% Main cites: Vlasceanu, -Ghassim, -Cappelen, Dannenberg 2010 (negotiators' equity pref), -Boneva (both), -Gustafson, -Enke, Fehr, Carattini, -Mildenberger, -Dechezlepretre, (Meilland, Beiser-McGrath 19,) Young-Brun, MacKay, Bechtel & Scheve, -Toews TODO!!
% Foreign affairs North/South divide, with question of transfers at the forefront . 1% of HIC GDP could double LICs. 
% Puzzle: majority support for global redistr yet seldom mentioned, let alone enacted; cite NHB, global>national for climate, Ghassim, Cappelen

In this paper, I conduct a pre-registered large-scale survey to examine whether global redistribution policies are robustly accepted and to investigate the reasons for their lack of prominence. I test several hypotheses. Surveying eleven HICs, I test whether international policies are accepted by majorities in countries considered conservative and not yet surveyed on this topic, such as Japan, Poland, Russia, Saudi Arabia, and Switzerland. Recognizing that some key countries would likely not participate if these policies were implemented, I test how acceptance is affected when a climate scheme or a wealth tax is international but not truly global. I also test for pluralistic ignorance. I analyze the salience of global redistribution, and whether it is a vote-determining issue. I test for ``warm glow'', whereby people delude themselves into supporting hypothetical policies in order to ease their conscience, notably by testing whether the support is only claimed for as long as the policies are deemed unlikely. Finally, I explore a variety of international policies, ranging from the plausible to the radical. % TODO: rewrite to problematize, see e.g. grant proposal
% Hypotheses: warm glow, salience, pluralistic ignorance, non-global or new countries

Throughout the paper, I make a distinction between \textit{support} and \textit{acceptance}. I use the term \textit{support} to refer to the absolute share of \textit{Somewhat} or \textit{Strong support} on Likert scales, and \textit{acceptance} to refer to relative support --- specifically, the support share among non-\textit{Indifferent} responses. Although binary (\textit{Yes}/\textit{No}) questions are typically worded in terms of ``support'', I generally report their results using the term \textit{acceptance}. This approach avoids mistaking passive consent for active support among respondents who could not choose a neutral option. 
% Throughout the paper, I use the term \textit{acceptance} to refer to positive responses to survey items phrased in terms of support. This better captures the idea of passive consent rather than active demand.
% TODO!! support or acceptance? =>
% Demand: active mobilization
% Preference: the policy is favored to alternatives (by a majority)
% Support: absolute support
% Acceptance: relative (= conditional) support. Also used for binary questions as acceptance would be closest in terms of result adding a neutral option. => I could also use support, sticking to question wording (pro: consistent with figures)
% Tolerance/Assent: non-opposition (support + indifference)
% Net support: support minus opposition
% => I stick to the above terminology for specific questions; when I talk about overall survey results I use acceptance rather than support

% TODO relative support => acceptance?
The results confirm earlier studies: I find majority acceptance in every country surveyed for almost all globally redistributive policies tested. Policies currently discussed in international negotiations are accepted by large majorities. The most supported policy is the 2\% tax on billionaire wealth proposed by \cite{zucman_blueprint_2024}, with 81\% acceptance in the pooled sample. Proposals such as debt relief for vulnerable countries, developed countries contributing 0.7\% of their GDP in foreign aid, an expansion of the UN Security Council, or the Bridgetown initiative (expanding sustainable investments at low interests rates in LICs) all garner at least 70\% acceptance overall. 

Radical proposals are also widely accepted. Majorities in every country agree that ``governments should actively cooperate to have all countries converge in terms of GDP per capita by the end of the century'', and that globally coordinated climate policies are preferable to the status quo, even if they entail completely electrifying cars by 2045 and doubling the prices of heating fuel, flights, and red meat. Overall, I find 64\% acceptance for a progressive income tax that would collect 5\% of world income from the global top~3\% to finance poverty reduction in the Global South, with tax rates ranging from 15\% above \$80,000 per year to 45\% above \$1 million. Relatedly, in an interactive task where respondents design their preferred global income redistribution, nearly half choose a redistribution that would make them poorer (versus less than 10\% choosing one that would make them richer). The average custom redistribution entails over 5\% of world income in transfers from the rich to the poor. 

Before respondents could infer the survey's topic, they had to complete a budget allocation task and a conjoint experiment. When asked to allocate the revenue from a %hypothetical 
global wealth tax among five spending items, 87\% of respondents allocate a positive amount to the global item (public services in LICs). This item receives an average preferred share of 17.5\% of the revenue, slightly below an equal split of 20\%. %This indicates that most people attach to sustainable development abroad a priority lower than for the average issue, but non-zero. 
This indicates that most people prioritize sustainable development abroad less than the average issue, but still consider it worthwhile. 

%  TODO country ranking?
While policies to address global inequality are widely accepted, they have low salience. Indeed, this topic is rarely mentioned in open-ended fields at the beginning of the survey, where respondents were asked to write about various considerations. Respondents' top concern is the cost of living, and their most frequent wish is for greater purchasing power. While inequality is most often regarded as the greatest injustice ---with some inconclusive indications that these responses relate to inequality at the global level--- global inequality almost never appears among issues respondents consider important but neglected in public debate. 

Despite its low salience, global redistribution may be a vote-determining issue, as %indicated by % revision
the conjoint experiment suggests. 
In this task, respondents express their preference between two political programs, each composed of policies randomly selected from those prominently debated in their country. When a program includes a \textit{global tax on millionaires with 30\% of the revenue funding LICs}, the likelihood of that program being preferred increases by 4~p.p., while \textit{cutting development aid} reduces it by 4~p.p.  A direct question confirms that some voters might change their vote intention if a candidate campaigned on sustainable development: 36\% of respondents report they would be more likely (versus 17\% less likely) to vote for a party if it participated in a global movement for climate action, taxes on millionaires, and poverty reduction in LICs. In a related question, 68\% of respondents (and 52\% of the 561 millionaires who responded) state they could actively participate in such a movement (either by signing a petition, attending a demonstration, going on strike, or donating to a strike fund). 

What if a sustainable development policy is international but not global? Acceptance decreases only slightly. In the case of a wealth tax with 30\% of revenue financing LICs, acceptance is reduced from 74\% to 68\% when the policy is implemented only by some countries (e.g. the EU, the UK, and Brazil) rather than all countries. Likewise, acceptance of an International Climate Scheme (ICS), defined as a cap-and-trade with equal per capita allocation of emissions rights, decreases from 68\% to 65\% when participating countries shrink from a group covering 72\% of world emissions to one covering 33\% of emissions.

I identify pluralistic ignorance through an incentivized question that asks respondents for their belief regarding support for the Global (version of the) Climate Scheme, either among their compatriots or in the U.S. In Japan and in European countries, there is majority support for the GCS, yet most people believe there is not. Overall, the median respondent underestimates support in their own country by 16~p.p. and non-American respondents underestimate support in the U.S. by 22~p.p. Pluralistic ignorance may be an important reason why global solidarity solutions are neglected.

To test whether support might drop if the prospect of global policies materializes (a form of \textit{warm glow}), I manipulate the belief that large international transfers are likely in the next fifteen years. More specifically, I inform a random half of the respondents that ``countries have agreed to demonstrate some
degree of solidarity in addressing global challenges'', providing diverse examples including the % recent % revision
adoption of a shipping levy at the International Maritime Organization that should partly finance LICs, developed countries' commitments to finance climate action in developing countries, and the study % consideration? exploration?
by the G20 of a coordinated % common % revision
tax on billionaires. The information treatment increases the belief that transfers are likely by 7~p.p. from a baseline of 33\%, and it %. If anything, it 
also \textit{increases} the share of global policies supported by 1~p.p. %, though this effect is not statistically significant ($p$-value = .12). 
An IV estimation shows that the share of policies supported causally increases by 18~p.p. when people believe that international transfers are likely, %; however,   the effect is not significant ($p$ = .11). 
% In an IV specification, the share of policies supported rises by 14~p.p. ($p$ = 0.11) when individuals believe that international transfers are likely%
% Furthermore, the IV estimate is consistent with the non-causal %effect estimated 
% by OLS estimate of 15~p.p. 
consistent with the non-causal effect estimated by OLS. 
In other words, I find no evidence of \textit{warm glow}. On the contrary, the effect tends to go in the opposite direction compared to the \textit{warm glow} hypothesis: if people believed that a global policy were likely, they would be more likely to support it (there is suggestive evidence of %which can be interpreted as
 a \textit{status quo} bias). 

Finally, I test respondents' broad values to verify their consistency with global redistribution. The majority of respondents agree that ``helping countries in need in the right thing to do''. However, only a minority is convinced that it is in HICs' long-term interest to do so, or that it is their historical responsibility. Similarly, there is no majority support for reparations for colonization and slavery in the pooled sample. These results suggest that support for global solidarity is driven by a sense of empathy and duty rather than guilt or interest. % TODO lire Cappelen 2025: pluralistic ignorance

Following the works of \cite{enke_moral_2020,cappelen_universalism_2025}, it is interesting to measure the degree of universalism. % TODO!! elaborate, read
I use a new question to measure universalism, asking respondents which group they advocate for when they vote. 45\% choose a universalist response (either ``Humans'' or ``Sentient beings (humans and animals)''), while 32\% opt for their fellow citizens. Using a variance decomposition, I find that universalism is a stronger predictor of policy attitudes than sociodemographic variables such as income, country or even vote choice, echoing the results of \cite{enke_universalism_2024}. % TODO for them it's not better than vote
Consistent with the cross-national differences observed in other measures, the share of universalists is lowest in Japan. % and the U.S. %, and highest in Saudi Arabia. 
Meanwhile, there is a majority of universalists in Europe, Saudi Arabia, and among left-wing voters. 
% Results: plausible as well as large changes accepted, low salience, lower-than-average but non-zero priority, vote-dermining (conjoint+movement), no warm glow, pluralistic ignorance, only slight decrease if non-global, duty main reason + linked to universalism

By studying in depth the support for global policies, this paper departs from the usual  %introduces a shift in the 
methodological approach of attitudinal surveys. In general, academic surveys focus on estimating effect sizes of some treatment on political attitudes, or identifying the socio-demographic factors and the beliefs that correlate with attitudes (e.g. \citealp{kuziemko_how_2015,douenne_yellow_2022,alesina_intergenerational_2018}). The magnitude of support for a given proposal is often deemed unsuitable for satisfactory estimation, %as problematic to estimate satisfactorily. %
% It is often argued that 
because such attitudes are viewed as weakly held, inconsistent, or unstable. 
The measure of support is usually left to non-academic pollsters, who rarely apply all academic best practices: transparency, representative sampling, neutral and precise question wording, comparison with existing literature, and the use of multiple questions and complementary methods to correctly interpret the results. Although estimating the extent of support is challenging, this question seems too important not to be addressed using scientific methods. 
Furthermore, \cite{ansolabehere_strength_2008} refute common perceptions regarding policy attitudes, showing that they are as stable and %this assess once measurement error has been reduced through the use of multiple measures,
nearly as predictive of vote choice as party identification.  
% Furthermore, %absent large scale measurements of public opinion like referenda, 
% representative surveys remain the best method to assess support or opposition to given policies. 
In this paper, I examine support for various policies, approach the question from diverse angles, and run a battery of pre-registered tests to check % assess, ascertain, confirm, corroborate, validate
the reliability of stated support estimates.

\paragraph{Related literature.} 
Previous cross-country surveys consistently find strong public support for globally redistributive policies \citep{cappelen_majority_2025,fabre_majority_2025} or global democratic governance \citep{ghassim_who_2024}. % re-cite universalism?

Although it focuses on multilateral policies, the paper relates to the literature on attitudes toward foreign aid. \cite{kaufmann_foreign_2019} and \cite{fabre_majority_2025} find, that despite an important overestimation of the amount of aid, desired aid is larger than perceived aid in most countries. \cite{hudson_mile_2012} provide a critical review of the literature and show that the strong support for poverty alleviation largely stems from intrinsic altruism, in line with Eurobarometer data \citep{cho_ideology_2024}. 

\cite{nair_misperceptions_2018} finds that US-Americans underestimate their rank by 27 percentiles on average and overestimate the global median income by a factor 10, lowering their support for foreign aid. Likewise, \cite{fehr_your_2022} finds that 9 out of 10 Germans want global redistribution, even though respondents underestimate their relative position in the global income distribution by an average of 15 percentiles. %, although they do not misperceive their rank in the national distribution on average. 

Finally, the paper contributes to strands of literature which analyzes dispositions towards free-riding on climate action and how support for climate agreements depends on their country coverage. 
Using conjoint analyses in Western countries, 
\cite{bechtel_mass_2013} and % AER revision
\cite{beiser-mcgrath_could_2019} show that a climate agreement with larger country coverage are %15 p.p. 
more likely to be preferred. In Germany and the U.S., \cite{gampfer_obtaining_2014} also find stronger support for funding climate action in low-income countries when cost is shared with other countries. Nevertheless, surveys consistently show that people support unilateral climate action in their country, even in the absence of such action in other countries \citep{bernauer_how_2015,mcgrath_how_2017,beiser-mcgrath_commitment_2019%,sivonen_attitudes_2022
}. \cite{aklin_prisoners_2020} show that the empirical evidence for free-riding is not compelling, and that climate inaction can be equally well explained by distributive conflicts. Still, survey evidence indicates some degree of conditional cooperation: support for domestic climate action increases if other countries join forces \citep{carlsson_importance_2025}. 
% TODO? cite fabre_majority_2025 as literature review

% \cite{dannenberg_equity_2010} elicit climate negotiators' equity preferences and find that regional differences in addressing climate change are driven more by national interests than by different equity concerns. 

% TODO
% literature:
% By employing a theoretical model and examining correlations between lobbying and actual aid (controling for desired aid), they argue that the gap between actual and desired aid stems from the political influence of the rich who defend their vested interests. \cite{gilens_political_2001} shows that even Americans with high political knowledge misperceive actual aid, and finds that 17\% fewer of them want to cut aid when we provide them specific information about the amount of aid. % same for Hurst et al
% \cite{hudson_mile_2012} provide a critical review of the literature and show that the strong support for poverty alleviation largely stems from intrinsic altruism. \cite{bauhr_does_2013} find that support for aid is reduced by the perception of corruption in recipient countries. However, this effect is mitigated by the aid-corruption paradox: countries with higher levels of corruption often need more help. %TD: `` most corrupt countries need more help.'' does not sound right to me, because ``most->more'' should be ``more->more''.
% Using a 2002 Gallup survey and the 2006 World Values Survey, and in line with \cite{heinrich_voters_2018} in the U.S., \cite{paxton_individual_2012} show that the main determinants for wanting more aid are trust, left-wing ideology, interest in politics, and being a woman (all positively associated). 

% foreign aid: revision?
% 0 attitudes global pol: Carattini 19 AER, Fehr,
% 0+3.1 unilateral/free-riding: Bernauer & Gampfer, Carlsson, Aklin, 
% 0+3.1 country coverage: Sivonen, Gampfer, Bechtel & Scheve, Beiser & Bernauer Could revenue, Beiser & Bernauer Commitment, 
% 0+3.1 pluralistic ignorance: Mildenberger & Tingley 19, Bursztyn
% 0+4.1 conjoint: Bergquist 20, 
% 0 review attitudes: Carattini 18, 
% 0? behavior interventions: Vlasceanu, 
% 0+5.2 universalism: Cappelen (global rates and correlates), Enke 20 (linked with Dems), Enke 24 (stronger predictor than sociodemographics)
% 6 conservative bias: Schleich 28-13% find their personal position well represented at international climate negotiations
% 6 negotiators defend country interest: Dannenberg 2010
% 3.1 tax & dividend: Mildenberger 22
% 3.1 or App sources, global price: MacKay, young-brun

% Another methodological novelty is the interactive...
% Method: interactivity, possibility to measure preferences

% Questions that do not prime global solidarity show that this topic is not a salient concern for most people, which may explain why global redistribution proposals receives little attention in public debate, despite widespread acceptance. 

% Paragraphs in Kuziemko et al. (many citations are in footnotes):
% Context; Puzzle (citing theory/paper/evidence + graph); Hypotheses/method; Original method (interactivity); Main results; Specific result; Mechanisms/robustness (at length); Literature; Outline.
% Data: representativeness, attrition; Results; Robustness; Conclusion

% Say multiple sources were used to construct the questions (PIP, WID, Climate models, wealth tax simulator, etc.)

% \paragraph{Literature.} 
% Conjoint: Example use: political candidates (e.g., Teele, Kalla, and Rosenbluth, 2018), immigrants (e.g., Hainmueller and Hopkins, 2015), and public policies (e.g., Ballard-Rosa, Martin, and Scheve, 2017), Hainmueller, Hopkins, and Yamamoto, 2014).
% Fields: Mondon corroborates (using Eurobarometer) than Immigration was the top concern in the UK over 2009-2019 but only #14 concern for "you, personally" (with cost of living being #1) 

\section{Data and Design\label{sec:data}}

\paragraph{Samples.}
I conducted an original survey of 12,001 respondents representative of the adult population in eleven high-income countries (see Figure \ref{fig:country_coverage}). The countries were chosen to span the diversity of high-income countries and the sample sizes to be commensurate with each country's population size.\footnote{The sample sizes are as follows: U.S.: 3,000; Japan: 2,000; Russia: 1,001; Saudi Arabia: 1,000; Europe: 5,000, split in proportion to the countries' adult population sizes (except for Switzerland), i.e. France: 798; Germany: 1,048; Italy: 756; Spain: 603; Poland: 500; Switzerland: 469. The maximum margins of error (at the 5\% threshold) for country samples range from $\pm$1.8 p.p. in the U.S. to $\pm$4.5 p.p. in Switzerland, with an intermediate value of $\pm$3.1 p.p. in Saudi Arabia. } 
The survey was fielded online in 2025 using the companies \textit{Yandex} (for Russia), \textit{Kantar} (for Saudi Arabia), and \textit{Bilendi} (for the other countries).\footnote{For all countries except Russia, responses were collected between April 15 and July 3, 2025. For Russia, responses were collected between September 19 and October 9, 2025. Each complete response was rewarded with approximately \euro{}3 in gift points.} 

In Russia, I could not administer the same questionnaire as in the other countries.\footnote{To the best of my knowledge, \cite{toews_learning_2025} were the first to manage surveying the Russian public on climate attitudes.} I had to curtail it for two reasons. First, I could not use the platform \textit{Qualtrics}, which prevented me from using certain question formats (such as constant sum scales) or embedding Javascript (used to design the interactive question). Second, I had to cut or reword some questions due to preventive censorship by the survey company. In the other countries, the questionnaires are almost identical, though the figures in the questions are adapted to the country-specific context (e.g. respondents are informed about the cost of the Global Climate Scheme to the average person in their country). Appendix \ref{subsec:country_features} lists the unique features of the questionnaire in each country.

% # 1? coverage map
\begin{figure}[h!]
    \caption[Country coverage]{Country coverage of the survey.
    }\label{fig:country_coverage}
    \makebox[\textwidth][c]{\includegraphics[width=\textwidth]{../figures/maps_participation/country_coverage_curtailed.png}} 
\end{figure}

\paragraph{Representativeness.}\label{par:representativeness}
The samples are stratified to be representative of the country's adult population based on the following quota variables (with some exceptions\footnote{In the U.S., I also use race (4 categories) as a quota variable. In Saudi Arabia, I do not use urbanicity, but I use citizenship (Saudi vs. non-Saudi). In Russia, I do not use region nor urbanicity.}): 
gender, age (5 brackets), income (4), diploma (3), region (2 to 5), and urbanicity (2 to 3). Tables \ref{tab:representativeness_0}-\ref{tab:representativeness_3} in Appendix \ref{app:representativeness} show that the samples closely match the actual population frequencies along these dimensions, except for Russia and Saudi Arabia, where individuals without a high school diploma are somewhat underrepresented, as well as low-income individuals in Russia and non-Saudis in Saudi Arabia. All results are reweighted to be fully representative of the population along the quotas, with weights trimmed between 0.25 and 4. Results aggregated at the global or European levels weigh each country in proportion to its adult population size. Descriptive results on a random branch use weights that are recomputed within that subsample.

As shown in Figure \ref{fig:lmg}, sociodemographic variables explain 10\% to 15\% of the variance in our main attitudinal outcomes, and this figure drops to 5\% after accounting for country and vote. In other words, although variables such as age and diploma are significantly correlated with attitudes (see Tables \ref{tab:determinant}-\ref{tab:determinants_custom_redistr}), differences in average acceptance of a policy between (say) age groups rarely exceed a dozen percentage points. In contrast, our measure of universalism is a stronger predictor than any sociodemographic variable. % TODO? move to subsec:debated ?

Appendix \ref{app:pol} shows how our main attitudinal outcomes vary by political leaning. Non-voters exhibit attitudes close to the center of the political spectrum. Besides, attitudes are much less polarized in Japan compared to Europe and the U.S. 
Figures \ref{fig:vote_pnr_out}-\ref{fig:vote_representativeness} show how our weighted samples compare to actual voting results in the most recent election. 
Although the proportion of self-reported non-voters is lower than in reality, voting patterns across the three main political leanings are similar to the actual distribution. 
Appendix \ref{app:vote} shows that our main results are robust to reweighting by vote. 

\paragraph{Data Quality.}\label{par:data_quality}
The median survey duration is 17 minutes (13 min in Russia). 
Best practices have been implemented  to ensure top-notch data quality \citep{stantcheva_how_2023}. 
The questionnaire was carefully worded in a neutral and informative way;\footnote{At the end of the survey, 70\% of the respondents find it politically unbiased (Figure \ref{fig:survey_biased}). The most common comment left by respondents in the feedback field is that the survey was ``interesting''; very few criticize it (Figure~\ref{fig:comment_field}).} tested on random people in public spaces to ensure correct comprehension; translated by professional translators, with figures converted into national currencies; and double-checked by native speakers.

Of all respondents who started the questionnaire, 23\% respondents were allowed to continue (as their quotas were not full). Among them, 17\% dropped out (including 10\% who dropped out after the socio-demographic questions). The final sample is obtained after excluding 16\% of respondents from the extended sample for suspicion of low quality: 9\% for failing an attention test and 13\% for completing the questionnaire in less than 6 minutes\footnote{6 minutes corresponds to 30\% of the expected median duration of 20 min. In Russia, the cutoff is 200 seconds, or 30\% of the expected median duration of 11 minutes.} (including 5\% for both reasons). 
Appendix \ref{app:attrition} checks for differential attrition and Appendix \ref{app:extended} shows that our main results replicate in the extended sample. 

The order of question items is randomized whenever possible. Appendix \ref{app:order} examines the effect of item order on answers. Item order generally has a significant but small effect (2 to 14~p.p.). The size of this effect helps identify questions for which opinions are strongly held (e.g. a preference for a sustainable scenario over the status quo) versus weakly held (e.g. the preferred amount of climate finance). % crystallized, well-formed, consolidated, stable, ambivalent

Appendix \ref{app:comparison} compares the responses to two attitudinal questions borrowed from other surveys. The overall averages differ by 2 to 4~p.p. and the cross-country correlations are high: .70 \citep{global_nation_global_2023} to .86 \citep{cappelen_majority_2025}. 

\paragraph{Incentives.}
The questionnaire includes three incentivized questions, each awarding a \$100 prize to one randomly selected winner. First, a comprehension question about the Global Climate Scheme (GCS) checks whether respondents understand the policy's cost. 
Second, a donation lottery allows respondents to choose what portion of the prize they would donate to a reforestation NGO, should they win. Third, a question assesses respondents' perception of the actual support for the GCS, rewarding a correct guess.

\paragraph{Survey Structure.}\label{par:flow}
While Appendix \ref{app:questionnaire} provides the full questionnaire, Figure \ref{fig:flow} depicts the survey flow with all random branches. The various treatments are independent and uniformly distributed. Whenever there is a treatment, the acceptance rates reported are computed using the control group. Appendix \ref{app:placebo} runs placebo tests to check if earlier treatments affect unrelated outcomes. % TODO in app:placebo, comment on effect size that overall support can be at most 1pp higher/lower due to unrelated treatments

% # 2. survey_flow
\begin{figure}[h!]
    \caption[Survey flow]{Survey flow.
    }\label{fig:flow}
    \makebox[\textwidth][c]{\includegraphics[width=\textwidth]{../figures/questionnaire/survey_flow.pdf}} 
\end{figure}

After collecting sociodemographic characteristics, the questionnaire begins with broad questions to assess the prioritization and salience of global solidarity before respondents become aware of the survey topic. First, respondents answer open-ended fields on either their main concerns, wants, issues of interest, or perceived injustices. Second, they complete a conjoint experiment where they have to select their preferred political program, or abstain. Both programs are randomly generated: each policy (or lack thereof) in five policy domains is selected at random from a pool of policies that are prominent in the country's public debate. Third, respondents allocate the revenue of a global wealth tax among five (national or global) spending items. 

Then follow attitudinal questions about the main policies studied: a \textit{Climate Scheme} at the national, global, or international level; an international wealth tax funding low-income countries; and ten plausible global solidarity policies. These questions include treatments that vary the international coverage of policies or test for warm glow. 

The final part of the questionnaire explores attitudes towards more radical global redistribution scenarios and includes more sophisticated questions, such as an interactive task in which respondents can choose their preferred custom redistribution of global incomes by manipulating sliders. 

The survey concludes with a comprehension question, synthetic questions (e.g. regarding one's moral circle), and a feedback field.

% \paragraph{Acceptance versus Support.} 

% Throughout the paper, I make a distinction between \textit{support} and \textit{acceptance}. I use the term \textit{support} to refer to the absolute share of \textit{Somewhat} or \textit{Strong support} on Likert scales, and \textit{acceptance} to refer to relative support --- specifically, the support share among non-\textit{Indifferent} responses. Although binary (\textit{Yes}/\textit{No}) questions are typically worded in terms of ``support'', I generally report their results using the term \textit{acceptance}. This approach avoids mistaking passive consent for active support among respondents who could not choose a neutral option.

\paragraph{Pre-registered Hypotheses and Data Availability.}
The project has been approved by the CIRED institutional review board (IRB-CIRED-2025-2) and %is approved by IRB at Harvard University (IRB21-0137), and 
preregistered in the Open Science Foundation registry (\href{https://osf.io/7mzn4}{osf.io/7mzn4}). The study did not deviate from the registration; the questionnaires and hypotheses tests used are the ones \href{https://osf.io/j5scn}{specified \textit{ex ante}}. All data, code, and figures from the paper are available at \href{https://github.com/bixiou/robustness_global_redistr}{github.com/bixiou/robustness\_global\_redistr}. 


\section{Salience and Prioritization of Global Solidarity\label{sec:salience}}
% - People view global poverty/ineq as big injustice though not salient concern (e.g. revenue_split)
In this section, I analyze the salience of global solidarity in undirected open-ended fields, and the prioritization of global programs in a budget allocation task.

\subsection{Top-of-mind Considerations}\label{subsec:considerations}

% TODO: check https://www.oecd.org/en/data/tools/oecd-better-life-index.html, literature on issue/concerns/wishes
At the beginning of the survey, respondents are randomly assigned one of four open-ended questions: their main concerns, their needs or wishes, an issue important to them but neglected in public debate, or the greatest injustice of all. The questions are deliberately broad and vague to let respondents express their top-of-mind considerations without any priming. 

To analyze the answers, I automatically translated each field into English.\footnote{I used \href{https://www.onlinedoctranslator.com/en/translationform}{onlinedoctranslator.com}, which is powered by \textit{Google Translate}.} 
Then, I used AI and my own reading of a few hundred answers to identify the most common concepts, from which I selected 27 categories. Next, I classified each answer into one or more of these categories, both manually (Figures \ref{fig:field_manual}-\ref{fig:injustice_field}) and automatically using AI (Figure \ref{fig:field_gpt}). Finally, I manually defined a list of 47 (conjunctions of) keywords and used it to automatically classify all responses.\footnote{The list of keywords is provided in Appendix \ref{subsec:keywords}.} Figure \ref{fig:field_keyword} reports the 24 most common keyword matches. 

The three different classification methods yield consistent results but differ in accuracy. While the keyword classification allows for an exact and reproducible search, the AI search is not limited to specific words and captures more matching responses. % Although the manual classification could be less consistent than the AI one (if e.g. the interpretation of the category changes over time or among coders), 
Overall, manual classification seems to provide the most accurate results, with a number of matches generally falling between those of the other two methods. 
For example, to the \textit{injustice} question, 1.2\% of answers match the keywords for \textit{global inequality} and ChatGPT identifies this category in 7.5\% of answers, versus 3.2\% according to my manual coding.\footnote{The keyword matching searches the regular expression ``global poverty|global inequal|hunger|drinking water|starv'', ignoring case. The automatic and manual classifications are based on the category definition ``Inequality at the inetrnational % inetrnational AER revision
level / Hunger or poverty in poor countries''.} 
Indeed, the AI incorrectly classifies unspecific answers like ``poverty'' in this category,\footnote{Interestingly, out of the 47  (one-word) answers ``poverty'', ChatGPT-4.1 only coded 42 of them as \textit{global inequality}, illustrating the lack of consistency of this classifier.} while the keyword search misses answers like ``inequality among humans''. 
Given this observation, I use the manual classification as the benchmark and the two other methods as robustness checks.

While less accurate than the classifications, word clouds (Figure \ref{fig:wordcloud}) provide a simple visualization of the most common concepts in each question. By far, the most frequent \textit{concerns} or \textit{wishes} of respondents relate to their purchasing power, with concepts such as ``money'', ``inflation'', the ``cost of living'', or ``financial stability'' appearing in 31\% of these fields. Within countries, the share of people concerned with money decreases with income: it ranges from 22\% in the top income decile to 35\% in the bottom one.\footnote{At the country level, the concern for money is significantly correlated with inequality (an additional point in the Gini index is associated with 0.8~p.p. more respondents concerned with money). %(a Gini index higher by 10 points is associated with 8~p.p. more respondents concerned with money). 
Interestingly, the concern for money is higher in richer countries, though the correlation vanishes once one controls for the Gini.} 
The next most frequent \textit{concerns} are health (or the healthcare system, 13\%), far-right governments (or related concepts such as ``Trump'' or ``trade tariffs, 10\%) and war (either in general or specific conflicts, such as the Gaza War, 9\%). Most \textit{wishes} are personal, with the next most frequent (after money) relating to one's own or one's relatives' health (21\%) or peace of mind (10\%). Interestingly, almost none of the responses mention relational considerations, such as love, friendships, loneliness, intimate life, or the desire to have children (except in Saudi Arabia, where the latter was mentioned). % shorten by cuttin the end?
Further research is needed to determine whether the predominance of materialistic considerations stems from the context (an impersonal survey) or truly reflects people's primary thoughts. % TODO check literature

% # 3. keywords in fields (taken jointly)
\begin{figure}[h!]
  \caption[Word cloud of open-ended field, per variant]{Most common concepts in open-ended fields. (Questions \ref{q:concerns_field}-\ref{q:injustice_field})} \label{fig:wordcloud} % Random answers can be found on \href{http://preferences-pol.fr/fields2025.html}{bit.ly/fields2025}
  \begin{subfigure}{.48\textwidth}
    \caption[]{``What are your main concerns these days?''}
    \includegraphics[height=.35\textheight]{../figures/all/concerns_field_en.pdf}
  \end{subfigure} \quad
  \begin{subfigure}{.48\textwidth}
    \caption[]{``What are your needs or wishes?''}
    \includegraphics[height=.35\textheight]{../figures/all/wish_field_en.pdf}
  \end{subfigure}
  \begin{subfigure}{.48\textwidth}
    \caption[]{``What according to you is the greatest injustice of all?''}
    \includegraphics[height=.35\textheight]{../figures/all/injustice_field_en.pdf}
  \end{subfigure} \quad
  \begin{subfigure}{.48\textwidth}
    \caption[]{``Can you name an issue that is important to you but is neglected in the public debate?''}
    \includegraphics[height=.35\textheight]{../figures/all/issue_field_en.pdf}
  \end{subfigure} 
\end{figure}

Asked about the greatest \textit{injustice}, the most frequent answers relate to ``inequality'' or ``poverty'', with 19\% of occurrences (28\% in Europe but only 9\% in the U.S.). It is unclear whether these respondents are thinking about inequality in their own country or at the global level, since only 11\% of them specify a geographical scope. One clue is that 2\% mention their own country versus 10\% the global level (or Global South issues such as ``clean water'' or ``starvation''). Italians, Poles, and Spaniards are the most likely to mention ``global inequality'' or ``global poverty'', while Japanese and Russian respondents are the least likely to do so. 
The next most common answers relate to ``discrimination'' (based on gender, race, or sexual orientation, 9\%), ``violence'' or ``wrongful convictions'' (many respondents denounce the unjust sentencing of innocents, 9\%), or their country's ``welfare state'' (with people criticizing either the lack of public services or the excessive welfare given to undeserving people, 8\%). 

Asking people about ``an issue important to them but neglected in the public debate'' fails to uncover unusual topics. 21\% of respondents are unable to identify such an issue. %Although ``immigration'' is the most frequent word according to the word cloud, only 5\% of respondents refer to this issue, which comes after
The most frequently mentioned concepts are ``public services'' (12\%), the ``cost of living'' (10\%), ``health'' (9\%), ``ageing'' (6\%), and the ``environment'' (6\%).\footnote{Although ``immigration'' is one of the most frequent words according to the word cloud, the issue is only mentioned in 5\% of cases.} 
The fact that the most frequently mentioned topics are already well-publicized suggests that public debate reflects or shapes what people have in mind. % ``taxes'' or the ``welfare State'' (12\%), the ``healthcare system'' (9\%), the ``cost of living'' (9\%), and the ``environment'' (6\%).

Reading and coding each field one by one took about 30 hours, but it was worthwhile: not only does it result in an arguably more accurate classification; it also provided first-hand insight into how people think. For example, most people reason from their own perspective (e.g. ``my pension is too low'', ``I want to buy a house'') and do not refer to the broader picture or to political reforms. To get a sense of people's own words, a random display of responses can be found at \href{http://preferences-pol.fr/fields2025.html}{bit.ly/fields2025}.

% Interestingly, the topics vary significantly across countries. Here are my impressions of each country's slant. Compared to other countries, the concepts overrepresented in each country are as follows:
% \begin{itemize}
%   \item France: insecurity, holidays or free time, the public deficit, equality, gender equality;
%   \item Germany: old age poverty, immigration, the return of growth or the economic situation, free time, war (in Europe), bureaucracy;
%   \item Italy: health, serenity or peace of mind, war, work stress and free time, world hunger, femicides; 
%   \item Poland: war, inequality, holidays, honesty, disabled people;
%   \item Spain: ``health, money and love'', housing, corruption, water access, global poverty, squatters;
%   \item the UK: the cost of living, immigration, having a comfortable life, mental health, the holocaust, roads dangerous for driving, being unjustly imprisoned, cut in winter fuel allowance;
%   \item Switzerland: equality, immigration, gender equality;
%   \item Japan: the level of pensions, a cut on the consumption tax, the price of rice, the declining birth rate, childcare support, reducing the number of parliament members, foreigners' preferential treatment, excessive social assistance or the lack of reward for hard work, stock prices;
%   \item Russia: metaphysical questions or profound interrogations, ``lies'', buying a house or a car, traveling, wanting to live;
%   \item Saudi Arabia: hobbies such as sports or soccer, the willingness to become millionaire (or billionaire), one's business project, buying a house, their car, that they are satisfied with their income, ``self-injustice'' or sin, raising children, Palestine, orphan's oppression, travels;
%   \item the U.S.: the economy, Trump, breaches to the Constitution, abortion, and gun control.
% \end{itemize}
The topics mentioned vary according to sociodemographic characteristics. For example, a respondent who mentions \textit{immigration} is 3.5 times more likely to vote for the far right (correlation of .17); one who mentions \textit{old age} is twice as likely to be 65 or older (correlation of .13). Beyond these examples, the strongest effects I find are that \textit{criticizing the far right} correlates with voting for the left (.16), mentioning \textit{health} with age (.11), \textit{employment} with being unemployed (.09), \textit{animals} with extendeding one's moral circle to sentient beings (.09), \textit{education} with being a student (.09), \textit{the environment} with voting for the left (.08), and \textit{money} with one's income ($-$.08).

Interestingly, the topics also vary significantly across countries. Below are my impressions of each country's slant. Compared to other countries, the concepts overrepresented in France are: insecurity, holidays or free time, the public deficit, equality, or gender equality; in Germany: old age poverty, immigration, the return of growth or the economic situation, free time, war (in Europe), and bureaucracy; in Italy: health, serenity or peace of mind, war, work stress and free time, world hunger, and femicides; in Poland: war, inequality, holidays, honesty, and disabled people; in Spain: ``health, money and love'', housing, corruption, water access, global poverty, and squatters; in the UK: the cost of living, immigration, having a comfortable life, mental health, the Holocaust, roads dangerous for driving, being unjustly imprisoned, and cuts to the winter fuel allowance; in Switzerland: equality, immigration, and gender equality; in Japan: the level of pensions, a cut to the consumption tax, the price of rice, the declining birth rate, childcare support, reducing the number of parliament members, foreigners' preferential treatment, excessive social assistance or the lack of reward for hard work, and stock prices; in Russia: metaphysical questions or profound interrogations, ``lies'', buying a house or a car, traveling, the desire to live; in Saudi Arabia: hobbies such as sports or soccer, the willingness to become millionaire (or billionaire), one's business project, buying a house, one's car, satisfaction with one's income, ``self-injustice'' or sin, raising children, Palestine, the oppression of orphans, and travel; in the U.S.: the economy, Donald Trump, breaches of the Constitution, abortion, and gun control. % TODO? shorten?

% Our topic of interest, \textit{global inequality}, does not emerge as an issue salient to most people. It is rarely mentioned as a neglected \textit{issue} or as a \textit{concern}, contrary to international issues such as war, climate change, or the rise of the far-right. And while \textit{global inequality} is mentioned as frequently as these other international issues in terms of \textit{injustice}, addressing it seldom appears as a \textit{wish}. Indeed, most considerations relate to issues that directly affect oneself or one's family, and political considerations (regarding e.g. public services, pensions or taxes) are often framed at the national level. %are self- or family-centered.
Our topic of interest, \textit{global inequality}, does not emerge as an issue salient to most people. Indeed, most considerations relate to issues that directly affect oneself or one's family, and political considerations (regarding e.g. public services, pensions, or taxes) are often framed at the national level. \textit{Global redistribution} almost never appears as a \textit{wish}. Furthermore, \textit{global inequality} is rarely mentioned as a neglected \textit{issue} or as a \textit{concern}, in contrast to international issues such as war, climate change, or the rise of the far right. However, it is mentioned as frequently as these other international issues in terms of \textit{injustice}. 

In summary, the low salience of global solidarity may explain why this topic fails to mobilize political forces, despite being referred to as a just cause and it being accepted by majorities (as shown below). 
% TODO far right or far-right

\subsection{Prioritization of Public Spending Items}\label{subsec:revenue_split}

% TODO? shorten/improve flow?
% revision: American -> US-American TODO AER
\cite{fabre_majority_2025} find that 58\% of US-Americans and 71\% of Western Europeans would support a global tax on millionaires funding low-income countries (LICs), with only 26\% and 14\% opposing it, respectively; around half of them would prefer to allocate half (rather than none) of the revenue from a global wealth tax to LICs; and, on average, respondents prefer to allocate 33.4\% of the revenue to LICs (versus domestic healthcare and education). 
% \cite{fabre_majority_2025} uses two random variants to study the preferred use of revenue from a global wealth tax between infrastructure and public services \textit{in
% low-income countries} versus \textit{domestic} healthcare and education. The ``binary'' variant shows that around half of US-Americans and Western Europeans prefer to allocate half (rather than none) of the revenue to low-income countries. The ``continuous'' variant shows that the median respondent prefers to allocate 30\% of the revenue to low-income countries (LICs).
In other words, while most people would prefer a globally redistributive tax to the status quo, the more leeway they are granted to allocate the revenue from such a tax, the less they would allocate to LICs. The greatest leeway tested by \cite{fabre_majority_2025} let the respondents select their preferred share for LICs versus one alternative spending item, and the average preference was 33.4\%---that is, 66.8\% of an equal split. % \footnote{The greatest leeway possible would be an open-ended field, but its interpretation would be challenging as answers that do not specify the geographical scope would be equivocal.}
Naturally, one expects respondents to split the revenue among all desirable spending items, so that the revenue allocated to one item depends on the number of items. Therefore, if LICs compete with not one but several national items, the share allocated to LICs is expected to diminish. 
If this share is less than 67\% of an equal split, it would mean that \citep{fabre_majority_2025} overestimated the prioritization of LICs, perhaps due to an excessive salience of LICs when only one alternative is proposed, or because the domestic alternative---healthcare and education---was not the most desired. % shorten by taking out this sentence? 
Conversely, if several items relate to a global issue, the total ``global spending'' should increase. 
If global spending increases by less than the proportionate amount for the number of global items, it could indicate that people view global items as a single entity and that some domestic issues were missing from the binary version. % shorten by taking out this sentence? 

To test whether the results in \citep{fabre_majority_2025} provide an accurate picture of the % relative
prioritization of global spending as well as to uncover the prioritization of different global causes, I conduct a revenue allocation task with five spending items. 
In each of the two variants of this task, respondents use sliders to allocate the revenue of a hypothetical global wealth tax (at a rate of 2\% on wealth in excess of \$5 million), after being informed of the revenue the tax would collect in their country (from \$1 billion in Poland to \$514 billion in the U.S.) versus in all LICs combined (\$1 billion). 

In the \textit{Few} variant, one global item (``Education, Healthcare and Renewable energy in LICs'') competes with four domestic ones. %(``Education and healthcare'', ``Social welfare programs'', ``Reduction in the income tax'', ``Reduction of the deficit''). 
In every country, the most prioritized item is ``Domestic: Education and healthcare'', with an average preferred share of 26\% (Figures \ref{fig:split_few_bars_nb0}, \ref{fig:split_few}). The global item is the least prioritized overall, at 17.5\% (from 14\% in Japan to 21\% in Saudi Arabia and Spain). However, global spending is the second most prioritized item in Europe (19\%) and Saudi Arabia. Furthermore, global spending is 31\% higher than the expected 13.4\% (that is, 66.8\% of % an equal split of 
20\%)\footnote{The one-sided test that global spending is equal to 33.4\% is rejected at the 1\% threshold in all countries except Japan, where it is rejected at the 10\% threshold. If one restricts the comparison to the countries surveyed by \citep{fabre_majority_2025}, the global item is allocated 17.8\%, which is 34\% more than expected. The most credible explanation for outperforming expectations is that the domestic item chosen by \citep{fabre_majority_2025} was the preferred one. Indeed, the global item is allocated 68\% of the ``Domestic: Healthcare and Education'' share, almost exactly as expected.} 
and only 13\% of respondents do not allocate any revenue to it (Figure \ref{fig:split}). 

%While these results reveal large acceptance of allocating a substantial share of the revenue from a global tax to LICs, this may result from framing the question with only two spending items (domestic versus LICs). 

% # 4. revenue_split: country_comparison/split_main_means_nolegend + country_comparison/split_main_nb0_nolabel
\begin{figure}[h!]
  \caption[Preferred split of revenue of an international wealth tax]{Preferred split of revenue from an international wealth tax. The first two items are from the \textit{Few} variant with 5 fixed items (the \textit{Global} one and the most preferred one are displayed); the last four items are from the \textit{Many} variant with 5 items taken at random out of 13 (the 4 \textit{Global} ones are displayed). \hfill (Questions \ref{q:revenue_split_few}-\ref{q:revenue_split_many})} \label{fig:split}

  % \begin{subfigure}{.45\textwidth}
  %   \caption[]{Average preferred allocation (in \%).}
  %   \begin{flushleft}
  %   \includegraphics[height=.36\textheight]{../figures/country_comparison/split_main_means_nolegend.pdf}\end{flushleft}
  % \end{subfigure} 
  % \begin{subfigure}{.55\textwidth}
  %   \caption[]{Share of respondents allocating 0 revenue (in \%).} 
  %   \begin{flushright}
  %   \includegraphics[height=.36\textheight]{../figures/country_comparison/split_main_nb0_nolabel.pdf}\end{flushright}

\begin{subfigure}{.62\textwidth}
    \caption[]{Average preferred allocation (in \%).}
    \includegraphics[height=.38\textheight]{../figures/country_comparison/split_main_means_nolegend.pdf}
    \end{subfigure} 
  \begin{subfigure}{.38\textwidth}
    \caption[]{Share of respondents allocating 0 revenue (in \%).} 
    \includegraphics[height=.38\textheight]{../figures/country_comparison/split_main_nb0_nolabel.pdf}

  \end{subfigure}
\end{figure}

In the \textit{Many} variant, five items are selected at random from a pool of four global and nine domestic items. While domestic healthcare (27\%) %(27.0\%) 
and education (22\%) %(22.5\%) 
are the most prioritized items, the average allocation for global items ranges from 16\% %16.5\% 
for ``Forestation and biodiversity projects'' to 19\% %18.6\% 
for ``Education and Healthcare in LICs'' (Figures \ref{fig:split}, \ref{fig:split_many}-\ref{fig:split_many_global_mean}). On average, tasks include 1.5 global items, which together receive 26.9\% of the revenue --- again, equivalent to 17.5\% per global item. Interestingly, there is no significant correlation between the number of global items and the average allocation per global item.\footnote{In Russia, the question could not be asked in the same way due to different software. Instead, respondents had to choose what share of the global tax revenue to allocate to sustainable development in low-income countries. On average, Russians allocate 12.2\% to LICs, with a median allocation of just 5\%, but only 12\% allocate nothing to LICs.} % TODO! table allocation per global item ~ number of global items

Overall, the revenue allocation tasks validate and confirm the findings from \citep{fabre_majority_2025}. Most people would favor using a substantial share of the revenue from a global wealth tax to finance sustainable development in LICs, even though global spending is somewhat less prioritized than domestic spending. 


\section{Acceptance of %International 
Policies as a Function of Country Coverage\label{sec:coverage}}
% - Strong support for global tax / GCS even with partial participation

While acceptance of global climate or redistributive policies is widespread \citep{fabre_majority_2025,cappelen_majority_2025}, acceptance may drop if policies are not truly \textit{global} but only \textit{international}, i.e. if key countries such as China, Russia, or the U.S. do not participate. Indeed, people may be concerned about a domestic loss of competitiveness resulting from the expatriation of taxpayers to low-tax jurisdictions; or about unfair burden-sharing if non-cooperating countries free-ride on decarbonization or sustainable development funding. % TODO! cite Gampfer, Carlsson, Bechtel Scheve 2013 May also depend on which countries are in
In this section, I examine the acceptance of globally redistributive policies depending on the coalition of countries that would implement them. I study, in turn, a carbon price and a wealth tax.

\subsection{International Climate Scheme}\label{subsec:ics}

% TODO!! cite Beiser-McGrath \& Bernauer (2019), Sivonen (2022): +15pp if all countries rather than just half in CN, US
% NB: I chose to present GCS on control. Trade-off between presenting control results (not contaminated by treatment though less precise) and overall results (more comparable to ICS as those mix treated and control, though less comparable to NCS). I prioritized presenting results for themselves rather than for comparison with ICS.

\paragraph{Presentation of the Schemes.} ``Cap and dividend'' is a reference climate policy \citep{grubb_greenhouse_1990,bertram_tradeable_1992,baer_equity_2000,barnes_creating_2008,blanchard_major_2021}, % TODO ,young-brun_within-country_2025 ? ,jamieson_climate_2001,rajan_global_2021 3554_economists_economists_2019
whereby fossil fuel companies at the source of emissions must buy emission permits on a carbon market, with the revenue from carbon pricing rebated equally to individuals. The limited and declining number of emission permits guarantees that emissions are capped according to the climate objective. As polluting companies pass the cost of emission permits down the value chain, the carbon price is ultimately paid by consumers, in proportion to their carbon footprint. Meanwhile, the equal cash transfer (or ``dividend'') offsets price increases for the average consumer. Those with a higher-than-average carbon footprint financially lose, while those with a lower carbon footprint (who are on average poorer) financially gain. 
% explanation cap & dividend

Using simple \textit{Yes}/\textit{No} questions, I test the acceptance of three types of ``cap and dividend'' (or ``Climate Scheme'') policies that differ by geographical scope: the National, Global, and International Climate Schemes (Figures \ref{fig:ics}, \ref{fig:ncs_gcs_ics}). While average consumers in a high-income country are financially unaffected by the National Climate Scheme (NCS), they lose out in the Global and International versions, since their carbon footprint is larger than the world (or climate coalition) average. % TODO? acceptance => support?

% One random third of the respondents are asked for their support of the National Climate Scheme (NCS). It % DONE put back? no
The National Climate Scheme (NCS) 
is accepted by 68\% of respondents (ranging from 56\% in Poland to 88\% in Saudi Arabia). 
% majority support NCS (TODO!!? cite Douenne & Fabre? Mildenberger?)

\paragraph{The Global Climate Scheme.} 
Before presenting the Global Climate Scheme (GCS), respondents are instructed to pay careful attention, with the incentive that they may win a \$100 lottery prize if they correctly answer a comprehension question at the end of the survey. 
When presented with the Global Climate Scheme (GCS), respondents are informed that the cash transfer would lift 600~million people out of extreme poverty, and the cost to them is made salient. Respondents are informed of the amount of the cash transfer, as well as the price increases and the net cost to the average person in their country (e.g. 2\% price increases and a net cost of \$90 per month in the U.S., or 2\% and \euro{}45 per month in Germany).%
\footnote{The computations use a carbon price of \$95/tCO$_\text{2}$. For Russia, Saudi Arabia, and the U.S., computations assume universal country coverage and the cash transfer is \$35 per month. 
For Europe and Japan, the net loss is computed in a non-universal but \textit{High} participation scenario, which implies a lower cash transfer (\euro{}20 per month) and a higher net cost (by about \$10 per month) since the coalition's average carbon footprint is lower than the world average.} 
The GCS is accepted by 55\% of respondents (from 49\% in the U.S. and Russia to 85\% in Saudi Arabia). The salience of costs in the GCS question may explain the somewhat lower acceptance of the GCS compared to the NCS.%
\footnote{Acceptance of the GCS is also around 10~p.p. lower than in \citep{fabre_majority_2025}. There may be different reasons for this: attitudes may have changed in the two-year interval; I added information on the price increases, which allows respondents to estimate the cost to themselves (rather than to their average fellow citizen)%; and the framing might have been slightly confusing in the earlier survey as the GCS was presented on the same page as a national redistribution scheme revision put back? 100 countries instead of 60%?
.} 

% is top3_tax 5.1% of world income or HIC income? => World, but overestimated (in reality it's 4.3%) due to "rounding" errors when interpolating from countries distrib to world distrib using too coarse an interpolation => give figures in PPP nominal (4.7%) terms + PPP (4.3%) in footnote
\paragraph{Pluralistic Ignorance.} 
After assessing support for the GCS, respondents are asked in an incentivized way about their belief concerning the actual support, either in their country or in the U.S. (Figure \ref{fig:ncs_gcs_ics}).\footnote{US-Americans are asked about either their country or the EU. In Russia, I was not permitted to enquire about beliefs in a foreign country.} 
In every country and for any variant of the question, actual support is underestimated. The median respondent underestimates the support in their own country by 16~p.p. and the support in the U.S. by 22~p.p. In Japan and in European countries, the underestimation is more severe, with most people wrongly believing that the GCS does not garner majority support in their country. Such pluralistic ignorance might explain why politicians do not dare to propose global climate justice policies. 

\paragraph{International Climate Scheme.}
To test how country coverage influences the acceptance of the International Climate Scheme (ICS), respondents are randomly assigned to one of four variants. They can visualize the country coverage on a map (see examples in Figure \ref{fig:ics_maps}), where their own country is striped to denote its potential participation. Respondents are also informed of the number of countries that would participate in the assigned scenario, the list of these countries or world regions, and their share of world emissions. 
% 5c.
\begin{figure}[h!]
\caption[Example maps of the International Climate Scheme]{Example maps of the International Climate Scheme question. (Question \ref{q:ics_support}).} \label{fig:ics_maps}
\begin{subfigure}{.49\textwidth}
  \caption[]{\textit{Low} variant for the U.S.}
  \includegraphics[height=.49\textwidth]{../figures/maps_participation/GCS_low_EN.png}
\end{subfigure} 
\begin{subfigure}{.49\textwidth}
  \caption[]{\textit{High color} variant for EU countries.}%\begin{flushright}
  \includegraphics[height=.49\textwidth]{../figures/maps_participation/GCS_high_color_EU.pdf}%\end{flushright}
\end{subfigure}
\end{figure}

The \textit{Mid} scenario covers 56\% of world emissions and includes China and Global South countries. The \textit{Low} scenario replaces China with the EU and covers 33\% of emissions. The \textit{High scenario} adds various high-income countries to the \textit{Mid} scenario, including the EU, Japan, Canada, and South Korea, and covers 72\% of emissions. The last variant, \textit{High color}, combines the \textit{High} participation scenario with a colored map that displays not only the country coverage, but also the net gain or cost for each country, with China appearing as neither gaining nor losing from the policy.\footnote{In a standard cap and dividend, China should lose, as its carbon footprint exceeds the world average. However, the Global Climate Scheme departs slightly from the standard policy so that middle-income countries do not lose out \citep{fabre_global_2025}.} 

% # 5a. ICS: \includegraphics[width=\textwidth]{../figures/country_comparison/variables_ncs_gcs_ics_by_country}
\begin{figure}[h!]
    \caption[Support for the NCS, GCS, ICS]{Percentage of support for the National, Global, and International Climate Schemes (\textit{Yes}/\textit{No} question). \hfill (Questions \ref{q:ncs_support}-\ref{q:ics_support}.)
    }\label{fig:ics}
    \makebox[\textwidth][c]{\includegraphics[width=\textwidth]{../figures/country_comparison/variables_ncs_gcs_ics_control_by_country.pdf}} 
\end{figure}

As expected, the wider the coverage, the higher the acceptance. However, this effect is relatively small, as acceptance is only 4~p.p. higher in the \textit{High} variant (at 68\%) compared to the \textit{Low} variant (65\%). Interestingly, acceptance among Europeans significantly increases when China is added to the coalition, but does not rise further 
when other HICs are also added. Conversely, for US-Americans and Japanese, the participation of the EU or China yields similar levels of acceptance, and only the combined participation of China, the EU, and other HICs significantly increases acceptance. 

The effect of country coverage is entirely driven by the 74\% of respondents who understand that the GCS would result in increased gasoline prices. It is worth noting that acceptance is higher among the minority of respondents who misunderstand the policy: by 5~p.p. for the GCS and 4~p.p. for the ICS. % gcs_support no longer majority in PL among understanders; but ics_high_color_support majority in every country among understanders

Acceptance is 6.6~p.p. lower in the \textit{High color} variant compared to the \textit{High} variant. Three reasons may explain this effect. First, the cost may be more salient with the colored map. %Indeed, the map appears to be informative even to the 74\% of respondents who understand that the GCS would result in increased gasoline prices, as the effect is just as high for them as for the other respondents. Second, some respondents may be concerned by the information (made explicit in the question) that China would neither gain nor lose from the policy. 
Third, with the colored map, respondents learn how their own country fares compared to others. In fact, the effect % TODO? (of lower support)
is %1~p.p. and 
no longer significant (and of opposite sign) for countries that appear to lose less than 0.5\% of their GDP (Spain and Switzerland). 
% Interestingly, acceptance of the GCS and of non-\textit{color} variants of the ICS are 5~p.p. lower for the 74\% of respondents who understand that the GCS would result in increased gasoline prices (after controlling for sociodemographics), but this effect is only 2~p.p. and no longer significant for the \textit{High color} ICS, indicating that the information provided by the map affects those who had not yet understood the policy's consequences. => no, the effect comes from understanders being more reactive to high country coverage.

Notice that acceptance of the ICS in the \textit{Low} coverage variant is similar to that of the NCS. This suggests that the average respondent is willing to pay the ICS's higher cost for the guarantee of poverty alleviation and decarbonization in the Global South. %This suggests that, for the average person, the guarantee of poverty alleviation and decarbonization in the Global South balances out the ICS's higher cost. 

Finally, the greater acceptance of the ICS compared to the GCS is somewhat puzzling. Perhaps people view the proposal as more credible when a list of participating countries is provided, compared to the GCS, which is framed as if all countries might join (or, on the contrary, as one in which the participation of any country is uncertain). Relatedly, acceptance may be stronger for more precise or more visual proposals, either because they are viewed as more advanced or because they induce an experimenter demand bias. The greater acceptance could also be due to costs being less salient in the ICS question (but acceptance is still greater than in the GCS in the \textit{High color} variant, where costs are visible). Unfortunately, the data does not allow testing these different hypotheses.

% support increases with coverage
% ICS low: similar to NCS: people willing to pay Global South for its decarbonization
% ICS mid, high: higher acceptance: people happier with less free-riding
% ICS high color: visualizing net gain stronger effect than country coverage
% why GCS < ICS? hypotheses: lower support for less concrete / more uncertain proposal, cost less salient, map induces more experimenter demand, people think it's more credible if it's about joining a club (rather than a GLOBAL cs), stronger effect of not-wanting-to-free-ride when see most countries are in

\subsection{Wealth Tax Funding LICs}\label{subsec:wealth_tax}

I test the effect of country coverage on the acceptance of an internationally redistributive wealth tax using a simple \textit{Yes}/\textit{No} question with three random variants. The policy is described as a 2\% tax on wealth above \$1~million, with 30\% of its revenue financing public services in LICs. In the \textit{Global} variant, all countries except the respondent's own are assumed to participate. The \textit{HIC} variant covers all HICs (except the respondent's country). The \textit{International} variant covers some countries and not others, with the precise coverage varying by respondent's country but always including Brazil and European countries (or the whole EU) and excluding China and the U.S.\footnote{More precisely, in the U.S., excluded countries differ and are \textit{China, Japan, and Canada}. As for included countries, in addition to \textit{Brazil}, they are: \textit{the EU and the UK} for Switzerland, Saudi Arabia, and the U.S.; \textit{the EU} for Russia and the UK; and \textit{France, Germany, Spain, and the UK (except one's own country)} for EU countries.}

Here again, acceptance increases with the country coverage, but the effect is small. The middle-ground \textit{HIC} variant garners 70\% acceptance (from 58\% in Switzerland to 81\% in Saudi Arabia). Compared to \textit{HIC}, acceptance is 5~p.p. higher with \textit{Global} coverage, while it is only 1.4~p.p. and non-significantly lower with \textit{International} coverage (Figure \ref{fig:wealth_tax}). 

% # 5b. wealth tax by coverage: \includegraphics[width=\textwidth]{../figures/country_comparison/variables_wealth_tax_support_by_country}
\begin{figure}[h!]
    \caption[Support for an international wealth tax depending on country coverage]{Percentage of support for an international wealth tax with 30\% of revenue funding LICs, depending on the country coverage (\textit{Yes}/\textit{No} question). \hfill (Questions \ref{q:global_tax_support}-\ref{q:intl_tax_support}).
    }\label{fig:wealth_tax}
    \makebox[\textwidth][c]{\includegraphics[width=.9\textwidth]{../figures/country_comparison/variables_wealth_tax_support_by_country.pdf}} 
\end{figure}

Overall, the results indicate that the acceptance of internationally redistributive policies is quite robust to country coverage. This suggests that the issues of competitiveness or free riding are not decisive factors in public support. % TODO!!? cite literature

\section{Sincerity of Support for Global Redistribution\label{sec:sincerity}}

Skeptics about the public's support for global redistribution would argue that this support is not reflected in real-stake decisions or that it mostly results from \textit{warm glow}. According to the \textit{warm glow} hypothesis, many people would express their support to enjoy moral comfort as long as the policy appears out of reach and supporting it seems harmless. In case of \textit{warm glow}, support would vanish if (\textit{i}) the prospect of implementation materialized or if (\textit{ii}) moral comfort could be obtained from a substitute. In this section, I test whether global redistribution is a vote-determining issue using a conjoint analysis, and I test both forms of \textit{warm glow} (\textit{i} and \textit{ii}) using two other survey experiments.
% - No (or little) evidence of warm glow or support due to unrealism; global redistr genuinely supported (conjoint analysis)

\subsection{Conjoint Analysis}\label{subsec:conjoint}

I conduct a conjoint experiment in all countries except Russia and Saudi Arabia. This question is positioned at the beginning of the survey, before respondents know the survey's topic. Respondents are presented with two random political programs, framed as the fictitious programs of the leading candidates in the next election, and are asked which candidate they would vote for (27\% of the respondents choose the outside option \textit{Neither of them}). Each program contains a policy or an absence of policy, chosen at random, for each of five policy domains (the order of which is also randomized). %: \textit{Economic issues}, \textit{Social issues}, \textit{Climate policy}, \textit{Tax system}, \textit{Foreign policy}. 
Our domain of interest is \textit{Foreign policy}, whose pool contains three policies: \textit{Cut development aid}, \textit{International tax on millionaires with 30\% financing healthcare and education in low-income countries}, and a country-specific policy. % (e.g. \textit{Support Ukraine militarily and financially} in Germany). 
I included both an \textit{anti}- and a \textit{pro}-global redistribution policy to capture a potential status quo bias or, on the contrary, a bias in favor of reform. 
The policies, except for these two of interest, have been selected from the programs of the main candidates in the country's most recent election, ensuring coverage of the entire political spectrum and the most prominent proposals in the national public debate. 

Figure \ref{fig:conjoint} shows the effect of including our policies of interest in a program on the likelihood that it is preferred (see Figures \ref{fig:conjoint_FR}-\ref{fig:conjoint_ES_original} for full country-by-country results\footnote{With a few exceptions, \textit{raising the minimum wage} is among the most popular policies, alongside \textit{redistributive taxes or transfers}, \textit{anti-immigration regulations}, and \textit{abortion rights}. 
Conversely, \textit{a ban on new combustion-engine cars} is among the least popular ones.}). 
% Why Japan millionaire significant in conjoint_JP but not in conjoint? => because in conjoint_JP the formula includes all domains vs. only foreign policy in conjoint. SEs are the same but the coef is lower in conjoint_JP. => In conjoint_JP, the coef is the marginal effect when every domain is "-", while in conjoint it's the marginal effect at the mean (compared to foreign at "-").
More specifically, Figure \ref{fig:conjoint} presents the results %estimates of coefficients $\beta_1$ and $\beta_2$ 
of the following regression, estimated by simple OLS with standard errors clustered by respondent: 
$$\text{Preferred}_{pi} = \beta_0 + \beta_1 \text{Cut\_aid}_{pi} + \beta_2 \text{Intl\_tax}_{pi} + \beta_3 \text{Foreign3}_{pi} + \varepsilon_{pi}$$
where $pi$ denotes the %either 
program $p$ faced by respondent $i$, and each variable is a dummy. 

% TODO! corresponding table; also table for other figures
% # 6. conjoint: foreign aid + global tax (per country + global)
\begin{figure}[h!]
\caption[Conjoint analysis: effect of development aid and millionaire tax]{Effect on the likelihood that a political program is preferred of containing the following policy (compared to no foreign policy in the program). No control is included, 95\% confidence intervals are shown. (See Figure \ref{fig:conjoint_vote} for effects by vote.) \hfill (Question \ref{q:conjoint})} \label{fig:conjoint}
\begin{subfigure}{.49\textwidth}
  \caption[]{Cut development aid}
  \includegraphics[height=.36\textheight]{../figures/country_comparison/program_preferred_by_cut_aid_in_program.pdf}
\end{subfigure} 
\begin{subfigure}{.49\textwidth}
  \caption[]{International tax on millionaires with 30\% financing health and education in low-income countries}%\begin{flushright}
  \includegraphics[height=.36\textheight]{../figures/country_comparison/program_preferred_by_millionaire_tax_in_program.pdf}%\end{flushright}
\end{subfigure}
\end{figure}
% TODO? include controls for consistency with warm glow?

Both policies significantly affect program choice: the internationally redistributive millionaire tax increases the likelihood that a program is preferred by 4~p.p., while cutting development aid decreases it by 4~p.p. At the country level, the effects are generally non-significant due to lack of power, but when significant, they are of the same sign as the global effect (except for \textit{cut aid} in Switzerland). Overall, the effects are of comparable size to the effects of other policies,\footnote{Overall, the average absolute effect size (weighted by countries' population sizes) is 6~p.p., and our effects of interest are not significantly lower than this average (at the 15\% threshold).} % \footnote{Compared to the average absolute effect size (weighted by countries' population sizes), our effects of interest are 31\% lower on (weighted) average, but the difference is not statistically significant.} %
suggesting that global redistribution issues may be as vote-determining as policies prominent in the national debate. 

One concern with this type of conjoint analysis is that it involves unrealistic political programs, namely programs that contain both left and far-right policies, which distorts the actual choices that voters may face. \cite{de_la_cuesta_improving_2022} showed that to fully address this issue, one should weigh each pair of programs by the probability that it would arise in a real election. Since this probability cannot be computed, the best practice is to bound the effects by estimating them with extreme probabilities. The results just presented are based on one extreme, the uniform distribution. To construct the other extreme, I classify each policy proposal according to its originating political party%as left, center, or right,%
\footnote{Interestingly, the most popular policies originate from left-wing parties, except in Germany and Switzerland. 
Indeed, the average deviation from the mean effect is highest for policies originating from the \textit{Left}, and lowest for those from the \textit{Center-right or Right}. 
%To study the popularity of policies as a function of the party that propose them, 
% In the main specification, I use a manual classification which assigns \textit{center} to policies actually supported by the left or far right when they seem compatible with any political leaning. 
% Interestingly, 
% % among the policies most ``popular'' in a given country, I have classified five of them as \textit{center} and four as \textit{left}. However, in reality, seven of these policies are proposed by a party from the \textit{left}, two from the \textit{far right}, and \textit{none} from the \textit{center}. Other analyses confirm that 
% policies originating from the \textit{left} are the most popular (except in Germany and Switzerland), even though I often classify them as \textit{center}. 
% Actually, policies originating from a \textit{center} party are the least popular, but policies I classify as \textit{center} are the most popular. % are overrepresented among most liked policies. %I classify as \textit{center} are overrepresented among most liked policies, but that considering the party they originate from, \textit{center} policies are \textit{under}represented among most liked policies and overrepresented among \textit{least} liked policies, while policies from the \textit{left} are overrepresented among most liked policies.
} 
and consider a program consistent if it does not contain policies from both the \textit{left} and the \textit{far right}. Then, I re-estimate the regression after dropping the 29\% of pairs with an inconsistent program, effectively assigning them a probability of zero. Effects are preserved: +4~p.p. for the tax and $-$4~p.p. for cutting aid.\footnote{In the main specification, I consider our policies of interest as consistent with any program. As an alternative, I classify them as either \textit{left} (for the tax) or \textit{far right} (for cutting aid). In that case, only 43\% of observations are retained, yet effects are still preserved (+4~p.p. for the tax and $-$5~p.p. for cutting aid).} 
This indicates that our results are robust to the critique of \cite{de_la_cuesta_improving_2022}. 
% Interestingly, the effects differ when I classify our policies of interest as \textit{left} (for the tax) and \textit{right} (for cutting aid), in which case only 39\% of observations are retained. In this case, the effect of the millionaire tax is preserved, but the effect of cutting aid vanishes to 0~p.p. In other words, the effect of cutting aid disappears when left-leaning programs containing it are removed, while the effect of the tax remains unchanged when right-leaning programs containing it are removed. Therefore, cutting aid is harmful only to left-leaning programs, while a wealth tax benefits any program. Interestingly, this does \textit{not} mean that only left-wing voters oppose cutting aid while an international wealth tax would be vote-determining for right-wing voters. Indeed, the results by vote show that the tax has a negative (though non-significant) effect among right and far-right voters, while cutting aid has a weakly significant effect of $-$2.4~p.p. ($p$ = .097) on \textit{Center-right or Right} voters (Figure \ref{fig:conjoint_vote}). 
% These observations can be reconciled by noting that some non-voters and right-wing voters may opt for left-wing programs, and vice versa. % shorten: cut last sentence

\subsection{Testing Warm Glow}\label{subsec:warm_glow}

Some people might claim to support a policy of global redistribution merely to ease their conscience. If support were mainly due to this psychological mechanism, called \textit{warm glow}, it might
dissipate when the prospect of the policy materializes or if the policy support could be replaced by a substitute with the same moral appeal.

\paragraph{Moral Substitute.}

Following \cite{nunes_identifying_2003}, warm glow would be revealed if support for the GCS decreased after respondents are offered the opportunity to express generosity towards the cause of climate change. To test this hypothesis, right before the GCS page, I assign a random subset of the respondents to a donation lottery, while the control group faces no question.\footnote{More precisely, right before the GCS question, the sample is split into three branches: the \textit{Donation} lottery, the \textit{NCS} question, and the control group. The NCS treatment is excluded from this analysis as it is unrelated to this experiment (restricting the NCS question to a subsample was done to prevent it from influencing responses to the GCS).} 
In the \textit{Donation} branch, respondents must decide how much they would donate to the reforestation NGO \textit{Just One Tree}, should they win the question's \$100 lottery. 
Lower support for the GCS in the treated group would be evidence of warm glow, as it would indicate that the support derives (at least partially) from moral satisfaction at having recently supported a just cause. 

% # 7. warm_glow: effect of info + display donation vs. control (per country + global)
    \begin{figure}[h!]
\caption[Testing warm glow]{Testing warm glow (negative effects would indicate the presence of warm glow). Regressions include controls, 95\% confidence intervals are shown.}\label{fig:warm_glow}
\begin{subfigure}{.45\textwidth}
  \caption[]{Effect of a \textit{Donation lottery} treatment on support for the Global Climate Scheme. (Questions \ref{q:donation}-\ref{q:gcs_support})\label{fig:warm_glow_substitute}}
  \includegraphics[height=.36\textheight]{../figures/country_comparison/gcs_support_by_variant_warm_glow.pdf}
\end{subfigure} \quad
\begin{subfigure}{.49\textwidth}
  \caption[]{Effect of information about ongoing global redistribution initiatives on the share of plausible global policies supported. (Questions \ref{q:info_solidarity}-\ref{q:solidarity_support})\label{fig:warm_glow_realism}}
  \includegraphics[height=.36\textheight]{../figures/country_comparison/share_solidarity_supported_by_info_solidarity.pdf}
\end{subfigure}
\end{figure}

On the contrary, support for the GCS is 3.5~p.p. higher in the \textit{Donation} branch compared to the control group, and the coefficient is positive in every country, though often not significant (Figure \ref{fig:warm_glow_substitute}). While the reason for this positive effect remains unclear,\footnote{Perhaps the \textit{Donation} question triggers thoughts favorable to the GCS, such as the realization that individual actions like donations are ill-suited to address climate change, so that we need a global policy, even if it is imperfect.} 
the results show no evidence of warm glow.

\paragraph{Realism Treatment.}

To test the hypothesis that some people express support for global redistribution only as long as its implementation seems unlikely, I randomly assign half of the respondents to receive information about ongoing negotiations on globally redistributive policies. Among other things, treated respondents are informed that the International Maritime Organization recently adopted a levy on maritime carbon emissions that should partly finance LICs; that the G20 considered introducing a global tax on billionaires; that the UN General Assembly recently agreed on the principle of expanding the UN Security Council to new members; and that the UN Secretary-General supports financial system reforms that would drive resources towards sustainable development (see Question~\ref{q:info_solidarity}). 
Then, respondents are asked ``how likely [it is] that international policies involving significant transfers from HICs to LICs will be introduced in the next 15 years'', right before their support for ten plausible global policies is tested.\footnote{Section \ref{subsec:debated} reports acceptance of these policies and Appendix~\ref{subsec:plausible_policies_sources} describes the corresponding international negotiations.}  Here, warm glow would be revealed if the information treatment increased the belief that global redistribution is likely but decreased support for global policies. %(as respondents would fear that such policies may materialize).

The treatment was designed to satisfy the exclusion restriction required for the instrumental variables (IV) strategy. The exclusion restriction states that the treatment affects support for global policies only through its impact on beliefs that global redistribution is likely. Table \ref{tab:iv} reports the corresponding regression results. Although the treatment is randomly assigned, the preferred specification includes the sociodemographic variables as controls to improve precision.\footnote{In fact, some effects are no longer significant in the specification without controls, with $p$-values for the IV and the direct effect at .11 and .12, respectively.} Informed respondents are 7~p.p. more likely to believe that global redistribution is likely, from a baseline of 33\% in the control group. With an effective F-statistic of 67, this highly significant effect provides a strong first stage for the IV estimation. Assuming that the exclusion restriction holds, the IV is well identified. The local average treatment effect estimated by 2SLS is 18~p.p., indicating that believing global redistribution is likely causally \textit{increases} the share of global policies supported. This estimate is consistent with both the non-causal OLS coefficient of 15~p.p. and the direct effect of the treatment on policy support, estimated at 1~p.p. (see Figure \ref{fig:warm_glow_realism}).

% I have designed the treatment to ensure the exclusion restriction in an instrumental variable (IV) strategy. The exclusion restriction states that the treatment should affect support for global policies solely through the belief that global redistribution is likely. 
% Table \ref{tab:iv} reports the results of regressions. Although the treatment is random, our preferred specification includes sociodemographic variables as controls to improve precision. 
% Informed respondents are 7~p.p. more likely to believe that global redistribution is likely, from a baseline of 33\% in the control group.
% With an effective F-statistic of 67, this highly significant effect provides a strong first stage for the IV. Assuming the exclusion restriction holds, the IV should be well identified. 
% The local average treatment effect estimated by 2SLS is 18~p.p., 
% indicating that believing that global redistribution is likely causally \textit{increases} the share of policies supported. 
% This effect is consistent with the non-causal OLS estimate of 15~p.p. and with the direct effect of the treatment on the share of policies supported, estimated at 1~p.p. (see Figure \ref{fig:warm_glow_realism}). % marginally ; $p$ = .12 %, though not significant ($p$ = .11).   % (see Table \ref{tab:iv}). 
% Given the strong first stage, the IV should be well identified, as the treatment has been designed to ensure the exclusion restriction; namely, that the information treatment affects support for global policies solely through the belief that global redistribution is likely. %As shown in Table \ref{tab:iv}, the 
% TODO? rewrite more professionally

% # 7bis: 2SLS 
\begin{table}[!htbp] 
  \caption[Effect on support for global redistribution of believing that it is likely]{Effect on support for global redistribution of believing that it is likely.}\label{tab:iv} 
  \makebox[\textwidth][c]{
% Table created by stargazer v.5.2.3 by Marek Hlavac, Social Policy Institute. E-mail: marek.hlavac at gmail.com
% Date and time: ven., nov. 14, 2025 - 15:00:07
\begin{tabular}{@{\extracolsep{5pt}}lccccc} 
\\[-1.8ex]\hline 
\hline \\[-1.8ex] 
\\[-1.8ex] & \multicolumn{2}{c}{\makecell{Believes global\\redistribution likely}} & \multicolumn{3}{c}{Share of plausible global policies supported} \\ 
 & IV 1st Stage & IV 1st Stage & IV 2nd Stage & OLS & Direct Effect \\ 
\\[-1.8ex] & (1) & (2) & (3) & (4) & (5)\\ 
\hline \\[-1.8ex] 
 Information treatment & 0.077$^{***}$ & 0.074$^{***}$ &  &  & 0.013$^{**}$ \\ 
  & (0.010) & (0.009) &  &  & (0.007) \\ 
  Believes global redistribution likely &  &  & 0.181$^{**}$ & 0.145$^{***}$ &  \\ 
  &  &  & (0.086) & (0.007) &  \\ 
  (Intercept) & 0.332$^{***}$ & 0.078 & 0.216$^{***}$ & 0.220$^{***}$ & 0.230$^{***}$ \\ 
  & (0.007) & (0.067) & (0.065) & (0.064) & (0.066) \\ 
 \hline \\[-1.8ex] 
Controls: sociodemos and vote &  & \checkmark & \checkmark & \checkmark & \checkmark \\ 
Effective F-statistic & 65.04 & 67.09 &  &  &  \\ 
Observations & 12,001 & 12,001 & 12,001 & 12,001 & 12,001 \\ 
R$^{2}$ & 0.006 & 0.134 & 0.174 & 0.176 & 0.141 \\ 
\hline 
\hline \\[-1.8ex] 
\end{tabular} 
}
  \textit{Note: Robust standard errors (HC1) are reported in parentheses. \hfill $^{*}$p$<$0.1; $^{**}$p$<$0.05; $^{***}$p$<$0.01. \\ As in Appendix~\ref{app:determinants}, control variables are: vote, gender, age, income, education, urbanity, likelihood of becoming millionaire, living with partner, employment status, foreign born, country region.}
\end{table}

Again, the effects go in the opposite direction to warm glow. In this case, increased support may stem from enhanced credibility of policies that are known to be discussed in international organizations. 
Overall, the results of these two experiments provide no evidence that support for global redistribution is affected by warm glow. On the contrary, they suggest that support is sincere and robust to the prospect of implementation or to the possibility of a moral substitute.


\section{Breadth of Accepted International Policies\label{sec:breadth}}

Knowing that some internationally redistributive policies are sincerely supported and may influence voting behavior, I now examine the range of international policies that could be accepted. In this section, I analyze, in turn, the support for global policies currently debated in the international community, as well as more radical proposals; I also assess broader willingness to defend global solidarity, the preferred channels to transfer resources to LICs, and a custom redistribution task designed to reveal the preferred extent of international transfers. 
% - People ready to sustainability and radical global redistr (for duty, not reparations)
% - People favor social protection to unconditional transfers

\subsection{Acceptance of Currently Debated Global Policies}\label{subsec:debated}

\paragraph{Plausible Global Policies.}

Figure \ref{fig:solidarity_support_share} reveals the acceptance of plausible global policies. %, i.e. the share of \textit{Somewhat} or \textit{Strong support} among non-\textit{Indifferent} answers. % TODO? (also known as relative support)
These policies are deemed ``plausible'' because they are debated in international organizations, as detailed in Appendix~\ref{subsec:plausible_policies_sources}. Almost every policy garners majority acceptance % TODO? relative majority support?
in each country. The only exception is the acceptance among Japanese respondents of a globally redistributive tax on carbon emissions from aviation, at 46\%. 
This proposal has the most salient cost: a 30\% increase in flight prices. It is the least % absolute
supported in every country. 
The policies with the most acceptance, above 70\% %of relative support 
in every country, are the 2\% minimum tax on billionaires' wealth proposed by \cite{zucman_blueprint_2024} and the expansion of low-interest-rate sustainable investments in LICs \citep{bridgetown_initiative_bridgetown_2025}. 
% TODO!? One sentence on absolute support? same for top_tax

% # 8. solidarity_support (on control): heatmap \includegraphics[height=.89\textheight]{../figures/country_comparison/solidarity_support_share}
\begin{figure}[h!] % TODO? relative support => acceptance?
    \caption[Acceptance of plausible global redistribution policies]{Acceptance of plausible global redistribution policies (Percentage of \textit{Somewhat} or \textit{Strongly support} among non-\textit{Indifferent} responses). See Figure~\ref{fig:solidarity_support_positive} for the absolute support. (Question \ref{q:solidarity_support}).
    }\label{fig:solidarity_support_share}
    \makebox[\textwidth][c]{\includegraphics[width=\textwidth]{../figures/country_comparison/solidarity_support_share.pdf}} 
\end{figure}

% The share of plausible policies supported ranges from 38\% in Japan to 66\% in Saudi Arabia and 65\% in Italy. Conversely, the share of policies opposed is highest in Switzerland (29\%), Poland %(23\%) 
% and the U.S. (23\%). The other questions exhibit a similar ranking of countries. 

\paragraph{Ranking of Countries in Terms of Support for Global Redistribution.}\label{par:ranking}

On average, respondents support 51\% of the plausible policies and oppose 21\% of them. This means they support +30~p.p. more policies than they oppose (Figure~\ref{fig:synthetic_indicators_mean}). The countries with the highest mean difference between support and opposition are Saudi Arabia (+50~p.p.), Italy (+49) and Spain (+39). In contrast, net support is lowest in Japan (+20), Switzerland (+24), Poland, and the U.S. (+25). % Actually, Poland has .08% higher net support than the U.S.
% TODO fig:synthetic_indicators_mean on control group

Other synthetic indicators of support for global redistribution show consistent country rankings. In particular, the countries previously identified as having the highest and lowest net support retain their rankings when ordered by average latent support for global redistribution (Figure~\ref{fig:synthetic_indicators_mean}). 
To construct this latent variable, I standardize all variables of support and average them, weighted by loadings obtained from an exploratory factor analysis (see details and loading weights in Appendix~\ref{subsec:efa}). 
% Methodologically, I construct this latent variable by averaging all variables of support in the survey, standardized and weighted by loadings obtained by an exploratory factor analysis (see details and loading weights in Appendix~\ref{subsec:efa}). 

One might wonder why the countries leading in support are the Saudi kingdom and right-wing-dominated Italy. Breaking down the support by political leaning and other selected sociodemographics, Figures \ref{fig:main_radical_redistr_pol}-\ref{fig:synthetic_indicators_pol_mean} shed some light on this question. In Saudi Arabia, half of the adult population is immigrant. % 51.5%
However, foreign workers do not drive the results, as Saudi citizens exhibit slightly higher support than non-Saudis.\footnote{Therefore, tentative explanations may rather come from Saudi society. While Saudis benefit from a generous welfare state, the Islamic principle of \textit{Zakat} (almsgiving) might further foster a culture of generosity.} 
As for Italy, it is both the country with the lowest gap in support between left-wing and far-right voters (along with Japan, at 33~p.p.) and the country with the highest support among left-wing voters.\footnote{While the former point may be linked to the vision of Italy's far-right leader of an Africa-Italy partnership (trading off foreign aid with cooperation ion fighting immigration), the Italian population might also be influenced by the Vatican's messages in favor of global solidarity.} 
% TODO: to analyze Saudi anomaly: correlation between optimism (well-being) and support: .13*** at country level
% TODO: correlation between growth and support: .07***

\paragraph{Climate Finance Goal.}\label{par:ncqg}

Climate finance refers to the financing of climate action in developing countries by developed countries. In 2024, countries agreed on a ``New Collective Quantified Goal'' (NCQG) of climate finance set at \$300~billion per year by 2035, which is triple the previous goal. However, while developing countries such as India called for \$600~billion in grants (or grant-equivalent funding), the NCQG does not specify the share of finance that should be provided as grants. Currently, the goal is being met with only \$26~billion in grants and the remainder in loans \citep{oecd_climate_2024}. 

I test the preferred amount for the NCQG in grant-equivalent terms, using two random variants. Both variants inform respondents of the current situation and the agreed goal, expressing amounts in both absolute terms and as a proportion of developed countries' GDP. The \textit{Short} variant uses qualitative, textual responses, and features a middle category of \$100~billion (namely, ``Meet the newly agreed goal by tripling grants and loans (\$100 billion in grants, or 0.15\% of GDP).''). The \textit{Long} variant provides more detailed explanations in the question text and then uses numerical answers, with a midpoint of \$300~billion. 

In both variants, the median preferred NCQG is \$100~billion, with 19\% of respondents choosing an amount of \$600~billion or larger (Figures~\ref{fig:ncqg}-\ref{fig:ncqg_full}). 

That differently framed variants yield consistent results suggests that, despite its length, the question was well understood. 
The median choice of a climate finance quantum in line with the internationally agreed NCQG can be interpreted in two distinct ways. Either diplomats of HICs are defending the level of generosity that reflects the median preferences of their compatriots, or respondents' attitudes are anchored in existing agreements (or in their governments' stance). The results presented below are more consistent with the latter interpretation, as they reveal majority acceptance of much larger international transfers.


\subsection{Support for Radical Proposals, Political Action, and Broad Values}\label{subsec:radical}

In the final part of the questionnaire, I pose a variety of questions to assess the range of global solidarity policies, actions or values that people may accept (Figure \ref{fig:radical_redistr_share}).

\paragraph{Sustainable Future versus Status Quo.}\label{par:sustainable_future}

Respondents were asked which scenario they would prefer for the next twenty years: a sustainable future or the status quo (note that scenarios were not labeled that way in the questionnaire, but were instead randomly named \textit{A} or \textit{B}). In the sustainable scenario, most countries cooperate to tax millionaires and meet the +2\textdegree{}C target, through the electrification of cars and the doubling of prices for heating fuel or gas, air travel, and beef. Although overall purchasing power is preserved (through a reduction in sales tax), people change their habits (e.g. flying and eating meat are cut by half). In the status quo, no policy is implemented, people maintain their lifestyles, and global warming reaches +3\textdegree{}C by 2100, causing more severe disasters.

Overall, 68\% of respondents prefer the sustainable future over the status quo.

\paragraph{Global Income Redistribution.}\label{par:global_income_redistr}

I test the support for a global tax on top incomes to finance poverty reduction in LICs, with the tax targeting either the global top~1\% or top~3\%, depending on random assignment. The top~1\% variant describes an additional 15\% tax on income in excess of \$120,000 per year (at Purchasing Power Parity), while the top~3\% variant features additional rates of 15\%, 30\%, and 45\% above \$80,000, \$120,000, and \$1~million, respectively. Each tax is calibrated to finance the poverty gap, with poverty defined using thresholds of \$250 and \$400 per month for the top~1\% and top~3\% variants, respectively. These taxes entail international transfers of 2\% and 5\% of world nominal income, respectively (see Appendix~\ref{subsec:country_features} for details). 
Two numerical examples explain to respondents how the tax would affect taxpayers' income. The question also states the share of affected taxpayers worldwide and in their country, as well as the share of their country's GDP that would be transferred. For example, in the U.S., the top~1\% tax would affect the top 8\% and transfer 3\% of GDP, while the top~3\% tax would affect the top 18\% and transfer 8\% of GDP (see Figure~\ref{fig:top_tax_share}). These figures are about half as high in Japan and Germany, and around four times lower in France and Spain.

Overall, 56\% (resp. 50\%) of the respondents support the top~1\% (resp. top~3\%) tax, and 25\% (resp. 28\%) oppose it (Figure~\ref{fig:top_tax_positive}). 
% There is majority acceptance %relative majority support for 
% of the policy 
The policy is accepted by a majority in every case except Switzerland for the top~3\% tax (in which case 18\% of Swiss people would be affected). Overall, the tax garners %relative majority support
majority acceptance 
even among the 6\% of respondents who would be affected, though this is not the case in every country for the top~3\% variant (Figure~\ref{fig:top_tax_affected_share}). 

\paragraph{Global Convergence.}\label{par:convergence}

A simple question captures the acceptance of global solidarity: ``Should governments actively cooperate to have all countries converge in terms of GDP per capita by the end of the century?'' Overall, 61\% answer \textit{Yes} and 26\% \textit{No}, with the lowest relative agreement (i.e. excluding people not responding%choosing not to answer
) in the U.S., at 56\%.

\paragraph{Willingness to Act.}\label{par:willingness_act}

Two questions asked the respondents how they would react to a ``worldwide movement in favor of a global program to tackle climate change, implement taxes on millionaires and fund poverty reduction in [LICs]''. 

% TODO! compare this to Cappelen et al
In a multiple-choice question (censored in Russia), 29\% report they could participate in the movement by either attending a demonstration (19\%), going on strike (7\%), or donating \$100 to a strike fund (10\%). This share rises to 68\% in favor of the movement when including the 52\% of respondents who ``could sign a petition and spread ideas'' for the movement (Figure \ref{fig:global_movement}). 
Taken at face value, these results would mean that a successful global solidarity movement could collect up to \$10~billion and organize some of the largest demonstrations in history, matching Earth Day mobilizations.\footnote{On April 22, 1970, 20 million US-Americans (10\% of the population) demonstrated for the environment. Since then, Earth Day events regularly mobilize hundreds of millions of people worldwide.} % TODO? cut for AER?
Interestingly, 52\% of the 584 millionaires\footnote{In the weighted subsample of millionaires, 60\% are US-Americans and 26\% Europeans.} who answered the survey would be in favor of such a movement.

When asked whether they would be more or less likely to vote for the political party they feel closest to if it were part of such a movement, 36\% of the respondents state they would be more likely versus 17\% less likely (Figure \ref{fig:vote_intl_coalition}). Among the 5\% of respondents who did not vote in the last election and feel closest to a left-wing party, the share more likely to vote in that case increases to 46\% (versus 10\% who are less likely). 

\paragraph{Reasons for Helping LICs.}\label{par:why_help_lic}

In a multiple-choice question, I asked respondents which reasons for HICs supporting LICs they agree with, among arguments involving \textit{duty}, \textit{long-term interest}, or \textit{historical responsibility}. At 54\%, the reason most frequently chosen in every country (except France) is \textit{duty}, specifically ``Helping countries in need is the right thing to do'' (Figure \ref{fig:why_hic_help_lic}). Additionally, 38\% select \textit{interest}, and 25\% \textit{responsibility}, with only 16\% disagreeing with every reason.

\paragraph{Reparations.}\label{par:reparations}
% TODO past or present when describing the survey?

In former colonial or slave States,\footnote{I did not ask this question in Japan or Russia, because these countries' historiographies do not present their past as colonial but rather  as an empire% larger ``empire''
.} 
I asked respondents whether they would support ``reparations for colonization and slavery to former colonies and descendants of slaves'', specifying that the reparations ``could take the form of funding education and facilitating technology transfers''. Consistent with the general disagreement that HICs have a \textit{historical} responsibility to support LICs, only a minority of 35\% of respondents support reparations (except in Italy where 56\% do), while 42\% oppose them (Figure~\ref{fig:reparations_support}). This suggests that framing global solidarity as a decolonial struggle might be counterproductive. % for its advocates. TODO? add for AER

\paragraph{Agreement That Own Taxes Should Solve Global Problems.}\label{par:my_tax}

% To crosscheck the findings and analyze the extent to which they could be affected by time, framing, or sampling, I reproduced a question asked in a previous study, the ``Global Solidarity Report'' \citep{global_nation_global_2023}. 
Overall, 41\% agree and 28\% disagree that ``[their] taxes should go towards solving global problem''. With 60\% relative agreement, there is a relative majority in favor of one's own taxes financing global solidarity, though a lower one than for specific proposals that would make the richest contribute.\footnote{This confirms that the willingness to pay for global solidarity, even through taxes, does not equate to acceptance of global redistribution proposals.%Mmeasure the same disposition as the support for global redistribution proposals. 
} 
As explained in Section \ref{sec:data} and shown in Appendix \ref{app:comparison}, the present results replicate well the ``Global Solidarity Report'' that first asked this question \citep{global_nation_global_2023}

% \paragraph{Comparison with existing surveys}

% To crosscheck the findings and analyze the extent to which they could be affected by time, framing, or sampling, I reproduced a question asked in a previous study, the ``Global Solidarity Report'' \citep{global_nation_global_2023}. Overall, 41\% agree and 28\% disagree that ``[Their] taxes should go towards solving global problem''. 
% The results replicate well the earlier study, as relative agreement, at 59\%, is just 3~p.p. higher in the present survey, and the cross-country correlation is .70 (see Figure \ref{fig:comparison} in Appendix \ref{app:comparison}). 

% Furthermore, although the question on the billionaire tax is not worded exactly as in \cite{cappelen_majority_2025}, the results are also very consistent, with an overall absolute support 4~p.p. lower in the present survey and a cross-country correlation of .86.

% # 9. radical_redistr: heatmap sustainability, top_tax, reparations, NCQG?, vote_intl_coalition, group_defended?, my_tax_global_nation, convergence_support # my_tax_global_nation other source? No, just mention corrs and means in appendix/footnote & say no sufficient confidence in their representativeness + compare w Stostad \includegraphics[width=\textwidth]{../figures/country_comparison/radical_redistr_few_share}
\begin{figure}[h!]
    \caption[Acceptance of broad or radical global redistribution]{Acceptance of broad action or radical proposals of global redistribution. \hfill (Questions \ref{q:sustainable_future}-\ref{q:top3_tax_support}, \ref{q:convergence_support}-\ref{q:vote_intl_coalition}, \ref{q:reparations_support}, \ref{q:my_tax_global_nation}).
    }\label{fig:radical_redistr_share} 
    \makebox[\textwidth][c]{\includegraphics[width=\textwidth]{../figures/country_comparison/radical_redistr_all_share.pdf}} % radical_redistr_share
\end{figure}

\paragraph{Moral Circle.}\label{par:group_defended}

Asked ``Which group of people do you advocate for when you vote?'',\footnote{In Russia and Saudi Arabia, the question was asked differently. It read: ``Which group do you advocate for? For example, if you were the richest person on Earth, which group would you predominantly help with your money?''\label{fn:group_defended}} 
45\% select a universalist answer (``Humans'' or ``Sentient beings (humans and animals)''), which is more than the most common answer, referring to one's fellow citizens (32\%). Universalists are fewer in Japan (30\%) %and the U.S. (43\%) 
but constitute a  majority in Europe (50\%) and Saudi Arabia (57\%), as shown in Figures \ref{fig:group_defended} and \ref{fig:group_defended_all}.
Among those who lean to the left, 59\% are universalists, compared to 32\% on the center-right or far-right. %, and 45\% of non-voters.

Following \cite{enke_universalism_2024}, I construct an alternative measure of universalism based on the vocabulary used in open-ended fields.\footnote{More specifically, I use the Moral Foundation Dictionary (MFD) 2.0 \citep{frimer_moral_2019} and define \textit{universalism} as the number of occurrences of \textit{care} or \textit{fairness} words minus the number of \textit{loyalty} or \textit{authority} words. I also test an alternative definition, based on the extended MFD \citep{hopp_extended_2021}, that uses weights rather than dummy variables to indicate a word's belonging to a moral dimension. The latter definition is even less (though still significantly) correlated with the latent support for global redistribution, at .03.} 
Although latent support for global redistribution is significantly correlated with this measure, the correlation is only .05, less than the correlation with manual, keyword, or AI classifications of a field as relating to inequality or poverty (at .09, .08, and .06, respectively), and much less than the correlation with our universalism variable based on moral circle (at .37). Furthermore, the correlation between the two measures of universalism is only .03. This observation demonstrates that the various indicators labeled as ``universalism'' by different authors may not all capture the same dimension.

% # 10. group_defended: barresN or barres? \includegraphics[height=.9\textheight]{../figures/all/group_defended}
\begin{figure}[h!]
    \caption[Moral circle]{``Which group of people do you advocate for when you vote?''\footref{fn:group_defended} (Question \ref{q:group_defended}).
    }\label{fig:group_defended}
    \makebox[\textwidth][c]{\includegraphics[width=.9\textwidth]{../figures/all/group_defended_nolabel.pdf}} 
\end{figure}

\subsection{Preferred Channels for Transferring Resources to LICs}\label{subsec:transfer_how}

Asked to evaluate ways of transferring resources to reduce poverty in LICs on a 4-point Likert scale, the most preferred option in every country is ``Cash transfers to parents (child allowances), to the disabled and to the elderly'', with 16\% selecting it as the \textit{Best way} overall, and 49\% as a \textit{Right way} or \textit{Best way} (Figures \ref{fig:transfer_how}, \ref{fig:transfer_how_above_one}-\ref{fig:transfer_how_negative}). ``Unconditional transfers to the national government'' is the only option seen as a \textit{Wrong way} by the majority, but this share falls from 51\% down to 21\% (becoming the third most supported option out of seven) when ``transfers to the national government [are] conditioned on the use of funds for poverty reduction programs''. Interestingly, ``unconditional cash transfers to each household'' are controversial: they are the second most chosen \textit{Best way}, yet 33\% view them as a \textit{Wrong way}. Conversely, ``transfers to public development aid agencies which then finance suitable projects'' is uncontroversial, with only 16\% considering it a \textit{Wrong way}, while 37\% rate it as a \textit{Right} or \textit{Best way}.

% # 11. transfer_how: heatmap (maybe just one row grouping all countries and options in columns) \includegraphics[width=\textwidth]{../figures/country_comparison/transfer_how_positive}
\begin{figure}[h!]
    \caption[\textit{Right} or \textit{Best way} to transfer resources to LICs (global average)]{``How do you evaluate each of these channels to transfer resources to reduce poverty in LICs?''\\ Percentage of \textit{Right} or \textit{Best way} (other options: \textit{Wrong} or \textit{Acceptable way}). (Question \ref{q:transfer_how}).
    }\label{fig:transfer_how}
    \makebox[\textwidth][c]{\includegraphics[width=\textwidth]{../figures/country_comparison/transfer_how_positive.pdf}} 
\end{figure}

\subsection{Custom Global Income Redistribution}\label{subsec:custom_redistr}

The last task of the questionnaire allowed respondents to manipulate the shape of the global income distribution.\footnote{Appendix \ref{subsec:country_features} details how I obtained the world distribution of PPP incomes. } The question text included the following instructions: 
\begin{quote}``Below you will find a graph of the world distribution of after-tax income and three sliders that vary it. The current distribution is in red, and your custom one is in green. The first two sliders control the proportion of winners and the proportion of losers, among all humans. The third slider controls the degree of redistribution from the richest to the poorest. 
If you do not want new policies to reduce global inequality, you can set the third slider to zero.''\end{quote} 
The interactive question is available at \href{https://www.centre-cired.fr/custom-global-redistr}{bit.ly/custom\_redistrib} and the algorithm translating slider positions into a redistribution is described in Appendix \ref{app:algo}. %and a video explainer on TODO.
Figure \ref{fig:custom_redistr_question} displays what respondents see below the instructions, including the interactive graph and a table summarizing how their custom redistribution would affect five example income levels (including their own, asked right before). To mitigate potential anchoring at the sliders' initial positions,%
\footnote{To test for anchoring, I regress responses on the sliders' initial positions. %The initial positions have a significant anchoring effect on the responses. %Relative to the other variant's %difference between the two variants' 
% initial positions, a given variant brings the responses closer to its position by 
I define the anchoring effect as the effect size relative to the difference between the initial positions of the two variants. It is always significant, at 36\% for the share of winners, 57\% for the share of losers, and 42\% for the degree of redistribution. While anchoring plays a role, the responses converge to a middle point between the two anchors, indicating that the anchors themselves may have been defined by the surveyor (myself) drifting away from a shared preference in opposite directions. % collective, common taste or natural focal point. 
} 
sliders are initialized in one of two random positions: %  (yielding similar rich-to-poor transfers of 4.3\% to 4.6\% of the world GDP)
either %the \textit{diffuse} redistribution with 
60\% of winners, 20\% of losers, and a degree of redistribution of 2 out of 10 (as in Figure \ref{fig:custom_redistr_question}); or %the \textit{concentrated} one with 
40\%, 10\%, and 7/10, respectively. 
Given the complexity of the task and its inconvenience on mobile devices, respondents are given the explicit option to skip it. 
% As the task is difficult to understand (and not so handy %to handle 
% on phone screens), respondents are instructed that they can skip it. 

% # 12. average custom_redistr \includegraphics[height=.8\textheight]{../figures/all/custom_redistr_mean.png}
\begin{figure}[h!]
   \caption[Custom redistribution task screenshot]{Custom global redistribution: screenshot of the bottom of the page. (Question \ref{q:custom_redistr}).
    }\label{fig:custom_redistr_question}
    \makebox[\textwidth][c]{\includegraphics[height=.85\textheight]{../figures/questionnaire/survey_custom_redistr_bottom.png}} % bottom2 is without the choice options satisfied/skip
\end{figure}

Overall, 56\% are satisfied with their custom redistribution and 43\% skip it. 
% Furthermore, 35\% of the respondents did not move the sliders from their original position, and while 39\% of them still state that they are satisfied with the redistribution, I exclude them from the main analysis, conducted over the 40\% of ``conforming'' % responsive, admitted, 
% respondents who touched the sliders and are satisfied with their custom redistribution. 
Although the non-response rate may seem high, it is relatively evenly spread across the population. Indeed, the share of satisfied respondents is 52\% for non-voters, 54\% for center-right or right-wing voters, 57\% for far-right voters, and 61\% for left-wing voters; while it ranges from 49\% for people without a %secondary education (i.e. 
high-school diploma to 57\% for those with a post-secondary diploma.
% While the non-response rate may seem high, it is relatively spread out over the population. Indeed, the share of conforming respondents is 37\% for non-voters, 39\% for center-right or right-wing voters, 40\% for far-right voters and 43\% for left-wing voters; while it ranges from 33\% for people without a high-school diploma to 42\% for those with a post-secondary diploma.  
The limited heterogeneity in response rates across crucial sociodemographic groups suggests that the task enabled motivated respondents to make an informed choice regarding their preferred redistribution with little sacrifice in terms of sample representativeness.

% \paragraph{Anchoring}
% Furthermore, the average responses are similar between satisfied and all respondents: between these groups, the share of winners and the degree of redistribution differ by at most 2\%, and the share of losers by at most 10\%. The similarity in results between satisfied respondents who moved the sliders and those who did not means that the average custom responses are similar to the average between the two possible sliders' initial positions. Two factors may explain this similarity. First, the responses might have been anchored at sliders' initial positions. Second, both the respondents and the surveyor (myself) might have a similar view of what a reasonable redistribution looks like. To test for anchoring, I regress responses on the sliders' initial position. The initial positions have a significant anchoring effect on the responses. %Relative to the other variant's %difference between the two variants' 
% % initial positions, a given variant brings the responses closer to its position by 
% The anchoring effect, defined as the effect size relative to the difference between the two variants' initial positions, is 
% 40\% for the share of winners, 55\% for the share of losers, and 42\% for the degree of redistribution. While anchoring plays a role, the responses converge to a middle point between the two anchors, indicating that the anchors themselves may have been defined by the surveyor drifting away (in opposite directions) from a shared preference. % collective, common taste or natural focal point. 

% To sum up, while this task enables respondents to make an informed choice regarding their preferred redistribution, it remains an imperfect method, and more appropriate ways to reveal citizens' preferred redistribution may be proposed. 

% With these caveats in mind, 
Figure \ref{fig:custom_redistr_median} shows the median preferred redistribution among satisfied respondents, i.e. the curve obtained by setting the sliders at their median preferred values: 49\% of winners, 18\% of losers, and a degree of redistribution of 5/10, resulting in a transfer of 5.4\% of world income from rich to poor and in a minimum income of \$287 per month. 
Interestingly, 48\% choose to lose from their custom redistribution while only 9\% choose to win; the median satisfied respondent selects parameters such that they neither win nor lose. Besides, 10\% of satisfied respondents opt for the status quo, preserving the current income distribution. 
Finally, Figure \ref{fig:custom_redistr_mean} presents the average preferred redistribution among satisfied respondents, obtained by pointwise averaging custom curves. 
The average preferred redistribution transfers 5.4\% of world income from the top 27\% to the bottom 73\% and entails a minimum income of \$247 per month.  
As shown in Figure \ref{fig:tax_radical_redistr}, at the top of the distribution, the average preferred redistribution can be achieved with a 7\% marginal income tax rate above \$25,000 and a 16\% rate above \$40,000 per year (at Purchasing Power Parity). 

Figures~\ref{fig:custom_redistr_satisfied_mean}-\ref{fig:custom_redistr_satisfied_median} reveal limited heterogeneity in custom redistributions across countries. However, Figures~\ref{fig:custom_redistr_winners_agg}-\ref{fig:custom_redistr_transfer_ceiling} show greater variation at the individual level, though the bulk of respondents favor a custom redistribution implying transfers of 4\% to 5\% of world income and a minimum income of \$150 to \$350 per month. 

\cite{fabre_french_2022} applied the same method to uncover French preferences regarding national income redistribution and tested support for the median and average % 42% losers in French average
preferred redistributions on a separate sample. Excluding the 22\% to 24\% of people not responding, 51\% of respondents accepted the average redistribution and 67\% the median one.%
\footnote{\cite{fabre_french_2022} also tested a redistribution obtained from median parameters and a 5\% lower aggregate income to account for adverse behavioral responses. This was accepted by 62\% of French respondents.} 
While one cannot be sure that these results would replicate in the context of a global redistribution, they suggest that a majority might accept the average or median redistribution described above.

\section{Conclusion} % Summary, conclusion

Applying the theory of optimal taxation, \cite{kopczuk_limitations_2005} show that the level of U.S. foreign aid could only be rationalized if the U.S. placed a value 2,000 times higher on the welfare of a US-American than on that of a foreigner (although this ratio should be reduced by the proportion of foreign aid transfers diverted or wasted). 
Our results contradict the notion that government action accurately reflects attitudes towards global redistribution and are consistent with a conservative bias among legislators \citep{gilens_testing_2014,broockman_bias_2018,pilet_politicians_2024}. 
Indeed, a majority of respondents in high-income countries support a global tax on top incomes to finance poverty reduction in low-income countries. Additionally, over two-thirds of respondents accept a tax on the wealth of millionaires with 30\% of the revenue financing LICs, even in the case of only a few countries implementing it. In every country, majorities accept an International Climate Scheme that is costly to them but beneficial to the poorest globally, showing that most people value climate action and poverty reduction. 

The revenue allocation task sheds light on how much people value global versus domestic public goods. On average, respondents allocate 17.5\% of the revenue from a hypothetical global wealth tax to sustainable development out of the five specified categories. 
This indicates that people are neither selfless universalists, who would allocate all the revenue from this tax to the poorest countries, nor devoid of altruism towards foreigners, as this would imply allocating nothing to global spending. 
The custom redistribution task confirms that most people would actually prefer much greater global redistribution than currently exists, as the average respondent opts for a global minimum income of \$247 per month, financed by transfers amounting to over 5\% of world income. 

An exploration of respondents' underlying values reveals that support for global redistribution primarily stems from a sense of duty and empathy towards the destitute. For some, this issue appears important enough to factor into their voting decision. Indeed, the likelihood that a political program is preferred increases if it includes a globally redistributive tax on millionaires and decreases if it includes cuts to foreign aid. Additionally, one-third of respondents report that they would be more likely to vote for a political party if it were part of a global movement for sustainable development, and a similar proportion state that they could themselves participate in such a movement. 

These results raise the question of why so few policymakers campaign on sustainable development proposals. The lack of supply of such campaigns might stem from pluralistic ignorance among policymakers and activists, consistent with the public's underestimation of support for a Global Climate Scheme. % as well as with previous studies \citep{}
Alternatively, it could be due to a lack of demand from constituents. Indeed, global inequality is rarely a top-of-mind consideration. People's most frequent concerns relate to self-interested issues such as their purchasing power or health; articulated political demands generally refer to national issues such as public services; and the most salient international issues are climate change, wars, and the rise of the far right. 

The low salience of global inequality may manifest as a lack of popular mobilization, resulting in it being a low priority for policymakers. Combined with the necessary trade-off between global redistribution and boosting fellow citizens' purchasing power, policymakers may prioritize the latter ---which is the primary concern of voters--- to the point of ignoring universalist attitudes. 
Status quo bias is a compounding factor: the weakness of global institutions and the primacy of national polities make international cooperation unlikely, % the lack of a worldwide polity 
which may discourage universalist thought and make it seem utopian. % mute the expression of universalist mindsets 
% Indeed, if people believed that global policies are likely, support for them would increase, as our information experiment demonstrated. 
Indeed, support for global policies is partly caused by the belief that they are likely, as our information experiment demonstrated. Therefore, the organization of the % solidarity confined at the national level 
world order based on nation-states might silence demands for % ``utopian''
universalist reforms and perpetuate a cycle where the low salience of universalist concerns and status quo institutions reinforce each other. 

Nevertheless, the survey results %widespread acceptance of sustainable development proposals 
suggest some untapped potential for global solidarity. It is unlikely that the public would resist global redistribution policies. This is especially true for balance sheet operations with expansionary impacts and indirect costs, such as debt restructuring, liquidity provision, %currency swaps, 
the expansion of lending by Multilateral Development Banks, and their recapitalization through the rechanneling of Special Drawing Rights. These reforms are widely accepted and are the natural focus of multilateral initiatives \citep{bridgetown_initiative_bridgetown_2025,ffd4_outcome_2025}. % Fair allocation of taxing rights, address illicit flows...
% Despite an unfavorable geopolitical context, the technicality of negotiations, and the (alleged or real) unreliability of institutions in recipient countries, multilateral initiatives such as the Sevilla Platform for Action \citep{ffd4_outcome_2025} make some (slow) progress on these issues. % institutional capacity % mismanagement of resources by 
% However, the current pace of progress is insufficient to achieve the Sustainable Development Goals. Our results suggest that accelerated progress would require increased salience of global inequality. 
% Inspired by the alleged role of the Intergovernmental Panel on Climate Change in making climate change salient, \cite{stiglitz_g20_2025} may have arrived at the same conclusion, when recommending to the G20 to establish an International Panel on Inequality.
Since public attitudes does not appear to be a limiting factor, further research is needed to understand policymakers' motivations and the obstacles they face in cooperating on sustainable development reforms. 

% Following the lead of the Intergovernmental Panel on Climate Change, the salience of global inequality may be fostered by an International Panel on Inequality, as recommended at the G20 by \citep{stiglitz_g20_2025}.

% coalition of the willing would work
% why  policymakers choose to go from lower priority to nothing? why not Bridgetown? debt restructuring and liquidity support? currency swaps? BEPS or ad tax? address illicit flows; IP waiver (check Stiglitz)
% mismanagement (or distrust) by LIC govts; complexity (hence slow progress) of technical negotations
% geopolitical  (outsized influence of the US, which requires bipartisan support); ipcc ineq
% research on whether low demand; status quo bias => vicious circle: utopian -> ignore it

% TODO absence of global polity; trade-off purchasing power / global redistr; policymakers choose to go from lower priority to nothing. Given results, it seems very unlikely that plausible global redistr would be unpopular/resisted by the public: why not Bridgetown? debt relief? BEPS or ad tax? billionaire tax (reinvesting MNE profits into global public investments, Glennie - education among the highest rate of return)
% Why not more? next hypothesis: pluralistic ignorance of elites (Mildenberger & Tingley); national structures / representing nations bias diplomacy/thinking towards national interest (Buchholz for theoretical argument in climate negots, Schleich show 28-13% find their personal position well represented at these nego) (perhaps mix of zero-sum thinking (helping others will be bad for me) and nationalism)
% Robustness?

% TODO? cite Andre et al? Conservative bias (Broockman & Skovron; Pilet et al)?

% \begin{methods}  % WPcomment
  % \begin{small} % NCCcomment
%Put methods in here.  If you are going to subsection it, use \subsection commands.  Methods section should be less than 800 words and if it is less than 200 words, it can be incorporated into the main text.
% \section*{\normalsize Methods}\label{sec:methods} % NCCcomment
% \addcontentsline{toc}{section}{\nameref{sec:methods}}

% The paper utilizes two sets of surveys: the \textit{global} survey and the \textit{Western} surveys. The \textit{global} surveys consist of two U.S. surveys, \textit{US1} and \textit{US2}, and one European survey, \textit{Eu}. The \textit{global} survey was conducted from March 2021 to March 2022 on 40,680 respondents from 20 countries (with 1,465 to 2,488 respondents per country). \textit{US1} collected responses from 3,000 respondents between January and March 2023, while \textit{US2} gathered data from 2,000 respondents between March and April 2023. \textit{Eu} included 3,000 respondents and was conducted from February to March 2023. We used the survey companies \emph{Dynata} and \emph{Bilendi}. To ensure representative samples, we employed stratified quotas based on gender, age (5 brackets), income (4), region (4), education level (3), and ethnicity (3) for the U.S. We also incorporated survey weights throughout the analysis to account for any remaining imbalances. These weights were constructed using the quota variables as well as the degree of urbanization, and trimmed between 0.25 and 4. Stratified quotas followed by reweighting is the usual method to reduce selection bias from opt-in online panels, when better sampling methods (such as compulsory participation of random dwellings) are unavailable.\cite{scherpenzeel_how_2010} By applying weights, the results are fully representative of the respective countries along the above mentioned dimensions. %. 
% Appendix \ref{app:balance} shows that the treatment branches are balanced. Appendix \ref{app:placebo} runs placebo tests of the effects of each treatment on unrelated outcomes. We do not find effects of earlier treatments on unrelated outcomes arriving later in the survey. Appendix \ref{app:extended} shows that our results are unchanged when including inattentive respondents.

\begin{small}
\section*{\normalsize Acknowledgements}
I received funding from the Agence Nationale de la Recherche for this research (ANR-24-CE03-7110). I am grateful to Alina Timoshkina and Gerhard Toews for helping me launch the survey in Russia. I thank researchers who kindly gave me access to their data: Félix Bajard for wealth distribution and Amory Gethin for income distribution. %, Gerhard Toews for Russian census. 
I am grateful to Armon Rezai and Wien Universität for helping me use Qualtrics. I am grateful to the translators and the numerous people who graciously tested the survey, proofread the translations, and gave feedback, including Gianluca Drappo, Valeria Glebova, Anne Guillemot, Marya Kayyal, Lu Li, Lilian Leupold, Guadalupe Manzo, Rintaro Matsuda. % Pauline? Samuel? Davide Pace? Mari Mohamad?
I am grateful to Erwan Akrour, Marius Le Hénaff, and Raquel Oliveira %and Alina Timoshkina 
for research assistance on the quotas and the pools of policies. I thank seminar, workshop or conference participants for useful feedback, at: the University of Brasília, Unicamp, the University of São Paulo, WZB Berlin, Paris-Saclay, the World Inequality Lab, MSE-Sorbonne, the Bundesbank, FSR Climate Conference, the New Economic School, CIRED, Fürth. 

% \section*{\normalsize Competing interests} Fabre declares that he also serves as treasurer of Global Redistribution Advocates.

% \end{methods} % WPcomment
\end{small}  % NCCcomment

% \bibliographystyle{naturemag_noURL} % nature class works only with style naturemag or naturemag_noURL, and naturemag bugs if there are certain URLs (like .pdf). Also, nature class only works with \cite, not \citet or \citep.  % WPcomment
\renewcommand{\url}[1]{\href{#1}{Link}} % NCCcomment
% \bibliographystyle{unsrtnat} % NCCcomment
% \bibliography{global_tax_attitudes}

\clearpage

\putbib
\end{bibunit}

\begin{bibunit}[plainnaturl_clean]

\appendix % NCCcomment
% \renewcommand{\thetable}{ED\arabic{table}}
% \renewcommand{\thefigure}{ED\arabic{figure}}
% \setcounter{figure}{0}
% \setcounter{table}{0}


\renewcommand{\thetable}{S\arabic{table}}
\renewcommand{\thefigure}{S\arabic{figure}}
\setcounter{figure}{0}
\setcounter{table}{0}

\clearpage
\section{Raw results% from the complementary surveys
}\label{app:raw_results}

% Country-specific raw results are also available as supplementary material files:  \href{https://github.com/bixiou/international_attitudes_toward_global_policies/raw/main/paper/app_desc_stats_US.pdf}{US}, \href{https://github.com/bixiou/international_attitudes_toward_global_policies/raw/main/paper/app_desc_stats_EU.pdf}{EU}, \href{https://github.com/bixiou/international_attitudes_toward_global_policies/raw/main/paper/app_desc_stats_FR.pdf}{FR}, \href{https://github.com/bixiou/international_attitudes_toward_global_policies/raw/main/paper/app_desc_stats_DE.pdf}{DE}, \href{https://github.com/bixiou/international_attitudes_toward_global_policies/raw/main/paper/app_desc_stats_ES.pdf}{ES}, \href{https://github.com/bixiou/international_attitudes_toward_global_policies/raw/main/paper/app_desc_stats_UK.pdf}{UK}.

\begin{figure}[h!]
    \caption[Keyword classification of open-ended fields]{Keyword classification of open-ended fields (matches with at least one keyword in a list). (Questions \ref{q:concerns_field}-\ref{q:injustice_field}).
    }\label{fig:field_keyword}
    \makebox[\textwidth][c]{\includegraphics[height=.85\textheight]{../figures/country_comparison/field_keyword_main_positive.pdf}} 
\end{figure}

\begin{figure}[h!]
    \caption[AI classification of open-ended fields]{AI classification of open-ended fields (using ChatGPT-4.1). (Questions \ref{q:concerns_field}-\ref{q:injustice_field}).
    }\label{fig:field_gpt}
    \makebox[\textwidth][c]{\includegraphics[height=.85\textheight]{../figures/country_comparison/field_gpt_positive.pdf}} 
\end{figure}

\begin{figure}[h!]
    \caption[Manual classification of open-ended fields]{Manual classification of open-ended fields. (Questions \ref{q:concerns_field}-\ref{q:injustice_field}).
    }\label{fig:field_manual}
    \makebox[\textwidth][c]{\includegraphics[height=.85\textheight]{../figures/country_comparison/field_manual_positive.pdf}} 
\end{figure}

\begin{figure}[h!]
    \caption[Manual classification of \textit{concerns} fields]{Manual classification of \textit{concerns} fields: ``What are your main concerns these days?'' (Question \ref{q:concerns_field}).
    }\label{fig:concerns_field}
    \makebox[\textwidth][c]{\includegraphics[height=.85\textheight]{../figures/country_comparison/field_concerns_manual_positive.pdf}} 
\end{figure}

\begin{figure}[h!]
    \caption[Manual classification of \textit{wish} fields]{Manual classification of \textit{wish} fields: ``What are your needs or wishes?'' (Question \ref{q:wish_field}).
    }\label{fig:wish_field}
    \makebox[\textwidth][c]{\includegraphics[height=.85\textheight]{../figures/country_comparison/field_wish_manual_positive.pdf}} 
\end{figure}

\begin{figure}[h!]
    \caption[Manual classification of \textit{issue} fields]{Manual classification of \textit{issue} fields: ``Can you name an issue that is important to you but is neglected in the public debate?'' (Question \ref{q:issue_field}).
    }\label{fig:issue_field}
    \makebox[\textwidth][c]{\includegraphics[height=.85\textheight]{../figures/country_comparison/field_issue_manual_positive.pdf}} 
\end{figure}

\begin{figure}[h!]
    \caption[Manual classification of \textit{injustice} fields]{Manual classification of \textit{injustice} fields: ``What according to you is the greatest injustice of all?'' (Question \ref{q:injustice_field}).
    }\label{fig:injustice_field}
    \makebox[\textwidth][c]{\includegraphics[height=.85\textheight]{../figures/country_comparison/field_concerns_manual_positive.pdf}} 
\end{figure}

\begin{figure}[h!]
    \caption[Conjoint analysis in France]{Conjoint analysis in France (Average Marginal Component Effect). \hfill (Question \ref{q:conjoint}).
    }\label{fig:conjoint_FR}
    \makebox[\textwidth][c]{\includegraphics[width=\textwidth]{../figures/FR/conjoint_EN-FR.pdf}} 
\end{figure}

\begin{figure}[h!]
    \caption[Conjoint analysis in Germany]{Conjoint analysis in Germany (Average Marginal Component Effect). \hfill (Question \ref{q:conjoint}).
    }\label{fig:conjoint_DE}
    \makebox[\textwidth][c]{\includegraphics[width=\textwidth]{../figures/DE/conjoint_EN-DE.pdf}} 
\end{figure}

\begin{figure}[h!]
    \caption[Conjoint analysis in Italy]{Conjoint analysis in Italy (Average Marginal Component Effect). \hfill (Question \ref{q:conjoint}).
    }\label{fig:conjoint_IT}
    \makebox[\textwidth][c]{\includegraphics[width=\textwidth]{../figures/IT/conjoint_EN-IT.pdf}} 
\end{figure}

\begin{figure}[h!]
    \caption[Conjoint analysis in Poland]{Conjoint analysis in Poland (Average Marginal Component Effect). \hfill (Question \ref{q:conjoint}).
    }\label{fig:conjoint_PL}
    \makebox[\textwidth][c]{\includegraphics[width=\textwidth]{../figures/PL/conjoint_EN-PL.pdf}} 
\end{figure}

\begin{figure}[h!]
    \caption[Conjoint analysis in Spain]{Conjoint analysis in Spain (Average Marginal Component Effect). \hfill (Question \ref{q:conjoint}).
    }\label{fig:conjoint_ES}
    \makebox[\textwidth][c]{\includegraphics[width=\textwidth]{../figures/ES/conjoint_EN-ES.pdf}} 
\end{figure}

\begin{figure}[h!]
    \caption[Conjoint analysis in the UK]{Conjoint analysis in the UK (Average Marginal Component Effect). \hfill (Question \ref{q:conjoint}).
    }\label{fig:conjoint_GB}
    \makebox[\textwidth][c]{\includegraphics[width=\textwidth]{../figures/GB/conjoint_EN-GB.pdf}} 
\end{figure}

\begin{figure}[h!]
    \caption[Conjoint analysis in Switzerland]{Conjoint analysis in Switzerland (Average Marginal Component Effect). \hfill (Question \ref{q:conjoint}).
    }\label{fig:conjoint_CH}
    \makebox[\textwidth][c]{\includegraphics[width=\textwidth]{../figures/CH/conjoint_EN-CH.pdf}} 
\end{figure} 

\begin{figure}[h!]
    \caption[Conjoint analysis in Japan]{Conjoint analysis in Japan (Average Marginal Component Effect). \hfill (Question \ref{q:conjoint}).
    }\label{fig:conjoint_JP}
    \makebox[\textwidth][c]{\includegraphics[width=\textwidth]{../figures/JP/conjoint_EN-JA.pdf}} 
\end{figure} 
\begin{figure}[h!]
    \caption[Conjoint analysis in the U.S.]{Conjoint analysis in the U.S. (Average Marginal Component Effect). \hfill (Question \ref{q:conjoint}).
    }\label{fig:conjoint_US}
    \makebox[\textwidth][c]{\includegraphics[width=\textwidth]{../figures/US/conjoint_EN.pdf}} 
\end{figure} 

\begin{figure}[h!]
    \caption[Conjoint analysis in France (French)]{Conjoint analysis in France (in French, Average Marginal Component Effect). \hfill (Question \ref{q:conjoint}).
    }\label{fig:conjoint_FR_original}
    \makebox[\textwidth][c]{\includegraphics[width=\textwidth]{../figures/FR/conjoint_FR.pdf}} 
\end{figure}

\begin{figure}[h!]
    \caption[Conjoint analysis in Germany (German)]{Conjoint analysis in Germany (in German, Average Marginal Component Effect). \hfill (Question \ref{q:conjoint}).
    }\label{fig:conjoint_DE_original}
    \makebox[\textwidth][c]{\includegraphics[width=\textwidth]{../figures/DE/conjoint_DE.pdf}} 
\end{figure}

\begin{figure}[h!]
    \caption[Conjoint analysis in Italy (Italian)]{Conjoint analysis in Italy (in Italian, Average Marginal Component Effect). \hfill (Question \ref{q:conjoint}).
    }\label{fig:conjoint_IT_original}
    \makebox[\textwidth][c]{\includegraphics[width=\textwidth]{../figures/IT/conjoint_IT.pdf}} 
\end{figure}

\begin{figure}[h!]
    \caption[Conjoint analysis in Poland (Polish)]{Conjoint analysis in Poland (in Polish, Average Marginal Component Effect). \hfill (Question \ref{q:conjoint}).
    }\label{fig:conjoint_PL_original}
    \makebox[\textwidth][c]{\includegraphics[width=\textwidth]{../figures/PL/conjoint_PL.pdf}} 
\end{figure}

\begin{figure}[h!]
    \caption[Conjoint analysis in Spain (Spanish)]{Conjoint analysis in Spain (in Spanish, Average Marginal Component Effect). \hfill (Question \ref{q:conjoint}).
    }\label{fig:conjoint_ES_original}
    \makebox[\textwidth][c]{\includegraphics[width=\textwidth]{../figures/ES/conjoint_ES-ES.pdf}} 
\end{figure}


\begin{figure}[h!]
    \caption[Average preferred revenue split (\textit{few})]{Average preferred revenue split for a global wealth tax (variant \textit{few}). (Question \ref{q:revenue_split_few}).
    }\label{fig:split_few_bars_nb0}
    \makebox[\textwidth][c]{\includegraphics[width=.8\textwidth]{../figures/country_comparison/split_few_bars_nb0.pdf}} 
\end{figure}

\begin{figure}[h!]
    \caption[Decomposition of preferred spendings in revenue split]{Decomposition of preferred shares for each spending item in the revenue split (\textit{All} countries together; variant \textit{few}). (Question \ref{q:revenue_split_few}).
    }\label{fig:split_few}
    \makebox[\textwidth][c]{\includegraphics[width=\textwidth]{../figures/all/split_few.pdf}} 
\end{figure}

\begin{figure}[h!]
    \caption[Decomposition of preferred spendings in revenue split]{Decomposition of preferred shares for each spending item in the revenue split (\textit{All} countries together; variant \textit{many}). (Question \ref{q:revenue_split_many}).
    }\label{fig:split_many}
    \makebox[\textwidth][c]{\includegraphics[width=.9\textwidth]{../figures/all/split_many.pdf}} 
\end{figure}

\begin{figure}[h!]
    \caption[Average preferred revenue split (\textit{many})]{Average preferred revenue split for a global wealth tax (variant \textit{many}). (Question \ref{q:revenue_split_many}).
    }\label{fig:split_many_global_mean}
    \makebox[\textwidth][c]{\includegraphics[width=\textwidth]{../figures/country_comparison/split_many_global_mean.pdf}} 
\end{figure} 

% \begin{figure}[h!]
%     \caption[]{. (Question \ref{q:gcs_support}).
%     }\label{fig:gcs_support}
%     \makebox[\textwidth][c]{\includegraphics[width=\textwidth]{../figures/country_comparison/split_few_mean.pdf}} 
% \end{figure}

% \begin{figure}[h!]
%     \caption[Amounts donated to plant trees.]{Amounts donated to plant trees, in the lottery. (Question \ref{q:donation}).
%     }\label{fig:donation}
%     \makebox[\textwidth][c]{\includegraphics[width=\textwidth]{../figures/country_comparison/donation.pdf}} 
% \end{figure}

% \begin{figure}[h!]
%     \caption[Support for the GCS and belief of support]{Support for the Global Climate Scheme and average belief regarding the support. (Questions \ref{q:gcs_support}-\ref{q:gcs_belief_own}).
%     }\label{fig:gcs_belief}
%     \makebox[\textwidth][c]{\includegraphics[width=\textwidth]{../figures/country_comparison/gcs_belief_mean.pdf}} 
% \end{figure} 

\begin{figure}[h!]
    \caption[Support for the NCS, GCS, ICS, and belief of support for GCS]{Support for the National, Global, and International Climate Schemes, and average belief regarding the support for the GCS. (Questions \ref{q:ncs_support}-\ref{q:ics_support}).
    }\label{fig:ncs_gcs_ics}
    \makebox[\textwidth][c]{\includegraphics[width=\textwidth]{../figures/country_comparison/ncs_gcs_ics_positive.pdf}} 
\end{figure} 

\begin{figure}[h!]
    \caption[Absolute support for plausible global redistribution policies]{Absolute support for plausible global redistribution policies (Percentage of \textit{Somewhat} or \textit{Strongly support}). (Question \ref{q:solidarity_support}).
    }\label{fig:solidarity_support_positive}
    \makebox[\textwidth][c]{\includegraphics[width=\textwidth]{../figures/country_comparison/solidarity_support_positive.pdf}} 
\end{figure}

\begin{figure}[h!]
    \caption[Preferred NCQG]{Preferred North-to-South climate grant funding in 2035 (NCQG, variant \textit{Short}). (Question \ref{q:ncqg}).
    }\label{fig:ncqg}
    \makebox[\textwidth][c]{\includegraphics[width=.9\textwidth]{../figures/all/ncqg.pdf}} 
\end{figure}

\begin{figure}[h!]
    \caption[Preferred NCQG]{Preferred North-to-South climate grant funding in 2035 (NCQG, variant \textit{Full}). (Question \ref{q:ncqg_full}).
    }\label{fig:ncqg_full}
    \makebox[\textwidth][c]{\includegraphics[width=.9\textwidth]{../figures/all/ncqg_full.pdf}} 
\end{figure}

\begin{figure}[h!]
    \caption[Support for an international wealth depending on country coverage]{Support for an international wealth tax with 30\% of revenue funding LICs, depending on the country coverage (\textit{Yes}/\textit{No} question). (Questions \ref{q:global_tax_support}-\ref{q:intl_tax_support}).
    }\label{fig:wealth_tax_heatmap}
    \makebox[\textwidth][c]{\includegraphics[width=\textwidth]{../figures/country_comparison/wealth_tax_support_positive.pdf}} 
\end{figure}

\begin{figure}[h!]
    \caption[Prefers a sustainable future]{Prefers a \textit{sustainable} rather than a \textit{business-as-usual} future. (Question \ref{q:sustainable_future}).
    }\label{fig:sustainable_future}
    \makebox[\textwidth][c]{\includegraphics[width=.9\textwidth]{../figures/all/sustainable_future.pdf}} 
\end{figure}

\begin{figure}[h!]
    \caption[Relative support for a global income tax on the richest to fund LICs]{Relative support for a global progressive income tax on the richest households to finance poverty reduction in the Global South (Percentage of \textit{Somewhat} or \textit{Strongly support} among non-\textit{Indifferent} responses). (Questions \ref{q:top1_tax_support}-\ref{q:top3_tax_support}).
    }\label{fig:top_tax_share}
    \makebox[\textwidth][c]{\includegraphics[width=\textwidth]{../figures/country_comparison/top_tax_share.pdf}} 
\end{figure}

\begin{figure}[h!]
    \caption[Absolute support for an income tax on top 1\% to fund LICs]{Absolute support for a global progressive income tax on the richest households to finance poverty reduction in the Global South (Percentage of \textit{Somewhat} or \textit{Strongly support}). (Questions \ref{q:top1_tax_support}-\ref{q:top3_tax_support}).
    }\label{fig:top_tax_positive}
    \makebox[\textwidth][c]{\includegraphics[width=\textwidth]{../figures/country_comparison/top_tax_positive.pdf}} 
\end{figure}

\begin{figure}[h!]
    \caption[\textit{Right} or \textit{Best} way to transfer resources to LICs]{``How do you evaluate each of these channels to transfer resources to reduce poverty in LICs?''\\ Percentage of \textit{Right} or \textit{Best} way (other options: \textit{Wrong} or \textit{Acceptable} way). (Question \ref{q:transfer_how}).
    }\label{fig:transfer_how_positive}
    \makebox[\textwidth][c]{\includegraphics[width=.9\textwidth]{../figures/country_comparison/transfer_how_positive.pdf}} 
\end{figure}

\begin{figure}[h!]
    \caption[\textit{Best} way to transfer resources to LICs]{``How do you evaluate each of these channels to transfer resources to reduce poverty in LICs?''\\ Percentage of \textit{Best} way (other options: \textit{Right}, \textit{Wrong} or \textit{Acceptable} way). (Question \ref{q:transfer_how}).
    }\label{fig:transfer_how_above_one}
    \makebox[\textwidth][c]{\includegraphics[width=.9\textwidth]{../figures/country_comparison/transfer_how_above_one.pdf}} 
\end{figure}

\begin{figure}[h!]
    \caption[\textit{Wrong} way to transfer resources to LICs]{``How do you evaluate each of these channels to transfer resources to reduce poverty in LICs?''\\ Percentage of \textit{Wrong} way (other options: \textit{Best}, \textit{Right} or \textit{Acceptable} way). (Question \ref{q:transfer_how}).
    }\label{fig:transfer_how_negative}
    \makebox[\textwidth][c]{\includegraphics[width=.9\textwidth]{../figures/country_comparison/transfer_how_negative.pdf}} 
\end{figure}

\begin{figure}[h!]
    \caption[Support for making all countries' GDP p.c. converge by 2100]{``Should governments actively cooperate to have all countries converge in terms of GDP per capita by the end of the century?'' (Question \ref{q:convergence_support}).
    }\label{fig:convergence_support}
    \makebox[\textwidth][c]{\includegraphics[width=.8\textwidth]{../figures/all/convergence_support.pdf}} 
\end{figure}

\begin{figure}[h!]
    \caption[Willingness to participate in a global movement for sustainable development]{``If there was a worldwide movement in favor of a global program to tackle climate change, implement taxes on millionaires and fund poverty reduction in low-income countries, to what extent would you be willing to be part of that movement? (Multiple answers possible)'' (Question \ref{q:global_movement}).
    }\label{fig:global_movement}
    \makebox[\textwidth][c]{\includegraphics[width=.8\textwidth]{../figures/country_comparison/global_movement_positive.pdf}} 
\end{figure}

\begin{figure}[h!]
    \caption[Would vote for a party in a global coalition for sustainable development]{``Let us call "your political party" the party you voted for in the last election, or the party that represents your views most closely.~\\\textbf{Imagine }there was \textbf{a worldwide coalition} of political parties in favor of a common program \textbf{to tackle climate change, implement taxes on millionaires and fund poverty reduction in low-income countries}.~\\\\\textbf{Would you be more likely to vote for your party if it were part of that coalition?}'' (Question \ref{q:vote_intl_coalition}).
    }\label{fig:vote_intl_coalition}
    \makebox[\textwidth][c]{\includegraphics[width=.7\textwidth]{../figures/all/vote_intl_coalition.pdf}} 
\end{figure}

\begin{figure}[h!]
    \caption[Agreement with rationales for global redistribution]{``Some people think that high-income countries should support low-income countries.~\\Among the different reasons given, which ones do you agree with? (Multiple answers possible)'' (Question \ref{q:why_hic_help_lic}).
    }\label{fig:why_hic_help_lic}
    \makebox[\textwidth][c]{\includegraphics[width=\textwidth]{../figures/country_comparison/why_hic_help_lic_positive.pdf}} 
\end{figure}

\begin{figure}[h!]
    \caption[Average custom redistribution]{Average custom global redistribution. (Question \ref{q:custom_redistr}).
    }\label{fig:custom_redistr_mean}
    \makebox[\textwidth][c]{\includegraphics[width=.9\textwidth]{../figures/all/custom_redistr_mean.png}} 
\end{figure}

\begin{figure}[h!]
    \caption[Median custom redistribution]{Global redistribution obtained from median custom parameters: 49\% of winners; 18\% of losers; degree of redistribution of 5 (out of 10). (Question \ref{q:custom_redistr}).
    }\label{fig:custom_redistr_median}
    \makebox[\textwidth][c]{\includegraphics[width=.9\textwidth]{../figures/questionnaire/survey_custom_redistr_median_zoom.png}} 
\end{figure} % TODO: tax rates: all/tax_radical_redistr

% \begin{figure}[h!]
%     \caption[Average well-being depending on the variant]{Average subjective well-being, depending on the variant. (Question \ref{q:well_being}).
%     }\label{fig:well_being}
%     \makebox[\textwidth][c]{\includegraphics[width=\textwidth]{../figures/country_comparison/well_being_mean.pdf}} 
% \end{figure}

\begin{figure}[h!]
    \caption[Comprehension question on GCS]{``\textit{Comprehension question: one respondent with the expected answer will get [amount\_lottery: \$100].}\\\\How would gasoline prices change as a result of the Global Climate Scheme? \\Gasoline prices would...'' (Correct answer: \textit{increase}) (Question \ref{q:gcs_comprehension}).
    }\label{fig:gcs_comprehension}
    \makebox[\textwidth][c]{\includegraphics[width=.7\textwidth]{../figures/all/gcs_comprehension.pdf}} 
\end{figure}

\begin{figure}[h!]
    \caption[Relative agreement: ``My taxes should go towards solving global problems'']{Relative agreement for: ``To what extent do you agree or disagree with the following statement? "My taxes should go towards solving global problems."'' (Percentage of \textit{Agree} or \textit{Strongly agree} among non-\textit{Neither agree nor disagree} responses). (Question \ref{q:my_tax_global_nation}).
    }\label{fig:my_tax_global_nation_share}
    \makebox[\textwidth][c]{\includegraphics[width=.9\textwidth]{../figures/country_comparison/my_tax_global_nation_share.pdf}} 
\end{figure}

\begin{figure}[h!]
    \caption[Absolute agreement: ``My taxes should go towards solving global problems'']{Absolute agreement for: ``To what extent do you agree or disagree with the following statement? "My taxes should go towards solving global problems."'' (Percentage of \textit{Agree} or \textit{Strongly agree}). (Question \ref{q:my_tax_global_nation}).
    }\label{fig:my_tax_global_nation_positive}
    \makebox[\textwidth][c]{\includegraphics[width=.9\textwidth]{../figures/country_comparison/my_tax_global_nation_positive.pdf}} 
\end{figure}

\begin{figure}[h!]
    \caption[Moral circle]{``Which group of people do you advocate for when you vote?'' (Question \ref{q:group_defended}).
    }\label{fig:group_defended_all}
    \makebox[\textwidth][c]{\includegraphics[width=.9\textwidth]{../figures/all/group_defended.pdf}} 
\end{figure}

% \begin{figure}[h!]
%     \caption[Moral circle (heatmap)]{``Which group of people do you advocate for when you vote?'' (Question \ref{q:group_defended}).
%     }\label{fig:group_defended_heatmap}
%     \makebox[\textwidth][c]{\includegraphics[width=.9\textwidth]{../figures/country_comparison/group_defended_5_positive.pdf}} 
% \end{figure}

\begin{figure}[h!]
    \caption[Feeling that the survey was politically biased]{``Do you feel that this survey was politically biased?'' (Question \ref{q:survey_biased}).
    }\label{fig:survey_biased}
    \makebox[\textwidth][c]{\includegraphics[width=.8\textwidth]{../figures/all/survey_biased.pdf}} 
\end{figure}

% \begin{figure}[h!]
%     \caption[]{. (Questions \ref{q:gcs_support}).
%     }\label{fig:radical_redistr_positive}
%     \makebox[\textwidth][c]{\includegraphics[width=.9\textwidth]{../figures/country_comparison/radical_redistr_positive.pdf}} 
% \end{figure} % TODO?

\begin{figure}[h!]
    \caption[Vote in the last election compared to actual results (entire population)]{Vote in the last election, compared to actual results on the entire population. (Questions \ref{q:voted}, \ref{q:vote}).
    }\label{fig:vote_representativeness}
    \makebox[\textwidth][c]{\includegraphics[width=.9\textwidth]{../figures/country_comparison/vote_representativeness.pdf}} 
\end{figure}

\begin{figure}[h!]
    \caption[Vote in the last election compared to actual results (among voters)]{Vote in the last election, compared to actual results among voters. (Questions \ref{q:voted}, \ref{q:vote}).
    }\label{fig:vote_pnr_out}
    \makebox[\textwidth][c]{\includegraphics[width=.9\textwidth]{../figures/country_comparison/vote_pnr_out.pdf}} 
\end{figure}

% \begin{figure}[h!]
%     \caption[]{. (Question \ref{q:gcs_support}).
%     }\label{fig:gcs_support}
%     \makebox[\textwidth][c]{\includegraphics[width=.9\textwidth]{../figures/country_comparison/gcs_support.pdf}} 
% \end{figure}


\renewcommand{\theenumi}{\arabic{enumi}}
\clearpage
\section{Questionnaire}\label{app:questionnaire}
The U.S. version of the questionnaire is presented. Features that vary across countries are put in square brackets within the question tex, as follows: [feature\_name: U.S. value]. Features values for each country are given in \href{https://github.com/bixiou/robustness_global_redistr/raw/main/questionnaire/sources.xlsx}{this spreadsheet}. 
Random branches or conditions for displaying the question are specified in square brackets before the question text (cf. Figure \ref{fig:flow} for the survey flow). The question text is followed by square brackets that refer to Figures and Tables presenting the question results, and the variable name(s) corresponding to the question. Finally, response options are displayed in italics. 
Unless otherwise specified, response is compulsory and a single response much be chosen.

\subsection*{Welcome} 
 \begin{enumerate} 
\item  \label{q:consent} Welcome to this survey!\\
This survey is \textbf{anonymous }and is conducted \textbf{for research} purposes on a representative sample of [sample\_size: 3,000] [nationality: American people].\\
~\\
It takes around 20 min to complete.\\
~\\
The survey contains lotteries and awards for those who get the correct answer to some comprehension questions.\\
If you are attentive and lucky, \textbf{you can win up to [amount\_lottery: \$100]}.\\
~\\
Please answer every question carefully.\\
~\\
By clicking on the button below, you consent to the terms and conditions.

\end{enumerate} 

 \subsection*{Socio-demographics} 
 \begin{enumerate}[resume] 
\item  \label{q:gender} What is your gender? [%\textit{Figure \ref{fig:gender}}; 
\verb|gender|]
  \\ \textit{Woman; Man; Other}

\item  \label{q:hidden_country} What is your country? [%\textit{Figure \ref{fig:hidden_country}}; 
\verb|hidden_country|]


\item  \label{q:age_exact} What is your age? [%\textit{Figure \ref{fig:age_exact}}; 
\verb|age_exact, age|]
  \\ \textit{Below 18; 18 to 20; 21 to 24; 25 to 29; 30 to 34; 35 to 39; 40 to 44; 45 to 49; 50 to 54; 55 to 59; 60 to 64; 65 to 69; 70 to 74; 75 to 79; 80 to 84; 85 to 89; 90 to 99; 100 or above}

\item  \label{q:foreign} Were you or your parents born in a foreign country?~ [\textit{Figure \ref{fig:foreign}}; 
\verb|foreign|]
  \\ \textit{Yes, I was born in a foreign country; Not me but both my parents were born in a foreign country; Not me but one of my parents was born in a foreign country; No, I was born in this country and my parents too}

\item  \label{q:couple} Do you live with your partner (if you have one)? [%\textit{Figure \ref{fig:couple}}; 
\verb|couple|]
  \\ \textit{Yes; No}

\item  \label{q:hh_size} How many people are there in your household? \\The household includes: \textbf{you}, your spouse, \textbf{your family members} who live with you, and your dependents (not flatmates). [%\textit{Figure \ref{fig:hh_size}}; 
\verb|hh_size|]
  \\ \textit{1; 2; 3; 4; 5 or more}

\item  \label{q:Nb_children__14} How many children under the age of 14 live with you? [%\textit{Figure \ref{fig:Nb_children__14}}; 
\verb|Nb_children__14|]
  \\ \textit{0; 1; 2; 3; 4 or more}

\item ~[new page] \label{q:race} [\textit{Only in: US}] What race or ethnicity do you identify with? (Multiple answers are possible) [%\textit{Figure \ref{fig:race}}; 
\verb|race|]
  \\ \textit{White; Black or African American; Hispanic; Asian; American Indian or Alaskan Native; Native Hawaiian or Pacific Islander; Other; Prefer not to say}

\item  \label{q:income} What is the \textbf{[periodicity\_text: monthly] [income\_type: gross] income of your household}, [income\_type\_long: after taxes and transfers]?

This includes all sources of income: wages, pensions, welfare payments, property income, dividends, self-employment earnings, Social Security benefits, and income from other sources. [%\textit{Figure \ref{fig:income}}; 
\verb|income|]
  \\ ~[\textit{All but RU, US}: Custom thresholds, taking into account household composition Questions \ref{q:couple}-\ref{q:Nb_children__14}, and corresponding to the country's deciles and quartiles of standard of living, cf. the sheet ``Income'' in \href{https://github.com/bixiou/robustness_global_redistr/raw/main/questionnaire/source.xlsx}{this spreadsheet}; \\ \textit{RU, US}: Items based on household total income deciles and quartiles, namely in US: \textit{Less than \$17,000; between \$17,001 and \$30,000; between \$30,001 and \$36,000; between \$36,001 and \$43,000; between \$43,001 and \$56,000; between \$56,001 and \$72,000; between \$72,001 and \$91,000; between \$91,001 and \$115,000; between \$115,001 and \$130,000; between \$130,001 and \$150,000; between \$150,001 and \$213,000; More than \$213,000; I prefer not to answer}]

\item  \label{q:education} What is your highest completed education level? [%\textit{Figure \ref{fig:education}}; 
\verb|education|]
  \\ ~[Country-specific, usually: 0-1 Primary or less; 2 Medium school; 2 Some high school; 3 High school diploma; 3-4 Vocational training; 5 Short-cycle tertiary; 6 Bachelor's; 7-8 Master's or higher]

\item  \label{q:employment_status} What is your employment status? [%\textit{Figure \ref{fig:employment_status}}; 
\verb|employment_status|]
  \\ \textit{Full-time employed; Part-time employed; Self-employed; Unemployed (searching for a job); Student; Retired; Inactive (not searching for a job)}

\item  \label{q:zipcode} [\textit{Only the first digits asked in RU, SA}] What is your zipcode?\\
We ask for the zipcode to balance the sample in terms of degree of urbanization (rural, town or city). The survey will be terminated if your zipcode is not recognized. [%\textit{Figure \ref{fig:zipcode}}; 
\verb|zipcode|]


\item  \label{q:home} Are you a homeowner or a tenant? (Multiple answers are possible) [%\textit{Figure \ref{fig:home}}; 
\verb|home_owner|]
  \\ \textit{Tenant; Owner; Landlord renting out property; Hosted free of charge}

\item ~[new page] \label{q:millionaire} How likely are you to become a millionaire at some point in your life? [\textit{Figure \ref{fig:millionaire}}; 
\verb|millionaire|]
  \\ \textit{Very unlikely; Unlikely; Likely; Very likely; I am already a millionaire}

\item  \label{q:voted} [\textit{Except in: RU, SA}] Did you vote in the [election: 2024 presidential election]? [\textit{Figures \ref{fig:vote_representativeness}-\ref{fig:vote_pnr_out}}; 
\verb|voted|]
  \\ \textit{Yes; No; Prefer not to say; I didn't have the right to vote in [country\_name: the United States].}

\end{enumerate} 

 \subsection*{Vote} 
 \begin{enumerate}[resume] 
\item  \label{q:nationality_SA} [\textit{Only in: SA}] What is your nationality?\\If you have both the Saudi and a foreign nationality, choose "Saudi". [%\textit{Figure \ref{fig:nationality_SA}}; 
\verb|nationality_SA|]
  \\ \textit{Saudi; India; Bangladesh; Syria; Yemen; Egypt; Pakistan; Indonesia; Philippines; Sudan; Myanmar; Jordan; Sri Lanka; Nepal; Turkey; Somalia; Lebanon; Other}

\item  \label{q:vote} [\textit{Except in: RU, SA}] [\textit{If voted}: Which candidate did you vote for in the [election: 2024 presidential election]?; \textit{Otherwise}: Even if you did not vote in the [election: 2024 presidential election], please indicate the candidate that you were most likely to have voted for or who represents your views more closely.] [\textit{Figures \ref{fig:vote_representativeness}-\ref{fig:vote_pnr_out}}; 
\verb|vote|]
  \\ ~[Candidates/parties with at least 1\% of votes, e.g. in US: \textit{Harris; Trump; Other; Prefer not to say}. In FR, IT, PL, ES, election is the 2024 European election]

% \item  \label{q:vote_GB} [text\_vote: Which candidate did you vote for in the [election: 2024 European Parliament election]?\\Even if you did not vote in the [election: 2024 European Parliament election], please indicate the candidate that you were most likely to have voted for or who represents your views more closely.] [\textit{Figure \ref{fig:vote_GB}}; 
% \verb|vote_GB|]
%   \\ \textit{Conservative; Labour; Liberal Democrats; SNP; Prefer not to say; Green; DUP; Sinn Féin; Other; Reform UK; Plaid Cymru; Alliance Party of Northern Ireland}

% \item  \label{q:vote_FR} [text\_vote: Which candidate did you vote for in the [election: 2024 European Parliament election]?\\Even if you did not vote in the [election: 2024 European Parliament election], please indicate the candidate that you were most likely to have voted for or who represents your views more closely.] [\textit{Figure \ref{fig:vote_FR}}; 
% \verb|vote_FR|]
%   \\ \textit{Renaissance, MoDem \& Horizons; Rassemblement National; La France insoumise; Les Écologistes – EÉLV; Préfère ne pas répondre; Les Républicains; Résistons (Jean Lassalle); Reconquête; Autre; Parti Socaliste \& Place publique; Parti Communiste Français; Parti animaliste}

% \item  \label{q:vote_CH} [text\_vote: Which candidate did you vote for in the [election: 2024 European Parliament election]?\\Even if you did not vote in the [election: 2024 European Parliament election], please indicate the candidate that you were most likely to have voted for or who represents your views more closely.] [\textit{Figure \ref{fig:vote_CH}}; 
% \verb|vote_CH|]
%   \\ \textit{Social Democratic Party; Swiss People's Party; The Centre; Green Liberal Party; Préfère ne pas répondre; Green Party; Evangelical People's Party; Autre; The Liberals; Federal Democratic Union}

% \item  \label{q:vote_PL} [text\_vote: Which candidate did you vote for in the [election: 2024 European Parliament election]?\\Even if you did not vote in the [election: 2024 European Parliament election], please indicate the candidate that you were most likely to have voted for or who represents your views more closely.] [\textit{Figure \ref{fig:vote_PL}}; 
% \verb|vote_PL|]
%   \\ \textit{United Right (Law and Justice, Sovereign Poland...); Civic Coalition (Civic Platform, Polish Initiative...); Polish People's Party; Prefer not to say; The Left (New Left...); Other; Confederation (New Hope, National Movement, Confederation of the Polish Crown...); Poland 2050}

% \item  \label{q:vote_IT} [text\_vote: Which candidate did you vote for in the [election: 2024 European Parliament election]?\\Even if you did not vote in the [election: 2024 European Parliament election], please indicate the candidate that you were most likely to have voted for or who represents your views more closely.] [\textit{Figure \ref{fig:vote_IT}}; 
% \verb|vote_IT|]
%   \\ \textit{PD; FdI; League; Prefer not to say; SUE; Azione; FI – NM; AVS; PTD; Libertà; M5S; Other}

% \item  \label{q:vote_ES} [text\_vote: Which candidate did you vote for in the [election: 2024 European Parliament election]?\\Even if you did not vote in the [election: 2024 European Parliament election], please indicate the candidate that you were most likely to have voted for or who represents your views more closely.] [\textit{Figure \ref{fig:vote_ES}}; 
% \verb|vote_ES|]
%   \\ \textit{PSOE; PP; Sumar; Prefer not to say; Podemos; Junts UE; Ahora Repúblicas; SALF; CEUS; Vox; Other}

% \item  \label{q:vote_DE} [text\_vote: Which candidate did you vote for in the [election: 2024 European Parliament election]?\\Even if you did not vote in the [election: 2024 European Parliament election], please indicate the candidate that you were most likely to have voted for or who represents your views more closely.] [\textit{Figure \ref{fig:vote_DE}}; 
% \verb|vote_DE|]
%   \\ \textit{AfD; CDU/CSU; BSW; Prefer not to say; Die Linke; FW; Grüne; FDP; Volt; Die Partei; SPD; Other; Tierschutzpartei}

% \item  \label{q:vote_JP} [text\_vote: Which candidate did you vote for in the [election: 2024 European Parliament election]?\\Even if you did not vote in the [election: 2024 European Parliament election], please indicate the candidate that you were most likely to have voted for or who represents your views more closely.] [\textit{Figure \ref{fig:vote_JP}}; 
% \verb|vote_JP|]
%   \\ \textit{CDP; LDP; Reiwa Shinsengumi; Prefer not to say; Sanseitō; CPJ; Komeito; JCP; SDP; Other; Ishin JIP; DPFP}

\end{enumerate} 

 \subsection*{Open-ended field} 
 [\textit{Four random branches}; \textit{Figures \ref{fig:field_keyword}-\ref{fig:injustice_field}}; 
 \verb|field, variant_field|] 
 \begin{enumerate}[resume] 
\item  \label{q:concerns_field} ~[Branch: concerns] What are your main concerns these days? [\textit{Figure \ref{fig:concerns_field}}; 
\verb|concerns_field|]


\item  \label{q:wish_field} ~[Branch: wish] What are your needs or wishes? [\textit{Figure \ref{fig:wish_field}}; 
\verb|wish_field|]


\item  \label{q:issue_field} ~[Branch: issue] Can you name an issue that is important to you but is neglected in the public debate? [\textit{Figure \ref{fig:issue_field}}; 
\verb|issue_field|]


\item  \label{q:injustice_field} ~[Branch: injustice] What according to you is the greatest injustice of all?\\ 
~[\textit{Figure \ref{fig:injustice_field}}; 
\verb|injustice_field|]


\end{enumerate} 

 \subsection*{Conjoint analysis} 
 \begin{enumerate}[resume] 
\item  \label{q:conjoint} [\textit{Except in: RU, SA}] Imagine if the two top candidates in your constituency in the next general election campaigned with the following policies in their party's platforms. \\\\Which of these candidates would you vote for?  
~\\

\begin{tabular}{@{\extracolsep{5pt}}|c|c|c|} 
    \hline \\[-1.8ex] 
    \textbf{Candidate A} & \textbf{Candidate B} & \\ \hline \\[-1.8ex]
    ~[Random policy] & [Random policy] & [Policy field in random order] \\ 
    ~[Random policy] & [Random policy] & [Policy field in random order] \\ 
    ~[Random policy] & [Random policy] & [Policy field in random order] \\ 
    ~[Random policy] & [Random policy] & [Policy field in random order] \\ 
    ~[Random policy] & [Random policy] & [Policy field in random order] \\ 
    \hline 
\end{tabular}  

~\\~[\textit{Figures \ref{fig:conjoint}, \ref{fig:conjoint_FR}-\ref{fig:conjoint_ES_original}}; 
\verb|conjoint|]
  \\ \textit{Candidate A; Candidate B; Neither of them}

\end{enumerate} 

 \subsection*{Revenue split of global tax} 
 [\textit{Two random branches};  \verb|field, variant_split|] 
 \begin{enumerate}[resume] 
\item ~[Branch: Few] \label{q:revenue_split_few} Imagine a wealth tax applied to households with a net worth above [tax\_threshold: \$5 million], implemented in every country around the world.
~\\\\ 
~[tax\_country\_name: In the U.S.], the tax revenues collected would be [tax\_revenue: \$514 billion] per year (that is, [tax\_revenue\_gdp: 2]\% of [tax\_country\_gdp: U.S. GDP]), while it would be [LIC\_revenue: \$1 billion] in all low-income countries combined (700 million people live in a low-income country, most of them in Africa).
Each country would retain part of the revenues it collects and use it for different domestic purposes. The remaining part would be pooled globally to finance sustainable development in low-income countries.
~\\\\\textbf{What percentage of the global wealth tax revenue should be allocated to each category?} \\\textbf{The total allocation must sum to 100\%.}\\\\ 
~[\textit{Figures \ref{fig:split}, \ref{fig:split_few_bars_nb0}-\ref{fig:split_few}}; 
\verb|revenue_split_few|]
  \\ \textit{Domestic: Education and Healthcare; Domestic: Social welfare programs; Domestic: Reduction in the federal income tax; Domestic: Reduction of the deficit; Global: Education, Healthcare and Renewable energy in low-income countries}

\item ~[Branch: Many] \label{q:revenue_split_many} Imagine a wealth tax applied to households with net worth above [tax\_threshold: \$5 million], implemented in all countries around the world.
~\\\\ 
~[tax\_country\_name: In the U.S.], the tax revenues collected would be [tax\_revenue: \$514 billion] per year (that is, [tax\_revenue\_gdp: 2]\% of [tax\_country\_gdp: U.S. GDP]), while it would be [LIC\_revenue: \$1 billion] in all low-income countries combined (700 million people live in a low-income country, most of them in Africa).
Each country would retain part of the revenues it collects and use it for different domestic purposes. The remaining part would be pooled globally to finance sustainable development.
~\\\\\textbf{What percentage of the global wealth tax revenue should be allocated to each category?}~\\\textbf{The total allocation must sum to
100\%.}\\\\ 
~[\textit{Figures \ref{fig:split}, \ref{fig:split_many}-\ref{fig:split_many_global_mean}}; 
\verb|revenue_split_many|]
  \\ ~[Five items are chosen at random among the 13 possible ones: \textit{Domestic: Education and Research; Domestic: Healthcare; Domestic: Defense; Domestic: Deficit reduction; Domestic: Justice and Police; Domestic: Retirement pensions; Domestic: Social welfare programs; Domestic: Infrastructure (public transport, water systems...); Domestic: Income tax reduction; Global: Education and Healthcare in low-income countries; Global: Renewable energy and infrastructure to cope with climate change; Global: Loss and Damage Fund (to rebuild after climate disasters); Global: Forestation and biodiversity projects}]


\end{enumerate} 

 \subsection*{Warm glow -- moral substitute} 
 [\textit{Three random branches: NCS; Donation; control group};  \verb|variant_warm_glow|] 
 \begin{enumerate}[resume] 
\item ~[Branch: NCS] \label{q:ncs_support} Do you agree with the following policy?
~\\
Climate Scheme:~\\
To meet the national climate target, a limited number of permits to emit greenhouse gases would be issued nationally. Polluting firms would be required to buy permits to cover their greenhouse gas emissions. Such a policy would~make fossil fuel companies pay~for their emissions and gradually raise the price of fossil fuels.~Higher prices would encourage people and companies to use less fossil fuels, reducing greenhouse gas emissions.\\
The revenues generated by the sale of permits would finance an equal cash transfer.\textbf{~}Each [country\_adjective: American] would receive [amount\_expenses: \$115][periodicity: per month], thereby offsetting~price increases for the average [country\_adjective: American].\\
~\\
\textbf{Do you support the Climate Scheme?} [\textit{Figures \ref{fig:ics}, \ref{fig:ncs_gcs_ics}}; 
\verb|ncs_support|]
  \\ \textit{Yes; No}

\item ~[Branch: Donation] \label{q:donation} By taking this survey, you will be automatically entered into a lottery to win up to [amount\_lottery: \$100]. \\Should you be selected in the lottery, you will have the option to channel a part of this additional compensation to the charity \textit{Just One Tree} to plant trees.\\\\\textbf{In case you win the lottery, what share of the [amount\_lottery: \$100 prize] would you donate to plant trees?} [\textit{Figures \ref{fig:warm_glow_substitute}, \ref{fig:donation}
}; 
\verb|donation|]
  \\ \textit{Share to plant trees}

\end{enumerate} 

 \subsection*{Cap \& Share} 
 \begin{enumerate}[resume] 
\item  \label{q:gcs_support} Do you support the following policy?\\
To ensure that you have attentively read the description,~we will ask some comprehension questions later in the survey: those who get correct answers can win [amount\_lottery: \$100].
~\\
Global Climate Scheme:~\\\\
In 2015, all countries agreed to contain global warming "well below +2~\textdegree{}C". To achieve this,~there is a maximum amount of greenhouse gases we can emit globally.~\\\\
To meet the climate target, a limited number of permits to emit greenhouse gases would be issued globally. Polluting firms would be required to buy permits to cover their greenhouse gas emissions. Such a policy would~make fossil fuel companies pay~for their emissions and gradually raise the price of fossil fuels.~Higher prices would encourage people and companies to use less fossil fuels, reducing greenhouse gas emissions.\\\\
In accordance with the principle that each human has an equal right to pollute, the revenues generated by the sale of permits could finance a global basic income.~Every adult would receive [amount\_bi: \$20][periodicity: per month], thereby lifting 600 million people who earn less than \$2 a day out of extreme poverty.\\
The typical [national: American] would lose out financially [amount\_lost: \$105][periodicity: per month]~(as he or she would face around [price\_increase: 2]\% in price increases, which is higher than the [amount\_bi: \$20][periodicity: per month] they would receive).\\\\
The policy could be implemented as soon as 100 countries agree to it. Countries that would refuse to take part in the policy could face sanctions (like tariffs) from the rest of the world and would be excluded from the basic income program.\\\\
~\\\textbf{
Do you support the Global Climate Scheme?
} [\textit{Figures \ref{fig:ics}, \ref{fig:warm_glow_substitute}, \ref{fig:ncs_gcs_ics}}; 
\verb|gcs_support|]
  \\ \textit{Yes; No}\\\\
~[new page] [\textit{Two random branches: own; US}; \textit{Figure \ref{fig:ncs_gcs_ics}}; \verb|gcs_belief, variant_belief|] 
\item ~[Branch: US] \label{q:gcs_belief_us} According to you, \textbf{what percentage of [belief\_nationality: \textit{All but US: Americans; US: Europeans}] would answer \textit{Yes }to the previous question} (considering that typical [belief\_nationality] would lose [belief\_loss: \$140] per month from the Global Climate Scheme)\textbf{?}\\ The respondent who is closest to the correct value will get [amount\_lottery: \$100]. %[\textit{Figure \ref{fig:gcs_belief_us}}; 
% \verb|gcs_belief_us|]
  \\ \textit{Percentage of [belief\_nationality] in favor of Global Climate Scheme}

\item ~[Branch: own] \label{q:gcs_belief_own} According to you, \textbf{what percentage of \textit{[nationality: fellow citizens]} would answer \textit{Yes }to the previous question?}\\ The respondent who is closest to the correct value will get [amount\_lottery: \$100]. %[\textit{Figure \ref{fig:gcs_belief_own}}; 
% \verb|gcs_belief_own|]
  \\ \textit{Percentage of [nationality: fellow citizens] in favor of Global Climate Scheme}

\end{enumerate} 

 \subsection*{Cap \& Share non-universal} 
 ~[\textit{Four random branches: low; mid; high; high\_color}; \textit{Figures \ref{fig:ics}, \ref{fig:ncs_gcs_ics}}; 
 \verb|ics_support|] 
 \begin{enumerate}[resume] 
\item ~[Branch: low]  \label{q:gcs_low} Below is a map showing a possible set of countries that would participate in the Global Climate Scheme previously described.\\
~\\
These countries include India, the European Union, as well as all Africa, Latin America, South-Asia and South-East Asia.\\
Collectively, these [nb\_countries\_low: 145] countries account for [emissions\_low\_without: 40]\% of global emissions (if [ics\_country: the U.S.] joined them, [emissions\_low\_with: 40]\% of global emissions would be covered).\\
~\\ 

\item ~[Branch: mid] \label{q:gcs_mid} Below is a map showing a possible set of countries that would participate in the Global Climate Scheme previously described.\\
~\\
These countries include China, India, as well as all Africa, Latin America, South-Asia and South-East Asia.\\
Collectively, these 119 countries account for 56\% of global emissions (if [ics\_country: the U.S.] joined them, [emissions\_mid\_with: 70]\% of global emissions would be covered).\\
~\\ 

\item ~[Branch: high]  \label{q:gcs_high} Below is a map showing a possible set of countries that would participate in the Global Climate Scheme previously described.\\
~\\
These countries include China, India, [text\_countries\_high: the European Union, Japan, the United Kingdom], Canada, South Korea, as well as all Africa, Latin America, South-Asia and South-East Asia.~\\
Collectively, these [nb\_countries\_high: 153] countries account for [emissions\_high\_without: 71]\% of global emissions (if [ics\_country: the U.S.] joined them, [emissions\_high\_with: 86]\% of global emissions would be covered).\\
~\\ 

\item ~[Branch: high\_color]  \label{q:gcs_high_color} Below is a map showing a possible set of countries that would participate in the Global Climate Scheme previously described.\\
~\\
These countries include China, India, [text\_countries\_high: the European Union, Japan, the United Kingdom], Canada, South Korea, as well as all Africa, Latin America, South-Asia and South-East Asia. \\
Collectively, these [nb\_countries\_high: 153] countries account for [emissions\_high\_without: 72]\% of global emissions (if [ics\_country: the U.S.] joined them, [emissions\_high\_with: 86]\% of global emissions would be covered).\\\\Note that a provision would prevent the Global Climate Scheme from harming low- and middle-income countries: this is why countries like China, Mexico, or Egypt are in white on the map (they would neither win nor lose financially).\\


\item  \label{q:ics_support} Do you support [ics\_country: the U.S.] joining the Global Climate Scheme, in case it is adopted by the above countries? [\textit{Figures \ref{fig:ics}, \ref{fig:ncs_gcs_ics}}; 
\verb|ics_support|]
  \\ \textit{Yes; No}

\end{enumerate} 

 \subsection*{Warm glow -- realism} 
 \begin{enumerate}[resume] 
\item ~[\textit{Two random branches: with or without this informational text.}] \label{q:info_solidarity} To ensure that you have attentively read the description below, we will ask some comprehension questions later in the survey: those who get correct answers can win \$100.

~\\\\In several international organizations, \textbf{countries have agreed to demonstrate some degree of solidarity in addressing global challenges}.\\
Negotiations are ongoing to implement specific mechanisms for sustainable development.\\\\Here are a few examples:\\🚢~In 2025, to reduce carbon emissions from shipping, \textbf{the International Maritime Organization adopted an international levy on excess emissions from maritime fuel, that should partly finance low-income countries}.\\📦~Since 1970, \textbf{developed countries have agreed to contribute 0.7\% of their GDP in foreign aid} and development assistance.\\
🌱 In international climate negotiations, \textbf{developed countries have committed to finance climate action in developing countries}. In 2009, they committed to provide \$100 billion per year by 2020. In 2023, all countries agreed to set up a fund to help vulnerable countries cope with loss and damage from climate change. In 2024, the \$100 billion goal was increased to \$300 billion per year by 2035.\\📈~In 2021, 136 countries adopted a minimum tax rate of 15\% on multinational profits.\\💎 In 2024, under the leadership of Brazil, \textbf{the G20 considered the introduction of a global tax} of 2\% \textbf{on }the wealth of \textbf{billionaires}.
~\\🌐~In 2024, the UN General Assembly adopted the Pact for the Future, which foresees a reform of the UN Security Council to limit the power of its five permanent member and expand it to new members.\\🔄 Led by the Prime Minister of Barbados and supported by the UN Secretary General, the Bridgetown initiative seeks a new financial system that would drive financial resources towards climate action and sustainable development. [\textit{Figure \ref{fig:warm_glow_realism}}; 
\verb|info_solidarity|]


\item  \label{q:likely_solidarity} According to you, how likely is it that international policies involving significant transfers from high-income countries to low-income countries will be introduced in the next 15 years? [\textit{Figure \ref{fig:warm_glow_realism}}; 
\verb|likely_solidarity|]
  \\ \textit{Very unlikely; Unlikely; Likely; Very likely}

\item  \label{q:solidarity_support} Do you support or oppose the following policies?\\
~\\ 
~[\textit{Only in PL, SA}: (As some items refer to ``developed countries'', note that we consider [Saudi Arabia] to be a developed country in this question.)] [\textit{Figures \ref{fig:solidarity_support_share}, \ref{fig:solidarity_support_positive}-\ref{fig:share_solidarity_opposed}}; 
\verb|solidarity_support|] \\
~[Item order is randomized]
\begin{itemize}
    \item Institutions like the World Bank investing in many more sustainable projects in lower-income countries, and offering lower interest rates (the Bridgetown initiative)
    \item Developed countries financing a fund to help vulnerable countries cope with loss and damage from climate change
    \item Expanding the UN Security Council (in charge of peacekeeping) to new permanent members such as India, Brazil, and the African Union, and restricting the use of the veto
    \item Raising the globally agreed minimum tax rate on profits of multinational firms from 15\% to 35\%, closing loopholes and allocating revenues to countries where sales are made
    \item Debt relief for vulnerable countries by suspending repayments until they are better able to repay, promoting their development
    \item An international levy on carbon emissions from shipping, funding national budgets in proportion to population
    \item An international levy on carbon emissions from aviation, raising ticket prices by 30\% and funding national budgets in proportion to population
    \item Developed countries providing \$300 billion a year (0.4\% of their GDP) to finance climate action in developing countries
    \item Developed countries contributing at least 0.7\% of their GDP in foreign aid and development assistance
    \item A minimum tax of 2\% on the wealth of billionaires, in voluntary countries
\end{itemize}
\textit{Strongly oppose; Somewhat oppose; Indifferent; Somewhat support; Strongly support}
\end{enumerate} 

 \subsection*{NCQG} 
 [\textit{Two random branches: Full; Short}; %\textit{Figure \ref{fig:field}}; 
 \verb|ncqg_fusion, variant_ncqg|] 
 \begin{enumerate}[resume] 
% \item  \label{q:maritime_split} This year, to meet the global climate targets, the International Maritime Organization is designing a global levy on shipping carbon emissions.\\\\\textbf{According to you, what percentage of the revenue from a maritime fuel levy should be allocated to each category below?} The total must be 100\%.\\ 
% ~[\textit{Figure \ref{fig:maritime_split}}; 
% \verb|maritime_split|]
%   \\ \textit{Fostering sustainable transition in the least developed countries and small islands states; Return revenues to shipping companies to prevent increases in shipping costs; Finance research, development and deployment for zero-emission fuels and ships}

\item ~[Branch: Full] \label{q:ncqg_full} \textbf{At international climate negotiations, developing countries call for larger provision of "climate finance": the financing of climate action from developed countries in developing countries.} [developed\_note: (Note that we consider Saudi Arabia to be a developed country in this question.)]\\\\\textbf{There are two kinds of climate finance: grants (that is, donations) and loans. In 2022, \$26 billion was provided as grants and the rest as loans, for a total of \$116 billion.~}\\\\In 2009, developed countries agreed to mobilize \$100 billion per year in climate finance by 2020. In 2024, they committed to raise this goal to \$300 billion by 2035. None of the goals specify which share should be provided as grants.\\\\Below are different positions on the amount of climate finance that should be provided in 2035, all expressed in grant-equivalent terms (that is, not counting loans):\\-~ ~ ~ ~ \$0: There should be no contributions from developed countries to climate action in developing countries.\\-~ ~ ~ \$26 billion (0.04\% of developed countries' GDP): The current amount, consistent with the old (2020) goal.\\-~ ~ \$100 billion (0.14\% of GDP): The old (2020) goal, if all climate finance were provided as grants.\\-~ ~ \$300 billion (0.43\% of GDP): The new (2035) goal, if all climate finance were provided as grants.\\-~ ~ \$600 billion (0.86\% of GDP):~The goal called for by India, a position shared by most developing countries.\\- \$1,000 billion (1.43\% of GDP): The goal called for by Climate Action Network (a network of NGOs including Greenpeace, Oxfam, and WWF).\\- \$5,000 billion (7.14\% of GDP): The goal called for by Demand Climate Justice (a network of NGOs including 350.org and~the World Council of Churches)\\\\\textbf{If you could choose the amount of climate finance provided by developed countries to developing countries in 2035, what amount would you choose (in grant-equivalent terms)?}\\ 
~[\textit{Figure \ref{fig:ncqg_full}}; 
\verb|ncqg_full|]\\
~[Item order is randomly reversed or not]
  \\ \textit{\$0; \$300 billion; \$600 billion; \$26 billion; \$100 billion; \$1,000 billion; \$5,000 billion}

\item ~[Branch: Short] \label{q:ncqg} \textbf{"Climate finance" designates the financing of climate action from developed countries in developing countries.} [developed\_note: (Note that we consider Saudi Arabia to be a developed country in this question.)]\\\\\textbf{There are two kinds of climate finance: grants (that is, donations) and loans. The large majority is currently provided as loans.~}\\\\In 2009, developed countries agreed to mobilize \$100 billion per year in climate finance. In 2024, they committed to triple this goal by 2035. None of the goals specify which share should be provided as grants.~\\At international climate negotiations, developing countries call for larger provision of climate finance, particularly in the form of grants.\\\\\textbf{If you could choose the level of climate finance provided by developed countries to developing countries in 2035, what would you choose?}\\ 
~[\textit{Figure \ref{fig:ncqg}}; 
\verb|ncqg|]\\
~[Item order is randomly flipped or not]
  \\ \textit{Stop all provision of climate finance.; \\Reduce the provision of climate finance.; \\Maintain current contributions (\$26 billion per year in grants, that is 0.04\% of developed countries' GDP, and \$80 billion in loans, or 0.1\% of GDP).; \\ Meet the newly agreed goal by tripling grants and loans (\$100 billion in grants, or 0.15\% of GDP).; \\ Increase climate finance to a level between what developed countries have agreed and what developing countries are asking for (\$300 billion in grants, or 0.45\% of GDP).; \\Increase climate finance to match what developing countries are asking for (\$600 billion in grants, or 0.9\% of GDP).; \\Increase climate finance to match what NGOs are asking for (at least \$1,000 billion per year in grants, that is 1.4\% of GDP, is what Greenpeace, Oxfam, WWF, and the World Council of Churches ask for).}

\end{enumerate} 

 \subsection*{Wealth tax depending on sets of countries} 
 [\textit{Three random branches: Global; HIC; Int'l}; \textit{Figures \ref{fig:wealth_tax}, \ref{fig:wealth_tax_heatmap}}; 
 \verb|wealth_tax_support, variant_wealth_tax|] 
 \begin{enumerate}[resume] % TODO: plus condensé ?
\item ~[Branch: Global] \label{q:global_tax_support} \textbf{Imagine an international tax on individuals with net worth above [wealth\_threshold: \$1 million].~}\\Only wealth above [wealth\_threshold: \$1 million] would be taxed, at a rate of 2\%. Each country would retain 70\% of the revenues it collects, while 30\% would be pooled at the global level to finance public services in low-income countries (in particular, access to drinking water, healthcare, and education in Africa). \\\\Say we are in 2030. \textbf{Imagine that all other countries in the world adopt this policy. \\Do you support [country\_name: the United States] adopting this international tax on millionaires?}
  \\ \textit{Yes; No}

\item ~[Branch: HIC] \label{q:hic_tax_support} \textbf{Imagine an international tax on individuals with net worth above [wealth\_threshold: \$1 million].~}\\Only wealth above [wealth\_threshold: \$1 million] would be taxed, at a rate of 2\%. Each country would retain 70\% of the revenues it collects, while 30\% would be pooled at the global level to finance public services in low-income countries (in particular, access to drinking water, healthcare, and education in Africa). \\\\Say we are in 2030. \textbf{[hic\_tax: Imagine that all other high-income countries (such as the European Union, Japan, and Canada) adopt this policy and some middle-income countries (such as China) do not.]}\textbf{~\\Do you support [country\_name: the United States] adopting this international tax on millionaires?}
  \\ \textit{Yes; No}

\item ~[Branch: Int'l] \label{q:intl_tax_support} \textbf{Imagine an international tax on individuals with net worth above [wealth\_threshold: \$1 million].~}\\Only wealth above [wealth\_threshold: \$1 million] would be taxed, at a rate of 2\%. Each country would retain 70\% of the revenues it collects, while 30\% would be pooled at the global level to finance public services in low-income countries (in particular, access to drinking water, healthcare, and education in Africa). \\\\Say we are in 2030.\textbf{ [intl\_tax: Imagine that some countries  (such as the European Union) adopt this policy and others (such as Japan, Canada, and China) do not.]\\Do you support [country\_name: the United States] adopting this international tax on millionaires?}
  \\ \textit{Yes; No}

\end{enumerate} 

 \subsection*{Scenarios \& radical tax} 
 [\textit{Scenario A \& B are randomly interverted.}]
 \begin{enumerate}[resume] 
\item  \label{q:sustainable_future} \textbf{Consider two possible scenarios for the world for the next 20 years.~\\\\Scenario A}: \\Most countries implement coordinated policies to limit global warming to +2\textdegree{}C and reduce inequality. The world greatly reduces greenhouse gas emissions and is on track to meet its climate target. Taxes on millionaires fund the installation of heat pumps, the thermal insulation of buildings, and improved public transportation. Yachts and private jets are phased out worldwide. Cars are all electric by 2045, and they are about the same price as internal combustion cars nowadays. By 2045, environmental regulations gradually double the price heating fuel or gas, air travel, and beef. As a result, people fly half as much, eat half as much meat, and use more public transportation in 2045 than they did in 2025. Despite higher prices for polluting goods, the overall purchasing power is preserved, thanks to a decrease in sales tax that reduces the prices of non-polluting goods.\\\\\textbf{Scenario B}:\\Since 2025, no additional policies are implemented to address climate change or inequality. People maintain the same lifestyles as in 2025. For example, most people continue to drive cars with internal combustion engines. Greenhouse gas emissions are stable. Global warming is expected to reach +3\textdegree{}C by 2100 and higher levels beyond that date. A warmer climate will cause more frequent and more severe droughts, heatwaves, wildfires, and floodings.\\\\Apart from the elements described, the two scenarios are the same (for example, in terms of unemployment or crime). \\\\\textbf{Which scenario do you prefer for the future?} [\textit{Figures \ref{fig:radical_redistr_share}, \ref{fig:sustainable_future}}; 
\verb|sustainable_future|]
  \\ \textit{Scenario A; Scenario B} \\\\
% \item  \label{q:sustainable_future_b} \textbf{Consider two possible scenarios for the world for the next 20 years.~\\\\Scenario A}:\\Since 2025, no additional policies are implemented to address climate change or inequality. People maintain the same lifestyles as in 2025. For example, most people continue to drive cars with internal combustion engines. Greenhouse gas emissions are stable. Global warming is expected to reach +3\textdegree{}C by 2100 and higher levels beyond that date. A warmer climate will cause more frequent and more severe droughts, heatwaves, wildfires, and floodings.\\\\\textbf{Scenario B}: \\Most countries implement coordinated policies to limit global warming to +2\textdegree{}C and reduce inequality. The world greatly reduces greenhouse gas emissions and is on track to meet its climate target. Taxes on millionaires fund the installation of heat pumps, the thermal insulation of buildings, and improved public transportation. Yachts and private jets are phased out worldwide. Cars are all electric by 2045, and they are about the same price as internal combustion cars nowadays. By 2045, environmental regulations gradually double the price of heating fuel or gas, air travel, and beef. As a result, people fly half as much, eat half as much meat, and use more public transportation in 2045 than they did in 2025. Despite higher prices for polluting goods, the overall purchasing power is preserved, thanks to a decrease in sales tax that reduces the prices of non-polluting goods.\\\\Apart from the elements described, the two scenarios are the same (for example, in terms of unemployment or crime). \\\\\textbf{Which scenario do you prefer for the future?} [\textit{Figure \ref{fig:sustainable_future_b}}; 
% \verb|sustainable_future_b|]
%   \\ \textit{Scenario A; Scenario B}
  ~[new page] [\textit{Two random branches: top1; top3}; \textit{Figures \ref{fig:radical_redistr_share}, \ref{fig:top_tax_share}-\ref{fig:top_tax_positive}}; 
\verb|top_tax_support|, \verb|variant_top_tax|]
\item ~[Branch: top1] \label{q:top1_tax_support} Currently, 2 billion people live in acute poverty, with less than [lcu\_250: \$250][periodicity: per month].\\The Sustainable Development Goals, adopted by all countries in 2015, aim to alleviate poverty and give access to healthcare, education, drinking water, and sanitation for all by 2030.~Due to lack of funding, the world is not on track to meet these poverty reduction goals.\\\\\textbf{Poverty reduction could be funded by a global tax on individual income above [lcu\_120k: \$120,000][periodicity\_tax: per year].~\\The tax rate would be 15\% for every [currency: dollar] over [lcu\_120k: \$120,000] of income} after existing taxes.~\\For example, a single person earning [lcu\_130k: \$130,000][periodicity\_tax: per year] after taxes would pay [lcu\_1500: \$1,500] in additional taxes, or 15\% of [lcu\_10k: \$10,000] = [lcu\_130k: \$130,000]~\&ndash;~[lcu\_120k: \$120,000]. Meanwhile, a married couple earning [lcu\_200k: \$200,000][periodicity\_tax: per year], [lcu\_100k: \$100,000] for each of them, would go untaxed.\\This tax would apply to the richest 1\% of the world's population. [tax\_country\_name: In the United States], it would affect the richest [affected\_top1: 8]\% and redistribute [transfer\_top1: 3]\% of GDP to lower-income countries.\\\\\textbf{Do you support or oppose such a global tax on the richest people to finance global poverty reduction?}\\ 
  \\ \textit{Strongly oppose; Somewhat support; Strongly support; Somewhat oppose; Indifferent}

\item ~[Branch: top3] \label{q:top3_tax_support} Currently, 3 billion people live in deep poverty, with less than [lcu\_400: \$400][periodicity: per month].\\The Sustainable Development Goals, adopted by all countries in 2015, aim to alleviate poverty and achieve access to healthcare, education, drinking water, and sanitation for all by 2030.~Due to lack of funding, the world is not on track to meet these poverty reduction goals.\\\\\textbf{Poverty reduction could be funded by a global tax on individual income above [lcu\_80k: \$80,000][periodicity\_tax: per year].~\\The tax rate would be 15\% for every [currency: dollar] over [lcu\_80k: \$80,000] of income} after existing taxes, \textbf{30\% over [lcu\_120k: \$120,000], and 45\% over [lcu\_1M: \$1 million].~}\\For example, a single person earning [lcu\_90k: \$90,000][periodicity\_tax: per year] after taxes would pay [lcu\_1500\_top3: \$1,500] in additional taxes, or 15\% of [lcu\_10k\_top3: \$10,000] = [lcu\_90k: \$90,000]~\&ndash;~[lcu\_80k: \$80,000]. Meanwhile, a married couple earning [lcu\_150k: \$150,000][periodicity\_tax: per year], [lcu\_75k: \$75,000] for each of them, would go untaxed.\\This tax would apply to the richest 3\% of the world's population. [tax\_country\_name: In the United States], it would affect the richest [affected\_top3: 18]\% and redistribute [transfer\_top3: 8]\% of GDP to lower-income countries.\\\\\textbf{Do you support or oppose such a global tax on the richest people to finance global poverty reduction?}\\ 
~[\textit{Figures \ref{fig:radical_redistr_share}, \ref{fig:top_tax_share}-\ref{fig:top_tax_positive}}; 
\verb|top3_tax_support|]
  \\ \textit{Strongly oppose; Somewhat support; Strongly support; Somewhat oppose; Indifferent}

\item  \label{q:attention_test} To show that you are attentive, please select "A little" in the following list: [%\textit{Figure \ref{fig:attention_test}}; 
\verb|attention_test|]
  \\ \textit{Not at all; A little; A lot; A great deal}

\end{enumerate} 

 \subsection*{Preferred transfer means to LICs} 
 \begin{enumerate}[resume] 
\item  \label{q:transfer_how} Below are different ways to transfer resources to help reduce poverty in a low-income country.~\\How do you evaluate each of these options?\\ 
~[\textit{Figures \ref{fig:transfer_how}, \ref{fig:transfer_how_positive}-\ref{fig:transfer_how_negative}}; 
\verb|transfer_how|]
~[Item order is randomly flipped or not]
\begin{itemize}
  \item Transfers to public development aid agencies which then finance suitable projects
  \item Transfers to the national government conditioned on the use of funds for poverty reduction programs
  \item Unconditional transfers to the national government
  \item Unconditional transfers to local authorities (municipality, village chief...)
  \item Transfers to local NGOs with democratic decision-making processes
  \item Cash transfers to parents (child allowances), to the disabled and to the elderly
  \item Unconditional cash transfers to each household
\end{itemize}
\textit{A wrong way; An acceptable way; A right way; The best way}

\end{enumerate} 

 \subsection*{Radical redistribution} 
 \begin{enumerate}[resume] 
\item  \label{q:convergence_support} Should governments actively cooperate to have all countries converge in terms of GDP per capita by the end of the century? [\textit{Figures \ref{fig:radical_redistr_share}, \ref{fig:convergence_support}}; 
\verb|convergence_support|]
  \\ \textit{Yes; No; I prefer not to answer}

\item  \label{q:global_movement} If there was a worldwide movement in favor of a global program to tackle climate change, implement taxes on millionaires and fund poverty reduction in low-income countries, to what extent would you be willing to be part of that movement? (Multiple answers possible) [\textit{Figures \ref{fig:radical_redistr_share}, \ref{fig:global_movement}}; 
\verb|global_movement|]
  \\ \textit{I would \textit{not} support such a movement.; I could sign a petition and spread ideas.; I could attend a demonstration.; I could go on strike.; I could donate [amount\_lottery: \$100] to a strike fund.}

\item ~[\textit{Except in: RU, SA}] \label{q:vote_intl_coalition} Let us call "your political party" the party you voted for in the last election, or the party that represents your views most closely.~\\\textbf{Imagine }there was \textbf{a worldwide coalition} of political parties in favor of a common program \textbf{to tackle climate change, implement taxes on millionaires and fund poverty reduction in low-income countries}.~\\\\\textbf{Would you be more likely to vote for your party if it were part of that coalition?}\\ 
~[\textit{Figures \ref{fig:radical_redistr_share}, \ref{fig:vote_intl_coalition}}; 
\verb|vote_intl_coalition|]
~[Item order is randomly flipped or not]
  \\ \textit{Yes, I would be \textbf{more likely} to vote for my party if it joined that coalition (or to vote for another party if only that other party joined the coalition).; \\My choice would \textbf{not depend} on which parties are part of that coalition.; \\No, I would be \textbf{less likely} to vote for my party if it joined that coalition.}

\item  \label{q:why_hic_help_lic} Some people think that high-income countries should support low-income countries.~\\Among the different reasons given, which ones do you agree with? (Multiple answers possible) [\textit{Figure \ref{fig:why_hic_help_lic}}; 
\verb|why_hic_help_lic|]
~[Order of the first three items is randomized]
  \\ \textit{High-income countries have a historical responsibility for the current situation in low-income countries.; \\In the long run, it is in the interest of high-income countries to help low-income countries.; \\Helping those in need is the right thing to do. This is also true at the international level.; \\None of the above.}

\item ~[\textit{Only in: FR, DE, IT, ES, GB, US}] \label{q:reparations_support} Some people argue that Western countries owe reparations for colonization and slavery to former colonies and descendants of slaves. \\Reparations could take the form of funding education and facilitating technology transfers, to address unequal opportunities passed down from the past. \\\\\textbf{Do you support or oppose reparations} of this kind \textbf{for colonization and slavery?~}\\ 
~[\textit{Figures \ref{fig:radical_redistr_share}, \ref{fig:reparations_support}}; 
\verb|reparations_support|]
  \\ \textit{Strongly oppose; Somewhat oppose; Indifferent; Somewhat support; Strongly support}

\end{enumerate} 

 \subsection*{[\textit{Except in: RU}] Custom redistribution} 
 \begin{enumerate}[resume] 
\item \label{q:income_exact} What is the \textit{[periodicity\_text: yearly]} income of your household \textbf{after taxes and social benefits}?\\This includes all sources of income: salaries, pensions, allowances, welfare benefits, property income, etc.\\My household earns ... [text\_unit: \$ per year] (answer with no comma, no space, no period):\\ 
~[%\textit{Figure \ref{fig:income_exact}}; % TODO
\verb|income_exact|]

\item ~[new page] \label{q:custom_redistr} If you could redistribute income at the global level, what would you do? In this question, we let you choose your preferred parameters for a redistribution of income at the world level.~\\If you prefer to skip this question, check the corresponding box at the bottom of the page.\\\\The worldwide redistribution of income would take the form of additional policies, taxes, and transfers, on top of existing ones.\\These policies would lower the income of the richest (the losers from the redistribution) and increase the income of the poorest (the winners).~\\\\Below you will find a graph of the world distribution of after-tax income and three sliders that vary it. The current distribution is in red, and your custom one is in green.~\\The first two sliders~control the proportion of winners and the proportion of losers, among all humans. The third slider controls the degree of redistribution from the richest to the poorest.~\\If you do not want new policies to reduce global inequality, you can set the third slider to zero.~\\\\\textbf{You need to move the sliders} (by holding the mouse down on the little squares and moving to the side) to make the green curve evolve: the idea is to move the sliders \textbf{until you get a green curve you are satisfied with}. \\\\

~\\Examples of income changes after your proposed redistribution:\\

\begin{tabular}{@{\extracolsep{5pt}}|c|c|} 
    \hline \\[-1.8ex] 
    \textbf{Now} & \textbf{After} \\\hline %\\[-1.8ex]
    0 [text\_unit: \$ per year] & [after\_0] [text\_unit: \$ per year] \\ 
    ~[now\_10k] [text\_unit] & [after\_10k] [text\_unit] \\ 
    ~[now\_60k] [text\_unit] & [after\_60k] [text\_unit] \\ 
    ~[now\_100k] [text\_unit] & [after\_100k] [text\_unit] \\ 
    \multicolumn{2}{c}{Your \textit{individual} income} \\ 
    ~[own] [text\_unit] & [after\_own] [text\_unit] \\ 
    \hline 
\end{tabular}  

% ~ [\textit{Figure \ref{fig:custom_redistr}}; 
% \verb|custom_redistr|]
~[\textit{Figures \ref{fig:custom_redistr_question}, \ref{fig:custom_redistr_mean}-\ref{fig:custom_redistr_median}} 
% \verb|variables_custom_redistr|
]
\textit{I am satisfied with my custom redistribution.; \\I want to skip this question.}

\end{enumerate} 

 \subsection*{Well-being (\textit{for another project})} 
  [\textit{Four random branches: gallup\_0; gallup\_1; wvs\_0; wvs\_1}; %\textit{Figure \ref{fig:well_being}}; 
 \verb|well_being, variant_well_being|] 
 \begin{enumerate}[resume] % TODO? condenser?
\item ~[Branch: gallup\_0] \label{q:well_being_gallup_0} Please imagine a ladder, with steps numbered from 0 at the bottom to 10 at the top. The top of the ladder represents the best possible life for you and the bottom of the ladder represents the worst possible life for you. \\\\On which step of the ladder would you say you personally feel you stand at this time? [%\textit{Figure \ref{fig:well_being}}; 
\verb|well_being_gallup_0|]
  \\ \textit{Worst possible 0; 1; 2; 3; 4; 5; 6; 7; 8; 9; Best possible 10}

\item ~[Branch: gallup\_1] \label{q:well_being_gallup_1} Please imagine a ladder, with steps numbered from 1 at the bottom to 10 at the top. The top of the ladder represents the best possible life for you and the bottom of the ladder represents the worst possible life for you. \\\\On which step of the ladder would you say you personally feel you stand at this time? [%\textit{Figure \ref{fig:well_being}}; 
\verb|well_being_gallup_1|]
  \\ \textit{Worst possible 1; 2; 3; 4; 5; 6; 7; 8; 9; Best possible 10}

\item ~[Branch: wvs\_0] \label{q:well_being_wvs_0} All things considered, how satisfied are you with your life as a whole these days? [%\textit{Figure \ref{fig:well_being}}; 
\verb|well_being_wvs_0|]
  \\ \textit{Completely dissatisfied 0; 1; 2; 3; 4; 5; 6; 7; 8; 9; Completely satisfied 10}

\item ~[Branch: wvs\_1] \label{q:well_being_wvs_1} All things considered, how satisfied are you with your life as a whole these days? [%\textit{Figure \ref{fig:well_being}}; 
\verb|well_being_wvs_1|]
  \\ \textit{Completely dissatisfied 1; 2; 3; 4; 5; 6; 7; 8; 9; Completely satisfied 10}

\end{enumerate} 

 \subsection*{Comprehension} 
 \begin{enumerate}[resume] 
\item  \label{q:gcs_comprehension} \textit{Comprehension question: one respondent with the expected answer will get [amount\_lottery: \$100].}\\\\How would gasoline prices change as a result of the Global Climate Scheme? \\Gasoline prices would... [\textit{Figure \ref{fig:gcs_comprehension}}; 
\verb|gcs_comprehension|]
~[Item order is randomly flipped or not]
  \\ \textit{increase; not be affected; decrease}

\end{enumerate} 

 \subsection*{Synthetic questions} 
 \begin{enumerate}[resume] 
\item  \label{q:my_tax_global_nation} To what extent do you agree or disagree with the following statement? "My taxes should go towards solving global problems." [\textit{Figures \ref{fig:radical_redistr_share}, \ref{fig:my_tax_global_nation_share}-\ref{fig:my_tax_global_nation_positive}}; 
\verb|my_tax_global_nation|]
  \\ \textit{Strongly agree; Agree; Neither agree nor disagree; Disagree; Strongly disagree}

\item  \label{q:group_defended} Which group of people do you advocate for when you vote? [\textit{Figures \ref{fig:group_defended}, \ref{fig:group_defended_all}}; 
\verb|group_defended|]
  \\ \textit{Sentient beings (humans and animals); Humans; [country\_adjective\_plural: Americans]; People from my community (for example my region, my religion, my gender…); My family and myself}

\end{enumerate} 

 \subsection*{Feedback} 
 \begin{enumerate}[resume] 
\item  \label{q:survey_biased} Do you feel that this survey was politically biased? [\textit{Figure \ref{fig:survey_biased}}; 
\verb|survey_biased|]
  \\ \textit{Yes, left-wing biased; Yes, right-wing biased; No, I do not feel it was biased}

\item  \label{q:comment_field} The survey is nearing completion. You can now enter any comments, thoughts, or suggestions in the field below. [%\textit{Figure \ref{fig:comment_field}}; % TODO
\verb|comment_field|]


% \item  \label{q:interview} Lastly, \textbf{would you be interested in participating in a 30-minute interview with a researcher (via videoconference)? }\\\textbf{If so}, please \textbf{leave your email}: [\textit{Figure \ref{fig:interview}}; 
% \verb|interview|] % TODO? leave?

 \end{enumerate} 



\clearpage
\section{Representativeness of the surveys}\label{app:representativeness}

\begin{table}[h!]
    \caption[Sample representativeness in FR, DE, IT]{Sample representativeness in France, Germany, Italy. %(Back to \ref{subsec:data}) 
    } \label{tab:representativeness_1}
    \makebox[\textwidth][c]{
        \resizebox*{!}{.60\textheight}{% 73 without notes cf. https://tex.stackexchange.com/questions/13809/resizing-a-table-by-textheight 
        
\begin{tabular}[t]{lccccccccc}
\toprule
\multicolumn{1}{c}{} & \multicolumn{3}{c}{France} & \multicolumn{3}{c}{Germany} & \multicolumn{3}{c}{Italy} \\
\cmidrule(l{3pt}r{3pt}){2-4} \cmidrule(l{3pt}r{3pt}){5-7} \cmidrule(l{3pt}r{3pt}){8-10}
  & Pop. & Sample & \makecell{Weighted\\sample} & Pop. & Sample & \makecell{Weighted\\sample} & Pop. & Sample & \makecell{Weighted\\sample}\\
\midrule
Sample size &  & 798 & 798 &  & 1,048 & 1,048 &  & 756 & 756\\
\addlinespace
Gender: Woman & .52 & .52 & .52 & .51 & .49 & .51 & .52 & .52 & .51\\
Gender: Man & .48 & .48 & .48 & .49 & .51 & .49 & .48 & .48 & .49\\
\addlinespace
Income\_quartile: Q1 & .25 & .26 & .25 & .25 & .27 & .25 & .25 & .26 & .25\\
Income\_quartile: Q2 & .25 & .26 & .25 & .25 & .27 & .25 & .25 & .26 & .25\\
Income\_quartile: Q3 & .25 & .23 & .25 & .25 & .20 & .25 & .25 & .22 & .25\\
Income\_quartile: Q4 & .25 & .25 & .25 & .25 & .26 & .25 & .25 & .25 & .25\\
\addlinespace
Age: 18-24 & .10 & .11 & .10 & .09 & .10 & .09 & .08 & .08 & .08\\
Age: 25-34 & .15 & .15 & .15 & .15 & .16 & .15 & .12 & .12 & .12\\
Age: 35-49 & .23 & .23 & .23 & .23 & .25 & .23 & .23 & .23 & .23\\
Age: 50-64 & .24 & .24 & .24 & .27 & .27 & .27 & .28 & .29 & .28\\
Age: 65+ & .27 & .27 & .27 & .27 & .22 & .27 & .29 & .28 & .29\\
\addlinespace
Diploma\_25-64: Below upper secondary & .10 & .09 & .10 & .11 & .11 & .11 & .22 & .19 & .22\\
Diploma\_25-64: Upper secondary & .26 & .26 & .26 & .32 & .32 & .32 & .28 & .28 & .28\\
Diploma\_25-64: Post secondary & .26 & .27 & .26 & .22 & .24 & .21 & .14 & .17 & .14\\
\addlinespace
Urbanity: Cities & .47 & .47 & .46 & .39 & .42 & .39 & .36 & .37 & .36\\
Urbanity: Towns and suburbs & .19 & .19 & .19 & .42 & .42 & .42 & .46 & .47 & .46\\
Urbanity: Rural & .34 & .33 & .34 & .19 & .17 & .19 & .18 & .16 & .18\\
\addlinespace
Region: 1 & .18 & .19 & .18 & .17 & .19 & .17 & .66 & .70 & .65\\
Region: 2 & .22 & .23 & .22 & .29 & .32 & .29 & .34 & .29 & .34\\
Region: 3 & .11 & .11 & .11 & .54 & .48 & .54 &  &  & \\
Region: 4 & .21 & .22 & .21 &  &  &  &  &  & \\
Region: 5 & .28 & .26 & .28 &  &  &  &  &  & \\
\bottomrule
\end{tabular}
        }
    }
    {\footnotesize \textit{Note}: This table displays summary statistics of the samples alongside actual population frequencies. 
    Detailed sources for each variable and country population frequencies, as well as the definitions of regions, diploma, urbanity, employment, and vote are available in \href{https://github.com/bixiou/robustness_global_redistr/raw/main/questionnaire/sources.xlsx}{this spreadsheet}. 
    } 
\end{table}

\begin{table}[h!]
    \caption[Sample representativeness in PL, ES, GB, CH]{Sample representativeness in Poland, Spain, the UK, Switzerland. %(Back to \ref{subsec:data}) 
    } \label{tab:representativeness_2}
    \makebox[\textwidth][c]{\resizebox*{!}{.60\textheight}{\input{../tables/PL_ES_GB_CH}}}
    {\footnotesize \textit{Note}: This table displays summary statistics of the samples alongside actual population frequencies. 
    Detailed sources for each variable and country population frequencies, as well as the definitions of regions, diploma, urbanity, employment, and vote are available in \href{https://github.com/bixiou/robustness_global_redistr/raw/main/questionnaire/sources.xlsx}{this spreadsheet}. 
    } 
\end{table}

\begin{table}[h!]
    \caption[Sample representativeness in JP, SA, US]{Sample representativeness in Japan, Saudi Arabia, the United States. %(Back to \ref{subsec:data}) 
    } \label{tab:representativeness_3}
    \makebox[\textwidth][c]{\resizebox*{!}{.80\textheight}{\input{../tables/JP_SA_US}}}
    {\footnotesize \textit{Note}: This table displays summary statistics of the samples alongside actual population frequencies. 
    Detailed sources for each variable and country population frequencies, as well as the definitions of regions, diploma, urbanity, employment, and vote are available in \href{https://github.com/bixiou/robustness_global_redistr/raw/main/questionnaire/sources.xlsx}{this spreadsheet}. 
    } 
\end{table}
\clearpage
\section{Determinants of support}\label{app:determinants}

\begin{table}[h]\label{tab:determinant}
    \caption[Correlates of support for global redistribution]{Correlates of support for global redistribution (multivariate OLS regressions). %(Back to \ref{subsec:gcs_stated_support})
    } 
    \makebox[\textwidth][c]{
\resizebox*{!}{.67\textheight}{ 
        
\begin{tabular}{@{\extracolsep{5pt}}lccccccc} 
\\[-1.8ex]\hline 
\hline \\[-1.8ex] 
\\[-1.8ex] & \makecell{Share of\\plausible\\policies\\supported} & \makecell{Supports\\the Global\\Climate\\Scheme} & \makecell{Universalist\\(Group\\defended:\\\textit{Humans} or\\\textit{Sentient beings})} & \makecell{More likely\\to vote\\for party\\in global\\coalition} & \makecell{Endorses\\convergence\\of all countries'\\GDP p.c.\\by 2100} & \makecell{Supports an\\international\\wealth tax\\funding LICs} & \makecell{Prefers a\\sustainable\\future} \\ 
\\[-1.8ex] & (1) & (2) & (3) & (4) & (5) & (6) & (7)\\ 
\hline \\[-1.8ex] 
Mean & 0.508 & 0.554 & 0.454 & 0.365 & 0.61 & 0.704 & 0.681  \\ \hline \\[-1.8ex]
 Vote: Center\mbox{-}right or Right & 0.015 & 0.008 & $-$0.083$^{***}$ & 0.029$^{**}$ & 0.041$^{***}$ & $-$0.026$^{*}$ & $-$0.061$^{***}$ \\ 
  & (0.010) & (0.015) & (0.014) & (0.013) & (0.014) & (0.014) & (0.014) \\ 
  Vote: Far right & $-$0.090$^{***}$ & $-$0.143$^{***}$ & $-$0.225$^{***}$ & $-$0.063$^{***}$ & $-$0.065$^{***}$ & $-$0.140$^{***}$ & $-$0.169$^{***}$ \\ 
  & (0.013) & (0.020) & (0.019) & (0.018) & (0.020) & (0.019) & (0.020) \\ 
  Vote: Left & 0.211$^{***}$ & 0.170$^{***}$ & 0.150$^{***}$ & 0.257$^{***}$ & 0.190$^{***}$ & 0.184$^{***}$ & 0.147$^{***}$ \\ 
  & (0.010) & (0.014) & (0.015) & (0.014) & (0.014) & (0.013) & (0.014) \\ 
  Gender: Man & 0.016$^{**}$ & 0.018$^{*}$ & $-$0.044$^{***}$ & 0.029$^{***}$ & 0.009 & $-$0.007 & $-$0.043$^{***}$ \\ 
  & (0.007) & (0.010) & (0.010) & (0.010) & (0.010) & (0.009) & (0.009) \\ 
  Age: 18\mbox{-}24 & 0.012 & 0.175$^{***}$ & 0.104$^{***}$ & 0.108$^{***}$ & 0.109$^{***}$ & 0.101$^{***}$ & 0.062$^{***}$ \\ 
  & (0.014) & (0.020) & (0.021) & (0.022) & (0.020) & (0.018) & (0.019) \\ 
  Age: 25\mbox{-}34 & 0.020$^{*}$ & 0.094$^{***}$ & 0.075$^{***}$ & 0.102$^{***}$ & 0.046$^{***}$ & 0.046$^{***}$ & 0.027$^{*}$ \\ 
  & (0.011) & (0.015) & (0.016) & (0.016) & (0.015) & (0.014) & (0.015) \\ 
  Age: 50\mbox{-}64 & $-$0.002 & $-$0.036$^{**}$ & $-$0.034$^{**}$ & $-$0.033$^{**}$ & $-$0.025$^{*}$ & $-$0.021 & $-$0.020 \\ 
  & (0.010) & (0.014) & (0.014) & (0.014) & (0.013) & (0.013) & (0.013) \\ 
  Age: 65+ & 0.041$^{***}$ & $-$0.020 & $-$0.010 & 0.002 & $-$0.021 & $-$0.018 & 0.016 \\ 
  & (0.012) & (0.018) & (0.018) & (0.017) & (0.017) & (0.016) & (0.016) \\ 
  Income quartile: Q2 & 0.018$^{*}$ & 0.004 & $-$0.025$^{*}$ & 0.016 & $-$0.014 & 0.013 & 0.010 \\ 
  & (0.010) & (0.015) & (0.015) & (0.015) & (0.014) & (0.013) & (0.014) \\ 
  Income quartile: Q3 & 0.007 & $-$0.010 & 0.019 & $-$0.009 & $-$0.024$^{*}$ & $-$0.018 & 0.002 \\ 
  & (0.010) & (0.015) & (0.015) & (0.015) & (0.014) & (0.014) & (0.014) \\ 
  Income quartile: Q4 & $-$0.010 & $-$0.042$^{***}$ & $-$0.004 & $-$0.032$^{*}$ & $-$0.075$^{***}$ & $-$0.078$^{***}$ & 0.007 \\ 
  & (0.011) & (0.016) & (0.016) & (0.017) & (0.015) & (0.015) & (0.015) \\ 
  Diploma: Upper secondary & 0.042$^{***}$ & 0.001 & 0.018 & 0.036$^{**}$ & 0.029$^{*}$ & 0.022 & 0.022 \\ 
  & (0.011) & (0.016) & (0.016) & (0.015) & (0.015) & (0.014) & (0.015) \\ 
  Diploma: Above upper secondary & 0.085$^{***}$ & 0.026 & 0.025 & 0.079$^{***}$ & 0.015 & 0.015 & 0.039$^{**}$ \\ 
  & (0.011) & (0.016) & (0.016) & (0.015) & (0.015) & (0.015) & (0.016) \\ 
  Urbanicity: Rural & $-$0.012 & $-$0.054$^{***}$ & 0.016 & $-$0.006 & $-$0.015 & $-$0.021 & $-$0.020 \\ 
  & (0.010) & (0.015) & (0.015) & (0.014) & (0.015) & (0.014) & (0.015) \\ 
  Urbanicity: Towns and suburbs & $-$0.014 & $-$0.039$^{**}$ & $-$0.022 & $-$0.023 & $-$0.016 & $-$0.024$^{*}$ & 0.026$^{*}$ \\ 
  & (0.010) & (0.015) & (0.015) & (0.015) & (0.015) & (0.014) & (0.014) \\ 
  Will become millionaire: Likely & 0.036$^{***}$ & 0.070$^{***}$ & $-$0.001 & 0.039$^{***}$ & 0.055$^{***}$ & $-$0.019$^{*}$ & $-$0.019 \\ 
  & (0.008) & (0.012) & (0.012) & (0.013) & (0.012) & (0.011) & (0.012) \\ 
  Will become millionaire: Already & $-$0.020 & $-$0.019 & 0.008 & $-$0.058$^{**}$ & $-$0.042$^{*}$ & $-$0.236$^{***}$ & $-$0.047$^{**}$ \\ 
  & (0.017) & (0.023) & (0.024) & (0.023) & (0.023) & (0.023) & (0.022) \\ 
  Foreign born & 0.065$^{***}$ & 0.083$^{***}$ & 0.088$^{***}$ & 0.051$^{**}$ & 0.037$^{*}$ & 0.040$^{**}$ & 0.030 \\ 
  & (0.014) & (0.020) & (0.021) & (0.022) & (0.020) & (0.018) & (0.019) \\ 
 \hline \\[-1.8ex] 

Observations & 12,001 & 12,001 & 12,001 & 10,000 & 12,001 & 12,001 & 12,001 \\ 
R$^{2}$ & 0.141 & 0.104 & 0.100 & 0.115 & 0.105 & 0.091 & 0.069 \\ 
\hline 
\hline \\[-1.8ex] 
\end{tabular} 
        }
    }
    {\footnotesize \textit{Note}: Robust standard errors are reported in parentheses. Covariates omitted in the Table: \textit{Country}; \textit{Employment}; \textit{Couple}; \textit{Region}; \textit{Constant}. Omitted variables are: \textit{Vote: Non-voter, PNR or Other}; \textit{Gender: Woman}; \textit{Age: 35-49}; \textit{Income\_quartile: Q1}; \textit{Diploma: Below upper secondary}; \textit{Urbanicity: City}. \hfill $^{*}$p$<$0.1; $^{**}$p$<$0.05; $^{***}$p$<$0.01.
    }
\end{table}


\clearpage
\section{Attrition analysis}\label{app:attrition}

\begin{table}[h!]\label{tab:attrition}
    \caption[Attrition analysis]{Attrition analysis.} 
    \makebox[\textwidth][c]{\resizebox*{!}{.87\textheight}{ % 73 is the max when there is a title
        
\begin{tabular}{@{\extracolsep{5pt}}lccccc} 
\\[-1.8ex]\hline 
\hline \\[-1.8ex] 
\\[-1.8ex] & \makecell{Dropped out} & \makecell{Dropped out\\after\\socio-eco} & \makecell{Failed\\attention test} & \makecell{Duration\\(in min)} & \makecell{Duration\\below\\6 min} \\ 
\\[-1.8ex] & (1) & (2) & (3) & (4) & (5)\\ 
\hline \\[-1.8ex] 
Mean & 0.166 & 0.102 & 0.088 & 53.896 & 0.087  \\ \hline \\[-1.8ex]
 Vote: Center\mbox{-}right or Right & $-$0.042$^{***}$ & $-$0.040$^{***}$ & $-$0.008 & $-$7.194 & $-$0.025$^{***}$ \\ 
  & (0.008) & (0.008) & (0.007) & (10.597) & (0.008) \\ 
  Vote: Far right & $-$0.051$^{***}$ & $-$0.050$^{***}$ & $-$0.012 & $-$18.804 & $-$0.021$^{**}$ \\ 
  & (0.010) & (0.010) & (0.009) & (15.429) & (0.009) \\ 
  Vote: Left & $-$0.029$^{***}$ & $-$0.027$^{***}$ & $-$0.013$^{*}$ & $-$16.126 & $-$0.041$^{***}$ \\ 
  & (0.008) & (0.008) & (0.007) & (10.494) & (0.008) \\ 
  Gender: Man & $-$0.043$^{***}$ & $-$0.042$^{***}$ & 0.026$^{***}$ & $-$14.234$^{**}$ & 0.004 \\ 
  & (0.005) & (0.005) & (0.005) & (7.063) & (0.005) \\ 
  Age: 18\mbox{-}24 & $-$0.030$^{***}$ & $-$0.030$^{***}$ & 0.028$^{**}$ & $-$19.220$^{***}$ & 0.089$^{***}$ \\ 
  & (0.010) & (0.010) & (0.012) & (6.900) & (0.013) \\ 
  Age: 25\mbox{-}34 & $-$0.031$^{***}$ & $-$0.031$^{***}$ & 0.018$^{**}$ & $-$4.123 & 0.051$^{***}$ \\ 
  & (0.007) & (0.007) & (0.009) & (8.929) & (0.009) \\ 
  Age: 50\mbox{-}64 & 0.011 & 0.011 & $-$0.033$^{***}$ & 9.254 & $-$0.059$^{***}$ \\ 
  & (0.008) & (0.008) & (0.007) & (10.045) & (0.007) \\ 
  Age: 65+ & 0.039$^{***}$ & 0.039$^{***}$ & $-$0.058$^{***}$ & 28.229$^{*}$ & $-$0.105$^{***}$ \\ 
  & (0.010) & (0.010) & (0.008) & (15.380) & (0.008) \\ 
  Income quartile: Q2 & $-$0.029$^{***}$ & $-$0.029$^{***}$ & $-$0.040$^{***}$ & $-$0.384 & $-$0.012 \\ 
  & (0.008) & (0.008) & (0.008) & (8.336) & (0.008) \\ 
  Income quartile: Q3 & $-$0.028$^{***}$ & $-$0.030$^{***}$ & $-$0.056$^{***}$ & $-$13.405$^{**}$ & $-$0.017$^{**}$ \\ 
  & (0.008) & (0.008) & (0.008) & (6.818) & (0.008) \\ 
  Income quartile: Q4 & $-$0.032$^{***}$ & $-$0.032$^{***}$ & $-$0.060$^{***}$ & 16.509 & $-$0.029$^{***}$ \\ 
  & (0.009) & (0.009) & (0.009) & (13.757) & (0.008) \\ 
  Diploma: Upper secondary & $-$0.018$^{**}$ & $-$0.018$^{**}$ & $-$0.052$^{***}$ & 7.862 & $-$0.004 \\ 
  & (0.009) & (0.009) & (0.009) & (9.883) & (0.008) \\ 
  Diploma: Above upper secondary & $-$0.042$^{***}$ & $-$0.043$^{***}$ & $-$0.065$^{***}$ & 1.479 & $-$0.017$^{**}$ \\ 
  & (0.009) & (0.009) & (0.009) & (10.935) & (0.008) \\ 
  Urbanicity: Rural & $-$0.004 & $-$0.005 & $-$0.008 & $-$4.275 & $-$0.003 \\ 
  & (0.008) & (0.008) & (0.007) & (8.305) & (0.007) \\ 
  Urbanicity: Towns and suburbs & 0.007 & 0.007 & $-$0.016$^{**}$ & 5.311 & 0.001 \\ 
  & (0.008) & (0.008) & (0.007) & (16.323) & (0.007) \\ 
  Country: Germany & $-$0.353 & $-$0.352 & $-$0.679$^{**}$ & 25.812 & $-$0.211$^{***}$ \\ 
  & (0.263) & (0.263) & (0.329) & (30.763) & (0.025) \\ 
  Country: Italy & $-$0.123 & $-$0.121 & $-$0.675$^{**}$ & 688.464 & $-$0.207$^{***}$ \\ 
  & (0.318) & (0.319) & (0.329) & (604.750) & (0.033) \\ 
  Country: Japan & $-$0.320 & $-$0.317 & $-$0.707$^{**}$ & 5.073 & 0.206 \\ 
  & (0.262) & (0.263) & (0.329) & (27.025) & (0.196) \\ 
  Country: Poland & $-$0.258 & $-$0.257 & $-$0.735$^{**}$ & $-$27.152 & 0.770$^{***}$ \\ 
  & (0.261) & (0.262) & (0.329) & (26.191) & (0.025) \\ 
  Country: Saudi Arabia & $-$0.072 & $-$0.069 & $-$0.260 & 18.800 & $-$0.320$^{***}$ \\ 
  & (0.280) & (0.281) & (0.354) & (31.927) & (0.038) \\ 
  Country: Spain & $-$0.299 & $-$0.296 & $-$0.704$^{**}$ & 95.626$^{***}$ & $-$0.323$^{***}$ \\ 
  & (0.262) & (0.262) & (0.329) & (26.178) & (0.026) \\ 
  Country: Switzerland & 0.001 & 0.001 & $-$0.001 & 2.543 & $-$0.004 \\ 
  & (0.025) & (0.025) & (0.020) & (13.928) & (0.020) \\ 
  Country: United Kingdom & $-$0.262 & $-$0.261 & $-$0.839$^{**}$ & 65.461$^{**}$ & $-$0.383$^{***}$ \\ 
  & (0.262) & (0.262) & (0.329) & (28.897) & (0.029) \\ 
  Country: USA & $-$0.024 & $-$0.021 & $-$0.134$^{***}$ & $-$2.372 & $-$0.147$^{***}$ \\ 
  & (0.023) & (0.023) & (0.020) & (18.225) & (0.020) \\ 
 \hline \\[-1.8ex] 

Observations & 15,013 & 15,013 & 13,261 & 12,031 & 12,031 \\ 
R$^{2}$ & 0.037 & 0.037 & 0.079 & 0.012 & 0.089 \\ 
\hline 
\hline \\[-1.8ex] 
\end{tabular} }}
    \centering {\footnotesize \textit{Note}: Robust standard errors are reported in parentheses. $^{*}$p$<$0.1; $^{**}$p$<$0.05; $^{***}$p$<$0.01.}
\end{table}

\clearpage
\section{Balance analysis}\label{app:balance}

\begin{table}[h]\label{tab:balance}
    \caption[Balance analysis]{Balance analysis.} 
    \makebox[\textwidth][c]{
\resizebox*{!}{.71\textheight}{ 
        
\begin{tabular}{@{\extracolsep{5pt}}lcccccccc} 
\\[-1.8ex]\hline 
\hline \\[-1.8ex] 
 & \multicolumn{8}{c}{Random branch:} \\ 
\cline{2-9} 
\\[-1.8ex] & \makecell{Wealth tax\\coverage:\\Global} & \makecell{Wealth tax\\coverage:\\Int'l} & \makecell{Int'l CS\\coverage:\\Low} & \makecell{Int'l CS\\coverage:\\High} & \makecell{Int'l CS\\coverage:\\High color} & \makecell{National\\CS\\asked} & \makecell{Warm glow\\substitute:\\Control} & \makecell{Warm glow\\realism: Info\\treatment} \\ 
\\[-1.8ex] & (1) & (2) & (3) & (4) & (5) & (6) & (7) & (8)\\ 
\hline \\[-1.8ex] 
Mean & 0.332 & 0.334 & 0.25 & 0.256 & 0.252 & 0.36 & 0.358 & 0.489  \\ \hline \\[-1.8ex]
 Vote: Center\mbox{-}right or Right & 0.001 & 0.018 & $-$0.011 & 0.001 & 0.011 & $-$0.013 & 0.003 & $-$0.013 \\ 
  & (0.013) & (0.013) & (0.013) & (0.012) & (0.012) & (0.014) & (0.013) & (0.014) \\ 
  Vote: Far right & $-$0.026 & 0.016 & $-$0.015 & 0.025 & 0.019 & $-$0.008 & 0.019 & $-$0.010 \\ 
  & (0.018) & (0.018) & (0.017) & (0.017) & (0.017) & (0.018) & (0.019) & (0.020) \\ 
  Vote: Left & $-$0.004 & 0.001 & $-$0.004 & $-$0.007 & 0.002 & $-$0.003 & $-$0.005 & $-$0.014 \\ 
  & (0.014) & (0.014) & (0.013) & (0.013) & (0.012) & (0.014) & (0.013) & (0.014) \\ 
  Gender: Man & $-$0.00003 & $-$0.015$^{*}$ & $-$0.013 & 0.005 & $-$0.00005 & 0.001 & $-$0.007 & 0.011 \\ 
  & (0.009) & (0.009) & (0.008) & (0.008) & (0.008) & (0.009) & (0.009) & (0.010) \\ 
  Age: 18\mbox{-}24 & $-$0.009 & 0.007 & $-$0.013 & $-$0.001 & 0.007 & 0.009 & $-$0.020 & 0.001 \\ 
  & (0.018) & (0.018) & (0.017) & (0.017) & (0.017) & (0.019) & (0.018) & (0.020) \\ 
  Age: 25\mbox{-}34 & $-$0.024$^{*}$ & 0.025$^{*}$ & $-$0.008 & 0.018 & $-$0.004 & $-$0.003 & $-$0.001 & $-$0.011 \\ 
  & (0.014) & (0.014) & (0.012) & (0.013) & (0.013) & (0.014) & (0.014) & (0.015) \\ 
  Age: 50\mbox{-}64 & $-$0.009 & 0.008 & 0.006 & 0.016 & $-$0.003 & 0.008 & 0.003 & $-$0.009 \\ 
  & (0.013) & (0.013) & (0.012) & (0.012) & (0.012) & (0.013) & (0.013) & (0.013) \\ 
  Age: 65+ & $-$0.013 & 0.031$^{*}$ & 0.011 & 0.030$^{**}$ & $-$0.023 & $-$0.014 & 0.020 & $-$0.0004 \\ 
  & (0.016) & (0.016) & (0.015) & (0.015) & (0.015) & (0.016) & (0.016) & (0.017) \\ 
  Income quartile: Q2 & $-$0.006 & 0.001 & 0.00001 & $-$0.018 & $-$0.008 & $-$0.003 & $-$0.010 & $-$0.012 \\ 
  & (0.013) & (0.013) & (0.012) & (0.012) & (0.012) & (0.013) & (0.013) & (0.014) \\ 
  Income quartile: Q3 & 0.002 & 0.0003 & 0.005 & $-$0.014 & 0.0001 & $-$0.010 & $-$0.013 & 0.004 \\ 
  & (0.013) & (0.013) & (0.012) & (0.012) & (0.012) & (0.014) & (0.014) & (0.014) \\ 
  Income quartile: Q4 & $-$0.015 & 0.012 & $-$0.006 & $-$0.001 & 0.010 & $-$0.012 & 0.001 & $-$0.006 \\ 
  & (0.015) & (0.015) & (0.013) & (0.014) & (0.013) & (0.015) & (0.015) & (0.016) \\ 
  Diploma: Upper secondary & 0.013 & 0.003 & $-$0.0003 & $-$0.010 & 0.012 & $-$0.013 & 0.018 & $-$0.021 \\ 
  & (0.014) & (0.014) & (0.013) & (0.013) & (0.013) & (0.014) & (0.014) & (0.015) \\ 
  Diploma: Above upper secondary & 0.030$^{**}$ & $-$0.008 & 0.002 & $-$0.015 & 0.020 & $-$0.006 & 0.003 & $-$0.004 \\ 
  & (0.015) & (0.014) & (0.013) & (0.014) & (0.013) & (0.015) & (0.015) & (0.015) \\ 
  Urbanicity: Rural & 0.010 & 0.012 & 0.009 & $-$0.013 & $-$0.005 & $-$0.003 & 0.012 & $-$0.006 \\ 
  & (0.014) & (0.014) & (0.013) & (0.013) & (0.013) & (0.014) & (0.014) & (0.015) \\ 
  Urbanicity: Towns and suburbs & 0.021 & $-$0.015 & 0.008 & $-$0.004 & 0.003 & 0.011 & $-$0.015 & 0.004 \\ 
  & (0.015) & (0.015) & (0.014) & (0.013) & (0.013) & (0.015) & (0.015) & (0.016) \\ 
  Will become millionaire: Likely & 0.016 & $-$0.015 & $-$0.004 & 0.014 & $-$0.013 & 0.005 & 0.0002 & 0.006 \\ 
  & (0.011) & (0.011) & (0.010) & (0.010) & (0.010) & (0.011) & (0.011) & (0.012) \\ 
  Will become millionaire: Already & 0.006 & $-$0.010 & $-$0.008 & $-$0.001 & $-$0.014 & 0.029 & $-$0.005 & $-$0.035 \\ 
  & (0.022) & (0.022) & (0.020) & (0.020) & (0.020) & (0.022) & (0.022) & (0.023) \\ 
  Foreign born & $-$0.010 & 0.017 & $-$0.014 & 0.035$^{**}$ & $-$0.012 & 0.014 & $-$0.006 & $-$0.014 \\ 
  & (0.017) & (0.017) & (0.016) & (0.017) & (0.016) & (0.018) & (0.018) & (0.018) \\ 
 \hline \\[-1.8ex] 

Observations & 12,001 & 12,001 & 11,993 & 11,993 & 11,993 & 12,001 & 12,001 & 12,001 \\ 
R$^{2}$ & 0.006 & 0.006 & 0.005 & 0.005 & 0.005 & 0.021 & 0.025 & 0.006 \\ 
\hline 
\hline \\[-1.8ex] 
\end{tabular} 
        }
    }
    \centering {\footnotesize \textit{Note}: Robust standard errors are in parentheses. \textit{CS}: \textit{Climate Scheme}. $^{*}$p$<$0.1; $^{**}$p$<$0.05; $^{***}$p$<$0.01.
    }
\end{table}
\clearpage

\section{Placebo tests}\label{app:placebo}

% \begin{table}[h]\label{tab:placebo}
%     \caption[Placebo tests]{Placebo tests.} 
%     \makebox[\textwidth][c]{
% % \resizebox*{!}{.73\textheight}{ % 73 is the max when there is a title
%         \input{../tables/placebo_tests.tex}
%         }
%     % }
%     {\footnotesize \textit{Note}: Standard errors are reported in parentheses.
%     }
% \end{table}

\section{Main results on the extended sample}\label{app:extended}

% As a robustness check, we reproduce our main results on the extended sample that includes the 14\% respondents who failed the attention check or rushed through the survey ($n = 9,318$). These results are non-weighted. They closely match the results in our main specification. For example, the support for the GCS is 54\% in the U.S. and 75\% in Europe, while the same coefficients are significant for the list experiment. % and the conjoint analyses. 

% \begin{figure}[h!] 
%     \caption[(Extended sample) Main attitudes]{[Extended sample] Main attitudes. \\ (Relative support ---unless *--- in percent in Questions \ref{q:gcs_support}, \ref{q:global_tax}, \ref{q:other_policies}, \ref{q:foreign_aid_raise_support}, \ref{q:negotiation}) \hfill (Back~to~Section~\ref{subsec:universalistic})}\label{fig:main_by_vote_alla}
%     \makebox[\textwidth][c]{\includegraphics[width=\textwidth]{../figures/country_comparison/main_alla_share.pdf}} 
% \end{figure}
 

\clearpage
\renewcommand{\url}[1]{\href{#1}{Link}} 
\putbib
\end{bibunit}

\listoftables
\listoffigures


\end{document}
