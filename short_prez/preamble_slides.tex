% !TeX root = robust_global_redistr.tex
% \documentclass[aspectratio=169,xcolor=dvipsnames, 11pt,mathserif]{beamer} 
% Put this line above % !TeX root = robust_global_redistr.tex
% \documentclass[aspectratio=169,xcolor=dvipsnames, 11pt,mathserif]{beamer} 
% Put this line above % !TeX root = robust_global_redistr.tex
% \documentclass[aspectratio=169,xcolor=dvipsnames, 11pt,mathserif]{beamer} 
% Put this line above % !TeX root = robust_global_redistr.tex
% \documentclass[aspectratio=169,xcolor=dvipsnames, 11pt,mathserif]{beamer} 
% Put this line above \input{preamble_slides.tex} otherwise it may cause bug

%\documentclass[xcolor=dvipsnames,mathserif]{beamer} % this option has curvier math
%\documentclass[xcolor=dvipsnames,11pt]{beamer}
% Note: the color structure needs to be added here in the title. Now it recognizes all beamer % colors.

%%%%%%%%  PRESENTATION LAYOUT:
\usepackage{appendixnumberbeamer} % this package does not count the appendix pages. /!\ Inconstant behavior, sometimes bug.
\mode<presentation>
{
  \usetheme{Boadilla}
  \usecolortheme{lily} % lily is nice or orchid but not for definition
  \setbeamercovered{invisible}
  \setbeamertemplate{footline}{\raggedleft\insertframenumber~/~\inserttotalframenumber\hspace*{3pt}\vskip3pt} %this command shows frame number, not page number at bottom (means if using overlays, % frame number does not change)
%\setbeamertemplate{footline}[page number]  % this puts only page number on bottom
 \setbeamertemplate{navigation symbols}{t}  % this erases navigation symbols.
 \setbeamersize{text margin left=0.1cm, text margin right=0.1cm} % left=0.2cm
 \setbeamertemplate{frametitle}[default][center]
 \setbeamercolor{frametitle}{fg=black}
 \setbeamerfont{frametitle}{size=\large, series=\bfseries} % Modifies frame title font.
 \setbeamercolor{button}{fg=blue, bg=white}
 \setbeamertemplate{itemize item}[circle]
 \setbeamercolor{itemize item}{fg=black}
 \setbeamercolor{itemize/enumerate body}{fg=black}
 \setbeamerfont{framesubtitle}{series=\mdseries}
}

\renewcommand{\familydefault}{\rmdefault} %Options here are \ttdefault \ssdefault or \rmdefault.

% This defines actual color palette; 
\definecolor{blue}{RGB}{0,114,178}
\definecolor{orangered}{RGB}{213,94,0}
\definecolor{yellow}{RGB}{240,228,66}
\definecolor{green}{RGB}{0,158,115}
\definecolor{orange}{RGB}{230,159,0}

\hypersetup{
  colorlinks = false,
  linkbordercolor = {white},
  linkcolor = {blue}
}

%%% Color customization: where these colors are used. 
\colorlet{cwords}{blue} %this defines a color, stored under the name cwords that will be used and recognized in document.
\colorlet{cwordsc}{red} %color for contrast with some other word.
\colorlet{cwords2}{green} %2nd color for contrast with some other word.
\colorlet{cmath}{blue} %color for math in text.
%\everymath{\color{blue}} % This in conjunction with the everysel package sets color of math
\everydisplay{\color{blue}}


%%%%%%%%  PACKAGES USED:
\usepackage{amsmath}
\usepackage{setspace} % Only needed for spacing
\usepackage{changepage} % Only needed for local margin setting
\usepackage{mathpazo}% font, is overwritten by times
%\usepackage[hypertexnames=false]{hyperref} %This makes hyperref ``dumber'', and, hence, more robust! (otherwise sometimes the appendix links don't work).
\usepackage{hyperref}
\usepackage{multimedia}
\usepackage[english]{babel}
\usepackage{graphicx}
\usepackage[justification=centering]{caption}
%\usepackage{subfig}
\usepackage{subfloat}
\usepackage[en-US]{datetime2}
\usepackage{tabulary}
\usepackage{tabularx}
\usepackage{makecell} % multiline cells in tables
\usepackage{array,booktabs} % Needed for esttab tables according to the "inequality survey."
\newcommand{\sym}[1]{{#1}} % for symbols in Table
\usepackage[T1]{fontenc}
\usepackage[utf8]{inputenc}
\usepackage{times} % This is a different font.
\usepackage[overlay,absolute]{textpos}
%%%\usepackage{animate} % Animate graphs BUG
\setlength{\TPHorizModule}{1cm}
\setlength{\TPVertModule}{1cm}
\captionsetup[figure]{labelformat=empty} % removes caption prefix figure
\setlength{\itemsep}{\fill} % this is supposed to stretch items across full frame.
%\setlength{\parskip}{0.8\baselineskip} % This affects spacing between normal lines (not itemized).
\usepackage{colortbl} % For cell colors
\usepackage[final]{pdfpages}
\usepackage{caption}
\usepackage{subcaption}
%\captionsetup{justification   = raggedright,
%              singlelinecheck = false}
%\usepackage{adjustbox} % to use resizebox for tables size.
\usepackage[export]{adjustbox} % to use resizebox for tables size.
\usepackage{eurosym}
\usepackage{gensymb}
\usepackage{dsfont}
%\usepackage{enumitem}
%\setlist[itemize]{leftmargin=*} % remove margin from itemize

%% TIKZ
\usepackage{tikz}
% \usetikzlibrary{er, positioning,decorations.pathmorphing,calc}
% \usepackage{tikzscale}
% %% TIKZIT
% \usetikzlibrary{backgrounds}
% \usetikzlibrary{arrows}
% \usetikzlibrary{shapes,shapes.geometric,shapes.misc}
% % this style is applied by default to any tikzpicture included via \tikzfig
% \tikzstyle{tikzfig}=[baseline=-0.25em,scale=0.5]
% % these are dummy properties used by TikZiT, but ignored by LaTex
% \pgfkeys{/tikz/tikzit fill/.initial=0}
% \pgfkeys{/tikz/tikzit draw/.initial=0}
% \pgfkeys{/tikz/tikzit shape/.initial=0}
% \pgfkeys{/tikz/tikzit category/.initial=0}
% % standard layers used in .tikz files
% \pgfdeclarelayer{edgelayer}
% \pgfdeclarelayer{nodelayer}
% \pgfsetlayers{background,edgelayer,nodelayer,main}
% % style for blank nodes
% \tikzstyle{none}=[inner sep=0mm]
% % include a .tikz file
% \newcommand{\tikzfig}[1]{%
% {\tikzstyle{every picture}=[tikzfig]
% \IfFileExists{#1.tikz}
%   {\input{#1.tikz}}
%   {%
%     \IfFileExists{../figures/#1.tikz}
%       {\input{../figures/#1.tikz}}
%       {\tikz[baseline=-0.5em]{\node[draw=red,font=\color{red},fill=red!10!white] {\textit{#1}};}}%
%   }}%
% }
% % the same as \tikzfig, but in a {center} environment
% \newcommand{\ctikzfig}[1]{%
% \begin{center}\rm
%   \tikzfig{#1}
% \end{center}}
% % fix strange self-loops, which are PGF/TikZ default
% \tikzstyle{every loop}=[]
% \newcommand*\halfcirc{%
% %  \begin{tikzpicture}
% %  \draw[fill] (2,1)-- (90:1ex) arc (90:270:1ex) -- cycle ;
% %  \draw (2,1) circle (1ex);
% %  \end{tikzpicture}
% }
  
% \input{../latex/default.tikzstyles}


% \tikzset{every entity/.style={draw=black, fill=white}}
% \tikzset{comment/.style={draw=white, fill=white}}
% \tikzset{
% 	invisible/.style={opacity=0},
% 	visible on/.style={alt=#1{}{invisible}},
% 	alt/.code args={<#1>#2#3}{%
% 		\alt<#1>{\pgfkeysalso{#2}}{\pgfkeysalso{#3}} % \pgfkeysalso doesn't change the path
% 	},
% }

% Commands from template
\newcommand{\alrt}[1]{{\color{alert} #1}}
\newcommand{\alrtl}[1]{{\color{alert}\large #1}}
\newcommand{\alrtL}[1]{{\color{alert}\Large #1}}
\newcommand{\struc}[1]{{\color{structure} #1}}
\newcommand{\strucL}[1]{{\color{structure}\Large #1}}
\newcommand{\strucl}[1]{{\color{structure}\large #1}}
\newcommand{\dred}[1]{{\color{darkred} #1}}
\newcommand{\dredl}[1]{{\color{darkred}\large #1}}
\newcommand{\dredL}[1]{{\color{darkred}\Large #1}}
\newcommand{\altc}[1]{{\color{darkgreen}\textbf{#1}}}
\newcommand{\altcl}[1]{{\color{darkgreen}\textbf{\large #1}}}
\newcommand{\altcL}[1]{{\color{darkgreen}\textbf{\Large #1}}}
\newcommand{\hush}{\hushit}
\newcommand{\hushalrt}[1]{\hushit{{\color{alert} #1}}}
\newcommand{\hushalrtl}[1]{\hushit{{\large\color{alert} #1}}}
\newcommand{\hushalrtL}[1]{\hushit{{\Large\color{alert} #1}}}
\newcommand{\hushstruc}[1]{\hushit{{\color{structure} #1}}}
\newcommand{\hushstrucl}[1]{\hushit{{\large\color{structure} #1}}}
\newcommand{\hushstrucL}[1]{\hushit{{\Large\color{structure} #1}}}


\setbeamertemplate{caption}[numbered]

%%%%%%% SIGN NOBEL PRIZE %%%%%%%%
\usepackage{wasysym}
\newcommand{\nobel}{$\taurus$} % other possibilities:\logof\odot\varocircle\taurus\kreuz
\newcommand{\Nobel}[1]{$\taurus_{#1}$}
\newcommand{\Leontief}[1]{$\logof_{#1}$}
\newcommand{\leontief}{$\logof_$}

%%%%%%%%%SECTION TITLES DISPLAYED ON FULL PAGE %%%%%%%%%%
%%%%%%%%%%%%%%%%%%%%%%%%%%%%%%%%%%%%%%%%%%%
\AtBeginSection[]{
  \begin{frame}
  \vfill
  \centering
  \begin{beamercolorbox}[sep=8pt,center,shadow=true,rounded=true]{title}
    \usebeamerfont{title}{\huge \color{orangered} \insertsectionhead} \par
  \end{beamercolorbox}
  \vfill
  \end{frame}
}

\AtBeginSubsection[]{
  \begin{frame}
  \vfill
  \centering
  \begin{beamercolorbox}[sep=8pt,center,shadow=true,rounded=true]{title}
    \usebeamerfont{title}{\huge \color{blue} \insertsubsectionhead} \par
  \end{beamercolorbox}
  \vfill
  \end{frame}
}

%
% Custom font for a frame.
%
\usepackage{environ}
\newcommand{\customframefont}[1]{
\setbeamertemplate{itemize/enumerate body begin}{#1}
\setbeamertemplate{itemize/enumerate subbody begin}{#1}
}

\NewEnviron{framefont}[1]{
\customframefont{#1} % for itemize/enumerate
{#1 % For the text outside itemize/enumerate
\BODY
}
\customframefont{\normalsize}
}

%%%%%%%%%%%%%%%%%%%%%%%%%%%%%%%%%%%%%%%%%%%%%%%%%%%%%%%
%%%%%% OTHER PIECES OF TEMPLATE FILE
%%%%%% DELETE ONCE CLEAR THAT NOTHING IS MISSING
%%%%%%%%%%%%%%%%%%%%%%%%%%%%%%%%%%%%%%%%%%%%%%%%%%%%%%%


%\documentclass[aspectratio=169]{beamer} % wide
%\usepackage{amsmath,amsthm,fancyhdr,setspace,graphicx,booktabs,pdflscape}
%\usepackage{geometry}
%\usepackage{etex}
%\usepackage{xcolor,colortbl}
%\usepackage{beamerprosper}
%\usepackage{url}
%\usepackage{enumerate}
%\usepackage{graphicx}
%\usepackage{hyperref}
%\usepackage{multicol}
%\usepackage{caption}
%\usepackage{beamerprosper}
%\usepackage{pgfpages, pdfpages}
%\usepackage{tikz-cd}
%
%\usepackage{tabularx}
%\usepackage{anyfontsize}
%\usepackage{multicol,tabto}
%
%\usepackage{float}
%\usepackage{soul}
%\usepackage{grffile}
%\usepackage{changepage}
%\usepackage{sansmathaccent}
%\pdfmapfile{+sansmathaccent.map}
%
%
%\usetikzlibrary{er,positioning,calc,decorations.pathreplacing}
%
%\mode<presentation>
%\usefonttheme{structuresmallcapsserif}
%\setbeamertemplate{footline}[frame number]{}
%\setbeamertemplate{navigation symbols}{}
%
%%change font
%\usefonttheme{default}
%\setbeamertemplate{footline}[frame number]{}
%\setbeamertemplate{navigation symbols}{}
%
%\newcommand{\fig}[3]{\begin{frame}\frametitle{#2}\centerline{\includegraphics[width=#3in]{#1}}\end{frame}}
%
%\newcommand{\blackslide}[1]{\beamersetaveragebackground{black}\begin{frame}\frametitle{}\end{frame}\beamersetaveragebackground{white}}
%
%% pause commands
%\newcommand{\m}[2]{\begin{frame}\frametitle{#1}{ #2}\end{frame}}
%\newcommand{\mm}[3]{\begin{frame}\frametitle{#1}\uncover<1->{ #2}\uncover<2->{ #3 }\end{frame}}
%\newcommand{\mmm}[4]{\begin{frame}\frametitle{#1}\uncover<1->{ #2}\uncover<2->{ #3 }\uncover<3->{ #4 }\end{frame}}
%\newcommand{\mmmm}[5]{\begin{frame}\frametitle{#1}\uncover<1->{ #2}\uncover<2->{ #3 }\uncover<3->{ #4 }\uncover<4->{ #5 }\end{frame}}
%
%\setlength{\footskip}{24pt}
%
%\newcommand{\ex}{\mathbf{E}}
%\newcommand{\cov}{\mathbb{C}}
%\newcommand{\var}{\mathbb{V}}
%\newcommand{\tu}{\overline{\theta}}
%\newcommand{\vu}{\overline{v}}
%\newcommand{\tl}{\underline{\theta}}
%\newcommand{\ab}{\bar{a}}
%\newcommand{\lb}{\bar{L}}
%\newcommand{\hb}{\bar{H}}
%
%% New colors.
%\definecolor{darkred}{rgb}{0.6,0,0}
%\definecolor{darkblue}{rgb}{.15,.25,.55}
%\definecolor{darkgreen}{rgb}{0,.35,.05}
%\definecolor{ltgreen}{rgb}{0,.05,.8}
%\definecolor{bred}{rgb}{1,0,.05}
%\definecolor{navy}{rgb}{.1,.1,.5}
%
%% from beamer lecture 
%
%\newtheorem{proposition}{Proposition}
%%\newtheorem{theorem}{Theorem}
%
%\newenvironment{changemargin}[2]{%
%    \begin{list}{}{ %
%%        \setlength{\topmargin}{#3}%
%            \setlength{\topsep}{0pt}%
%            \setlength{\leftmargin}{#1}%
%            \setlength{\rightmargin}{#2}%
%            \setlength{\listparindent}{\parindent}%
%            \setlength{\itemindent}{\parindent}%
%            \setlength{\parsep}{\parskip}%
%        }%
%\item[]}{\end{list}}
%
%%Allows us to force a column width in array environment
%\newcolumntype{C}[1]{>{\centering\arraybackslash}p{#1}}
%\newcolumntype{L}[1]{>{\raggedright\arraybackslash}p{#1}}
%\newcolumntype{R}[1]{>{\raggedleft\arraybackslash}p{#1}}


%%SHORTCUTS

%%%%%%%%%%%%%%%%%%%%%%%%%
%% Bullets
%%%%%%%%%%%%%%%%%%%%%%%%%
%% These six lines allows to define "myItemize" environment where nested lists appear at once in overlays
%\let\oldItemize\itemize
%\let\endoldItemize\enditemize
%\newcommand{\myItemize}[1][<1->]{\oldItemize[#1]}
%\def\endmyItemize{\endoldItemize}
%\let\itemize\myItemize
%\let\enditemize\endmyItemize

\newcommand{\p}{\item}
\newcommand{\ip}{\item[]} % Invisible items.
\newcommand{\bb}{\begin{itemize}\itemsep15pt}
\newcommand{\bbs}{\medskip \begin{itemize}[<1->]\itemsep10pt}
\newcommand{\bbsp}{\medskip \begin{itemize}[<+->]\itemsep10pt}% items appear progressively
\newcommand{\bbvs}{
	\settowidth{\leftmargini}{\usebeamertemplate{itemize item}} % Remove identation of itemize 
	\medskip \begin{itemize}[<1->]\itemsep3pt  
	\setlength{\leftmargini}{0.4cm} % to adjust left margin of items
 	\setlength{\leftmarginii}{0.3cm} % same for nested items \setlength{\leftmarginiii}{0cm}
} %  \setlength\itemsep{3pt}
\newcommand{\bbvsp}{ % items appear progressively
	\settowidth{\leftmargini}{\usebeamertemplate{itemize item}}
	\medskip \begin{itemize}[<+->]\itemsep3pt % replace myItemize by itemize to have nested items also appear one by one
	\setlength{\leftmargini}{0.4cm} 
 	\setlength{\leftmarginii}{0.3cm}
}
\newcommand{\bbvsn}{ % no margin
	\settowidth{\leftmargini}{\usebeamertemplate{itemize item}} %  
	\medskip \begin{itemize}[<1->]\itemsep3pt  
	\setlength{\leftmargini}{0.4cm}
 	\setlength{\leftmarginii}{-0cm} 
} %  \setlength\itemsep{3pt}
\newcommand{\bbvsnp}{ % items appear progressively
	\settowidth{\leftmargini}{\usebeamertemplate{itemize item}}
	\medskip \begin{itemize}[<+->]\itemsep3pt
	\setlength{\leftmargini}{0.4cm} 
 	\setlength{\leftmarginii}{-0cm}
}
\newcommand{\ee}{\end{itemize} 
\smallskip}
\newcommand{\ees}{\end{itemize} }

\newcommand{\ben}{\begin{enumerate}}
\newcommand{\een}{\end{enumerate}}

\newcommand{\gras}[1]{\textbf{#1}}
\newcommand{\blue}[1]{\textcolor{blue}{#1}}
\newcommand{\green}[1]{\textcolor{green}{#1}}
\newcommand{\red}[1]{\textcolor{red}{#1}}
\newcommand{\orangered}[1]{\textcolor{orangered}{#1}}
\newcommand{\orange}[1]{\textcolor{orange}{#1}}
\newcommand{\yellow}[1]{\textcolor{yellow}{#1}}
\newcommand{\magenta}[1]{\textcolor{magenta}{#1}}
\newcommand{\rose}[1]{\textcolor{magenta}{#1}}

\newcommand{\can}{\citeasnoun}
\newcommand{\ican}{\iciteasnoun}

\newcommand{\non}{\nonumber}

%%%%%%%%%%%%%%%%%%%%%%%%%
%% Derivatives and partials. 
%%%%%%%%%%%%%%%%%%%%%%%%%
%% Duplicate: two ways to get partials
\newcommand{\pa}[2]{\frac{\partial #1}{\partial #2}} % Stef
\newcommand{\pder}[2]{\frac{\partial #1}{\partial #2}} % Doug

\newcommand{\dneu}{\mbox{d}}
\newcommand{\di}[2]{\frac{\dneu #1}{\dneu #2}}
%% Duplicate: two ways to get derivatives.
\newcommand{\dd}[2]{\frac{d #1}{d #2}} % Stef version
\newcommand{\der}[2]{\frac{d#1}{d#2}} % Doug version

%%%%%%%%%%%%%%%%%%%%%%%%%
%% Brackets and fractions
%%%%%%%%%%%%%%%%%%%%%%%%%


\newcommand{\fr}[2]{\frac{#1}{#2}}
\newcommand{\pfr}[2]{\left(\frac{#1}{#2}\right)}
\newcommand{\bfr}[2]{\left[\frac{#1}{#2}\right]}
\newcommand{\cfr}[2]{\left\{\frac{#1}{#2}\right\}}

\newcommand{\pr}[1]{\left(#1\right)}
\newcommand{\br}[1]{\left[#1\right]}
\newcommand{\cb}[1]{\left\{#1\right\}}
\newcommand{\qand}{\quad\text{and}\quad}


%%%%%%%%%%%%%%%%%%%%%%%%%
%% ARROWS
%%%%%%%%%%%%%%%%%%%%%%%%%

\newcommand{\Ra}{\Rightarrow}
\newcommand{\ra}{\rightarrow}
\newcommand{\Ras}{\ \Rightarrow\ }
\newcommand{\ras}{\ \rightarrow\ }
\newcommand{\Raq}{\quad\Rightarrow\quad}
\newcommand{\raq}{\quad\rightarrow\quad}


%%%%%%%%%%%%%%%%%%%%%%%%%
%% MATH
%%%%%%%%%%%%%%%%%%%%%%%%%

\newcommand{\E}[1]{\mathbb{E}\br{#1}}
\newcommand{\eps}{\varepsilon}

\newcommand{\be}{\begin{equation}}
\newcommand{\eeq}{\end{equation}}

\newcommand{\bea}{\begin{eqnarray}}
\newcommand{\eea}{\end{eqnarray}}

\newcommand{\bean}{\begin{eqnarray*}}
\newcommand{\eean}{\end{eqnarray*}}

\newcommand{\ba}{\begin{array}}
\newcommand{\ea}{\end{array}}


\newcommand{\lb}{\linebreak}
\newcommand{\strich}[2]{\left. #1 \right|_{#2}}
% use \left. before


%\newcommand{\bean}{\begin{multline*}}
%\newcommand{\eean}{\end{multline*}}

\newcommand{\nl}{\newline}
\newcommand{\np}{\newpage}

\newcommand{\rf}[1]{(\ref{#1})}

\newcommand{\1}{\mathds{1}}

\newcommand{\ds}{\displaystyle}
\newcommand{\fn}{\footnote}

\newcommand{\var}{\mbox{Var}}
\newcommand{\cov}{\mbox{Cov}}

\newcommand{\il}{\int\limits}

\newcommand{\li}{\left}
\newcommand{\re}{\right}

\newcommand{\s}{\right|_}

\newcommand{\ambigo}{\ba{c}>\\[-3mm]<\ea}
\newcommand{\ambigu}{\ba{c}<\\[-3mm]>\ea}

\newcommand{\ul}{\underline}
\newcommand{\ol}{\overline}

\newcommand{\e}{\mbox{\euro{} }}
 otherwise it may cause bug

%\documentclass[xcolor=dvipsnames,mathserif]{beamer} % this option has curvier math
%\documentclass[xcolor=dvipsnames,11pt]{beamer}
% Note: the color structure needs to be added here in the title. Now it recognizes all beamer % colors.

%%%%%%%%  PRESENTATION LAYOUT:
\usepackage{appendixnumberbeamer} % this package does not count the appendix pages. /!\ Inconstant behavior, sometimes bug.
\mode<presentation>
{
  \usetheme{Boadilla}
  \usecolortheme{lily} % lily is nice or orchid but not for definition
  \setbeamercovered{invisible}
  \setbeamertemplate{footline}{\raggedleft\insertframenumber~/~\inserttotalframenumber\hspace*{3pt}\vskip3pt} %this command shows frame number, not page number at bottom (means if using overlays, % frame number does not change)
%\setbeamertemplate{footline}[page number]  % this puts only page number on bottom
 \setbeamertemplate{navigation symbols}{t}  % this erases navigation symbols.
 \setbeamersize{text margin left=0.1cm, text margin right=0.1cm} % left=0.2cm
 \setbeamertemplate{frametitle}[default][center]
 \setbeamercolor{frametitle}{fg=black}
 \setbeamerfont{frametitle}{size=\large, series=\bfseries} % Modifies frame title font.
 \setbeamercolor{button}{fg=blue, bg=white}
 \setbeamertemplate{itemize item}[circle]
 \setbeamercolor{itemize item}{fg=black}
 \setbeamercolor{itemize/enumerate body}{fg=black}
 \setbeamerfont{framesubtitle}{series=\mdseries}
}

\renewcommand{\familydefault}{\rmdefault} %Options here are \ttdefault \ssdefault or \rmdefault.

% This defines actual color palette; 
\definecolor{blue}{RGB}{0,114,178}
\definecolor{orangered}{RGB}{213,94,0}
\definecolor{yellow}{RGB}{240,228,66}
\definecolor{green}{RGB}{0,158,115}
\definecolor{orange}{RGB}{230,159,0}

\hypersetup{
  colorlinks = false,
  linkbordercolor = {white},
  linkcolor = {blue}
}

%%% Color customization: where these colors are used. 
\colorlet{cwords}{blue} %this defines a color, stored under the name cwords that will be used and recognized in document.
\colorlet{cwordsc}{red} %color for contrast with some other word.
\colorlet{cwords2}{green} %2nd color for contrast with some other word.
\colorlet{cmath}{blue} %color for math in text.
%\everymath{\color{blue}} % This in conjunction with the everysel package sets color of math
\everydisplay{\color{blue}}


%%%%%%%%  PACKAGES USED:
\usepackage{amsmath}
\usepackage{setspace} % Only needed for spacing
\usepackage{changepage} % Only needed for local margin setting
\usepackage{mathpazo}% font, is overwritten by times
%\usepackage[hypertexnames=false]{hyperref} %This makes hyperref ``dumber'', and, hence, more robust! (otherwise sometimes the appendix links don't work).
\usepackage{hyperref}
\usepackage{multimedia}
\usepackage[english]{babel}
\usepackage{graphicx}
\usepackage[justification=centering]{caption}
%\usepackage{subfig}
\usepackage{subfloat}
\usepackage[en-US]{datetime2}
\usepackage{tabulary}
\usepackage{tabularx}
\usepackage{makecell} % multiline cells in tables
\usepackage{array,booktabs} % Needed for esttab tables according to the "inequality survey."
\newcommand{\sym}[1]{{#1}} % for symbols in Table
\usepackage[T1]{fontenc}
\usepackage[utf8]{inputenc}
\usepackage{times} % This is a different font.
\usepackage[overlay,absolute]{textpos}
%%%\usepackage{animate} % Animate graphs BUG
\setlength{\TPHorizModule}{1cm}
\setlength{\TPVertModule}{1cm}
\captionsetup[figure]{labelformat=empty} % removes caption prefix figure
\setlength{\itemsep}{\fill} % this is supposed to stretch items across full frame.
%\setlength{\parskip}{0.8\baselineskip} % This affects spacing between normal lines (not itemized).
\usepackage{colortbl} % For cell colors
\usepackage[final]{pdfpages}
\usepackage{caption}
\usepackage{subcaption}
%\captionsetup{justification   = raggedright,
%              singlelinecheck = false}
%\usepackage{adjustbox} % to use resizebox for tables size.
\usepackage[export]{adjustbox} % to use resizebox for tables size.
\usepackage{eurosym}
\usepackage{gensymb}
\usepackage{dsfont}
%\usepackage{enumitem}
%\setlist[itemize]{leftmargin=*} % remove margin from itemize

%% TIKZ
\usepackage{tikz}
% \usetikzlibrary{er, positioning,decorations.pathmorphing,calc}
% \usepackage{tikzscale}
% %% TIKZIT
% \usetikzlibrary{backgrounds}
% \usetikzlibrary{arrows}
% \usetikzlibrary{shapes,shapes.geometric,shapes.misc}
% % this style is applied by default to any tikzpicture included via \tikzfig
% \tikzstyle{tikzfig}=[baseline=-0.25em,scale=0.5]
% % these are dummy properties used by TikZiT, but ignored by LaTex
% \pgfkeys{/tikz/tikzit fill/.initial=0}
% \pgfkeys{/tikz/tikzit draw/.initial=0}
% \pgfkeys{/tikz/tikzit shape/.initial=0}
% \pgfkeys{/tikz/tikzit category/.initial=0}
% % standard layers used in .tikz files
% \pgfdeclarelayer{edgelayer}
% \pgfdeclarelayer{nodelayer}
% \pgfsetlayers{background,edgelayer,nodelayer,main}
% % style for blank nodes
% \tikzstyle{none}=[inner sep=0mm]
% % include a .tikz file
% \newcommand{\tikzfig}[1]{%
% {\tikzstyle{every picture}=[tikzfig]
% \IfFileExists{#1.tikz}
%   {\input{#1.tikz}}
%   {%
%     \IfFileExists{../figures/#1.tikz}
%       {\input{../figures/#1.tikz}}
%       {\tikz[baseline=-0.5em]{\node[draw=red,font=\color{red},fill=red!10!white] {\textit{#1}};}}%
%   }}%
% }
% % the same as \tikzfig, but in a {center} environment
% \newcommand{\ctikzfig}[1]{%
% \begin{center}\rm
%   \tikzfig{#1}
% \end{center}}
% % fix strange self-loops, which are PGF/TikZ default
% \tikzstyle{every loop}=[]
% \newcommand*\halfcirc{%
% %  \begin{tikzpicture}
% %  \draw[fill] (2,1)-- (90:1ex) arc (90:270:1ex) -- cycle ;
% %  \draw (2,1) circle (1ex);
% %  \end{tikzpicture}
% }
  
% \input{../latex/default.tikzstyles}


% \tikzset{every entity/.style={draw=black, fill=white}}
% \tikzset{comment/.style={draw=white, fill=white}}
% \tikzset{
% 	invisible/.style={opacity=0},
% 	visible on/.style={alt=#1{}{invisible}},
% 	alt/.code args={<#1>#2#3}{%
% 		\alt<#1>{\pgfkeysalso{#2}}{\pgfkeysalso{#3}} % \pgfkeysalso doesn't change the path
% 	},
% }

% Commands from template
\newcommand{\alrt}[1]{{\color{alert} #1}}
\newcommand{\alrtl}[1]{{\color{alert}\large #1}}
\newcommand{\alrtL}[1]{{\color{alert}\Large #1}}
\newcommand{\struc}[1]{{\color{structure} #1}}
\newcommand{\strucL}[1]{{\color{structure}\Large #1}}
\newcommand{\strucl}[1]{{\color{structure}\large #1}}
\newcommand{\dred}[1]{{\color{darkred} #1}}
\newcommand{\dredl}[1]{{\color{darkred}\large #1}}
\newcommand{\dredL}[1]{{\color{darkred}\Large #1}}
\newcommand{\altc}[1]{{\color{darkgreen}\textbf{#1}}}
\newcommand{\altcl}[1]{{\color{darkgreen}\textbf{\large #1}}}
\newcommand{\altcL}[1]{{\color{darkgreen}\textbf{\Large #1}}}
\newcommand{\hush}{\hushit}
\newcommand{\hushalrt}[1]{\hushit{{\color{alert} #1}}}
\newcommand{\hushalrtl}[1]{\hushit{{\large\color{alert} #1}}}
\newcommand{\hushalrtL}[1]{\hushit{{\Large\color{alert} #1}}}
\newcommand{\hushstruc}[1]{\hushit{{\color{structure} #1}}}
\newcommand{\hushstrucl}[1]{\hushit{{\large\color{structure} #1}}}
\newcommand{\hushstrucL}[1]{\hushit{{\Large\color{structure} #1}}}


\setbeamertemplate{caption}[numbered]

%%%%%%% SIGN NOBEL PRIZE %%%%%%%%
\usepackage{wasysym}
\newcommand{\nobel}{$\taurus$} % other possibilities:\logof\odot\varocircle\taurus\kreuz
\newcommand{\Nobel}[1]{$\taurus_{#1}$}
\newcommand{\Leontief}[1]{$\logof_{#1}$}
\newcommand{\leontief}{$\logof_$}

%%%%%%%%%SECTION TITLES DISPLAYED ON FULL PAGE %%%%%%%%%%
%%%%%%%%%%%%%%%%%%%%%%%%%%%%%%%%%%%%%%%%%%%
\AtBeginSection[]{
  \begin{frame}
  \vfill
  \centering
  \begin{beamercolorbox}[sep=8pt,center,shadow=true,rounded=true]{title}
    \usebeamerfont{title}{\huge \color{orangered} \insertsectionhead} \par
  \end{beamercolorbox}
  \vfill
  \end{frame}
}

\AtBeginSubsection[]{
  \begin{frame}
  \vfill
  \centering
  \begin{beamercolorbox}[sep=8pt,center,shadow=true,rounded=true]{title}
    \usebeamerfont{title}{\huge \color{blue} \insertsubsectionhead} \par
  \end{beamercolorbox}
  \vfill
  \end{frame}
}

%
% Custom font for a frame.
%
\usepackage{environ}
\newcommand{\customframefont}[1]{
\setbeamertemplate{itemize/enumerate body begin}{#1}
\setbeamertemplate{itemize/enumerate subbody begin}{#1}
}

\NewEnviron{framefont}[1]{
\customframefont{#1} % for itemize/enumerate
{#1 % For the text outside itemize/enumerate
\BODY
}
\customframefont{\normalsize}
}

%%%%%%%%%%%%%%%%%%%%%%%%%%%%%%%%%%%%%%%%%%%%%%%%%%%%%%%
%%%%%% OTHER PIECES OF TEMPLATE FILE
%%%%%% DELETE ONCE CLEAR THAT NOTHING IS MISSING
%%%%%%%%%%%%%%%%%%%%%%%%%%%%%%%%%%%%%%%%%%%%%%%%%%%%%%%


%\documentclass[aspectratio=169]{beamer} % wide
%\usepackage{amsmath,amsthm,fancyhdr,setspace,graphicx,booktabs,pdflscape}
%\usepackage{geometry}
%\usepackage{etex}
%\usepackage{xcolor,colortbl}
%\usepackage{beamerprosper}
%\usepackage{url}
%\usepackage{enumerate}
%\usepackage{graphicx}
%\usepackage{hyperref}
%\usepackage{multicol}
%\usepackage{caption}
%\usepackage{beamerprosper}
%\usepackage{pgfpages, pdfpages}
%\usepackage{tikz-cd}
%
%\usepackage{tabularx}
%\usepackage{anyfontsize}
%\usepackage{multicol,tabto}
%
%\usepackage{float}
%\usepackage{soul}
%\usepackage{grffile}
%\usepackage{changepage}
%\usepackage{sansmathaccent}
%\pdfmapfile{+sansmathaccent.map}
%
%
%\usetikzlibrary{er,positioning,calc,decorations.pathreplacing}
%
%\mode<presentation>
%\usefonttheme{structuresmallcapsserif}
%\setbeamertemplate{footline}[frame number]{}
%\setbeamertemplate{navigation symbols}{}
%
%%change font
%\usefonttheme{default}
%\setbeamertemplate{footline}[frame number]{}
%\setbeamertemplate{navigation symbols}{}
%
%\newcommand{\fig}[3]{\begin{frame}\frametitle{#2}\centerline{\includegraphics[width=#3in]{#1}}\end{frame}}
%
%\newcommand{\blackslide}[1]{\beamersetaveragebackground{black}\begin{frame}\frametitle{}\end{frame}\beamersetaveragebackground{white}}
%
%% pause commands
%\newcommand{\m}[2]{\begin{frame}\frametitle{#1}{ #2}\end{frame}}
%\newcommand{\mm}[3]{\begin{frame}\frametitle{#1}\uncover<1->{ #2}\uncover<2->{ #3 }\end{frame}}
%\newcommand{\mmm}[4]{\begin{frame}\frametitle{#1}\uncover<1->{ #2}\uncover<2->{ #3 }\uncover<3->{ #4 }\end{frame}}
%\newcommand{\mmmm}[5]{\begin{frame}\frametitle{#1}\uncover<1->{ #2}\uncover<2->{ #3 }\uncover<3->{ #4 }\uncover<4->{ #5 }\end{frame}}
%
%\setlength{\footskip}{24pt}
%
%\newcommand{\ex}{\mathbf{E}}
%\newcommand{\cov}{\mathbb{C}}
%\newcommand{\var}{\mathbb{V}}
%\newcommand{\tu}{\overline{\theta}}
%\newcommand{\vu}{\overline{v}}
%\newcommand{\tl}{\underline{\theta}}
%\newcommand{\ab}{\bar{a}}
%\newcommand{\lb}{\bar{L}}
%\newcommand{\hb}{\bar{H}}
%
%% New colors.
%\definecolor{darkred}{rgb}{0.6,0,0}
%\definecolor{darkblue}{rgb}{.15,.25,.55}
%\definecolor{darkgreen}{rgb}{0,.35,.05}
%\definecolor{ltgreen}{rgb}{0,.05,.8}
%\definecolor{bred}{rgb}{1,0,.05}
%\definecolor{navy}{rgb}{.1,.1,.5}
%
%% from beamer lecture 
%
%\newtheorem{proposition}{Proposition}
%%\newtheorem{theorem}{Theorem}
%
%\newenvironment{changemargin}[2]{%
%    \begin{list}{}{ %
%%        \setlength{\topmargin}{#3}%
%            \setlength{\topsep}{0pt}%
%            \setlength{\leftmargin}{#1}%
%            \setlength{\rightmargin}{#2}%
%            \setlength{\listparindent}{\parindent}%
%            \setlength{\itemindent}{\parindent}%
%            \setlength{\parsep}{\parskip}%
%        }%
%\item[]}{\end{list}}
%
%%Allows us to force a column width in array environment
%\newcolumntype{C}[1]{>{\centering\arraybackslash}p{#1}}
%\newcolumntype{L}[1]{>{\raggedright\arraybackslash}p{#1}}
%\newcolumntype{R}[1]{>{\raggedleft\arraybackslash}p{#1}}


%%SHORTCUTS

%%%%%%%%%%%%%%%%%%%%%%%%%
%% Bullets
%%%%%%%%%%%%%%%%%%%%%%%%%
%% These six lines allows to define "myItemize" environment where nested lists appear at once in overlays
%\let\oldItemize\itemize
%\let\endoldItemize\enditemize
%\newcommand{\myItemize}[1][<1->]{\oldItemize[#1]}
%\def\endmyItemize{\endoldItemize}
%\let\itemize\myItemize
%\let\enditemize\endmyItemize

\newcommand{\p}{\item}
\newcommand{\ip}{\item[]} % Invisible items.
\newcommand{\bb}{\begin{itemize}\itemsep15pt}
\newcommand{\bbs}{\medskip \begin{itemize}[<1->]\itemsep10pt}
\newcommand{\bbsp}{\medskip \begin{itemize}[<+->]\itemsep10pt}% items appear progressively
\newcommand{\bbvs}{
	\settowidth{\leftmargini}{\usebeamertemplate{itemize item}} % Remove identation of itemize 
	\medskip \begin{itemize}[<1->]\itemsep3pt  
	\setlength{\leftmargini}{0.4cm} % to adjust left margin of items
 	\setlength{\leftmarginii}{0.3cm} % same for nested items \setlength{\leftmarginiii}{0cm}
} %  \setlength\itemsep{3pt}
\newcommand{\bbvsp}{ % items appear progressively
	\settowidth{\leftmargini}{\usebeamertemplate{itemize item}}
	\medskip \begin{itemize}[<+->]\itemsep3pt % replace myItemize by itemize to have nested items also appear one by one
	\setlength{\leftmargini}{0.4cm} 
 	\setlength{\leftmarginii}{0.3cm}
}
\newcommand{\bbvsn}{ % no margin
	\settowidth{\leftmargini}{\usebeamertemplate{itemize item}} %  
	\medskip \begin{itemize}[<1->]\itemsep3pt  
	\setlength{\leftmargini}{0.4cm}
 	\setlength{\leftmarginii}{-0cm} 
} %  \setlength\itemsep{3pt}
\newcommand{\bbvsnp}{ % items appear progressively
	\settowidth{\leftmargini}{\usebeamertemplate{itemize item}}
	\medskip \begin{itemize}[<+->]\itemsep3pt
	\setlength{\leftmargini}{0.4cm} 
 	\setlength{\leftmarginii}{-0cm}
}
\newcommand{\ee}{\end{itemize} 
\smallskip}
\newcommand{\ees}{\end{itemize} }

\newcommand{\ben}{\begin{enumerate}}
\newcommand{\een}{\end{enumerate}}

\newcommand{\gras}[1]{\textbf{#1}}
\newcommand{\blue}[1]{\textcolor{blue}{#1}}
\newcommand{\green}[1]{\textcolor{green}{#1}}
\newcommand{\red}[1]{\textcolor{red}{#1}}
\newcommand{\orangered}[1]{\textcolor{orangered}{#1}}
\newcommand{\orange}[1]{\textcolor{orange}{#1}}
\newcommand{\yellow}[1]{\textcolor{yellow}{#1}}
\newcommand{\magenta}[1]{\textcolor{magenta}{#1}}
\newcommand{\rose}[1]{\textcolor{magenta}{#1}}

\newcommand{\can}{\citeasnoun}
\newcommand{\ican}{\iciteasnoun}

\newcommand{\non}{\nonumber}

%%%%%%%%%%%%%%%%%%%%%%%%%
%% Derivatives and partials. 
%%%%%%%%%%%%%%%%%%%%%%%%%
%% Duplicate: two ways to get partials
\newcommand{\pa}[2]{\frac{\partial #1}{\partial #2}} % Stef
\newcommand{\pder}[2]{\frac{\partial #1}{\partial #2}} % Doug

\newcommand{\dneu}{\mbox{d}}
\newcommand{\di}[2]{\frac{\dneu #1}{\dneu #2}}
%% Duplicate: two ways to get derivatives.
\newcommand{\dd}[2]{\frac{d #1}{d #2}} % Stef version
\newcommand{\der}[2]{\frac{d#1}{d#2}} % Doug version

%%%%%%%%%%%%%%%%%%%%%%%%%
%% Brackets and fractions
%%%%%%%%%%%%%%%%%%%%%%%%%


\newcommand{\fr}[2]{\frac{#1}{#2}}
\newcommand{\pfr}[2]{\left(\frac{#1}{#2}\right)}
\newcommand{\bfr}[2]{\left[\frac{#1}{#2}\right]}
\newcommand{\cfr}[2]{\left\{\frac{#1}{#2}\right\}}

\newcommand{\pr}[1]{\left(#1\right)}
\newcommand{\br}[1]{\left[#1\right]}
\newcommand{\cb}[1]{\left\{#1\right\}}
\newcommand{\qand}{\quad\text{and}\quad}


%%%%%%%%%%%%%%%%%%%%%%%%%
%% ARROWS
%%%%%%%%%%%%%%%%%%%%%%%%%

\newcommand{\Ra}{\Rightarrow}
\newcommand{\ra}{\rightarrow}
\newcommand{\Ras}{\ \Rightarrow\ }
\newcommand{\ras}{\ \rightarrow\ }
\newcommand{\Raq}{\quad\Rightarrow\quad}
\newcommand{\raq}{\quad\rightarrow\quad}


%%%%%%%%%%%%%%%%%%%%%%%%%
%% MATH
%%%%%%%%%%%%%%%%%%%%%%%%%

\newcommand{\E}[1]{\mathbb{E}\br{#1}}
\newcommand{\eps}{\varepsilon}

\newcommand{\be}{\begin{equation}}
\newcommand{\eeq}{\end{equation}}

\newcommand{\bea}{\begin{eqnarray}}
\newcommand{\eea}{\end{eqnarray}}

\newcommand{\bean}{\begin{eqnarray*}}
\newcommand{\eean}{\end{eqnarray*}}

\newcommand{\ba}{\begin{array}}
\newcommand{\ea}{\end{array}}


\newcommand{\lb}{\linebreak}
\newcommand{\strich}[2]{\left. #1 \right|_{#2}}
% use \left. before


%\newcommand{\bean}{\begin{multline*}}
%\newcommand{\eean}{\end{multline*}}

\newcommand{\nl}{\newline}
\newcommand{\np}{\newpage}

\newcommand{\rf}[1]{(\ref{#1})}

\newcommand{\1}{\mathds{1}}

\newcommand{\ds}{\displaystyle}
\newcommand{\fn}{\footnote}

\newcommand{\var}{\mbox{Var}}
\newcommand{\cov}{\mbox{Cov}}

\newcommand{\il}{\int\limits}

\newcommand{\li}{\left}
\newcommand{\re}{\right}

\newcommand{\s}{\right|_}

\newcommand{\ambigo}{\ba{c}>\\[-3mm]<\ea}
\newcommand{\ambigu}{\ba{c}<\\[-3mm]>\ea}

\newcommand{\ul}{\underline}
\newcommand{\ol}{\overline}

\newcommand{\e}{\mbox{\euro{} }}
 otherwise it may cause bug

%\documentclass[xcolor=dvipsnames,mathserif]{beamer} % this option has curvier math
%\documentclass[xcolor=dvipsnames,11pt]{beamer}
% Note: the color structure needs to be added here in the title. Now it recognizes all beamer % colors.

%%%%%%%%  PRESENTATION LAYOUT:
\usepackage{appendixnumberbeamer} % this package does not count the appendix pages. /!\ Inconstant behavior, sometimes bug.
\mode<presentation>
{
  \usetheme{Boadilla}
  \usecolortheme{lily} % lily is nice or orchid but not for definition
  \setbeamercovered{invisible}
  \setbeamertemplate{footline}{\raggedleft\insertframenumber~/~\inserttotalframenumber\hspace*{3pt}\vskip3pt} %this command shows frame number, not page number at bottom (means if using overlays, % frame number does not change)
%\setbeamertemplate{footline}[page number]  % this puts only page number on bottom
 \setbeamertemplate{navigation symbols}{t}  % this erases navigation symbols.
 \setbeamersize{text margin left=0.1cm, text margin right=0.1cm} % left=0.2cm
 \setbeamertemplate{frametitle}[default][center]
 \setbeamercolor{frametitle}{fg=black}
 \setbeamerfont{frametitle}{size=\large, series=\bfseries} % Modifies frame title font.
 \setbeamercolor{button}{fg=blue, bg=white}
 \setbeamertemplate{itemize item}[circle]
 \setbeamercolor{itemize item}{fg=black}
 \setbeamercolor{itemize/enumerate body}{fg=black}
 \setbeamerfont{framesubtitle}{series=\mdseries}
}

\renewcommand{\familydefault}{\rmdefault} %Options here are \ttdefault \ssdefault or \rmdefault.

% This defines actual color palette; 
\definecolor{blue}{RGB}{0,114,178}
\definecolor{orangered}{RGB}{213,94,0}
\definecolor{yellow}{RGB}{240,228,66}
\definecolor{green}{RGB}{0,158,115}
\definecolor{orange}{RGB}{230,159,0}

\hypersetup{
  colorlinks = false,
  linkbordercolor = {white},
  linkcolor = {blue}
}

%%% Color customization: where these colors are used. 
\colorlet{cwords}{blue} %this defines a color, stored under the name cwords that will be used and recognized in document.
\colorlet{cwordsc}{red} %color for contrast with some other word.
\colorlet{cwords2}{green} %2nd color for contrast with some other word.
\colorlet{cmath}{blue} %color for math in text.
%\everymath{\color{blue}} % This in conjunction with the everysel package sets color of math
\everydisplay{\color{blue}}


%%%%%%%%  PACKAGES USED:
\usepackage{amsmath}
\usepackage{setspace} % Only needed for spacing
\usepackage{changepage} % Only needed for local margin setting
\usepackage{mathpazo}% font, is overwritten by times
%\usepackage[hypertexnames=false]{hyperref} %This makes hyperref ``dumber'', and, hence, more robust! (otherwise sometimes the appendix links don't work).
\usepackage{hyperref}
\usepackage{multimedia}
\usepackage[english]{babel}
\usepackage{graphicx}
\usepackage[justification=centering]{caption}
%\usepackage{subfig}
\usepackage{subfloat}
\usepackage[en-US]{datetime2}
\usepackage{tabulary}
\usepackage{tabularx}
\usepackage{makecell} % multiline cells in tables
\usepackage{array,booktabs} % Needed for esttab tables according to the "inequality survey."
\newcommand{\sym}[1]{{#1}} % for symbols in Table
\usepackage[T1]{fontenc}
\usepackage[utf8]{inputenc}
\usepackage{times} % This is a different font.
\usepackage[overlay,absolute]{textpos}
%%%\usepackage{animate} % Animate graphs BUG
\setlength{\TPHorizModule}{1cm}
\setlength{\TPVertModule}{1cm}
\captionsetup[figure]{labelformat=empty} % removes caption prefix figure
\setlength{\itemsep}{\fill} % this is supposed to stretch items across full frame.
%\setlength{\parskip}{0.8\baselineskip} % This affects spacing between normal lines (not itemized).
\usepackage{colortbl} % For cell colors
\usepackage[final]{pdfpages}
\usepackage{caption}
\usepackage{subcaption}
%\captionsetup{justification   = raggedright,
%              singlelinecheck = false}
%\usepackage{adjustbox} % to use resizebox for tables size.
\usepackage[export]{adjustbox} % to use resizebox for tables size.
\usepackage{eurosym}
\usepackage{gensymb}
\usepackage{dsfont}
%\usepackage{enumitem}
%\setlist[itemize]{leftmargin=*} % remove margin from itemize

%% TIKZ
\usepackage{tikz}
% \usetikzlibrary{er, positioning,decorations.pathmorphing,calc}
% \usepackage{tikzscale}
% %% TIKZIT
% \usetikzlibrary{backgrounds}
% \usetikzlibrary{arrows}
% \usetikzlibrary{shapes,shapes.geometric,shapes.misc}
% % this style is applied by default to any tikzpicture included via \tikzfig
% \tikzstyle{tikzfig}=[baseline=-0.25em,scale=0.5]
% % these are dummy properties used by TikZiT, but ignored by LaTex
% \pgfkeys{/tikz/tikzit fill/.initial=0}
% \pgfkeys{/tikz/tikzit draw/.initial=0}
% \pgfkeys{/tikz/tikzit shape/.initial=0}
% \pgfkeys{/tikz/tikzit category/.initial=0}
% % standard layers used in .tikz files
% \pgfdeclarelayer{edgelayer}
% \pgfdeclarelayer{nodelayer}
% \pgfsetlayers{background,edgelayer,nodelayer,main}
% % style for blank nodes
% \tikzstyle{none}=[inner sep=0mm]
% % include a .tikz file
% \newcommand{\tikzfig}[1]{%
% {\tikzstyle{every picture}=[tikzfig]
% \IfFileExists{#1.tikz}
%   {\input{#1.tikz}}
%   {%
%     \IfFileExists{../figures/#1.tikz}
%       {\input{../figures/#1.tikz}}
%       {\tikz[baseline=-0.5em]{\node[draw=red,font=\color{red},fill=red!10!white] {\textit{#1}};}}%
%   }}%
% }
% % the same as \tikzfig, but in a {center} environment
% \newcommand{\ctikzfig}[1]{%
% \begin{center}\rm
%   \tikzfig{#1}
% \end{center}}
% % fix strange self-loops, which are PGF/TikZ default
% \tikzstyle{every loop}=[]
% \newcommand*\halfcirc{%
% %  \begin{tikzpicture}
% %  \draw[fill] (2,1)-- (90:1ex) arc (90:270:1ex) -- cycle ;
% %  \draw (2,1) circle (1ex);
% %  \end{tikzpicture}
% }
  
% \input{../latex/default.tikzstyles}


% \tikzset{every entity/.style={draw=black, fill=white}}
% \tikzset{comment/.style={draw=white, fill=white}}
% \tikzset{
% 	invisible/.style={opacity=0},
% 	visible on/.style={alt=#1{}{invisible}},
% 	alt/.code args={<#1>#2#3}{%
% 		\alt<#1>{\pgfkeysalso{#2}}{\pgfkeysalso{#3}} % \pgfkeysalso doesn't change the path
% 	},
% }

% Commands from template
\newcommand{\alrt}[1]{{\color{alert} #1}}
\newcommand{\alrtl}[1]{{\color{alert}\large #1}}
\newcommand{\alrtL}[1]{{\color{alert}\Large #1}}
\newcommand{\struc}[1]{{\color{structure} #1}}
\newcommand{\strucL}[1]{{\color{structure}\Large #1}}
\newcommand{\strucl}[1]{{\color{structure}\large #1}}
\newcommand{\dred}[1]{{\color{darkred} #1}}
\newcommand{\dredl}[1]{{\color{darkred}\large #1}}
\newcommand{\dredL}[1]{{\color{darkred}\Large #1}}
\newcommand{\altc}[1]{{\color{darkgreen}\textbf{#1}}}
\newcommand{\altcl}[1]{{\color{darkgreen}\textbf{\large #1}}}
\newcommand{\altcL}[1]{{\color{darkgreen}\textbf{\Large #1}}}
\newcommand{\hush}{\hushit}
\newcommand{\hushalrt}[1]{\hushit{{\color{alert} #1}}}
\newcommand{\hushalrtl}[1]{\hushit{{\large\color{alert} #1}}}
\newcommand{\hushalrtL}[1]{\hushit{{\Large\color{alert} #1}}}
\newcommand{\hushstruc}[1]{\hushit{{\color{structure} #1}}}
\newcommand{\hushstrucl}[1]{\hushit{{\large\color{structure} #1}}}
\newcommand{\hushstrucL}[1]{\hushit{{\Large\color{structure} #1}}}


\setbeamertemplate{caption}[numbered]

%%%%%%% SIGN NOBEL PRIZE %%%%%%%%
\usepackage{wasysym}
\newcommand{\nobel}{$\taurus$} % other possibilities:\logof\odot\varocircle\taurus\kreuz
\newcommand{\Nobel}[1]{$\taurus_{#1}$}
\newcommand{\Leontief}[1]{$\logof_{#1}$}
\newcommand{\leontief}{$\logof_$}

%%%%%%%%%SECTION TITLES DISPLAYED ON FULL PAGE %%%%%%%%%%
%%%%%%%%%%%%%%%%%%%%%%%%%%%%%%%%%%%%%%%%%%%
\AtBeginSection[]{
  \begin{frame}
  \vfill
  \centering
  \begin{beamercolorbox}[sep=8pt,center,shadow=true,rounded=true]{title}
    \usebeamerfont{title}{\huge \color{orangered} \insertsectionhead} \par
  \end{beamercolorbox}
  \vfill
  \end{frame}
}

\AtBeginSubsection[]{
  \begin{frame}
  \vfill
  \centering
  \begin{beamercolorbox}[sep=8pt,center,shadow=true,rounded=true]{title}
    \usebeamerfont{title}{\huge \color{blue} \insertsubsectionhead} \par
  \end{beamercolorbox}
  \vfill
  \end{frame}
}

%
% Custom font for a frame.
%
\usepackage{environ}
\newcommand{\customframefont}[1]{
\setbeamertemplate{itemize/enumerate body begin}{#1}
\setbeamertemplate{itemize/enumerate subbody begin}{#1}
}

\NewEnviron{framefont}[1]{
\customframefont{#1} % for itemize/enumerate
{#1 % For the text outside itemize/enumerate
\BODY
}
\customframefont{\normalsize}
}

%%%%%%%%%%%%%%%%%%%%%%%%%%%%%%%%%%%%%%%%%%%%%%%%%%%%%%%
%%%%%% OTHER PIECES OF TEMPLATE FILE
%%%%%% DELETE ONCE CLEAR THAT NOTHING IS MISSING
%%%%%%%%%%%%%%%%%%%%%%%%%%%%%%%%%%%%%%%%%%%%%%%%%%%%%%%


%\documentclass[aspectratio=169]{beamer} % wide
%\usepackage{amsmath,amsthm,fancyhdr,setspace,graphicx,booktabs,pdflscape}
%\usepackage{geometry}
%\usepackage{etex}
%\usepackage{xcolor,colortbl}
%\usepackage{beamerprosper}
%\usepackage{url}
%\usepackage{enumerate}
%\usepackage{graphicx}
%\usepackage{hyperref}
%\usepackage{multicol}
%\usepackage{caption}
%\usepackage{beamerprosper}
%\usepackage{pgfpages, pdfpages}
%\usepackage{tikz-cd}
%
%\usepackage{tabularx}
%\usepackage{anyfontsize}
%\usepackage{multicol,tabto}
%
%\usepackage{float}
%\usepackage{soul}
%\usepackage{grffile}
%\usepackage{changepage}
%\usepackage{sansmathaccent}
%\pdfmapfile{+sansmathaccent.map}
%
%
%\usetikzlibrary{er,positioning,calc,decorations.pathreplacing}
%
%\mode<presentation>
%\usefonttheme{structuresmallcapsserif}
%\setbeamertemplate{footline}[frame number]{}
%\setbeamertemplate{navigation symbols}{}
%
%%change font
%\usefonttheme{default}
%\setbeamertemplate{footline}[frame number]{}
%\setbeamertemplate{navigation symbols}{}
%
%\newcommand{\fig}[3]{\begin{frame}\frametitle{#2}\centerline{\includegraphics[width=#3in]{#1}}\end{frame}}
%
%\newcommand{\blackslide}[1]{\beamersetaveragebackground{black}\begin{frame}\frametitle{}\end{frame}\beamersetaveragebackground{white}}
%
%% pause commands
%\newcommand{\m}[2]{\begin{frame}\frametitle{#1}{ #2}\end{frame}}
%\newcommand{\mm}[3]{\begin{frame}\frametitle{#1}\uncover<1->{ #2}\uncover<2->{ #3 }\end{frame}}
%\newcommand{\mmm}[4]{\begin{frame}\frametitle{#1}\uncover<1->{ #2}\uncover<2->{ #3 }\uncover<3->{ #4 }\end{frame}}
%\newcommand{\mmmm}[5]{\begin{frame}\frametitle{#1}\uncover<1->{ #2}\uncover<2->{ #3 }\uncover<3->{ #4 }\uncover<4->{ #5 }\end{frame}}
%
%\setlength{\footskip}{24pt}
%
%\newcommand{\ex}{\mathbf{E}}
%\newcommand{\cov}{\mathbb{C}}
%\newcommand{\var}{\mathbb{V}}
%\newcommand{\tu}{\overline{\theta}}
%\newcommand{\vu}{\overline{v}}
%\newcommand{\tl}{\underline{\theta}}
%\newcommand{\ab}{\bar{a}}
%\newcommand{\lb}{\bar{L}}
%\newcommand{\hb}{\bar{H}}
%
%% New colors.
%\definecolor{darkred}{rgb}{0.6,0,0}
%\definecolor{darkblue}{rgb}{.15,.25,.55}
%\definecolor{darkgreen}{rgb}{0,.35,.05}
%\definecolor{ltgreen}{rgb}{0,.05,.8}
%\definecolor{bred}{rgb}{1,0,.05}
%\definecolor{navy}{rgb}{.1,.1,.5}
%
%% from beamer lecture 
%
%\newtheorem{proposition}{Proposition}
%%\newtheorem{theorem}{Theorem}
%
%\newenvironment{changemargin}[2]{%
%    \begin{list}{}{ %
%%        \setlength{\topmargin}{#3}%
%            \setlength{\topsep}{0pt}%
%            \setlength{\leftmargin}{#1}%
%            \setlength{\rightmargin}{#2}%
%            \setlength{\listparindent}{\parindent}%
%            \setlength{\itemindent}{\parindent}%
%            \setlength{\parsep}{\parskip}%
%        }%
%\item[]}{\end{list}}
%
%%Allows us to force a column width in array environment
%\newcolumntype{C}[1]{>{\centering\arraybackslash}p{#1}}
%\newcolumntype{L}[1]{>{\raggedright\arraybackslash}p{#1}}
%\newcolumntype{R}[1]{>{\raggedleft\arraybackslash}p{#1}}


%%SHORTCUTS

%%%%%%%%%%%%%%%%%%%%%%%%%
%% Bullets
%%%%%%%%%%%%%%%%%%%%%%%%%
%% These six lines allows to define "myItemize" environment where nested lists appear at once in overlays
%\let\oldItemize\itemize
%\let\endoldItemize\enditemize
%\newcommand{\myItemize}[1][<1->]{\oldItemize[#1]}
%\def\endmyItemize{\endoldItemize}
%\let\itemize\myItemize
%\let\enditemize\endmyItemize

\newcommand{\p}{\item}
\newcommand{\ip}{\item[]} % Invisible items.
\newcommand{\bb}{\begin{itemize}\itemsep15pt}
\newcommand{\bbs}{\medskip \begin{itemize}[<1->]\itemsep10pt}
\newcommand{\bbsp}{\medskip \begin{itemize}[<+->]\itemsep10pt}% items appear progressively
\newcommand{\bbvs}{
	\settowidth{\leftmargini}{\usebeamertemplate{itemize item}} % Remove identation of itemize 
	\medskip \begin{itemize}[<1->]\itemsep3pt  
	\setlength{\leftmargini}{0.4cm} % to adjust left margin of items
 	\setlength{\leftmarginii}{0.3cm} % same for nested items \setlength{\leftmarginiii}{0cm}
} %  \setlength\itemsep{3pt}
\newcommand{\bbvsp}{ % items appear progressively
	\settowidth{\leftmargini}{\usebeamertemplate{itemize item}}
	\medskip \begin{itemize}[<+->]\itemsep3pt % replace myItemize by itemize to have nested items also appear one by one
	\setlength{\leftmargini}{0.4cm} 
 	\setlength{\leftmarginii}{0.3cm}
}
\newcommand{\bbvsn}{ % no margin
	\settowidth{\leftmargini}{\usebeamertemplate{itemize item}} %  
	\medskip \begin{itemize}[<1->]\itemsep3pt  
	\setlength{\leftmargini}{0.4cm}
 	\setlength{\leftmarginii}{-0cm} 
} %  \setlength\itemsep{3pt}
\newcommand{\bbvsnp}{ % items appear progressively
	\settowidth{\leftmargini}{\usebeamertemplate{itemize item}}
	\medskip \begin{itemize}[<+->]\itemsep3pt
	\setlength{\leftmargini}{0.4cm} 
 	\setlength{\leftmarginii}{-0cm}
}
\newcommand{\ee}{\end{itemize} 
\smallskip}
\newcommand{\ees}{\end{itemize} }

\newcommand{\ben}{\begin{enumerate}}
\newcommand{\een}{\end{enumerate}}

\newcommand{\gras}[1]{\textbf{#1}}
\newcommand{\blue}[1]{\textcolor{blue}{#1}}
\newcommand{\green}[1]{\textcolor{green}{#1}}
\newcommand{\red}[1]{\textcolor{red}{#1}}
\newcommand{\orangered}[1]{\textcolor{orangered}{#1}}
\newcommand{\orange}[1]{\textcolor{orange}{#1}}
\newcommand{\yellow}[1]{\textcolor{yellow}{#1}}
\newcommand{\magenta}[1]{\textcolor{magenta}{#1}}
\newcommand{\rose}[1]{\textcolor{magenta}{#1}}

\newcommand{\can}{\citeasnoun}
\newcommand{\ican}{\iciteasnoun}

\newcommand{\non}{\nonumber}

%%%%%%%%%%%%%%%%%%%%%%%%%
%% Derivatives and partials. 
%%%%%%%%%%%%%%%%%%%%%%%%%
%% Duplicate: two ways to get partials
\newcommand{\pa}[2]{\frac{\partial #1}{\partial #2}} % Stef
\newcommand{\pder}[2]{\frac{\partial #1}{\partial #2}} % Doug

\newcommand{\dneu}{\mbox{d}}
\newcommand{\di}[2]{\frac{\dneu #1}{\dneu #2}}
%% Duplicate: two ways to get derivatives.
\newcommand{\dd}[2]{\frac{d #1}{d #2}} % Stef version
\newcommand{\der}[2]{\frac{d#1}{d#2}} % Doug version

%%%%%%%%%%%%%%%%%%%%%%%%%
%% Brackets and fractions
%%%%%%%%%%%%%%%%%%%%%%%%%


\newcommand{\fr}[2]{\frac{#1}{#2}}
\newcommand{\pfr}[2]{\left(\frac{#1}{#2}\right)}
\newcommand{\bfr}[2]{\left[\frac{#1}{#2}\right]}
\newcommand{\cfr}[2]{\left\{\frac{#1}{#2}\right\}}

\newcommand{\pr}[1]{\left(#1\right)}
\newcommand{\br}[1]{\left[#1\right]}
\newcommand{\cb}[1]{\left\{#1\right\}}
\newcommand{\qand}{\quad\text{and}\quad}


%%%%%%%%%%%%%%%%%%%%%%%%%
%% ARROWS
%%%%%%%%%%%%%%%%%%%%%%%%%

\newcommand{\Ra}{\Rightarrow}
\newcommand{\ra}{\rightarrow}
\newcommand{\Ras}{\ \Rightarrow\ }
\newcommand{\ras}{\ \rightarrow\ }
\newcommand{\Raq}{\quad\Rightarrow\quad}
\newcommand{\raq}{\quad\rightarrow\quad}


%%%%%%%%%%%%%%%%%%%%%%%%%
%% MATH
%%%%%%%%%%%%%%%%%%%%%%%%%

\newcommand{\E}[1]{\mathbb{E}\br{#1}}
\newcommand{\eps}{\varepsilon}

\newcommand{\be}{\begin{equation}}
\newcommand{\eeq}{\end{equation}}

\newcommand{\bea}{\begin{eqnarray}}
\newcommand{\eea}{\end{eqnarray}}

\newcommand{\bean}{\begin{eqnarray*}}
\newcommand{\eean}{\end{eqnarray*}}

\newcommand{\ba}{\begin{array}}
\newcommand{\ea}{\end{array}}


\newcommand{\lb}{\linebreak}
\newcommand{\strich}[2]{\left. #1 \right|_{#2}}
% use \left. before


%\newcommand{\bean}{\begin{multline*}}
%\newcommand{\eean}{\end{multline*}}

\newcommand{\nl}{\newline}
\newcommand{\np}{\newpage}

\newcommand{\rf}[1]{(\ref{#1})}

\newcommand{\1}{\mathds{1}}

\newcommand{\ds}{\displaystyle}
\newcommand{\fn}{\footnote}

\newcommand{\var}{\mbox{Var}}
\newcommand{\cov}{\mbox{Cov}}

\newcommand{\il}{\int\limits}

\newcommand{\li}{\left}
\newcommand{\re}{\right}

\newcommand{\s}{\right|_}

\newcommand{\ambigo}{\ba{c}>\\[-3mm]<\ea}
\newcommand{\ambigu}{\ba{c}<\\[-3mm]>\ea}

\newcommand{\ul}{\underline}
\newcommand{\ol}{\overline}

\newcommand{\e}{\mbox{\euro{} }}
 otherwise it may cause bug

%\documentclass[xcolor=dvipsnames,mathserif]{beamer} % this option has curvier math
%\documentclass[xcolor=dvipsnames,11pt]{beamer}
% Note: the color structure needs to be added here in the title. Now it recognizes all beamer % colors.

%%%%%%%%  PRESENTATION LAYOUT:
\usepackage{appendixnumberbeamer} % this package does not count the appendix pages. /!\ Inconstant behavior, sometimes bug.
\mode<presentation>
{
  \usetheme{Boadilla}
  \usecolortheme{lily} % lily is nice or orchid but not for definition
  \setbeamercovered{invisible}
  \setbeamertemplate{footline}{\raggedleft\insertframenumber~/~\inserttotalframenumber\hspace*{3pt}\vskip3pt} %this command shows frame number, not page number at bottom (means if using overlays, % frame number does not change)
%\setbeamertemplate{footline}[page number]  % this puts only page number on bottom
 \setbeamertemplate{navigation symbols}{t}  % this erases navigation symbols.
 \setbeamersize{text margin left=0.1cm, text margin right=0.1cm} % left=0.2cm
 \setbeamertemplate{frametitle}[default][center]
 \setbeamercolor{frametitle}{fg=black}
 \setbeamerfont{frametitle}{size=\large, series=\bfseries} % Modifies frame title font.
 \setbeamercolor{button}{fg=blue, bg=white}
 \setbeamertemplate{itemize item}[circle]
 \setbeamercolor{itemize item}{fg=black}
 \setbeamercolor{itemize/enumerate body}{fg=black}
 \setbeamerfont{framesubtitle}{series=\mdseries}
}

\renewcommand{\familydefault}{\rmdefault} %Options here are \ttdefault \ssdefault or \rmdefault.

% This defines actual color palette; 
\definecolor{blue}{RGB}{0,114,178}
\definecolor{orangered}{RGB}{213,94,0}
\definecolor{yellow}{RGB}{240,228,66}
\definecolor{green}{RGB}{0,158,115}
\definecolor{orange}{RGB}{230,159,0}

\hypersetup{
  colorlinks = false,
  linkbordercolor = {white},
  linkcolor = {blue}
}

%%% Color customization: where these colors are used. 
\colorlet{cwords}{blue} %this defines a color, stored under the name cwords that will be used and recognized in document.
\colorlet{cwordsc}{red} %color for contrast with some other word.
\colorlet{cwords2}{green} %2nd color for contrast with some other word.
\colorlet{cmath}{blue} %color for math in text.
%\everymath{\color{blue}} % This in conjunction with the everysel package sets color of math
\everydisplay{\color{blue}}


%%%%%%%%  PACKAGES USED:
\usepackage{amsmath}
\usepackage{setspace} % Only needed for spacing
\usepackage{changepage} % Only needed for local margin setting
\usepackage{mathpazo}% font, is overwritten by times
%\usepackage[hypertexnames=false]{hyperref} %This makes hyperref ``dumber'', and, hence, more robust! (otherwise sometimes the appendix links don't work).
\usepackage{hyperref}
\usepackage{multimedia}
\usepackage[english]{babel}
\usepackage{graphicx}
\usepackage[justification=centering]{caption}
%\usepackage{subfig}
\usepackage{subfloat}
\usepackage[en-US]{datetime2}
\usepackage{tabulary}
\usepackage{tabularx}
\usepackage{makecell} % multiline cells in tables
\usepackage{array,booktabs} % Needed for esttab tables according to the "inequality survey."
\newcommand{\sym}[1]{{#1}} % for symbols in Table
\usepackage[T1]{fontenc}
\usepackage[utf8]{inputenc}
\usepackage{times} % This is a different font.
\usepackage[overlay,absolute]{textpos}
%%%\usepackage{animate} % Animate graphs BUG
\setlength{\TPHorizModule}{1cm}
\setlength{\TPVertModule}{1cm}
\captionsetup[figure]{labelformat=empty} % removes caption prefix figure
\setlength{\itemsep}{\fill} % this is supposed to stretch items across full frame.
%\setlength{\parskip}{0.8\baselineskip} % This affects spacing between normal lines (not itemized).
\usepackage{colortbl} % For cell colors
\usepackage[final]{pdfpages}
\usepackage{caption}
\usepackage{subcaption}
%\captionsetup{justification   = raggedright,
%              singlelinecheck = false}
%\usepackage{adjustbox} % to use resizebox for tables size.
\usepackage[export]{adjustbox} % to use resizebox for tables size.
\usepackage{eurosym}
\usepackage{gensymb}
\usepackage{dsfont}
%\usepackage{enumitem}
%\setlist[itemize]{leftmargin=*} % remove margin from itemize

% To embed videos: none of these work
% /!\ You must place StrobeMediaPlayback.swf in the same folder
% \usepackage{multimedia}%
% \usepackage{media9}%
% \newcommand{\includemovie}[3]{%
% \includemedia[%
% width=#1,height=#2,%
% activate=pagevisible,%
% deactivate=pageclose,%
% addresource=#3,%
% flashvars={%
% src=#3 % same path as in addresource!
% &autoPlay=false % default: false; if =true, automatically starts playback after activation (see option ‘activation)’
% &loop=false % if loop=true, media is played in a loop
% &controlBarAutoHideTimeout=0 %  time span before auto-hide
% }%
% ]{}{StrobeMediaPlayback.swf}%
% }

%% TIKZ
\usepackage{tikz}
% \usetikzlibrary{er, positioning,decorations.pathmorphing,calc}
% \usepackage{tikzscale}
% %% TIKZIT
% \usetikzlibrary{backgrounds}
% \usetikzlibrary{arrows}
% \usetikzlibrary{shapes,shapes.geometric,shapes.misc}
% % this style is applied by default to any tikzpicture included via \tikzfig
% \tikzstyle{tikzfig}=[baseline=-0.25em,scale=0.5]
% % these are dummy properties used by TikZiT, but ignored by LaTex
% \pgfkeys{/tikz/tikzit fill/.initial=0}
% \pgfkeys{/tikz/tikzit draw/.initial=0}
% \pgfkeys{/tikz/tikzit shape/.initial=0}
% \pgfkeys{/tikz/tikzit category/.initial=0}
% % standard layers used in .tikz files
% \pgfdeclarelayer{edgelayer}
% \pgfdeclarelayer{nodelayer}
% \pgfsetlayers{background,edgelayer,nodelayer,main}
% % style for blank nodes
% \tikzstyle{none}=[inner sep=0mm]
% % include a .tikz file
% \newcommand{\tikzfig}[1]{%
% {\tikzstyle{every picture}=[tikzfig]
% \IfFileExists{#1.tikz}
%   {\input{#1.tikz}}
%   {%
%     \IfFileExists{../figures/#1.tikz}
%       {\input{../figures/#1.tikz}}
%       {\tikz[baseline=-0.5em]{\node[draw=red,font=\color{red},fill=red!10!white] {\textit{#1}};}}%
%   }}%
% }
% % the same as \tikzfig, but in a {center} environment
% \newcommand{\ctikzfig}[1]{%
% \begin{center}\rm
%   \tikzfig{#1}
% \end{center}}
% % fix strange self-loops, which are PGF/TikZ default
% \tikzstyle{every loop}=[]
% \newcommand*\halfcirc{%
% %  \begin{tikzpicture}
% %  \draw[fill] (2,1)-- (90:1ex) arc (90:270:1ex) -- cycle ;
% %  \draw (2,1) circle (1ex);
% %  \end{tikzpicture}
% }
  
% \input{../latex/default.tikzstyles}


% \tikzset{every entity/.style={draw=black, fill=white}}
% \tikzset{comment/.style={draw=white, fill=white}}
% \tikzset{
% 	invisible/.style={opacity=0},
% 	visible on/.style={alt=#1{}{invisible}},
% 	alt/.code args={<#1>#2#3}{%
% 		\alt<#1>{\pgfkeysalso{#2}}{\pgfkeysalso{#3}} % \pgfkeysalso doesn't change the path
% 	},
% }

% Commands from template
\newcommand{\alrt}[1]{{\color{alert} #1}}
\newcommand{\alrtl}[1]{{\color{alert}\large #1}}
\newcommand{\alrtL}[1]{{\color{alert}\Large #1}}
\newcommand{\struc}[1]{{\color{structure} #1}}
\newcommand{\strucL}[1]{{\color{structure}\Large #1}}
\newcommand{\strucl}[1]{{\color{structure}\large #1}}
\newcommand{\dred}[1]{{\color{darkred} #1}}
\newcommand{\dredl}[1]{{\color{darkred}\large #1}}
\newcommand{\dredL}[1]{{\color{darkred}\Large #1}}
\newcommand{\altc}[1]{{\color{darkgreen}\textbf{#1}}}
\newcommand{\altcl}[1]{{\color{darkgreen}\textbf{\large #1}}}
\newcommand{\altcL}[1]{{\color{darkgreen}\textbf{\Large #1}}}
\newcommand{\hush}{\hushit}
\newcommand{\hushalrt}[1]{\hushit{{\color{alert} #1}}}
\newcommand{\hushalrtl}[1]{\hushit{{\large\color{alert} #1}}}
\newcommand{\hushalrtL}[1]{\hushit{{\Large\color{alert} #1}}}
\newcommand{\hushstruc}[1]{\hushit{{\color{structure} #1}}}
\newcommand{\hushstrucl}[1]{\hushit{{\large\color{structure} #1}}}
\newcommand{\hushstrucL}[1]{\hushit{{\Large\color{structure} #1}}}


\setbeamertemplate{caption}[numbered]

%%%%%%% SIGN NOBEL PRIZE %%%%%%%%
\usepackage{wasysym}
\newcommand{\nobel}{$\taurus$} % other possibilities:\logof\odot\varocircle\taurus\kreuz
\newcommand{\Nobel}[1]{$\taurus_{#1}$}
\newcommand{\Leontief}[1]{$\logof_{#1}$}
\newcommand{\leontief}{$\logof_$}

%%%%%%%%%SECTION TITLES DISPLAYED ON FULL PAGE %%%%%%%%%%
%%%%%%%%%%%%%%%%%%%%%%%%%%%%%%%%%%%%%%%%%%%
\AtBeginSection[]{
  \begin{frame}
  \vfill
  \centering
  \begin{beamercolorbox}[sep=8pt,center,shadow=true,rounded=true]{title}
    \usebeamerfont{title}{\huge \color{orangered} \insertsectionhead} \par
  \end{beamercolorbox}
  \vfill
  \end{frame}
}

\AtBeginSubsection[]{
  \begin{frame}
  \vfill
  \centering
  \begin{beamercolorbox}[sep=8pt,center,shadow=true,rounded=true]{title}
    \usebeamerfont{title}{\huge \color{blue} \insertsubsectionhead} \par
  \end{beamercolorbox}
  \vfill
  \end{frame}
}

%
% Custom font for a frame.
%
\usepackage{environ}
\newcommand{\customframefont}[1]{
\setbeamertemplate{itemize/enumerate body begin}{#1}
\setbeamertemplate{itemize/enumerate subbody begin}{#1}
}

\NewEnviron{framefont}[1]{
\customframefont{#1} % for itemize/enumerate
{#1 % For the text outside itemize/enumerate
\BODY
}
\customframefont{\normalsize}
}

%%%%%%%%%%%%%%%%%%%%%%%%%%%%%%%%%%%%%%%%%%%%%%%%%%%%%%%
%%%%%% OTHER PIECES OF TEMPLATE FILE
%%%%%% DELETE ONCE CLEAR THAT NOTHING IS MISSING
%%%%%%%%%%%%%%%%%%%%%%%%%%%%%%%%%%%%%%%%%%%%%%%%%%%%%%%


%\documentclass[aspectratio=169]{beamer} % wide
%\usepackage{amsmath,amsthm,fancyhdr,setspace,graphicx,booktabs,pdflscape}
%\usepackage{geometry}
%\usepackage{etex}
%\usepackage{xcolor,colortbl}
%\usepackage{beamerprosper}
%\usepackage{url}
%\usepackage{enumerate}
%\usepackage{graphicx}
%\usepackage{hyperref}
%\usepackage{multicol}
%\usepackage{caption}
%\usepackage{beamerprosper}
%\usepackage{pgfpages, pdfpages}
%\usepackage{tikz-cd}
%
%\usepackage{tabularx}
%\usepackage{anyfontsize}
%\usepackage{multicol,tabto}
%
%\usepackage{float}
%\usepackage{soul}
%\usepackage{grffile}
%\usepackage{changepage}
%\usepackage{sansmathaccent}
%\pdfmapfile{+sansmathaccent.map}
%
%
%\usetikzlibrary{er,positioning,calc,decorations.pathreplacing}
%
%\mode<presentation>
%\usefonttheme{structuresmallcapsserif}
%\setbeamertemplate{footline}[frame number]{}
%\setbeamertemplate{navigation symbols}{}
%
%%change font
%\usefonttheme{default}
%\setbeamertemplate{footline}[frame number]{}
%\setbeamertemplate{navigation symbols}{}
%
%\newcommand{\fig}[3]{\begin{frame}\frametitle{#2}\centerline{\includegraphics[width=#3in]{#1}}\end{frame}}
%
%\newcommand{\blackslide}[1]{\beamersetaveragebackground{black}\begin{frame}\frametitle{}\end{frame}\beamersetaveragebackground{white}}
%
%% pause commands
%\newcommand{\m}[2]{\begin{frame}\frametitle{#1}{ #2}\end{frame}}
%\newcommand{\mm}[3]{\begin{frame}\frametitle{#1}\uncover<1->{ #2}\uncover<2->{ #3 }\end{frame}}
%\newcommand{\mmm}[4]{\begin{frame}\frametitle{#1}\uncover<1->{ #2}\uncover<2->{ #3 }\uncover<3->{ #4 }\end{frame}}
%\newcommand{\mmmm}[5]{\begin{frame}\frametitle{#1}\uncover<1->{ #2}\uncover<2->{ #3 }\uncover<3->{ #4 }\uncover<4->{ #5 }\end{frame}}
%
%\setlength{\footskip}{24pt}
%
%\newcommand{\ex}{\mathbf{E}}
%\newcommand{\cov}{\mathbb{C}}
%\newcommand{\var}{\mathbb{V}}
%\newcommand{\tu}{\overline{\theta}}
%\newcommand{\vu}{\overline{v}}
%\newcommand{\tl}{\underline{\theta}}
%\newcommand{\ab}{\bar{a}}
%\newcommand{\lb}{\bar{L}}
%\newcommand{\hb}{\bar{H}}
%
%% New colors.
%\definecolor{darkred}{rgb}{0.6,0,0}
%\definecolor{darkblue}{rgb}{.15,.25,.55}
%\definecolor{darkgreen}{rgb}{0,.35,.05}
%\definecolor{ltgreen}{rgb}{0,.05,.8}
%\definecolor{bred}{rgb}{1,0,.05}
%\definecolor{navy}{rgb}{.1,.1,.5}
%
%% from beamer lecture 
%
%\newtheorem{proposition}{Proposition}
%%\newtheorem{theorem}{Theorem}
%
%\newenvironment{changemargin}[2]{%
%    \begin{list}{}{ %
%%        \setlength{\topmargin}{#3}%
%            \setlength{\topsep}{0pt}%
%            \setlength{\leftmargin}{#1}%
%            \setlength{\rightmargin}{#2}%
%            \setlength{\listparindent}{\parindent}%
%            \setlength{\itemindent}{\parindent}%
%            \setlength{\parsep}{\parskip}%
%        }%
%\item[]}{\end{list}}
%
%%Allows us to force a column width in array environment
%\newcolumntype{C}[1]{>{\centering\arraybackslash}p{#1}}
%\newcolumntype{L}[1]{>{\raggedright\arraybackslash}p{#1}}
%\newcolumntype{R}[1]{>{\raggedleft\arraybackslash}p{#1}}


%%SHORTCUTS

%%%%%%%%%%%%%%%%%%%%%%%%%
%% Bullets
%%%%%%%%%%%%%%%%%%%%%%%%%
%% These six lines allows to define "myItemize" environment where nested lists appear at once in overlays
%\let\oldItemize\itemize
%\let\endoldItemize\enditemize
%\newcommand{\myItemize}[1][<1->]{\oldItemize[#1]}
%\def\endmyItemize{\endoldItemize}
%\let\itemize\myItemize
%\let\enditemize\endmyItemize

\newcommand{\p}{\item}
\newcommand{\ip}{\item[]} % Invisible items.
\newcommand{\bb}{\begin{itemize}\itemsep15pt}
\newcommand{\bbs}{\medskip \begin{itemize}[<1->]\itemsep10pt}
\newcommand{\bbsp}{\medskip \begin{itemize}[<+->]\itemsep10pt}% items appear progressively
\newcommand{\bbvs}{
	\settowidth{\leftmargini}{\usebeamertemplate{itemize item}} % Remove identation of itemize 
	\medskip \begin{itemize}[<1->]\itemsep3pt  
	\setlength{\leftmargini}{0.4cm} % to adjust left margin of items
 	\setlength{\leftmarginii}{0.3cm} % same for nested items \setlength{\leftmarginiii}{0cm}
} %  \setlength\itemsep{3pt}
\newcommand{\bbvsp}{ % items appear progressively
	\settowidth{\leftmargini}{\usebeamertemplate{itemize item}}
	\medskip \begin{itemize}[<+->]\itemsep3pt % replace myItemize by itemize to have nested items also appear one by one
	\setlength{\leftmargini}{0.4cm} 
 	\setlength{\leftmarginii}{0.3cm}
}
\newcommand{\bbvsn}{ % no margin
	\settowidth{\leftmargini}{\usebeamertemplate{itemize item}} %  
	\medskip \begin{itemize}[<1->]\itemsep3pt  
	\setlength{\leftmargini}{0.4cm}
 	\setlength{\leftmarginii}{-0cm} 
} %  \setlength\itemsep{3pt}
\newcommand{\bbvsnp}{ % items appear progressively
	\settowidth{\leftmargini}{\usebeamertemplate{itemize item}}
	\medskip \begin{itemize}[<+->]\itemsep3pt
	\setlength{\leftmargini}{0.4cm} 
 	\setlength{\leftmarginii}{-0cm}
}
\newcommand{\ee}{\end{itemize} 
\smallskip}
\newcommand{\ees}{\end{itemize} }

\newcommand{\ben}{\begin{enumerate}}
\newcommand{\een}{\end{enumerate}}

\newcommand{\gras}[1]{\textbf{#1}}
\newcommand{\blue}[1]{\textcolor{blue}{#1}}
\newcommand{\green}[1]{\textcolor{green}{#1}}
\newcommand{\red}[1]{\textcolor{red}{#1}}
\newcommand{\orangered}[1]{\textcolor{orangered}{#1}}
\newcommand{\orange}[1]{\textcolor{orange}{#1}}
\newcommand{\yellow}[1]{\textcolor{yellow}{#1}}
\newcommand{\magenta}[1]{\textcolor{magenta}{#1}}
\newcommand{\rose}[1]{\textcolor{magenta}{#1}}

\newcommand{\can}{\citeasnoun}
\newcommand{\ican}{\iciteasnoun}

\newcommand{\non}{\nonumber}

%%%%%%%%%%%%%%%%%%%%%%%%%
%% Derivatives and partials. 
%%%%%%%%%%%%%%%%%%%%%%%%%
%% Duplicate: two ways to get partials
\newcommand{\pa}[2]{\frac{\partial #1}{\partial #2}} % Stef
\newcommand{\pder}[2]{\frac{\partial #1}{\partial #2}} % Doug

\newcommand{\dneu}{\mbox{d}}
\newcommand{\di}[2]{\frac{\dneu #1}{\dneu #2}}
%% Duplicate: two ways to get derivatives.
\newcommand{\dd}[2]{\frac{d #1}{d #2}} % Stef version
\newcommand{\der}[2]{\frac{d#1}{d#2}} % Doug version

%%%%%%%%%%%%%%%%%%%%%%%%%
%% Brackets and fractions
%%%%%%%%%%%%%%%%%%%%%%%%%


\newcommand{\fr}[2]{\frac{#1}{#2}}
\newcommand{\pfr}[2]{\left(\frac{#1}{#2}\right)}
\newcommand{\bfr}[2]{\left[\frac{#1}{#2}\right]}
\newcommand{\cfr}[2]{\left\{\frac{#1}{#2}\right\}}

\newcommand{\pr}[1]{\left(#1\right)}
\newcommand{\br}[1]{\left[#1\right]}
\newcommand{\cb}[1]{\left\{#1\right\}}
\newcommand{\qand}{\quad\text{and}\quad}


%%%%%%%%%%%%%%%%%%%%%%%%%
%% ARROWS
%%%%%%%%%%%%%%%%%%%%%%%%%

\newcommand{\Ra}{\Rightarrow}
\newcommand{\ra}{\rightarrow}
\newcommand{\Ras}{\ \Rightarrow\ }
\newcommand{\ras}{\ \rightarrow\ }
\newcommand{\Raq}{\quad\Rightarrow\quad}
\newcommand{\raq}{\quad\rightarrow\quad}


%%%%%%%%%%%%%%%%%%%%%%%%%
%% MATH
%%%%%%%%%%%%%%%%%%%%%%%%%

\newcommand{\E}[1]{\mathbb{E}\br{#1}}
\newcommand{\eps}{\varepsilon}

\newcommand{\be}{\begin{equation}}
\newcommand{\eeq}{\end{equation}}

\newcommand{\bea}{\begin{eqnarray}}
\newcommand{\eea}{\end{eqnarray}}

\newcommand{\bean}{\begin{eqnarray*}}
\newcommand{\eean}{\end{eqnarray*}}

\newcommand{\ba}{\begin{array}}
\newcommand{\ea}{\end{array}}


\newcommand{\lb}{\linebreak}
\newcommand{\strich}[2]{\left. #1 \right|_{#2}}
% use \left. before


%\newcommand{\bean}{\begin{multline*}}
%\newcommand{\eean}{\end{multline*}}

\newcommand{\nl}{\newline}
\newcommand{\np}{\newpage}

\newcommand{\rf}[1]{(\ref{#1})}

\newcommand{\1}{\mathds{1}}

\newcommand{\ds}{\displaystyle}
\newcommand{\fn}{\footnote}

\newcommand{\var}{\mbox{Var}}
\newcommand{\cov}{\mbox{Cov}}

\newcommand{\il}{\int\limits}

\newcommand{\li}{\left}
\newcommand{\re}{\right}

\newcommand{\s}{\right|_}

\newcommand{\ambigo}{\ba{c}>\\[-3mm]<\ea}
\newcommand{\ambigu}{\ba{c}<\\[-3mm]>\ea}

\newcommand{\ul}{\underline}
\newcommand{\ol}{\overline}

\newcommand{\e}{\mbox{\euro{} }}
